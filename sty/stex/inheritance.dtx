% \iffalse meta-comment
% An Infrastructure for Semantic Macros and Module Scoping
% Copyright (c) 2019 Michael Kohlhase, all rights reserved
%                this file is released under the
%                LaTeX Project Public License (LPPL)
% 
% The original of this file is in the public repository at 
% http://github.com/sLaTeX/sTeX/
%
% TODO update copyright  
%
%<*driver>
\providecommand\bibfolder{../../lib/bib}
\RequirePackage{paralist}
\documentclass[full,kernel]{l3doc}
\usepackage[dvipsnames]{xcolor}
\usepackage[utf8]{inputenc}
\usepackage[T1]{fontenc}
%\usepackage{document-structure}
\usepackage[showmods,debug=all,lang={de,en}, mathhub=./tests]{stex}
\usepackage{url,array,float,textcomp}
\usepackage[show]{ed}
\usepackage[hyperref=auto,style=alphabetic]{biblatex}
\addbibresource{\bibfolder/kwarcpubs.bib}
\addbibresource{\bibfolder/extpubs.bib}
\addbibresource{\bibfolder/kwarccrossrefs.bib}
\addbibresource{\bibfolder/extcrossrefs.bib}
\usepackage{amssymb}
\usepackage{amsfonts}
\usepackage{xspace}
\usepackage{hyperref}

\usepackage{morewrites}


\makeindex
\floatstyle{boxed}
\newfloat{exfig}{thp}{lop}
\floatname{exfig}{Example}

\usepackage{stex-tests}

\MakeShortVerb{\|}

\def\scsys#1{{{\sc #1}}\index{#1@{\sc #1}}\xspace}
\def\mmt{\textsc{Mmt}\xspace}
\def\xml{\scsys{Xml}}
\def\mathml{\scsys{MathML}}
\def\omdoc{\scsys{OMDoc}}
\def\openmath{\scsys{OpenMath}}
\def\latexml{\scsys{LaTeXML}}
\def\perl{\scsys{Perl}}
\def\cmathml{Content-{\sc MathML}\index{Content {\sc MathML}}\index{MathML@{\sc MathML}!content}}
\def\activemath{\scsys{ActiveMath}}
\def\twin#1#2{\index{#1!#2}\index{#2!#1}}
\def\twintoo#1#2{{#1 #2}\twin{#1}{#2}}
\def\atwin#1#2#3{\index{#1!#2!#3}\index{#3!#2 (#1)}}
\def\atwintoo#1#2#3{{#1 #2 #3}\atwin{#1}{#2}{#3}}
\def\cT{\mathcal{T}}\def\cD{\mathcal{D}}

\def\fileversion{3.0}
\def\filedate{\today}

\RequirePackage{pdfcomment}

\ExplSyntaxOn\makeatletter
\cs_set_protected:Npn \@comp #1 #2 {
  \pdftooltip {
    \textcolor{blue}{#1}
  } { #2 }
}

\cs_set_protected:Npn \@defemph #1 #2 {
  \pdftooltip { 
    \textbf{\textcolor{magenta}{#1}}
  } { #2 }
}
\makeatother\ExplSyntaxOff

\begin{document}
  \DocInput{\jobname.dtx}
\end{document}
%</driver>
% \fi
%
% \title{ \sTeX-Module Inheritance
% 	\thanks{Version {\fileversion} (last revised {\filedate})} 
% }
%
% \author{Michael Kohlhase, Dennis Müller\\
% 	FAU Erlangen-Nürnberg\\
% 	\url{http://kwarc.info/}
% }
%
% \maketitle
%
% \begin{documentation}\label{pkg:inheritance:doc}
%
% Code related to Module Inheritance, in particular \emph{sms mode}.
%
% \section{Macros and Environments}\label{pkg:inheritance:doc:macros}
%
%    \subsection{SMS Mode}
% ``SMS Mode'' is used when loading modules from external tex files.
% It deactivates any output and ignores all \TeX\ commands
% not explicitly allowed via the following lists:
%
% \begin{variable}{\g_stex_smsmode_allowedmacros_tl}
%   Macros that are executed as is; i.e. with the category code scheme
%   used in SMS mode.
% \end{variable}
%
% \begin{variable}{\g_stex_smsmode_allowedmacros_escape_tl}
%   Macros that are executed with the category codes restored.
%
%   Importantly, these macros need to call \cs{stex_smsmode_set_codes:}
%   after reading all arguments. Note, that 
%   \cs{stex_smsmode_set_codes:} takes
%   care of checking whether we are in SMS mode in the first place, 
%   so calling this function eagerly is unproblematic.
% \end{variable}
%
% \begin{variable}{\g_stex_smsmode_allowedenvs_seq}
%   The names of environments that should be allowed in SMS mode.
%   The corresponding \cs{begin}-statements are treated like
%   the macros in \cs{g_stex_smsmode_allowedmacros_escape_tl}, so
%   \cs{stex_smsmode_set_codes:} should be called at the end of the
%   \cs{begin}-code. Since \cs{end}-statements take no arguments anyway,
%   those are called with the SMS mode category code scheme active.
% \end{variable}
%
% \begin{function}[pTF]{\stex_if_smsmode:}
%   Tests whether SMS mode is currently active.
% \end{function}
%
% \begin{function}{\stex_smsmode_set_codes:}
%   Sets the current category code scheme to that of the SMS mode, if
%   SMS mode is currently active and if necessary.
%
%   This method should be called at the end of every macro or 
%   \cs{begin} environment code that are allowed in SMS mode.
%
% \end{function}
%
% \begin{function}{\stex_in_smsmode:nn}
%   \begin{syntax} \cs{stex_in_smsmode:nn} \Arg{name} \Arg{code} \end{syntax}
%   Executes \meta{code} in SMS mode. \meta{name} can be arbitrary,
%   but should be distinct, since it allows for nesting 
%   \cs{stex_in_smsmode:nn} without spuriously terminating SMS mode.
% \end{function}
%
%\stextest{
% \immediate\openout\testfile=./tests/sometest.tex
% \immediate\write\testfile{\detokenize{\this is \a test}^^J}
% \immediate\write\testfile{\detokenize{this \is a \test}}
% \immediate\closeout\testfile
% \ExplSyntaxOn
% \stex_in_smsmode:nn { foo } { 
%   \this is \a test

this \is a \test 
 
% }
% \ExplSyntaxOff
%}
%
%
%    \subsection{Imports and Inheritance}
%
% \begin{function}{\importmodule}
%   \begin{syntax} \cs{importmodule}|[|\meta{archive-ID}|]|\Arg{module-path} \end{syntax}
%   Imports a module by reading it from a file and ``activating'' it.
%   \sTeX determines the module and its containing file by passing its
%   arguments on to \cs{stex_import_module_path:nn}.
%   
% \end{function}
%
%\stextest{
%   \begin{module}{Foo}
%      \symdecl[name=foo, args=3]{bar}
%      \symdecl[args=bai]{foobar}
%      Meaning:~\present\bar\\
%   \end{module}
%      Meaning:~\present\bar\\
%   \begin{module}{Importtest}
%     \importmodule{Foo}
%      Meaning:~\present\bar\\
%   \end{module}
%   \begin{module}{Importtest2}
%     \importmodule{Importtest}
%      Meaning:~\present\bar\\
%   \end{module}
%}
%
% \begin{function}{\usemodule}
%   \begin{syntax} \cs{importmodule}|[|\meta{archive-ID}|]|\Arg{module-path} \end{syntax}
%   Like \cs{importmodule}, but does not export its contents;
%   i.e. including the current module will not activate the used module
%   
% \end{function}
%
%\stextest{
%   \begin{module}{UseTest1}
%      \symdecl{foo}
%   \end{module}
%   \begin{module}{UseTest2}
%     \usemodule{UseTest1}
%      \symdecl{bar}
%      Meaning:~\present\foo\\
%   \end{module}
%   \begin{module}{UseTest3}
%     \importmodule{UseTest2}
%     Meaning:~\present\foo\\
%     Meaning:~\present\bar\\
%
%     All modules: \ExplSyntaxOn
%       \seq_use:Nn \l_stex_all_modules_seq {,~} \\
%      All~symbols:~
%        \seq_use:Nn \l_stex_all_symbols_seq {,~}
%     \ExplSyntaxOff 
%   \end{module}
%}
%
%
%\stextest{
% Circular dependencies:
% \begin{module}{CircDep1}
%   \importmodule[Foo/Bar]{circular1?Circular1}
%   \importmodule[Bar/Foo]{circular2?Circular2}
%   \present\fooA\\
%   \present\fooB
% \end{module}
%}
%
% \begin{function}{\stex_import_module_uri:nn}
%   \begin{syntax} \cs{stex_import_module_uri:nn} \Arg{archive-ID} \Arg{module-path} \end{syntax}
%   Determines the URI of a module by splitting
%   \meta{module-path} into \meta{path}|?|\meta{name}. If \meta{module-path}
%   does \emph{not} contain a |?|-character, we consider it to be the \meta{name},
%   and \meta{path} to be empty.
%
%   If \meta{archive-ID} is empty, it is automatically set to the
%   ID of the current archive (if one exists).
%
%   \begin{enumerate}
%   \item If \meta{archive-ID} is empty:
%     \begin{enumerate}
%     \item If \meta{path} is empty, then
%         \meta{name} must have been declared earlier in the same file
%         and retrievable from \cs{g_stex_modules_in_file_seq}, or
%         a file with name \meta{name}|.|\meta{lang}|.tex| must exist
%         in the same folder, containing a module \meta{name}.
%
%         That module should have the same namespace as the current one.
%     \item If \meta{path} is not empty, it must point to the relative
%         path of the containing file as well as the namespace.
%     \end{enumerate}
%   \item Otherwise:
%      \begin{enumerate}
%         \item If \meta{path} is empty, then
%         \meta{name} must have been declared earlier in the same file
%         and retrievable from \cs{g_stex_modules_in_file_seq}, or
%         a file with name \meta{name}|.|\meta{lang}|.tex| must exist
%         in the top |source| folder of the archive, 
%         containing a module \meta{name}.
%
%         That module should lie directly in the namespace 
%         of the archive.
%     \item If \meta{path} is not empty, it must point to the
%         path of the containing file as well as the namespace,
%         relative to the namespace of the archive.
%
%         If a module by that namespace exists, it is returned.
%         Otherwise, we call \cs{stex_require_module:nn} 
%         on the |source| directory of the archive to find the
%         file.
%       \end{enumerate}
%   \end{enumerate}
% \end{function}
%
% \begin{function}{\stex_import_require_module:nnnn}
%    \begin{syntax} \Arg{ns} \Arg{archive-ID} \Arg{path} \Arg{name} \end{syntax}
%     Checks whether a module with URI \meta{ns}|?|\meta{name} already
%     exists. If not, it looks for a plausible file that declares
%     a module with that URI.
%
%     Finally, activates that module by executing its |content|-field.
% \end{function}
%
%
% \end{documentation}
%
% \begin{implementation}\label{pkg:inheritance:impl}
%
% \section{\sTeX-Module Inheritance Implementation}
%
%    \begin{macrocode}
%<*package>

%%%%%%%%%%%%%   inheritance.dtx   %%%%%%%%%%%%%

%    \end{macrocode}
%
% \subsection{SMS Mode}
%    \begin{macrocode}
%<@@=stex_smsmode>
%    \end{macrocode}
%
% \begin{variable}{
%   \g_stex_smsmode_allowedmacros_tl,
%   \g_stex_smsmode_allowedmacros_escape_tl,
%   \g_stex_smsmode_allowedenvs_seq
% }
%    \begin{macrocode}
\tl_new:N \g_stex_smsmode_allowedmacros_tl
\tl_new:N \g_stex_smsmode_allowedmacros_escape_tl
\seq_new:N \g_stex_smsmode_allowedenvs_seq

\tl_set:Nn \g_stex_smsmode_allowedmacros_tl {
  \makeatletter
  \makeatother
  \ExplSyntaxOn
  \ExplSyntaxOff
}

\tl_set:Nn \g_stex_smsmode_allowedmacros_escape_tl {
  \symdef
  \importmodule
  \notation
  \symdecl
  \STEXexport
}

\exp_args:NNx \seq_set_from_clist:Nn \g_stex_smsmode_allowedenvs_seq {
  \tl_to_str:n {
    module,
    @module
  }
}
%    \end{macrocode}
% \end{variable}
%
% \begin{macro}[pTF]{\stex_if_smsmode:}
%    \begin{macrocode}
\bool_new:N \g_@@_bool
\bool_set_false:N \g_@@_bool
\prg_new_conditional:Nnn \stex_if_smsmode: { p, T, F, TF } {
  \bool_if:NTF \g_@@_bool \prg_return_true: \prg_return_false:
}
%    \end{macrocode}
% \end{macro}
%
%
% \begin{macro}[pTF]{\_@@_if_catcodes:}
% Checks whether the SMS mode category code scheme is active.
%    \begin{macrocode}
\bool_new:N \g_@@_catcode_bool
\bool_set_false:N \g_@@_catcode_bool
\prg_new_conditional:Nnn \_@@_if_catcodes: { p, T, F, TF } {
  \bool_if:NTF \g_@@_catcode_bool 
    \prg_return_true: \prg_return_false:
}
%    \end{macrocode}
% \end{macro}
%
% \begin{macro}{\stex_smsmode_set_codes:}
%    \begin{macrocode}
\cs_new_protected:Nn \stex_smsmode_set_codes: {
  \stex_if_smsmode:T {
    \_@@_if_catcodes:F {
      \bool_gset_true:N \g_@@_catcode_bool
      \exp_after:wN \char_gset_active_eq:NN 
        \c_backslash_str \_@@_cs:
      \tex_global:D \char_set_catcode_active:N \\
      \tex_global:D \char_set_catcode_other:N $
      \tex_global:D \char_set_catcode_other:N ^
      \tex_global:D \char_set_catcode_other:N _
      \tex_global:D \char_set_catcode_other:N &
      \tex_global:D \char_set_catcode_other:N ##
    }
  }
} \iffalse $ \fi % to make syntax highlighting work again
%    \end{macrocode}
% \end{macro}
%
% \begin{macro}{\_@@_unset_codes:}
%   Sets category code scheme back from the one used in SMS mode.
%    \begin{macrocode}
\cs_new_protected:Nn \_@@_unset_codes: {
  \_@@_if_catcodes:T {
    \bool_gset_false:N \g_@@_catcode_bool
    \exp_after:wN \tex_global:D \exp_after:wN 
      \char_set_catcode_escape:N \c_backslash_str
    \tex_global:D \char_set_catcode_math_toggle:N $
    \tex_global:D \char_set_catcode_math_superscript:N ^
    \tex_global:D \char_set_catcode_math_subscript:N _
    \tex_global:D \char_set_catcode_alignment:N &
    \tex_global:D \char_set_catcode_parameter:N ##
  }
} \iffalse $ \fi % to make syntax highlighting work again
%    \end{macrocode}
% \end{macro}
%
% \begin{macro}{\stex_in_smsmode:nn}
%    \begin{macrocode}
\cs_new_protected:Nn \stex_in_smsmode:nn {
  \vbox_set:Nn \l_tmpa_box {
    \bool_set_eq:cN { l_@@_#1_bool } \g_@@_bool
    \bool_gset_true:N \g_@@_bool
    \stex_smsmode_set_codes:
    #2
    \bool_gset_eq:Nc \g_@@_bool { l_@@_#1_bool }
    \stex_if_smsmode:F {
      \_@@_unset_codes:
    }
  }
  \box_clear:N \l_tmpa_box
}
%    \end{macrocode}
% \end{macro}
%
% \begin{macro}{\_@@_cs:}
% is executed on encountering |\| in smsmode.
% It checks whether the corresponding command is allowed and executes
% or ignores it accordingly:
%    \begin{macrocode}
\cs_new_protected:Nn \_@@_cs: {
  \str_clear:N \l_tmpa_str
  \peek_analysis_map_inline:n {
    % #1: token (one expansion)
    % #2: charcode
    % #3 catcode
    \token_if_eq_charcode:NNTF ##3 B {
      % token is a letter
      \exp_args:NNo \str_put_right:Nn \l_tmpa_str { ##1 }
    } {
      \str_if_empty:NTF \l_tmpa_str {
        % we don't allow (or need) single non-letter CSs
        % for now
        \peek_analysis_map_break: 
      }{
        \str_if_eq:onTF \l_tmpa_str { begin } {
          \peek_analysis_map_break:n { 
            \exp_after:wN \_@@_checkbegin:n ##1
          }
        } {
          \str_if_eq:onTF \l_tmpa_str { end } {
            \peek_analysis_map_break:n { 
              \exp_after:wN \_@@_checkend:n ##1
            }
          } {
          \tl_set:Nn \l_tmpa_tl { \use:c{\l_tmpa_str} }
          \exp_args:NNNo \exp_args:NNo \tl_if_in:NnTF 
            \g_stex_smsmode_allowedmacros_tl 
              { \use:c{\l_tmpa_str} } {
              \stex_debug:nn{modules}{Executing~1:~\l_tmpa_str}
              \peek_analysis_map_break:n { 
                \exp_after:wN \l_tmpa_tl ##1
              }
            } {
              \exp_args:NNNo \exp_args:NNo \tl_if_in:NnTF 
              \g_stex_smsmode_allowedmacros_escape_tl 
                { \use:c{\l_tmpa_str} } {
                \_@@_unset_codes:
                \stex_debug:nn{modules}{Executing~2:~\l_tmpa_str}
                % TODO \_@@_rescan_cs:
%                \int_compare:nNnTF {##2} = {92} {
%                  \peek_analysis_map_break:n {
%                    \_@@_unset_codes:
%                    \_@@_rescan_cs:
%                  }
%                } {
                  \peek_analysis_map_break:n {
                    \exp_after:wN \l_tmpa_tl ##1
                  }
%                }
              } {
                  \int_compare:nNnTF {##2} = {92} {
                    \peek_analysis_map_break:n { \_@@_cs: }
                  }{
                    \peek_analysis_map_break:n { \exp_after:wN\relax ##1 }
                  }
              }
            }
          }
        }
      }
    }
  }
}
%    \end{macrocode}
% \end{macro}
%
%
% \begin{macro}{\_@@_rescan_cs:}
% If the last token gobbled by |\stex_smsmode_cs:| happened to be
% a |\|, we need to rescan the cs name and reinsert it into the input
% stream:
%    \begin{macrocode}
\cs_new_protected:Nn \_@@_rescan_cs: {
  \str_clear:N \l_tmpb_str
  \peek_analysis_map_inline:n {
    \token_if_eq_charcode:NNTF ##3 B {
      % token is a letter
      \exp_args:NNo \str_put_right:Nn \l_tmpb_str { ##1 }
    } {
      \peek_analysis_map_break:n {
        \exp_after:wN \use:c \exp_after:wN { 
          \exp_after:wN \l_tmpa_str\exp_after:wN 
        } \use:c { \l_tmpb_str \exp_after:wN } ##1
      }
    }
  }
}
%    \end{macrocode}
% \end{macro}
%
% \begin{macro}{\_@@_checkbegin:n}
% called on |\begin|; checks whether the environment being opened
% is allowed in SMS mode. 
%    \begin{macrocode}
\cs_new_protected:Nn \_@@_checkbegin:n {
  \str_set:Nn \l_tmpa_str { #1 }
  \seq_if_in:NoT \g_stex_smsmode_allowedenvs_seq \l_tmpa_str {
    \_@@_unset_codes:
    \begin{#1}
  }
}
%    \end{macrocode}
% \end{macro}
%
% \begin{macro}{\_@@_checkend:n}
% called on |\end|; checks whether the environment being opened
% is allowed in SMS mode. 
%    \begin{macrocode}
\cs_new_protected:Nn \_@@_checkend:n {
  \str_set:Nn \l_tmpa_str { #1 }
  \seq_if_in:NoT \g_stex_smsmode_allowedenvs_seq \l_tmpa_str {
    \end{#1}
  }
}
%    \end{macrocode}
% \end{macro}
%
% \subsection{Inheritance}
%    \begin{macrocode}
%<@@=stex_importmodule>
%    \end{macrocode}
%
% \begin{macro}{\stex_import_module_uri:nn}
%    \begin{macrocode}
\cs_new_protected:Nn \stex_import_module_uri:nn {
  \str_set:Nx \l_@@_archive_str { #1 }
  \str_set:Nn \l_@@_path_str { #2 }

  \exp_args:NNNo \seq_set_split:Nnn \l_tmpb_seq ? { \l_@@_path_str }
  \seq_pop_right:NN \l_tmpb_seq \l_@@_name_str
  \str_set:Nx \l_@@_path_str { \seq_use:Nn \l_tmpb_seq ? }

  \stex_modules_current_namespace:
  \bool_lazy_all:nTF {
    {\str_if_empty_p:N \l_@@_archive_str}
    {\str_if_empty_p:N \l_@@_path_str}
    {\stex_if_module_exists_p:n { \l_stex_module_ns_str ? \l_@@_name_str } }
  }{
    \str_set_eq:NN \l_@@_path_str \l_stex_modules_subpath_str
    \str_set_eq:NN \l_stex_module_ns
  }{
    \str_if_empty:NT \l_@@_archive_str {
      \prop_if_empty:NF \l_stex_current_repository_prop {
        \prop_get:NnN \l_stex_current_repository_prop { id } \l_@@_archive_str
      }
    }  
    \str_if_empty:NTF \l_@@_archive_str {
      \str_if_empty:NF \l_@@_path_str {
        \str_set:Nx \l_stex_module_ns_str {
          \l_stex_module_ns_str / \l_@@_path_str
        }
      }
    }{
      \stex_require_repository:n \l_@@_archive_str
      \prop_get:cnN { c_stex_mathhub_\l_@@_archive_str _manifest_prop } { ns }
        \l_stex_module_ns_str
      \str_if_empty:NF \l_@@_path_str {
        \str_set:Nx \l_stex_module_ns_str {
          \l_stex_module_ns_str / \l_@@_path_str
        }
      }
    }
  }
}
%    \end{macrocode}
% \end{macro}
%
% \begin{variable}{
%   \l_@@_name_str,\l_@@_archive_str,\l_@@_path_str,\l_@@_file_str
% }
%   Store the return values of \cs{stex_import_module_uri:nn}.
%    \begin{macrocode}
\str_new:N \l_@@_name_str
\str_new:N \l_@@_archive_str
\str_new:N \l_@@_path_str
\str_new:N \g_@@_file_str
%    \end{macrocode}
% \end{variable}
%
% \begin{macro}{\stex_import_require_module:nnnn}
%    \begin{syntax} \Arg{ns} \Arg{archive-ID} \Arg{path} \Arg{name} \end{syntax}
%    \begin{macrocode}
\cs_new_protected:Nn \stex_import_require_module:nnnn {
  \exp_args:Nx \stex_if_module_exists:nF { #1 ? #4 } {

    % archive
    \str_set:Nx \l_tmpa_str { #2 }
    \str_if_empty:NTF \l_tmpa_str {
      \seq_set_eq:NN \l_tmpa_seq \g_stex_currentfile_seq
    } {
      \stex_path_from_string:Nn \l_tmpb_seq { \l_tmpa_str }
      \seq_concat:NNN \l_tmpa_seq \c_stex_mathhub_seq \l_tmpb_seq
      \seq_put_right:Nn \l_tmpa_seq { source }
    }

    % path
    \str_set:Nx \l_tmpb_str { #3 }
    \str_if_empty:NTF \l_tmpb_str {
      \str_set:Nx \l_tmpa_str { \stex_path_to_string:N \l_tmpa_seq / #4 }
      
      \ltx@ifpackageloaded{babel} { 
        \exp_args:NNx \prop_get:NnNF \c_stex_language_abbrevs_prop 
            { \languagename } \l_tmpb_str {
              \msg_error:nnx{stex}{error/unknownlanguage}{\languagename}
            }
      } {
        \str_clear:N \l_tmpb_str
      }

      \stex_debug:nn{modules}{Checking~\l_tmpa_str.\l_tmpb_str.tex}
      \IfFileExists{ \l_tmpa_str.\l_tmpb_str.tex }{
        \str_gset:Nx \g_@@_file_str { \l_tmpa_str.\l_tmpb_str.tex }
      }{
        \stex_debug:nn{modules}{Checking~\l_tmpa_str.tex}
        \IfFileExists{ \l_tmpa_str.tex }{
          \str_gset:Nx \g_@@_file_str { \l_tmpa_str.tex }
        }{
          % try english as default
          \stex_debug:nn{modules}{Checking~\l_tmpa_str.en.tex}
          \IfFileExists{ \l_tmpa_str.en.tex }{
            \str_gset:Nx \g_@@_file_str { \l_tmpa_str.en.tex }
          }{
            \msg_error:nnx{stex}{error/unknownmodule}{#1?#4}
          }
        }
      }

    } {
      \seq_set_split:NnV \l_tmpb_seq / \l_tmpb_str
      \seq_concat:NNN \l_tmpa_seq \l_tmpa_seq \l_tmpb_seq
      
      \ltx@ifpackageloaded{babel} { 
        \exp_args:NNx \prop_get:NnNF \c_stex_language_abbrevs_prop 
            { \languagename } \l_tmpb_str {
              \msg_error:nnx{stex}{error/unknownlanguage}{\languagename}
            }
      } {
        \str_clear:N \l_tmpb_str
      }

      \stex_path_to_string:NN \l_tmpa_seq \l_tmpa_str

      \stex_debug:nn{modules}{Checking~\l_tmpa_str/#4.\l_tmpb_str.tex}
      \IfFileExists{ \l_tmpa_str/#4.\l_tmpb_str.tex }{
        \str_gset:Nx \g_@@_file_str { \l_tmpa_str/#4.\l_tmpb_str.tex }
      }{
        \stex_debug:nn{modules}{Checking~\l_tmpa_str/#4.tex}
        \IfFileExists{ \l_tmpa_str/#4.tex }{
          \str_gset:Nx \g_@@_file_str { \l_tmpa_str/#4.tex }
        }{
          % try english as default
          \stex_debug:nn{modules}{Checking~\l_tmpa_str/#4.en.tex}
          \IfFileExists{ \l_tmpa_str/#4.en.tex }{
            \str_gset:Nx \g_@@_file_str { \l_tmpa_str/#4.en.tex }
          }{
            \stex_debug:nn{modules}{Checking~\l_tmpa_str.\l_tmpb_str.tex}
            \IfFileExists{ \l_tmpa_str.\l_tmpb_str.tex }{
              \str_gset:Nx \g_@@_file_str { \l_tmpa_str.\l_tmpb_str.tex }
            }{
              \stex_debug:nn{modules}{Checking~\l_tmpa_str.tex}
              \IfFileExists{ \l_tmpa_str.tex }{
                \str_gset:Nx \g_@@_file_str { \l_tmpa_str.tex }
              }{
                % try english as default
                \stex_debug:nn{modules}{Checking~\l_tmpa_str.en.tex}
                \IfFileExists{ \l_tmpa_str.en.tex }{
                  \str_gset:Nx \g_@@_file_str { \l_tmpa_str.en.tex }
                }{
                  \msg_error:nnx{stex}{error/unknownmodule}{#1?#4}
                }
              }
            }
          }
        }
      }
    }

%    \seq_set_eq:NN \l_tmpa_seq \g_stex_modules_in_file_seq
%    \seq_clear:N \g_stex_modules_in_file_seq
%    \exp_args:Nnx \use:nn {
      \exp_args:No \stex_in_smsmode:nn { \g_@@_file_str } {
        \seq_clear:N \l_stex_all_modules_seq
        \str_clear:N \l_stex_current_module_str
        \str_set:Nx \l_tmpb_str { #2 }
        \str_if_empty:NF \l_tmpb_str {
          \stex_set_current_repository:n { #2 }
        }
        \stex_debug:nn{modules}{Loading~\g_@@_file_str}
        \input { \g_@@_file_str }
      }
%    }{

%    }
%    \prop_gput:Noo \g_stex_module_files_prop
%    \g_@@_file_str \g_stex_modules_in_file_seq
%    \seq_set_eq:NN \g_stex_modules_in_file_seq \l_tmpa_seq

    \stex_if_module_exists:nF { #1 ? #4 } {
      \msg_error:nnx{stex}{error/unknownmodule}{
        #1?#4~(in~file~\g_@@_file_str)
      }
    }
  }
  \stex_activate_module:n { #1 ? #4 }
}
%    \end{macrocode}
% \end{macro}
%
% \begin{macro}{\importmodule}
%    \begin{macrocode}
\NewDocumentCommand \importmodule { O{} m } {
  \stex_import_module_uri:nn { #1 } { #2 }
  \stex_debug:nn{modules}{Importing~module:~
    \l_stex_module_ns_str ? \l_@@_name_str
  }
  \stex_if_smsmode:F {
    \stex_import_require_module:nnnn 
    { \l_stex_module_ns_str } { \l_@@_archive_str } 
    { \l_@@_path_str } { \l_@@_name_str }
    \stex_annotate_invisible:nnn 
      {import} {\l_stex_module_ns_str ? \l_@@_name_str} {}
  }
  \exp_args:Nx \stex_add_to_current_module:n {
    \stex_import_require_module:nnnn 
    { \l_stex_module_ns_str } { \l_@@_archive_str } 
    { \l_@@_path_str } { \l_@@_name_str }
  }
  \exp_args:Nx \stex_add_import_to_current_module:n {
    \l_stex_module_ns_str ? \l_@@_name_str
  }
  \stex_smsmode_set_codes:
}
\stex_deactivate_macro:Nn \importmodule {module~environments}
%    \end{macrocode}
% \end{macro}
%
% \begin{macro}{\usemodule}
%    \begin{macrocode}
\NewDocumentCommand \usemodule { O{} m } {
  \stex_if_smsmode:F {
    \stex_import_module_uri:nn { #1 } { #2 }
    \stex_import_require_module:nnnn 
    { \l_stex_module_ns_str } { \l_@@_archive_str }
    { \l_@@_path_str } { \l_@@_name_str }
    \stex_annotate_invisible:nnn 
      {usemodule} {\l_stex_module_ns_str ? \l_@@_name_str} {}
  }
  \stex_smsmode_set_codes:
}
%    \end{macrocode}
% \end{macro}
%
%    \begin{macrocode}
%</package>
%    \end{macrocode}
%
% \end{implementation}
%
% \PrintIndex
