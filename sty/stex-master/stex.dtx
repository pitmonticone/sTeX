%^^A An Infrastructure for Semantic Macros and Module Scoping
%^^A Copyright (c) 2019 Michael Kohlhase, all rights reserved
%^^A               this file is released under the
%^^A               LaTeX Project Public License (LPPL)
%^^A
%^^A The original of this file is in the public repository at 
%^^A http://github.com/sLaTeX/sTeX/
%
%^^A TODO update copyright
%  
% \iffalse
%<cls>\ProvidesClass{stex}[2021/03/08 v2.0 Semantic Markup]
%<package>\NeedsTeXFormat{LaTeX2e}[1999/12/01]
%<package>\ProvidesPackage{stex}[2020/11/10 v2.0 Semantic Markup]
%
%<*driver>
\documentclass{ltxdoc}
\usepackage[utf8]{inputenc}
\usepackage[T1]{fontenc}
\usepackage[showmods,debug]{stex}
\usepackage{url,array,float,textcomp}
\usepackage[show]{ed}
\usepackage[hyperref=auto,style=alphabetic]{biblatex}
\addbibresource{kwarcpubs.bib}
\addbibresource{extpubs.bib}
\addbibresource{kwarccrossrefs.bib}
\addbibresource{extcrossrefs.bib}
\usepackage{listings,amssymb}
\usepackage{amsfonts}
\usepackage{ctangit}
\usepackage{xspace}
\usepackage{hyperref}
\makeindex
\floatstyle{boxed}
\newfloat{exfig}{thp}{lop}
\floatname{exfig}{Example}
\def\githubissue#1{\cite{sTeX:github:on}, \hyperlink{https://github.com/sLaTeX/sTeX/issues/#1}{issue #1}}
\begin{document}
\RecordChanges
\DocInput{stex.dtx}
\end{document} 
%</driver>
% \fi
% 
% \iffalse\CheckSum{802}\fi
% 
% \GetFileInfo{stex.sty}
% 
% \MakeShortVerb{\|}
%\def\scsys#1{{{\sc #1}}\index{#1@{\sc #1}}\xspace}
% \def\xml{\scsys{Xml}}
% \def\mathml{\scsys{MathML}}
% \def\omdoc{\scsys{OMDoc}}
% \def\openmath{\scsys{OpenMath}}
% \def\latexml{\scsys{LaTeXML}}
% \def\perl{\scsys{Perl}}
% \def\cmathml{Content-{\sc MathML}\index{Content {\sc MathML}}\index{MathML@{\sc MathML}!content}}
% \def\activemath{\scsys{ActiveMath}}
% \def\twin#1#2{\index{#1!#2}\index{#2!#1}}
% \def\twintoo#1#2{{#1 #2}\twin{#1}{#2}}
% \def\atwin#1#2#3{\index{#1!#2!#3}\index{#3!#2 (#1)}}
% \def\atwintoo#1#2#3{{#1 #2 #3}\atwin{#1}{#2}{#3}}
% \def\cT{\mathcal{T}}\def\cD{\mathcal{D}}
% \title{{\texttt{stex.sty}}: {\stex2.0}\thanks{Version {\fileversion} (last revised
% {\filedate})}}
% \author{Michael Kohlhase, Dennis M\"uller\\
%    FAU Erlangen-N\"urnberg\\
%    \url{http://kwarc.info/}}
% \maketitle
%
% \makeatletter
% \def\testwrite{\begingroup\catcode`\\=12\catcode`\#=12\catcode`\&=12\testwrite@}
% \def\testwrite@#1{\ex\gdef\ex\test@content\ex{\detokenize{#1}}\ex\immediate\ex\write\ex\testoutfile\ex{\test@content}\endgroup}
% \def\testwriteI{\begingroup\catcode`\\=12\catcode`\#=12\catcode`\&=12\testwriteI@}
% \newcount\test@counter\test@counter=0
% \def\testwriteI@#1{\ex\gdef\ex\test@content\ex{\detokenize{#1}}\immediate\write\testoutfile{^^J\detokenize{\advance\test@counter by 1\begingroup\paragraph{Test \the\test@counter:}}^^J}\ex\immediate\ex\write\ex\testoutfile\ex{\test@content}\immediate\write\testoutfile{^^J\endgroup^^J}\endgroup\@test}
% \newwrite\testoutfile\immediate\openout\testoutfile=alltests.tex
% \testwrite{\documentclass{article}^^J\usepackage[showmods]{stex}^^J\begin{document}\makeatletter^^J}
% \testwrite{\newcount\test@counter\test@counter=0}
% \def\test{\begingroup\makeatletter\testwriteI}
% \long\def\@test{\global\advance\test@counter by 1\paragraph{\textcolor{blue}{Test \the\test@counter:}}\textcolor{blue}{\ex\scantokens\ex{\test@content}}\endgroup}
% \makeatother
%
% \begin{abstract}
%    TODO
% \end{abstract}
%
% \newpage\setcounter{tocdepth}{2}\tableofcontents\newpage
%
% \section{Introduction}\label{sec:intro}
% TODO
%
% \section{User commands}
%
% \begin{itemize}
%   \item[\checkmark] |\sTeX|
%   \item[\checkmark] |module|
%   \item[\checkmark] |\importmodule|
%   \item[\checkmark] |\usemodule|
%   \item[\checkmark] |\symdecl|
%   \item[\checkmark] |\notation|
%   \item[?]          |\inputref|
%   \item[?]          |\libinput|
%   \item[$\times$]   |\defi|
%   \item[$\times$]   |\tref|
%   \item[$\times$]   |omgroup|/|omtext|
% \end{itemize}
%
% \section{Implementation}\label{sec:impl}
%    \begin{macrocode}
%<*cls>
\RequirePackage{xkeyval}
\DeclareOptionX{lang}{\def\stex@babeloption{#1}}
\DeclareOptionX*{\PassOptionsToPackage{\CurrentOption}{stex}}
\ProcessOptionsX
\LoadClass[border=1px,varwidth]{standalone}
\setlength\textwidth{15cm}
\g@addto@macro{\@parboxrestore}{\setlength\parskip{\baselineskip}}
\usepackage{stex}
%</cls>
%<*package>

\let\ex\expandafter

\ifcsname if@latexml\endcsname\else
    \ex\newif\csname if@latexml\endcsname\@latexmlfalse
\fi

\RequirePackage{amsfonts,amsmath,xkeyval,xspace,standalone,scalatex}
\RequirePackageWithOptions{stex-metakeys}
\if@latexml\else\RequirePackage{xstring}\fi
\RequirePackage{etoolbox}

\scalatex@addNamespaceAbbrev{stex}{http://kwarc.info/ns/sTeX}

\def\defemph#1{\textbf{#1}}

% TODO
\newif\if@stex@debugmode\@stex@debugmodefalse
\DeclareOptionX{debug}{\@stex@debugmodetrue}
\def\stex@debug#1{\if@stex@debugmode\message{^^J#1^^J}\fi}
% Modules:
\newif\ifmod@show\mod@showfalse
\DeclareOptionX{showmods}{\mod@showtrue}
% sref:
\newif\ifextrefs\extrefsfalse
\DeclareOptionX{extrefs}{\extrefstrue}
\DeclareOptionX{lang}{\def\stex@babeloption{#1}}
%
\ProcessOptionsX
%\ProcessOptions


\def\stex@babel@set#1#2{
  \csdef{stex@babel@short#1}{#2}
  \csdef{stex@babel@long#2}{#1}
}
\stex@babel@set{english}{en}
\stex@babel@set{ngerman}{de}
\stex@babel@set{arabic}{ar}
\stex@babel@set{bulgarian}{bg}
\stex@babel@set{russian}{ru}
\stex@babel@set{finnish}{fi}
\stex@babel@set{romanian}{ro}
\stex@babel@set{turkish}{tr}
\stex@babel@set{french}{fr}
% \stex@babel@set{chinese simplified}{zhs}
% \stex@babel@set{chinese traditional}{zht}

\ifcsvoid{stex@babeloption}{}{
  \ifcsvoid{stex@babel@long\stex@babeloption}{
    \PackageError{stex}{Unknown language shortcode: \stex@babeloption}{}
  }{
    \RequirePackage[\csname stex@babel@long\stex@babeloption\endcsname]{babel}
  }
}

\if@stex@debugmode\stex@debug{sTeX debug mode on}\fi
%    \end{macrocode}
% A conditional for LaTeXML:
%    \begin{macrocode}
%
%    \end{macrocode}
% The following macro and environment generate LaTeXML annotations as a |<span>| node with the first and second arguments as 
% |property| and |resource| attributes respectively, and the third argument as content. In math mode, the first two arguments are
% instead used as the |class| attribute, separated by an underscore.
%    \begin{macrocode}
\protected\long\def\latexml@annotate#1#2#3{%
  \def\latexml@annotate@bodyarg{#3}%
  \if@latexml\ifmmode\latexml@annotate@math{#1}{#2}{\ifx\latexml@annotate@bodyarg\@empty\ \else#3\fi}\else\latexml@annotate@text{#1}{#2}{#3}\fi\else#3\fi%
}
\protected\long\def\latexml@invisible#1{%
  \if@latexml\ifmmode\latexml@invisible@math{#1}\else\latexml@invisible@text{#1}\fi\fi%
}
\protected\long\def\latexml@annotate@text#1#2#3{}
\protected\long\def\latexml@annotate@math#1#2#3{}
\protected\long\def\latexml@annotate@invisible#1#2#3{}
\newenvironment{latexml@annotateenv}[2]{}{}

\protected\long\def\scalatex@invisible#1{%
\def\scalatex@annotate@bodyarg{#1}%
  \if@scalatex\scalatex@annotateHTML{stex:visible="false" style:display="none"}\ifx\scalatex@annotate@bodyarg\@empty\scalatex@directHTML{\@Ampersand lrm;}\else\scalatex@annotate@bodyarg\fi\scalatex@annotateHTMLEnd\fi%
}
\protected\long\def\scalatex@annotate#1#2#3{%
  \def\scalatex@annotate@bodyarg{#3}%
  \scalatex@annotateHTML{property="stex:#1" resource="#2"}\ifx\scalatex@annotate@bodyarg\@empty\scalatex@directHTML{\@Ampersand lrm;}\else\scalatex@annotate@bodyarg\fi\scalatex@annotateHTMLEnd%
}
\protected\long\def\scalatex@annotate@invisible#1#2#3{%
  \def\scalatex@annotate@bodyarg{#3}%
  \scalatex@annotateHTML{property="stex:#1" resource="#2" stex:visible="false" style:display="none"}\ifx\scalatex@annotate@bodyarg\@empty\scalatex@directHTML{\@Ampersand lrm;}\else\scalatex@annotate@bodyarg\fi\scalatex@annotateHTMLEnd%
}
\newenvironment{scalatex@annotateenv}[2]{\scalatex@annotateHTML{property="stex:#1" resource="#2"}}{\scalatex@annotateHTMLEnd}

\protected\long\def\stex@invisible{\if@scalatex\expandafter\scalatex@invisible\else\expandafter\latexml@invisible\fi}
\protected\long\def\stex@annotate{\if@scalatex\expandafter\scalatex@annotate\else\expandafter\latexml@annotate\fi}
\protected\long\def\stex@annotate@invisible{\if@scalatex\expandafter\scalatex@annotate@invisible\else\expandafter\latexml@annotate@invisible\fi}
\newenvironment{stex@annotateenv}[2]{\par\if@scalatex\begin{scalatex@annotateenv}{#1}{#2}\else\begin{latexml@annotateenv}{#1}{#2}\fi}{\if@scalatex\end{scalatex@annotateenv}\else\end{latexml@annotateenv}\fi}
\newif\if@htmlmode
\if@latexml\@htmlmodetrue\else\if@scalatex\@htmlmodetrue\else\@htmlmodefalse\fi\fi
%    \end{macrocode}
%
% \subsection{sTeX base}
% The \sTeX logo:
%    \begin{macrocode}
\protected\def\stex{%
  \@ifundefined{texorpdfstring}%
  {\let\texorpdfstring\@firstoftwo}%
  {}%
  \texorpdfstring{\raisebox{-.5ex}S\kern-.5ex\TeX}{sTeX}\xspace%
}
\def\sTeX{\stex}
%    \end{macrocode}
%
% \subsection{Paths and URIs}\label{sec:impl:pathsuris}
%
% We define two macros for changing the category codes of common
% characters in URIs, in particular \#.
%    \begin{macrocode}
\def\pathsuris@setcatcodes{%
    \edef\pathsuris@oldcatcode@hash{\the\catcode`\#}%
    \catcode`\#=12\relax%
    \edef\pathsuris@oldcatcode@slash{\the\catcode`\/}%
    \catcode`\/=12\relax%
    \edef\pathsuris@oldcatcode@colon{\the\catcode`\:}%
    \catcode`\:=12\relax%
    \edef\pathsuris@oldcatcode@qm{\the\catcode`\?}%
    \catcode`\?=12\relax%
}
\def\pathsuris@resetcatcodes{%
    \catcode`\#\pathsuris@oldcatcode@hash\relax%
    \catcode`\/\pathsuris@oldcatcode@slash\relax%
    \catcode`\:\pathsuris@oldcatcode@colon\relax%
    \catcode`\?\pathsuris@oldcatcode@qm\relax%
}
%    \end{macrocode}
%
% \begin{macro}{\defpath}
%  |\defpath{macro name}{base path}| defines a new macro which can 
%  take another path to form one integrated path. For example, |\MathHub|
%  is defined as:
% \begin{center}
% |\defpath{MathHub}{/path/to/localmh/MathHub}|
% \end{center}
% then we can use |\MathHub| to form other paths, for example, 
% \begin{center}
% |\MathHub{source/smglom/sets}|
% \end{center}
%  will generate |/path/to/localmh/MathHub/source/smglom/sets|.
%    \begin{macrocode}
\def\namespace@read#1{%
  \edef\namespace@read@path{#1}%
  \edef\namespace@read@path{\ex\detokenize\ex{\namespace@read@path}}%
  \namespace@continue%
}
\def\namespace@continue{%
  \pathsuris@resetcatcodes%
  \ex\edef\csname\namespace@macroname\endcsname##1{%
    \namespace@read@path\@Slash##1%
  }%
}
\protected\def\namespace#1{%
  \def\namespace@macroname{#1}%
  \pathsuris@setcatcodes%
  \namespace@read%
}
\let\defpath\namespace
%    \end{macrocode}
% \end{macro}
%
% \subsubsection{Path Canonicalization}\label{sec:impl:cpath}
%
% We define some macros for later comparison.
%    \begin{macrocode}
\pathsuris@setcatcodes
\def\@ToTop{..}
\def\@Slash{/}
\def\@Colon{:}
\def\@Space{ }
\def\@QuestionMark{?}
\def\@Dot{.}
\catcode`\&=12
\def\@Ampersand{&}
\catcode`\&=4
\def\@Fragment{#}
\pathsuris@resetcatcodes
\catcode`\.=0
.catcode`.\=12
.let.@BackSlash\
.catcode`.\=0
\catcode`\.=12
\edef\old@percent@catcode{\the\catcode`\%}
\catcode`\%=12
\let\@Percent%
\catcode`\%=\old@percent@catcode
%    \end{macrocode}
%
% \begin{macro}{\@cpath}
% Canonicalizes (file) paths:
%    \begin{macrocode}
\def\@cpath#1{%
    \edef\pathsuris@cpath@temp{#1}%
    \def\@cpath@path{}%
    \IfBeginWith\pathsuris@cpath@temp\@Slash{%
      \@cpath@loop%
      \edef\@cpath@path{\@Slash\@cpath@path}%
    }{%
        \IfBeginWith\pathsuris@cpath@temp{\@Dot\@Slash}{%
            \StrGobbleLeft\pathsuris@cpath@temp2[\pathsuris@cpath@temp]%
            \@cpath@loop%
        }{%
            \ifx\pathsuris@cpath@temp\@Dot\else%
            \@cpath@loop\fi%
        }%
    }%
    \IfEndWith\@cpath@path\@Slash{%
      \ifx\@cpath@path\@Slash\else%
        \StrGobbleRight\@cpath@path1[\@cpath@path]%
      \fi%
    }{}%
}

\def\@cpath@loop{%
    \IfSubStr\pathsuris@cpath@temp\@Slash{%
        \StrCut\pathsuris@cpath@temp\@Slash%
          \pathsuris@cpath@temp@a\pathsuris@cpath@temp%
        \ifx\pathsuris@cpath@temp@a\@ToTop%
            \ifx\@cpath@path\@empty%
                \edef\@cpath@path{\@ToTop}%
            \else%
                \edef\@cpath@path{\@cpath@path\@Slash\@ToTop}%
            \fi%
            \@cpath@loop%
        \else%
        \ifx\pathsuris@cpath@temp@a\@Dot%
            \@cpath@loop%
        \else%
        \IfBeginWith\pathsuris@cpath@temp\@ToTop{%
            \StrBehind{\pathsuris@cpath@temp}{\@ToTop}%
              [\pathsuris@cpath@temp]%
            \IfBeginWith\pathsuris@cpath@temp\@Slash{%
                \edef\pathsuris@cpath@temp%
                  {\@cpath@path\pathsuris@cpath@temp}%
            }{%
                \ifx\@cpath@path\@empty\else%
                    \edef\pathsuris@cpath@temp%
                      {\@cpath@path\@Slash\pathsuris@cpath@temp}%
                \fi%
            }%
            \def\@cpath@path{}%
            \@cpath@loop%
        }{%
            \ifx\@cpath@path\@empty%
                \edef\@cpath@path{\pathsuris@cpath@temp@a}%
            \else%
                \edef\@cpath@path%
                  {\@cpath@path\@Slash\pathsuris@cpath@temp@a}%
            \fi%
            \@cpath@loop%
        }%
        \fi\fi%
    }{%
        \ifx\@cpath@path\@empty%
            \edef\@cpath@path{\pathsuris@cpath@temp}%
        \else%
            \edef\@cpath@path{\@cpath@path\@Slash\pathsuris@cpath@temp}%
        \fi%
    }%
}
%    \end{macrocode}
% \test{
%\gdef\present#1{\texttt{>>\meaning#1<<}}
%\gdef\printltx#1{\texttt{\detokenize{#1}}}^^J
% \begin{center}
% \begin{tabular}{|l|l|l|}\hline
%   path & canonicalized path & expected\\\hline
%   aaa & \cpath@print{aaa} & aaa \\
%   ../../aaa & \cpath@print{../../aaa} &  ../../aaa\\
%   aaa/bbb & \cpath@print{aaa/bbb} & aaa/bbb \\
%   aaa/.. & \cpath@print{aaa/..} &\\
%   ../../aaa/bbb & \cpath@print{../../aaa/bbb} & ../../aaa/bbb\\
%   ../aaa/../bbb & \cpath@print{../aaa/../bbb} & ../bbb \\
%   ../aaa/bbb & \cpath@print{../aaa/bbb} &  ../aaa/bbb\\
%   aaa/bbb/../ddd & \cpath@print{aaa/bbb/../ddd} & aaa/ddd\\
%   aaa/bbb/./ddd & \cpath@print{aaa/bbb/./ddd} & aaa/bbb/ddd\\
%   ./ & \cpath@print{./} & \\
%   aaa/bbb/../.. & \cpath@print{aaa/bbb/../..} & \\\hline
% \end{tabular}
% \end{center}
% }
%
% \end{macro}
%
% \begin{macro}{\cpath@print}
% Implement |\cpath@print| to print the canonicalized path.
%    \begin{macrocode}
\newcommand\cpath@print[1]{%
    \@cpath{#1}%
    \@cpath@path%
}
%    \end{macrocode}
% \end{macro}
%
% \begin{macro}{\path@filename}
%
%    \begin{macrocode}
\def\path@filename#1#2{%
    \edef\filename@oldpath{#1}%
    \StrCount\filename@oldpath\@Slash[\filename@lastslash]%
    \ifnum\filename@lastslash>0%
        \StrBehind[\filename@lastslash]\filename@oldpath%
          \@Slash[\filename@oldpath]%
        \edef#2{\filename@oldpath}%
    \else%
        \edef#2{\filename@oldpath}%
    \fi%
}
%    \end{macrocode}
% \test{
%  \edef\testpath{/foo/bar/baz.tex}
% \path@filename\testpath\testfilename
% Path: \testpath\\
% Filename: \testfilename\\
% }
% \end{macro}
% \begin{macro}{\path@filename@noext}
%
%    \begin{macrocode}
\def\path@filename@noext#1#2{%
    \path@filename{#1}{#2}%
    \edef\filename@oldpath{#2}%
    \StrCount\filename@oldpath\@Dot[\filename@lastdot]%
    \ifnum\filename@lastdot>0%
        \StrBefore[\filename@lastdot]\filename@oldpath%
          \@Dot[\filename@oldpath]%
        \edef#2{\filename@oldpath}%
    \else%
        \edef#2{\filename@oldpath}%
    \fi%
}
%    \end{macrocode}
% \test{
%  \edef\testpath{/foo/bar/baz.tex}
% \path@filename@noext\testpath\testfilename
% Path: \testpath\\
% Filename: \testfilename\\
% }
% \end{macro}
%
% \subsubsection{Windows}
%
% First, a conditional that tells us whether we have to use windows
% or unix file paths:
%    \begin{macrocode}
\newif\if@iswindows@\@iswindows@false
\IfFileExists{nul:}{\IfFileExists{/dev/null}{}{\@iswindows@true}}{}
%    \end{macrocode}
% \test{We are on windows: \if@iswindows@ yes\else no\fi.}
%
% \begin{macro}{\windows@to@path}
% Converts a windows-style file path to a unix-style file path:
%    \begin{macrocode}
\newif\if@windowstopath@inpath@
\def\windows@to@path#1{%
    \@windowstopath@inpath@false%
    \def\windows@temp{}%
    \edef\windows@path{#1}%
    \ifx\windows@path\@empty\else%
        \ex\windows@path@loop\windows@path\windows@path@end%
    \fi%
    \let#1\windows@temp%
}
\def\windows@path@loop#1#2\windows@path@end{%
    \def\windows@temp@b{#2}%
    \ifx\windows@temp@b\@empty%
        \def\windows@continue{}%
    \else%
        \def\windows@continue{\windows@path@loop#2\windows@path@end}%
    \fi%
    \if@windowstopath@inpath@%
        \ifx#1\@BackSlash%
            \edef\windows@temp{\windows@temp\@Slash}%
        \else%
            \edef\windows@temp{\windows@temp#1}%
        \fi%
    \else%
        \ifx#1:%
            \edef\windows@temp{\@Slash\windows@temp}%
            \@windowstopath@inpath@true%
        \else%
            \edef\windows@temp{\windows@temp#1}%
        \fi%
    \fi%
    \windows@continue%
}
%    \end{macrocode}
% \test{\edef\testpath{\detokenize{C:\foo\bar.baz}}
% Input: \texttt{\testpath}\\
% \windows@to@path\testpath
% Output: \testpath}
% \end{macro}
%
% \begin{macro}{\path@to@windows}
% Converts a unix-style file path to a windows-style file path:
%    \begin{macrocode}
\def\path@to@windows#1{%
    \@windowstopath@inpath@false%
    \def\windows@temp{}%
    \edef\windows@path{#1}%
    \edef\windows@path{\expandafter\@gobble\windows@path}%
    \ifx\windows@path\@empty\else%
        \expandafter\path@windows@loop\windows@path\windows@path@end%
    \fi%
    \let#1\windows@temp%
}
\def\path@windows@loop#1#2\windows@path@end{%
    \def\windows@temp@b{#2}%
    \ifx\windows@temp@b\@empty%
        \def\windows@continue{}%
    \else%
        \def\windows@continue{\path@windows@loop#2\windows@path@end}%
    \fi%
    \if@windowstopath@inpath@%
        \ifx#1/%
            \edef\windows@temp{\windows@temp\@BackSlash}%
        \else%
            \edef\windows@temp{\windows@temp#1}%
        \fi%
    \else%
        \ifx#1/%
            \edef\windows@temp{\windows@temp:\@BackSlash}%
            \@windowstopath@inpath@true%
        \else%
            \edef\windows@temp{\windows@temp#1}%
        \fi%
    \fi%
    \windows@continue%
}
%    \end{macrocode}
% \test{\def\testpath{/C/foo/bar.baz}
% Input: \testpath\\
% \path@to@windows\testpath
% Output: \testpath}
% \end{macro}
%
% \subsubsection{Auxiliary methods}
%
% \begin{macro}{\path@trimstring}
% Removes initial and trailing spaces from a string:
%    \begin{macrocode}
\def\path@trimstring#1{%
    \edef\pathsuris@trim@temp{#1}%
    \IfBeginWith\pathsuris@trim@temp\@Space{%
        \StrGobbleLeft\pathsuris@trim@temp1[#1]%
        \path@trimstring{#1}%
    }{%
        \IfEndWith\pathsuris@trim@temp\@Space{%
            \StrGobbleRight\pathsuris@trim@temp1[#1]%
            \path@trimstring{#1}%
        }{%
            \edef#1{\pathsuris@trim@temp}%
        }%
    }%
}
%    \end{macrocode}
% \test{\edef\tempcs{     foo   bar     }\path@trimstring\tempcs
% \texttt{>>\tempcs<<}}
% \end{macro}
%
% \begin{macro}{\@kpsewhich}
% Calls \texttt{kpsewhich} to get e.g. system variables:
%    \begin{macrocode}
%\if@latexml\else
\def\@kpsewhich#1#2{\begingroup%
  \edef\kpsewhich@cmd{"|kpsewhich #2"}%
  \everyeof{\noexpand}%
  \catcode`\\=12%
  \edef#1{\@@input\kpsewhich@cmd\@Space}%
  \path@trimstring#1%
  \if@iswindows@\windows@to@path#1\fi%
  \xdef#1{\ex\detokenize\expandafter{#1}}%
\endgroup}
%\fi
%    \end{macrocode}
% \test{\@kpsewhich\tempcs{etoolbox.sty} \tempcs}
% \end{macro}
%
% \subsubsection{\protect\sTeX input hooks}
% We determine the \texttt{PWD} of the current main document:
%    \begin{macrocode}
\edef\pwd@cmd{\if@iswindows@ -expand-var \@Percent%
  CD\@Percent\else -var-value PWD\fi}
\@kpsewhich\stex@PWD\pwd@cmd
\edef\stex@mainfile{\stex@PWD\@Slash\jobname}
\edef\stex@mainfile{\ex\detokenize\ex{\stex@mainfile}}
%    \end{macrocode}
% \test{\stex@PWD}
%
% We keep a stack of \string\input ed files:
%    \begin{macrocode}
\def\stex@currfile@stack{}

\def\stex@currfile@push#1{%
    \edef\stex@temppath{#1}%
    \edef\stex@temppath{\ex\detokenize\ex{\stex@temppath}}%
  \edef\stex@currfile@stack{\stex@currfile%
    \ifx\stex@currfile@stack\@empty\else,\stex@currfile@stack\fi}
  \IfBeginWith\stex@temppath\@Slash{\@cpath{\stex@temppath}}{%
    \@cpath{\stex@PWD\@Slash#1}%
  }
  \let\stex@currfile\@cpath@path%
  \path@filename\stex@currfile\stex@currfilename%
  \StrLen\stex@currfilename[\stex@currfile@tmp]%
  \StrGobbleRight\stex@currfile{\the\numexpr%
    \stex@currfile@tmp+1 }[\stex@currpath]%
  \global\let\stex@currfile\stex@currfile%
  \global\let\stex@currpath\stex@currpath%
  \global\let\stex@currfilename\stex@currfilename%
}
\def\stex@currfile@pop{%
  \ifx\stex@currfile@stack\@empty%
    \global\let\stex@currfile\stex@mainfile%
    \global\let\stex@currpath\stex@PWD%
    \global\let\stex@currfilename\jobname%
  \else%
    \StrCut\stex@currfile@stack,\stex@currfile\stex@currfile@stack%
    \path@filename\stex@currfile\stex@currfilename%
    \StrLen\stex@currfilename[\stex@currfile@tmp]%
    \StrGobbleRight\stex@currfile{\the\numexpr%
      \stex@currfile@tmp+1 }[\stex@currpath]%
    \global\let\stex@currfile\stex@currfile%
    \global\let\stex@currpath\stex@currpath%
    \global\let\stex@currfilename\stex@currfilename%
  \fi%
}
%    \end{macrocode}
%
% \begin{macro}{\stexinput}
% Inputs a file by (if necessary) converting its path to a windows path
% first, and adding the file path to the input stack above:
%    \begin{macrocode}
\def\stexinput#1{%
    \stex@iffileexists{#1}{%
      \stex@currfile@push\stex@temp@path%
      \input{\stex@currfile}%
      \stex@currfile@pop%
    }%
    {%
        \PackageError{stex}{File does not exist %
          (#1): \stex@temp@path}{}%
    }%
}
\def\stex@iffileexists#1#2#3{%
  \edef\stex@temp@path{#1}%
  \if@iswindows@\path@to@windows\stex@temp@path\fi%
  \IfFileExists\stex@temp@path{#2}{#3}%
}
\stex@currfile@pop
%    \end{macrocode}
% \test{
% \global\newwrite\testfile\immediate\openout\testfile=testfile.tex
% \immediate\write\testfile{A test file: \string\stex@currfile}
% \immediate\closeout\testfile
% This file: \stex@currfile\\
% \stexinput{testfile.tex}
% \\Back: \stex@currfile}
% \end{macro}
%
% \subsubsection{MathHub repositories}
% We read the |MATHHUB| system variable and set |\MathHub| accordingly:
%    \begin{macrocode} 
\@kpsewhich\mathhub@path{--var-value MATHHUB}
\if@iswindows@\windows@to@path\mathhub@path\fi
\ifx\mathhub@path\@empty
  \PackageWarning{stex}{MATHHUB system variable not %
    found or wrongly set}{}
  \defpath{MathHub}{}
\else\defpath{MathHub}\mathhub@path\fi
\stex@debug{MathHub: \mathhub@path}
%    \end{macrocode}
% \test{\mathhub@path}
%
%\begin{macro}{\mathhub@findmanifest}
% |\mathhub@findmanifest{|\meta{path}|}| searches for a file |MANIFEST.MF| up and over \meta{path} in
% the file system tree.
%
%    \begin{macrocode}
\def\mathhub@findmanifest#1{%
  \@cpath{#1}%
  \ifx\@cpath@path\@Slash%
    \def\manifest@mf{}%
  \else\ifx\@cpath@path\@empty%
      \def\manifest@mf{}%
  \else%
    \edef\@findmanifest@path{\@cpath@path/MANIFEST.MF}%
    \if@iswindows@\path@to@windows\@findmanifest@path\fi%
    \IfFileExists{\@findmanifest@path}{%
      \edef\manifest@mf{\@findmanifest@path}%
      \xdef\temp@archive@dir{\ex\detokenize\ex{\@cpath@path}}%
    }{%
    \edef\@findmanifest@path{\@cpath@path/META-INF/MANIFEST.MF}%
    \if@iswindows@\path@to@windows\@findmanifest@path\fi%
    \IfFileExists{\@findmanifest@path}{%
      \edef\manifest@mf{\@findmanifest@path}%
      \xdef\temp@archive@dir{\ex\detokenize\ex{\@cpath@path}}%
    }{%
    \edef\@findmanifest@path{\@cpath@path/meta-inf/MANIFEST.MF}%
    \if@iswindows@\path@to@windows\@findmanifest@path\fi%
    \IfFileExists{\@findmanifest@path}{%
      \edef\manifest@mf{\@findmanifest@path}%
      \xdef\temp@archive@dir{\ex\detokenize\ex{\@cpath@path}}%
    }{%
      \mathhub@findmanifest{\@cpath@path/..}%
    }}}%
  \fi\fi%
}
%    \end{macrocode}
% \test{In \MathHub{smglom/mv/source}:\\\mathhub@findmanifest{\MathHub{smglom/mv/source}}\manifest@mf}
%\end{macro}
%
% the next macro is a helper function for parsing |MANIFEST.MF|
%
%    \begin{macrocode}
\def\split@manifest@key{%
  \IfSubStr{\manifest@line}{\@Colon}{%
      \StrBefore{\manifest@line}{\@Colon}[\manifest@key]%
      \StrBehind{\manifest@line}{\@Colon}[\manifest@line]%
      \path@trimstring\manifest@line%
      \path@trimstring\manifest@key%
  }{%
      \def\manifest@key{}%
  }%
}
%    \end{macrocode}
%
% the next helper function iterates over lines in |MANIFEST.MF|
%
%    \begin{macrocode}
\def\parse@manifest@loop{%
  \ifeof\@manifest%
  \else%
    \read\@manifest to \manifest@line\relax%
    \split@manifest@key%
    % id
    \IfStrEq\manifest@key{id}{%
        \xdef\manifest@mf@id{\manifest@line}%
    }{%
    % narration-base
    \IfStrEq\manifest@key{narration-base}{%
        \xdef\manifest@mf@narr{\manifest@line}%
    }{%
    % namespace
    \IfStrEq\manifest@key{source-base}{%
        \xdef\manifest@mf@ns{\manifest@line}%
    }{%
    \IfStrEq\manifest@key{ns}{%
        \xdef\manifest@mf@ns{\manifest@line}%
    }{%
    % dependencies
    \IfStrEq\manifest@key{dependencies}{%
        \xdef\manifest@mf@deps{\manifest@line}%
    }{%
    }}}}}%
    \parse@manifest@loop%
  \fi%
}
%    \end{macrocode}
%\begin{macro}{\mathhub@parsemanifest}
% |\mathhub@parsemanifest{|\meta{macroname}|}{|\meta{path}|}| finds |MANIFEST.MF| via |\mathhub@findmanifest{|\meta{path}|}|,
% and parses the file, storing the individual fields (|id|, |narr|, |ns| and |dependencies|) in
% \meta{macroname}|id|, \meta{macroname}|narr|, etc.
%    \begin{macrocode}
\newread\@manifest
\def\mathhub@parsemanifest#1#2{%
  \gdef\temp@archive@dir{}%
  \mathhub@findmanifest{#2}%
  \begingroup%
    \newlinechar=-1%
    \endlinechar=-1%
    \gdef\manifest@mf@id{}%
    \gdef\manifest@mf@narr{}%
    \gdef\manifest@mf@ns{}%
    \gdef\manifest@mf@deps{}%
    \immediate\openin\@manifest=\manifest@mf\relax%
    \parse@manifest@loop%
    \immediate\closein\@manifest%
  \endgroup%
  \if@iswindows@\windows@to@path\manifest@mf\fi%
  \cslet{#1id}\manifest@mf@id%
  \cslet{#1narr}\manifest@mf@narr%
  \cslet{#1ns}\manifest@mf@ns%
  \cslet{#1deps}\manifest@mf@deps%
  \ifcsvoid{manifest@mf@id}{}{%
    \cslet{#1dir}\temp@archive@dir%
  }%
}
%    \end{macrocode}
% \test{
% \immediate\openout\testfile=FOO/MANIFEST.MF
% \immediate\write\testfile{id: FOO/BAR}
% \immediate\write\testfile{ns: http://mathhub.info/FOO/BAR}
% \immediate\closeout\testfile
%\mathhub@parsemanifest{test@}{./FOO}
% id: \test@id\\ns: \test@ns\\dir: \test@dir
% }
%\end{macro}
%\begin{macro}{\mathhub@setcurrentreposinfo}
% |\mathhub@setcurrentreposinfo{|\meta{id}|}| sets the current repository to \meta{id}, checks if the
% |MANIFEST.MF| of this repository has already been read, and if not, finds it, parses it and stores
% the values in |\currentrepos@|\meta{key}|@|\meta{id} for later retrieval.
%    \begin{macrocode}
\def\mathhub@setcurrentreposinfo#1{%
  \edef\mh@currentrepos{#1}%
  \ifx\mh@currentrepos\@empty%
    \edef\currentrepos@dir{\@Dot}%
    \def\currentrepos@narr{}%
    \def\currentrepos@ns{}%
    \def\currentrepos@id{}%
    \def\currentrepos@deps{}%
  \else%
  \ifcsdef{mathhub@dir@\mh@currentrepos}{%
    \@inmhrepostrue
    \ex\let\ex\currentrepos@dir\csname mathhub@dir@#1\endcsname%
    \ex\let\ex\currentrepos@narr\csname mathhub@narr@#1\endcsname%
    \ex\let\ex\currentrepos@ns\csname mathhub@ns@#1\endcsname%
    \ex\let\ex\currentrepos@deps\csname mathhub@deps@#1\endcsname%
  }{%
    \mathhub@parsemanifest{currentrepos@}{\MathHub{#1}}%
    \@setcurrentreposinfo%
    \ifcsvoid{currentrepos@dir}{\PackageError{stex}{No archive with %
      name #1 found!}{make sure that #1 is directly in your MATHHUB folder %
      and contains a MANIFEST.MF, either directly in #1 or in a meta-inf %
      subfolder.}}{\@inmhrepostrue}%
  }%
  \fi%
}

\def\@setcurrentreposinfo{%
  \edef\mh@currentrepos{\currentrepos@id}%
  \ifcsvoid{currentrepos@dir}{}{%
    \csxdef{mathhub@dir@\currentrepos@id}{\currentrepos@dir}%
    \csxdef{mathhub@narr@\currentrepos@id}{\currentrepos@narr}%
    \csxdef{mathhub@ns@\currentrepos@id}{\currentrepos@ns}%
    \csxdef{mathhub@deps@\currentrepos@id}{\currentrepos@deps}%
  }%
}
%    \end{macrocode}
%\end{macro}
% Finally -- and that is the ultimate goal of all of the above, we set the current repos.  
%    \begin{macrocode}
\newif\if@inmhrepos\@inmhreposfalse
\ifcsvoid{stex@PWD}{}{
\mathhub@parsemanifest{currentrepos@}\stex@PWD
\@setcurrentreposinfo
\ifcsvoid{currentrepos@dir}{\message{sTeX: Not currently in a MathHub repository}}{%
  \message{Current sTeX repository: \mh@currentrepos}
}
}
%    \end{macrocode}
%
% \subsection{Modules}
%
%    \begin{macrocode}
\ifmod@show\if@latexml\else\RequirePackage{mdframed}\fi\fi
%    \end{macrocode}
%
% Aux:
%
%    \begin{macrocode}
%\def\ignorespacesandpars{\begingroup\catcode13=10%
%  \@ifnextchar\relax{\endgroup}{\endgroup}}
%    \end{macrocode}
% and more adapted from \url{http://tex.stackexchange.com/questions/179016/ignore-spaces-and-pars-after-an-environment}
%    \begin{macrocode}
%\def\ignorespacesandparsafterend#1\ignorespaces\fi{#1%
%  \fi\ignorespacesandpars}
%\def\ignorespacesandpars{\ifhmode\unskip\fi\@ifnextchar\par%
%  {\ex\ignorespacesandpars\@gobble}{}}
%    \end{macrocode}
%
% Options for the |module|-environment:
%
%    \begin{macrocode}
\addmetakey*{module}{title}
\addmetakey*{module}{name}
\addmetakey*{module}{creators}
\addmetakey*{module}{contributors}
\addmetakey*{module}{srccite}
\addmetakey*{module}{ns}
\addmetakey*{module}{narr}
\addmetakey*{module}{lang}
\addmetakey*{module}{sig}
\addmetakey*{module}{meta}
%    \end{macrocode}
%
% \begin{macro}{module@heading}
%   We make a convenience macro for the module heading. This can be customized. 
%    \begin{macrocode}
\ifdef{\thesection}{\newcounter{module}[section]}{\newcounter{module}}%
\newrobustcmd\module@heading{%
  \stepcounter{module}%
  \ifmod@show%
  \noindent{\textbf{Module} \thesection.\themodule [\module@name]}%
  \sref@label@id{Module \thesection.\themodule [\module@name]}%
    \ifx\module@title\@empty :\quad\else\quad(\module@title)\hfill\\\fi%
  \fi%
}%
%    \end{macrocode}
% \test{
% \begin{module}[name=Test]
%    Foo
% \end{module}
% }
% \makeatletter\mod@showfalse\makeatother
%
% \end{macro}
%
% \begin{environment}{module}
%   Finally, we define the begin module command for the module environment. Much of the
%   work has already been done in the keyval bindings, so this is quite simple.
%    \begin{macrocode}

% meta-theory
\def\metatheory@none{None}
\def\stex@metatheory#1{\stex@annotate@invisible{metatheory}{#1}{}}%
\def\stex@defaultmeta{}
\ifcsdef{MathHub}{
  \stex@iffileexists{\mathhub@path/FoMID/Foundation/source/Meta}{
    \def\stex@defaultmeta{[FoMID/Foundation]?Meta}
  }{}
}{}
\stex@debug{Default meta theory: \stex@defaultmeta}

\def\stex@do@metatheory{%
  \edef\metatheory@path{\module@meta}%
  \ifx\metatheory@path\metatheory@none%
    \stex@metatheory{}%
  \else%
    \ifx\metatheory@path\@empty%
      \ifx\stex@defaultmeta\@empty\else%
        \let\metatheory@path\stex@defaultmeta%
        \metatheory@splitoptions%
        \stex@debug{Metatheory: \metatheory@options:\metatheory@path \if@inimport(in import)\fi}%
        \if@inimport\ex\importmodule\ex[\metatheory@options]\metatheory@path\else%
          \if@htmlmode\stex@annotate@invisible{metatheory}{}{\ex\importmodule\ex[\metatheory@options]\metatheory@path}\else
            \ex\importmodule\ex[\metatheory@options]\metatheory@path
          \fi%
        \fi%
      \fi%
    \else%
      \metatheory@splitoptions%
      \stex@debug{Metatheory: \metatheory@options:\metatheory@path \if@inimport(in import)\fi}%
      \if@inimport\ex\importmodule\ex[\metatheory@options]\metatheory@path\else%
        \if@htmlmode\stex@annotate@invisible{metatheory}{}{\ex\importmodule\ex[\metatheory@options]\metatheory@path}\else
          \ex\importmodule\ex[\metatheory@options]\metatheory@path
        \fi%
      \fi%
    \fi%
  \fi%
  \parsemodule@maybesetcodes%
}
\def\metatheory@splitoptions{%
  \IfBeginWith\metatheory@path{[}{%
    \ex\metatheory@parseoptions\metatheory@path\end%
  }{%
    \def\metatheory@options{}%
  }%
}

\def\metatheory@parseoptions[#1]#2\end{%
  \def\metatheory@options{#1}%
  \def\metatheory@path{#2}%
}

\newenvironment{module}[1][]{%
  \begin{@module}[#1]%
  \module@heading% make the headings
  %\ignorespacesandpars
  \parsemodule@maybesetcodes}{%
  \end{@module}%
  \ignorespacesafterend%
}%
\ifmod@show\surroundwithmdframed{module@om@common}\fi%
%    \end{macrocode}
% \end{environment}
%
% Some auxiliary methods:
%    \begin{macrocode}
\def\g@addto@macro@safe#1#2{\ifx#1\relax\def#1{}\fi\g@addto@macro#1{#2}}
\def\addto@thismodule#1{%
  \@ifundefined{this@module}{}{%
    \ex\g@addto@macro@safe\this@module{#1}%
  }%
}
\def\addto@thismodulex#1{%
\@ifundefined{module@name}{}{%
  %\stex@debug{adding \detokenize{#1} to \module@name}%
  \edef\addto@thismodule@exp{#1}%
  \ex\ex\ex\g@addto@macro@safe%
  \ex\this@module\ex{\addto@thismodule@exp}%
}}
%    \end{macrocode}
%
% \begin{environment}{@module}
%   A variant of the |module| environment that does not create printed representations (in
%   particular no frames).\medskip
%
%   To compute the \meta{uri} of a module, |\set@default@ns| computes the namespace, if
%   none is provided as an optional argument, as follows:
%
%   If the file of the module is |/some/path/file.tex| and we are not in a |MathHub|
%   repository, the namespace is |file:///some/path|.
%
%   If the file of the module is |/some/path/in/mathhub/repo/sitory/source/sub/file.tex|
%   and |repo/sitory| is an archive in the |MathHub| root, and the |MANIFEST.MF| of
%   |repo/sitory| declares a namespace |http://some.namespace/foo|, then the namespace of
%   the module is |http://some.namespace/foo/sub|.
%    \begin{macrocode}
\newif\ifarchive@ns@empty@\archive@ns@empty@false
\def\set@default@ns{%
  \edef\@module@ns@temp{\stex@currpath}%
  \if@iswindows@\windows@to@path\@module@ns@temp\fi%
  \archive@ns@empty@false%
  %\stex@debug{Generate new namespace^^J  Filepath: \@module@ns@temp}%
  \ifcsvoid{mh@currentrepos}{\archive@ns@empty@true}%
  {\ex\ifx\csname mathhub@ns@\mh@currentrepos\endcsname\@empty\archive@ns@empty@true\fi%
  }%
  %\stex@debug{  \ifarchive@ns@empty@ Namespace empty\else Namespace not empty\fi}%
  \ifarchive@ns@empty@%
    \edef\@module@ns@tempuri{file\@Colon\@Slash\@Slash\@module@ns@temp}%
  \else%
    \edef\@module@filepath@temppath{\@module@ns@temp}%
    \edef\@module@ns@tempuri{\csname mathhub@ns@\mh@currentrepos\endcsname}%
    \edef\@module@archivedirpath{\csname mathhub@dir@\mh@currentrepos\endcsname\@Slash source}%
    \edef\@module@archivedirpath{\ex\detokenize\ex{\@module@archivedirpath}}%
    \IfBeginWith\@module@filepath@temppath\@module@archivedirpath{%
      \StrLen\@module@archivedirpath[\ns@temp@length]%
      \StrGobbleLeft\@module@filepath@temppath\ns@temp@length[\@module@filepath@temprest]%
      \edef\@module@ns@tempuri{\@module@ns@tempuri\@module@filepath@temprest}%
    }{}%
  \fi%
  \IfEndWith\@module@ns@tempuri\@Slash{\StrGobbleRight\@module@ns@tempuri1[\@module@ns@tempuri]}{}%
  \setkeys{module}{ns=\@module@ns@tempuri}%
}
%    \end{macrocode}
% \test{\set@default@ns\module@ns}
%
% If the module is not given a |name|, |\set@next@moduleid| computes one by 
% enumeration via the filename, e.g. |stex|, |stex1|, etc.
%
%    \begin{macrocode}
\def\set@next@moduleid{%
  \path@filename@noext\stex@currfile\stex@next@moduleid@filename%
  \edef\set@nextmoduleid@csname{namespace@\module@ns\@QuestionMark\stex@next@moduleid@filename @unnamedmodules}%
  \unless\ifcsname\set@nextmoduleid@csname\endcsname%
      \csgdef{\set@nextmoduleid@csname}{0}%
  \fi%
  \edef\namespace@currnum{\csname\set@nextmoduleid@csname\endcsname}%
  \edef\module@temp@setidname{\noexpand\setkeys{module}{name=%
    \stex@next@moduleid@filename\ex\unless\ex\ifnum\csname\set@nextmoduleid@csname\endcsname=0.\namespace@currnum\fi}}%
  \module@temp@setidname%
  \csxdef{\set@nextmoduleid@csname}{\the\numexpr\namespace@currnum+1}%
}
%    \end{macrocode}
% \test{\set@next@moduleid\module@name\\\set@next@moduleid\module@name}
%
% Finally, the |@module| environment does the actual work, i.e. setting metakeys, computing
% namespace/id, defining |\this@module|, etc.
% 
%  
%   For a module with name \meta{name} (|\module@name|) and uri \meta{uri} (|\module@uri|), this defines the following macros: 
%   \begin{itemize}
%      \item |\module@defs@|\meta{uri}
%   that acts as a repository for semantic macros of the current module. It will be called
%   by |\importmodule| to activate them. 
%     \item We will add the internal forms of the semantic
%   macros whenever |\symdef| is invoked. To do this, we will need an unexpanded form
%   |\this@module| that expands to |\module@defs@|\meta{uri}; we define it first and then
%   initialize |\module@defs@|\meta{uri} as empty. 
%      \item |\module@names@|\meta{uri} will store all symbol names declared in this module.
%      \item |\module@imports@|\meta{uri} will store the URIs of all modules directly included
%         in this module
%     \item |\|\meta{uri} that expands to |\invoke@module{|\meta{uri}|}| (see below).
%     \item |\stex@module@|\meta{name} that expands to \meta{uri}, if unambiguous, otherwise
%        to |ambiguous|.
%   \end{itemize}
%   If we are currently in a mathhub repository, this information will also
%   be stored in |\module@defs@|\meta{uri}, so we can resolve includes properly when this
%   module is activated.
%
%    \begin{macrocode}
\edef\stex@dot@tex{\detokenize{.tex}}
\newenvironment{@module}[1][]{%
  \metasetkeys{module}{#1}%
  \ifcsvoid{module@lang}{}{%
    \ifcsvoid{main@language}{}{%
      \ifcsvoid{stex@babel@long\module@lang}{
        \PackageError{stex}{unknown language: \module@lang}{}
      }{%
        \edef\module@current@language{\csname stex@babel@long\module@lang\endcsname}%
        \ex\selectlanguage\ex{\module@current@language}%
      }%
    }%
  }%
  \ifcsvoid{module@name}{\let\module@name\module@id}{}% % TODO deprecate
  \ifcsvoid{module@name}{\set@next@moduleid}{}%
  \let\module@id\module@name% % TODO deprecate
  \ifcsvoid{currentmodule@uri}{%
    \ifx\module@ns\@empty\set@default@ns\fi%
    \ifx\module@narr\@empty%
      \setkeys{module}{narr=\module@ns}%
    \fi%
  }{%
    \if@smsmode%
      \ifx\module@ns\@empty\set@default@ns\fi%
      \ifx\module@narr\@empty%
        \setkeys{module}{narr=\module@ns}%
      \fi%
    \else%
      % Nested Module:
      \stex@debug{Nested module! Parent: \currentmodule@uri}%
      \setkeys{module}{name=\currentmodule@name\@Slash\module@name}%
      \let\module@id\module@name % TODO deprecate
      \setkeys{module}{ns=\currentmodule@ns}%
    \fi%
  }%
  \edef\module@uri{\module@ns\@QuestionMark\module@name}%
  \csgdef{module@names@\module@uri}{}%
  \csgdef{module@imports@\module@uri}{}%
  \csxdef{\module@uri}{\noexpand\@invoke@module{\module@uri}}%
  \begin{stex@annotateenv}{theory}{\module@uri}%
  \ifcsvoid{module@sig}{%
    \ifcsvoid{stex@module@\module@name}{%
      \ex\global\ex\let\csname stex@module@\module@name\ex\endcsname\csname\module@uri\endcsname%
    }{%
      \ex\edef\csname stex@module@\module@name\endcsname{\detokenize{ambiguous}}%
    }%
    \csdef{module@defs@\module@uri}{}%
  }{%
    %\stex@debug{Module in file \stex@currfile}%
    \IfEndWith\stex@currfile\stex@dot@tex{}{%
      \edef\stex@currfile{\stex@currfile\stex@dot@tex}%
      %\stex@debug{updated to \stex@currfile}%
    }%
    \path@filename\stex@currfile\module@lang@filename%
    \ifcsvoid{module@lang}{%
      \PackageError{stex}{Module \module@uri: Signature \module@sig %
        provided, but no language set.}{Use lang=<language> in module}%
    }{}%
    \edef\module@lang{\ex\detokenize\ex{\module@lang}}%
    \IfEndWith\module@lang@filename{\module@lang\stex@dot@tex}{}{%
      \PackageError{stex}{multilingual module \module@uri~not in file ending with %
        >>\module@lang\stex@dot@tex<<^^JFilename:>>\module@lang@filename<<}{}%
    }%
    \StrLen{\module@lang\stex@dot@tex}[\module@lang@length]%
    \StrGobbleRight\stex@currfile{\module@lang@length}[\module@sig@file]%
    \requiremodules{\module@sig@file\module@sig\stex@dot@tex}%
    %\let\old@ifinimport\if@inimport%
    %\@inimporttrue%
    \csname module@defs@\module@uri\endcsname%
    %\let\if@inimport\old@ifinimport%
  }%
  \edef\this@module{%
    \ex\noexpand\csname module@defs@\module@uri\endcsname%
  }%
  \ex\xdef\csname stex@lastmodule@\module@name\endcsname{\module@uri}%
  \csdef{module@defs@\module@uri}{}%
  \ifcsvoid{mh@currentrepos}{}{%
    \@inmhrepostrue%
    \addto@thismodulex{\ex\edef\ex\noexpand\csname mh@old@repos@\module@uri\endcsname%
      {\noexpand\mh@currentrepos}}%
    \addto@thismodulex{\noexpand\mathhub@setcurrentreposinfo{\mh@currentrepos}}%
  }%
  \let\currentmodule@name\module@name%
  \let\currentmodule@ns\module@ns%
  \let\currentmodule@uri\module@uri%
  \ex\let\csname\module@name\ex\endcsname\csname\module@uri\endcsname%
  \addto@thismodulex{%
    \let\ex\noexpand\csname\module@name\endcsname\ex\noexpand\csname\module@uri\endcsname%
  }%
  \stex@debug{New module: \module@uri}%
  \stex@debug{Signature : \module@sig  Language : \module@lang}%
  \stex@annotate{header}{}{%
    \stex@annotate@invisible{signature}{\module@sig}{}%
    \stex@annotate@invisible{language}{\module@lang}{}%
    \stex@do@metatheory%
  }%
  \parsemodule@maybesetcodes%
}{%
  \if@inmhrepos%
  \@inmhreposfalse%
  \addto@thismodulex{\noexpand\mathhub@setcurrentreposinfo{\ex\noexpand\csname mh@old@repos@\module@uri\endcsname}}%
  \fi%
  \end{stex@annotateenv}%
}%

\newenvironment{@structural@feature}[3]{%
  \ifcsvoid{currentmodule@uri}{%
    \set@default@ns\let\currentmodule@ns\module@ns%
    \set@next@moduleid\let\currentmodule@name\module@name%
  }{}%
  \edef\currentmodule@name{\currentmodule@name\@Slash#2-feature}%
  \begin{stex@annotateenv}{feature:#1}{\currentmodule@uri\@QuestionMark#2}%
  \edef\currentmodule@uri{\currentmodule@ns\@QuestionMark\currentmodule@name}%
  \let\module@uri\currentmodule@uri%
  \csdef{module@defs@\module@uri}{}%
  \edef\this@module{%
    \ex\noexpand\csname module@defs@\module@uri\endcsname%
  }%
  \stex@annotate@invisible{header}{}{#3}%
  \parsemodule@maybesetcodes%
}{%
  \end{stex@annotateenv}%
}%

\newcommand\structural@feature[4]{\begingroup%
  \ifcsvoid{currentmodule@uri}{%
    \set@default@ns\let\currentmodule@ns\module@ns%
    \set@next@moduleid\let\currentmodule@name\module@name%
  }{}%
  \edef\currentmodule@name{\currentmodule@name\@Slash#2-feature}%
  \stex@annotate{feature:#1}{\currentmodule@uri\QuestionMark#2}{%
    \edef\currentmodule@uri{\currentmodule@ns\@QuestionMark\currentmodule@name}%
    \let\module@uri\currentmodule@uri%
    \csdef{module@defs@\module@uri}{}%
    \edef\this@module{%
      \ex\noexpand\csname module@defs@\module@uri\endcsname%
    }%
  \stex@annotate@invisible{header}{}{#3}%
  #4}%
\endgroup}

\def\term@annotate{\@ifnextchar*{\term@annotate@star}{\term@annotate@nostar}}%

\def\term@annotate@nostar#1#2{%
  \stex@annotate{term}{#1}{#2}%
}
\def\term@annotate@star*#1#2{%
  \stex@annotate@invisible{term}{#1}{#2}%
}


\def\variable@annotate{\@ifnextchar*{\variable@annotate@star}{\variable@annotate@nostar}}%

\def\variable@annotate@nostar#1#2{%
  \stex@annotate@invisible{variable}{#1}{#2}%
}

\def\variable@annotate@star*#1#2{%
  \stex@annotate@invisible{variable:implicit}{#1}{#2}%
}

\def\module@export#1#2{%
  \stex@debug{Exporting \unexpanded{#2}}%
  \addto@thismodule{#2}%
  #1#2\parsemodule@maybesetcodes%
}
%    \end{macrocode}
% \test{
%    \begin{module}[name=Foo]
%      Name: \module@name\\
%      URI: \module@uri\\
%      this@module: \expandafter\present\csname module@defs@\module@uri\endcsname
%    \end{module}
%}
%
% \test{
%  Faking a MathHub archive Foo/Bar with URI \texttt{\detokenize{http://foo.bar/baz}}:\\
%  \edef\currentrepos@id{Foo/Bar}
%  \edef\currentrepos@dir{.}
%  \edef\currentrepos@ns{http://foo.bar/baz}
%  \edef\currentrepos@narr{}
%  \edef\currentrepos@deps{}
%  \@setcurrentreposinfo
%    \begin{module}[name=Foo2]\\
%      Name: \module@name\\
%      URI: \module@uri\\
%      this@module: \ex\present\csname module@defs@\module@uri\endcsname
%    \end{module}
%}
%
% \test{
%  Removing the |\MathHub| system variable first: \let\MathHub\@undefined\\
%    \begin{module}[name=Foo]\\
%      Name: \module@name\\
%      URI: \module@uri\\
%      this@module: \ex\present\csname module@defs@\module@uri\endcsname
%    \end{module}}
%
% \test{
%  Faking a MathHub archive Foo/Bar with URI \texttt{\detokenize{http://foo.bar/baz}}:\\
%  \edef\currentrepos@id{Foo/Bar}
%  \edef\currentrepos@dir{.}
%  \edef\currentrepos@ns{http://foo.bar/baz}
%  \edef\currentrepos@narr{}
%  \edef\currentrepos@deps{}
%  \@setcurrentreposinfo
%    \begin{module}[name=Foo2]\\
%      Name: \module@name\\
%      URI: \module@uri\\
%      this@module: \expandafter\present\csname module@defs@\module@uri\endcsname
%    \end{module}
%}
%
% A module with URI \meta{uri} and id \meta{id} creates two macros |\|\meta{uri} and |\stex@module@|\meta{id},
% that ultimately expand to |\@invoke@module{|\meta{uri}|}|. Currently, the only functionality is
% |\@invoke@module{|\meta{uri}|}\@URI|, which expands to the full uri of a module (i.e. via
% |\stex@module@|\meta{id}|\@URI|). In the future, this macro can be extended with additional functionality,
% e.g. accessing symbols in a macro for overloaded (macro-)names.
%
%    \begin{macrocode}
\def\@URI{uri} % TODO check this
\def\@invoke@module#1#2{%
  \ifx\@URI#2%
    #1%
  \else%
    % TODO something else
    %\def\invoke@module@next{\csname#1\@QuestionMark#2\endcsname}%
    %\ex\invoke@module@next%
    \csname#1\@QuestionMark#2\ex\endcsname%
  \fi%
}
%    \end{macrocode}
% \end{environment}
%
% \subsection{Inheritance}
%
% \subsubsection{Selective Inclusion}
%
% The next great goal is to establish the |\requiremodules| macro, which reads an \sTeX
% file and processes all the module signature information in them, but does not produce
% any output. This is a tricky business, as we need to ``parse'' the modules and treat the
% module signature macros specially (we refer to this as ``\defemph{sms mode}'', since it
% is equivalent to what the -- now deprecated -- |sms| utility did).
%
% In the following we introduce a lot of auxiliary functionality before we can define
% |\requiremodules|.
%
% \begin{macro}{\parsemodule@allow*}
%   The first step is setting up a functionality for registering |\sTeX| macros and
%   environments as part of a module signature. 
%    \begin{macrocode}
\newif\if@smsmode\@smsmodefalse
\def\parsemodule@allow#1{%
  \ex\def\csname parsemodule@allowedmacro@#1\ex\endcsname\ex{\csname#1\endcsname}%
}
\def\parsemodule@allowenv#1{%
  \ex\def\csname parsemodule@allowedenv@#1\endcsname{#1}%
}
\def\parsemodule@replacemacro#1#2{%
  \ex\def\csname parsemodule@allowedmacro@#1\ex\endcsname\ex{\csname#2\endcsname}%
}
\def\parsemodule@replaceenv#1#2{%
  \ex\def\csname parsemodule@allowedenv@#1\endcsname{#2}%
}
\def\parsemodule@escapechar@beginstring{begin}
\def\parsemodule@escapechar@endstring{end}
%    \end{macrocode}
% \end{macro}
% 
% and now we use that to actually register all the \sTeX functionality as relevant for
% |sms| mode. 
%
%    \begin{macrocode}
\def\parsemodule@oldmakeatletter{\parsemodule@maybesetcodes\catcode`@=11}
\def\parsemodule@oldmakeatother{\parsemodule@maybesetcodes\catcode`@=12}
\parsemodule@replacemacro{makeatletter}{parsemodule@oldmakeatletter}
\parsemodule@replacemacro{makeatother}{parsemodule@oldmakeatother}
\parsemodule@allow{symdef}
\parsemodule@allow{abbrdef}
\parsemodule@allow{module@export}
\parsemodule@allow{importmodule}
\parsemodule@allow{mmt@symdecl}
\parsemodule@allow{instantiates}
\parsemodule@allow{setnotation}
\parsemodule@allowenv{module}
\parsemodule@allowenv{@module}
\parsemodule@allow{importmhmodule}
\parsemodule@allow{gimport}
\parsemodule@allowenv{modsig}
\parsemodule@allowenv{mhmodsig}
\parsemodule@allowenv{mhmodnl}
\parsemodule@allowenv{modnl}
\parsemodule@allowenv{@structural@feature}
\parsemodule@allow{symvariant}
\parsemodule@allow{structural@feature}
\parsemodule@allow{symi}
\parsemodule@allow{symii}
\parsemodule@allow{symiii}
\parsemodule@allow{symiv}
\parsemodule@allow{notation}
\parsemodule@allow{symdecl}

% to deprecate:

\parsemodule@allow{defi}
\parsemodule@allow{defii}
\parsemodule@allow{defiii}
\parsemodule@allow{defiv}
\parsemodule@allow{adefi}
\parsemodule@allow{adefii}
\parsemodule@allow{adefiii}
\parsemodule@allow{adefiv}
\parsemodule@allow{defis}
\parsemodule@allow{defiis}
\parsemodule@allow{defiiis}
\parsemodule@allow{defivs}
\parsemodule@allow{Defi}
\parsemodule@allow{Defii}
\parsemodule@allow{Defiii}
\parsemodule@allow{Defiv}
\parsemodule@allow{Defis}
\parsemodule@allow{Defiis}
\parsemodule@allow{Defiiis}
\parsemodule@allow{Defivs}
%    \end{macrocode}
% 
% To read external modules without producing output, |\requiremodules| redefines the \textbackslash-character
% to be an \emph{active} character that, instead of executing a macro, checks whether a macro name
% has been registered using |\parsemodule@allow| before selectively executing the corresponding macro or ignoring it. 
% To produce the relevant code, we therefore
% define a macro |\@active@slash| that produces a \textbackslash-character with category code 13 (\emph{active}),
% as well as |\@open@brace| and |\@close@brace|, which produce open and closing braces with
% category code 12 (\emph{other}).
%
%    \begin{macrocode}
\catcode`\.=0
.catcode`.\=13
.def.@active@slash{\}
.catcode`.<=1
.catcode`.>=2
.catcode`.{=12
.catcode`.}=12
.def.@open@brace<{>
.def.@close@brace<}>
.catcode`.\=0
\catcode`\.=12
\catcode`\{=1
\catcode`\}=2
\catcode`\<=12
\catcode`\>=12
%    \end{macrocode}
%
% The next two macros set and reset the category codes before/after |sms| mode.
%
% \begin{macro}{\set@parsemodule@catcodes}
%    \begin{macrocode}
  \def\parsemodule@ignorepackageerrors{,inputenc,}
  \let\parsemodule@old@PackageError\PackageError
  \def\parsemodule@packageerror#1#2#3{%
    \IfSubStr\parsemodule@ignorepackageerrors{,#1,}{}{%
      \parsemodule@old@PackageError{#1}{#2}{#3}%
    }%
  }
  \def\set@parsemodule@catcodes{%
      \ifnum\catcode`\\=0%
      \global\catcode`\\=13%
      \global\catcode`\#=12%
      \global\catcode`\{=12%
      \global\catcode`\}=12%
      \global\catcode`\$=12%$
      \global\catcode`\^=12%
      \global\catcode`\_=12%
      \global\catcode`\&=12%
      \ex\global\ex\let\@active@slash\parsemodule@escapechar%
      \global\let\parsemodule@old@PackageError\PackageError%
      \global\let\PackageError\parsemodule@packageerror%
      \fi%
  }
%    \end{macrocode}
% \end{macro}
%
% \begin{macro}{\reset@parsemodule@catcodes}
%    \begin{macrocode}
  \def\reset@parsemodule@catcodes{%
      \ifnum\catcode`\\=13%
      \global\catcode`\\=0%
      \global\catcode`\#=6%
      \global\catcode`\{=1%
      \global\catcode`\}=2%
      \global\catcode`\$=3%$
      \global\catcode`\^=7%
      \global\catcode`\_=8%
      \global\catcode`\&=4%
      \global\let\PackageError\parsemodule@old@PackageError%
      \fi%
  }
%    \end{macrocode}
% \end{macro}
%
% \begin{macro}{\parsemodule@maybesetcodes}
%   Before a macro is executed in |sms|-mode, the category codes will be reset to normal,
%   to ensure that all macro arguments are parsed correctly. Consequently, the macros
%   need to set the category codes back to |sms| mode after having read all arguments iff
%   the macro got executed in
%   |sms| mode. |\parsemodule@maybesetcodes| takes care of that.
%    \begin{macrocode}
  \def\parsemodule@maybesetcodes{%
    \if@smsmode\set@parsemodule@catcodes\fi%
  }
%    \end{macrocode}
% \end{macro}
%
% 
% \begin{macro}{\parsemodule@escapechar}
% This macro gets called whenever a \textbackslash-character occurs in |sms| mode.
% It is split into several macros that parse and store characters in
% |\parsemodule@escape@currcs| until a character
% with category code $\neq$ 11 occurs (i.e. the macro name is complete), check
% whether the macro is allowed in |sms| mode, and then either ignore it or execute it
% after setting category codes back to normal. Special care needs to be taken to make
% sure that braces have the right category codes (1 and 2 for open and closing braces,
% respectively) when delimiting macro arguments.
%
% Entry point:
%    \begin{macrocode}

\def\parsemodule@escapechar{%
    \def\parsemodule@escape@currcs{}%
    \parsemodule@escape@parse@nextchar@%
}%
%    \end{macrocode}
% The next macro simply reads the next character and checks whether it has category code 11.
% If so, it stores it in |\parsemodule@escape@currcs|. Otherwise, the macro name
% is complete, it stores the last
% character in |\parsemodule@last@char| and calls |\parsemodule@escapechar@checkcs|.
%
%    \begin{macrocode}
\long\def\parsemodule@escape@parse@nextchar@#1{%
    \ifcat a#1\relax%
        \edef\parsemodule@escape@currcs{\parsemodule@escape@currcs#1}%
        \let\parsemodule@do@next\parsemodule@escape@parse@nextchar@%
    \else%
      \def\parsemodule@last@char{#1}%
      \ifx\parsemodule@escape@currcs\@empty%
        \def\parsemodule@do@next{}%
      \else%
        \def\parsemodule@do@next{\parsemodule@escapechar@checkcs}%
      \fi%
    \fi%
    \parsemodule@do@next%
}
%    \end{macrocode}
%
% The next macro checks whether the currently stored macroname is allowed in |sms| mode.
% There are four cases that need to be considered: |\begin|, |\end|, allowed macros, and others.
% In the first two cases, we reinsert |\parsemodule@last@char| and continue with |\parsemodule@escapechar@checkbeginenv|
% or |\parsemodule@escapechar@checkendenv| respectively, to check whether the environment
% being openend/closed is allowed in |sms| mode. In both cases, |\parsemodule@last@char|
% is an open brace with category code 12. In the third case, we need to check whether
% |\parsemodule@last@char| is an open brace, in which case we call |\parsemodule@converttoproperbraces|,
% otherwise, we set category codes to normal and execute the macro.
% In the fourth case, we just reinsert |\parsemodule@last@char| and continue.
%    \begin{macrocode}
\def\parsemodule@escapechar@checkcs{%
    \ifx\parsemodule@escape@currcs\parsemodule@escapechar@beginstring%
        \edef\parsemodule@do@next{\noexpand\parsemodule@escapechar@checkbeginenv\parsemodule@last@char}%
    \else%
        \ifx\parsemodule@escape@currcs\parsemodule@escapechar@endstring%
          \edef\parsemodule@do@next{\noexpand\parsemodule@escapechar@checkendenv\parsemodule@last@char}%
        \else%
            \ifcsvoid{parsemodule@allowedmacro@\parsemodule@escape@currcs}{%
              \def\parsemodule@do@next{\relax\parsemodule@last@char}%
            }{%
              \ifx\parsemodule@last@char\@open@brace%
                \ex\let\ex\parsemodule@do@next@ii\csname parsemodule@allowedmacro@\parsemodule@escape@currcs\endcsname%
                \edef\parsemodule@do@next{\noexpand\parsemodule@converttoproperbraces\@open@brace}%
              \else%
                \reset@parsemodule@catcodes%
                \edef\parsemodule@do@next{\ex\noexpand\csname parsemodule@allowedmacro@\parsemodule@escape@currcs\endcsname\ex\noexpand\parsemodule@last@char}%
              \fi%
            }%
        \fi%
    \fi%
    \parsemodule@do@next%
}
%    \end{macrocode}
% This macro simply takes an argument in braces (with category codes 12), reinserts it
% with ``proper'' braces (category codes 1 and 2), sets category codes back to normal and calls |\parsemodule@do@next@ii|,
% which has been |\let| as the macro to be executed.
%
%    \begin{macrocode}
\ex\ex\ex\def%
\ex\ex\ex\parsemodule@converttoproperbraces%
\ex\@open@brace\ex#\ex1\@close@brace{%
  \reset@parsemodule@catcodes%
  \parsemodule@do@next@ii{#1}%
}
%    \end{macrocode}
%
% The next two macros apply in the |\begin| and |\end| cases. They check whether the
% environment is allowed in |sms| mode, if so, open/close the environment, and otherwise
% do nothing.
%
% Notably, |\parsemodule@escapechar@checkendenv| does not set category codes back to normal,
% since |\end{environment}| never takes additional arguments that need to be parsed anyway.
%
%    \begin{macrocode}
\ex\ex\ex\def%
\ex\ex\ex\parsemodule@escapechar@checkbeginenv%
\ex\@open@brace\ex#\ex1\@close@brace{%
    \ifcsvoid{parsemodule@allowedenv@#1}{%
      \def\parsemodule@do@next{#1}%
    }{%
      \reset@parsemodule@catcodes%
      \edef\parsemodule@envname{\csname parsemodule@allowedenv@#1\endcsname}%
      \ex\def\ex\parsemodule@do@next\ex{%
        \ex\begin\ex{\parsemodule@envname}%
      }%
    }%
    \parsemodule@do@next%
}
\ex\ex\ex\def%
\ex\ex\ex\parsemodule@escapechar@checkendenv%
\ex\@open@brace\ex#\ex1\@close@brace{%
  \ifcsvoid{parsemodule@allowedenv@#1}{%
      \def\parsemodule@do@next{#1}%
    }{%
      \edef\parsemodule@envname{\csname parsemodule@allowedenv@#1\endcsname}%
      \ex\def\ex\parsemodule@do@next\ex{%
        \ex\end\ex{\parsemodule@envname}%
      }%
    }%
    \parsemodule@do@next%
}
%    \end{macrocode}
% \end{macro}
% 
% \begin{macro}{\@requiremodules}
%   the internal version of |\requiremodules| for use in the |*.aux| file. We disable it
%   at the end of the document, so that when the |aux| file is read again, nothing is
%   loaded.
%    \begin{macrocode}
\newrobustcmd\@requiremodules[1]{%
  \if@tempswa\requiremodules{#1}\fi%
}%
%    \end{macrocode}
% \end{macro}
%
% \begin{macro}{\requiremodules}
%   This macro loads the module signatures in a file using the |\requiremodules@smsmode|
%   above. We set the flag |\mod@showfalse| in the local group, so that the macros know
%   now to pollute the  result.
%    \begin{macrocode}
  \newrobustcmd\requiremodules[1]{%
    \mod@showfalse%
    \edef\mod@path{#1}%
    \edef\mod@path{\ex\detokenize\ex{\mod@path}}%
    \stex@debug{Reading in file \mod@path}%
    \requiremodules@smsmode{#1}%
  }%
%    \end{macrocode}
% \end{macro}
%
% \begin{macro}{\requiremodules@smsmode}
%   this reads \sTeX modules by setting the category codes for |sms| mode,
%   |\input|ting the required file and wrapping it in a |\vbox| that gets stored away
%   and ignored, in order to not produce any output. It also sets |\hbadness|, |\hfuzz|
%   and friends to values that suppress overfull and underfull hbox messages.
%    \begin{macrocode}
  \newbox\modules@import@tempbox
  \def\requiremodules@smsmode#1{%
    \setbox\modules@import@tempbox\vbox{%
      \@smsmodetrue%
      \set@parsemodule@catcodes%
      \hbadness=100000\relax%
      \hfuzz=10000pt\relax%
      \vbadness=100000\relax%
      \vfuzz=10000pt\relax%
      \edef\requiremodules@path@temp{#1}%
      \IfEndWith\requiremodules@path@temp\stex@dot@tex%
        {}{\edef\requiremodules@path@temp{\requiremodules@path@temp.tex}}%
      \message{^^J(\requiremodules@path@temp}%
      \stex@currfile@push\requiremodules@path@temp%
      \requiremodules@readfile%
      \ex\scantokens\ex{\requiremodules@filecontent^^J}%
      \message{)}%
      \stex@currfile@pop%
      \reset@parsemodule@catcodes%
    }%
    \parsemodule@maybesetcodes%
  }
  \newread\usemodules@fileread%
  \def\requiremodules@readfile{%
  \begingroup%
    \immediate\openin\usemodules@fileread=\requiremodules@path@temp\relax%
    \catcode`\\=12%
    \catcode`\%=12%
    \def\requiremodules@filecontent{}%
    \requiremodules@readfileI%
    \immediate\closein\usemodules@fileread%
    \xdef\requiremodules@filecontent{\requiremodules@filecontent}%
  \endgroup%
  }
  \def\requiremodules@readfileI{%
    \ifeof\usemodules@fileread\else%
    \read\usemodules@fileread to \requiremodules@filecontent@temp\relax%
    \ex\ex\ex\def\ex\ex\ex\requiremodules@filecontent\ex\ex\ex{%
      \ex\requiremodules@filecontent\requiremodules@filecontent@temp^^J}%
    \ex\requiremodules@readfileI%
    \fi%
  }
%    \end{macrocode}
% \test{
% \immediate\openout\testfile=FOO/testmodule.tex
% \edef\firststring{\detokenize{\begin}}
% \path@trimstring\firststring
% \edef\firststring{\firststring{module}[name=testmodule]}
% \immediate\write\testfile{\firststring}
% \immediate\write\testfile{  foo bar boo far}
% \edef\firststring{\detokenize{\end}}
% \path@trimstring\firststring
% \edef\firststring{\firststring{module}}
% \immediate\write\testfile{\firststring}
% \immediate\closeout\testfile
% \mathhub@parsemanifest{test@}{./FOO}
% parsing |FOO/testmodule.tex|\\
% \requiremodules@smsmode{FOO/testmodule}
% \present\stex@module@testmodule
% }
% \end{macro}
%
% \subsubsection{importmodule}
%
% \begin{macro}{\importmodule@bookkeeping}
%    \begin{macrocode}
\newif\if@importmodule@switchrepos\@importmodule@switchreposfalse
\def\importmodule@bookkeeping#1#2#3{%
  \@importmodule@switchreposfalse%
  \stex@debug{Importmodule: #1^^J  #2}%
  \edef\importmodule@mhrepos{#1}%
  \ifcsvoid{importmodule@mhrepos}{%
    \ifcsvoid{currentrepos@dir}{%
      %\stex@debug{Importmodule: Set importmodule@dir to \stex@PWD}%
      \let\importmodule@dir\stex@PWD%
    }{%
      %\stex@debug{Importmodule: Set importmodule@dir to \currentrepos@dir\@Slash source}%
      \edef\importmodule@dir{\currentrepos@dir\@Slash source}%
    }%
  }{%
    \@importmodule@switchrepostrue%
    %\stex@debug{Importmodule: Repository switch to \importmodule@mhrepos}%
    %\stex@debug{Importmodule: Current repos: \mh@currentrepos}%
    \ex\let\csname importmodule@oldrepos@#2\endcsname\mh@currentrepos%
    \mathhub@setcurrentreposinfo\importmodule@mhrepos%
    %\stex@debug{Importmodule: New repos: \mh@currentrepos^^J  Namespace: \currentrepos@ns}%
    \edef\importmodule@dir{\currentrepos@dir\@Slash source}%
  }%
  \StrCut{#2}\@QuestionMark\importmodule@subdir\importmodule@modulename%
  \ifx\importmodule@modulename\@empty%
    \let\importmodule@modulename\importmodule@subdir%
    \let\importmodule@subdir\@empty%
  \else%
    \ifx\importmodule@subdir\@empty\else%
      \edef\importmodule@dir{\importmodule@dir\@Slash\importmodule@subdir}%
    \fi%
  \fi%
  #3%
  \if@importmodule@switchrepos%
    \ex\mathhub@setcurrentreposinfo\csname importmodule@oldrepos@#2\endcsname%
    %\stex@debug{Importmodule: switched back to: \mh@currentrepos}%
  \fi%
  %\ignorespacesandpars%
}
%    \end{macrocode}
% \end{macro}
%
% \begin{macro}{\importmodule}
%    \begin{macrocode}
%\srefaddidkey{importmodule}
\newcommand\importmodule[2][]{\@@importmodule[#1]{#2}{export}}
\newcommand\@@importmodule[3][]{%
  \importmodule@bookkeeping{#1}{#2}{%
    \@importmodule[\importmodule@dir]\importmodule@modulename{#3}%
  }%
  \parsemodule@maybesetcodes%
}
%    \end{macrocode}
% \end{macro}
%
% \begin{macro}{\@importmodule} 
%   |\@importmodule[|\meta{filepath}|]{|\meta{mod}|}{|\meta{export?}|}|
%   loads \meta{filepath}|.tex| and activates the module \meta{mod}. If \meta{export?} is
%   |export|, then it also re-exports the |\symdef|s from \meta{mod}.
%
%   First |\@load| will store the base file name with full path, then check if
%   |\module@|\meta{mod}|@path| is defined. If this macro is defined, a module of this
%   name has already been loaded, so we check whether the paths coincide, if they do, all
%   is fine and we do nothing otherwise we give a suitable error. If this macro is
%   undefined we load the path by |\requiremodules|.
%    \begin{macrocode}
\newcommand\@importmodule[3][]{%
  {%
    \edef\@load{#1}%
    \edef\@importmodule@name{#2}%
    \stex@debug{Loading #2 in #1}%
    \ifcsvoid{stex@module@\@importmodule@name}{% TODO check this
      \stex@iffileexists{\@load.tex}{
        \stex@debug{Exists: #1}%
        \requiremodules\@load%
      }{%
        \stex@debug{Does not exist: #1^^JTrying \@load\@Slash\@importmodule@name}%
        \edef\@nload{\@load\@Slash\@importmodule@name}%
        \stex@iffileexists{\@nload.tex}{%
          \requiremodules{\@nload}%
        }{%
          \stex@debug{Does not exist: \@nload.tex}%
          \ifcsvoid{main@language}{% Try english as default
            \stex@iffileexists{\@load.en.tex}{%
              \requiremodules{\@load.en}%
            }{%
              \requiremodules{\@nload.en}%
            }%
          }{%
            \ifcsvoid{stex@babel@short\languagename}{%
              \PackageError{stex}{Module #1?#2 not found and unknown %
                language \languagename}{}%
            }{%
              \edef\stex@implang{\csname stex@babel@short\languagename\endcsname}%
              \stex@iffileexists{\@load.\stex@implang.tex}{%
                \requiremodules{\@load.\stex@implang}%
              }{%
                \stex@iffileexists{\@load.en.tex}{%
                  \requiremodules{\@load.en}%
                }{%
                  \stex@iffileexists{\@nload.\stex@implang.tex}{%
                    \requiremodules{\@nload.\stex@implang}%
                  }{%
                    \requiremodules{\@nload.en}%
                  }%
                }%
              }%
            }%
          }%
        }%
      }%
    }{}%
    \ifx\@load\@empty\else%
      {% TODO
  %      \edef\@path{\csname module@#2@path\endcsname}%
  %      \IfStrEq\@load\@path{\relax}% if the known path is the same as the requested one do nothing
  %      {\PackageError{stex}% else signal an error
  %        {Module Name Clash\MessageBreak%
  %          A module with name #2 was already loaded under the path "\@path"\MessageBreak%
  %          The imported path "\@load" is probably a different module with the\MessageBreak%
  %          same name; this is dangerous -- not importing}%
  %        {Check whether the Module name is correct}%
  %      }%
      }%
    \fi%
    \global\let\@importmodule@load\@load%
  }%
  \edef\@export{#3}\def\@@export{export}%prepare comparison
  %\ifx\@export\@@export\export@defs{#2}\fi% export the module
  \ifx\@export\@@export\addto@thismodulex{%
    \noexpand\@importmodule[\@importmodule@load]{#2}{noexport}%
  }\fi%
  \ifcsvoid{this@module}{}{%
    \ifcsvoid{module@imports@\module@uri}{
      \csxdef{module@imports@\module@uri}{%
        \csname stex@module@#2\endcsname\@URI% TODO check this
      }%
    }{%
      \csxdef{module@imports@\module@uri}{%
        \csname stex@module@#2\endcsname\@URI,% TODO check this
        \csname module@imports@\module@uri\endcsname%
      }%
    }%
  }%
    \edef\activate@module@name{#2}%
    \StrCount\activate@module@name\@Slash[\activate@module@lastslash]%
    \ifnum\activate@module@lastslash>0%
    \StrCut[\activate@module@lastslash]\activate@module@name\@Slash\activate@module@temp\activate@module@name%
    \fi%
    \ifcsvoid{stex@lastmodule@\activate@module@name}{%
      \PackageError{stex}{No module with name \activate@module@name found}{}%
    }{%
      \ifx\@export\@@export%
        \stex@annotate@invisible{import}{\csname stex@lastmodule@\activate@module@name\endcsname}{}%
      \else\if@inimport\else%
        \stex@annotate@invisible{usemodule}{\csname stex@lastmodule@\activate@module@name\endcsname}{}%
      \fi\fi%
      \ex\ex\ex\activate@defs\ex\ex\ex{\csname stex@lastmodule@\activate@module@name\endcsname}%
    }%
  % activate the module
}%
%    \end{macrocode}
%
% \test{
% \immediate\openout\testfile=testmoduleimporta.tex
% \gdef\macrostring#1#2{
%   \edef\firststring{\detokenize{#1}}
%   \path@trimstring\firststring
%   \edef\firststring{\firststring#2}
%   \immediate\write\testfile{\firststring}
% }
% \macrostring{\begin}{{module}[name=testmoduleimporta]}
% \immediate\write\testfile{  foo bar boo far}
% \macrostring{\symdecl}{{foo}}
% \macrostring{\end}{{module}}
% \immediate\closeout\testfile
%  \printltx{\importmodule{testmoduleimporta}}:
%    \importmodule{testmoduleimporta}\\
% \present\stex@module@testmoduleimporta\\
% \present\foo
% }
% \test{
% \immediate\openout\testfile=testmoduleimportb.tex
% \macrostring{\begin}{{module}[name=importb]}
% \immediate\write\testfile{  foo bar boo far}
% \macrostring{\symdecl}{{bar}}
% \macrostring{\end}{{module}}
% \immediate\closeout\testfile
%    \printltx{\importmodule{testmoduleimportb?importb}}:
%    \importmodule{testmoduleimportb?importb}\\
% \present\stex@module@importb\\
% \present\bar
% }
% 
%
% \end{macro}
%
% Default document module:
%
%    \begin{macrocode}
\AtBeginDocument{%
  \set@default@ns%
  \ifx\module@narr\@empty\setkeys{module}{narr=\module@ns}\fi%
  \let\module@name\jobname%
  \let\module@id\module@name % TODO deprecate
  \edef\module@uri{\module@ns\@QuestionMark\module@name}%
  \csgdef{module@names@\module@uri}{}%
  \csgdef{module@imports@\module@uri}{}%
  \csxdef{\module@uri}{\noexpand\@invoke@module{\module@uri}}%
  \ex\global\ex\let\csname stex@module@\module@name\ex\endcsname\csname\module@uri\endcsname%
  \edef\this@module{%
    \ex\noexpand\csname module@defs@\module@uri\endcsname%
  }%
  \csdef{module@defs@\module@uri}{}%
  \ifcsvoid{mh@currentrepos}{}{%
    \@inmhrepostrue%
    \addto@thismodulex{\ex\edef\ex\noexpand\csname mh@old@repos@\module@uri\endcsname%
      {\noexpand\mh@currentrepos}}%
    \addto@thismodulex{\noexpand\mathhub@setcurrentreposinfo{\mh@currentrepos}}%
  }%
}
%    \end{macrocode}
%
% \test{\module@uri}
%
% \begin{macro}{\activate@defs}
%   To activate the |\symdef|s from a given module \meta{mod}, we call the macro
%   |\module@defs@|\meta{mod}. But to make sure that every module is activated only once,
%   we only activate if the macro |\module@defs@|\meta{mod} is undefined, and define it
%   directly afterwards to prohibit further activations.
%    \begin{macrocode}
\newif\if@inimport\@inimportfalse
\def\activate@defs#1{%
  \if@inimport\else%
    \def\inimport@module{#1}%
%    \stex@debug{Entering import #1}%
    \@inimporttrue%
  \fi%
  \edef\activate@defs@uri{#1}%
  \ifcsundef{module@defs@\activate@defs@uri}{%
    \PackageError{stex}{No module with URI \activate@defs@uri loaded}{Probably missing an
      \detokenize{\importmodule} (or variant) somewhere?
    }
  }{%
    \ifcsundef{module@\activate@defs@uri @activated}%
      {%
        \csdef{module@\activate@defs@uri @activated}{true}%
        \if@smsmode\else\stex@debug{Activating import \activate@defs@uri}\fi%
        %\stex@debug{\ex\meaning\csname module@defs@\activate@defs@uri\endcsname}%
        \csname module@defs@\activate@defs@uri\endcsname%
      }{%
        \stex@debug{Already activated: \activate@defs@uri}%
      }%
  }%
  \def\inimport@thismodule{#1}%
%  \stex@debug{End of import #1}%
  \ifx\inimport@thismodule\inimport@module\@inimportfalse%
%    \stex@debug{Leaving import #1}%
  \fi%
}%
%    \end{macrocode}
% \end{macro}
% 
% \begin{macro}{\usemodule} 
%   |\usemodule| acts like |\importmodule|, except that it does not re-export the semantic
%   macros in the modules it loads. 
%    \begin{macrocode}
\newcommand\usemodule[2][]{%
  \@@importmodule[#1]{#2}{noexport}%
}
%    \end{macrocode}
%
% \test{
%  \begin{module}[name=Foo]
%    \symdecl{foo}
%  \end{module}
%  \begin{module}[name=Bar]
%    \usemodule{Foo}
%    \present\foo
%    \symdecl{bar}
%  \end{module}
%  \begin{module}[name=Baz]
%    \importmodule{Bar}
%    Should be undefined: \present\foo\\
%    Should be defined:   \present\bar
%  \end{module}
%}
%
% \end{macro}
%
%
% \begin{macro}{\inputref@*skip}
% hooks for spacing customization, they are empty by default. 
%    \begin{macrocode}
\def\inputref@preskip{}
\def\inputref@postskip{}
%    \end{macrocode}
% \end{macro}
%
% \begin{macro}{\inputref}
%   |\inputref{|\meta{path to the current file without extension}|}| supports both
%   absolute path and relative path, meanwhile, records the path and the extension (not
%   for relative path).
%    \begin{macrocode}
\newrobustcmd\inputref[2][]{%
  \importmodule@bookkeeping{#1}{#2}{%
    %\inputreftrue
    \inputref@preskip%
    \stexinput{\importmodule@dir\@Slash\importmodule@modulename.tex}%
    \inputref@postskip%
  }%
}%
%    \end{macrocode}
%
%
% \end{macro}
%
% \subsection{Symbols/Notations/Verbalizations}
%
% \begin{macro}{\if@symdeflocal}
% A flag whether a symbol declaration is local (i.e. does not get exported)
% or not.
%    \begin{macrocode}
\newif\if@symdeflocal\@symdeflocalfalse
%    \end{macrocode}
% \end{macro}
%
% \begin{macro}{\define@in@module}
% calls |\edef\#1{#2}| and adds the macro definition to |\this@module|
%    \begin{macrocode}
\def\define@in@module#1#2{
  \expandafter\edef\csname #1\endcsname{#2}%
  \edef\define@in@module@temp{%
    \def\expandafter\noexpand\csname#1\endcsname%
    {#2}%
  }%
  \if@symdeflocal\else%
    \expandafter\g@addto@macro@safe\csname module@defs@\module@uri%
    \expandafter\endcsname\expandafter{\define@in@module@temp}%
  \fi%
}
%    \end{macrocode}
% \end{macro}
%
%\begin{macro}{\symdecl}
% |\symdecl[name=foo]{bar}| Declares a new symbol in the current module with URI 
% \meta{module-uri}|?foo| and defines new macros
% |\|\meta{uri} and |\bar|. If no optional name is given, |bar| is used as a name.
%
%    \begin{macrocode}
\define@key{symdecl}{name}{\def\symdecl@name{#1}}%
\define@key{symdecl}{type}{\def\symdecl@type{#1}}%
\define@key{symdecl}{args}{\def\symdecl@args{#1}}%
\define@key{symdecl}{local}[true]{\def\symdecl@local{#1}}%

\addmetakey[false]{symdecl}{local}[true]%

\newcommand\symdecl[2][]{%
  \def\symdecl@local{false}%
  \def\symdecl@name{}%
  \def\symdecl@type{}%
  \def\symdecl@args{}%
  \ifcsdef{this@module}{%
    \setkeys{symdecl}{#1}%
    \ifcsvoid{symdecl@name}{
      \edef\symdecl@name{#2}%
    }{}%
    \edef\symdecl@uri{\module@uri\@QuestionMark\symdecl@name}%
    \stex@debug{Symdecl \symdecl@uri^^Jtype: \meaning\symdecl@type}%
    \ifcsvoid{stex@symbol@\symdecl@name}{%
      \expandafter\edef\csname stex@symbol@\symdecl@name\endcsname{\symdecl@uri}%
    }{%
      \expandafter\def\csname stex@symbol@\symdecl@name\endcsname{\detokenize{ambiguous}}%
    }%
    \edef\symdecl@symbolmacro{%
      \noexpand\ifcsvoid{stex@symbol@\symdecl@name}{%
        \expandafter\edef\expandafter\noexpand\csname stex@symbol@\symdecl@name\endcsname{\symdecl@uri}%
      }{%
        \expandafter\def\expandafter\noexpand\csname stex@symbol@\symdecl@name\endcsname{\detokenize{ambiguous}}%
      }%
    }%
    \if@inimport\else\if@smsmode\else\ifcsvoid{symdecl@type}{}{%
      %\stex@debug{HERE1: \symdecl@uri}%
      \setbox\modules@import@tempbox\hbox{$\symdecl@type$}% only to have latex check this
    }\fi\fi%
    \ifcsvoid{symdecl@args}{\csgdef{\symdecl@uri\@QuestionMark args}{}}{%
      \IfInteger\symdecl@args{\notation@num@to@ia@\symdecl@args\csxdef{\symdecl@uri\@QuestionMark args}{\notation@curr@args}}{%
        \ex\global\ex\let\csname\symdecl@uri\@QuestionMark args\endcsname\symdecl@args%
      }%
    }%
    \expandafter\g@addto@macro@safe\csname module@defs@\module@uri%
    \expandafter\endcsname\expandafter{\symdecl@symbolmacro}%
    \ifcsvoid{\symdecl@uri}{%
      \ifcsvoid{module@names@\module@uri}{%
        \csxdef{module@names@\module@uri}{\symdecl@name}%
      }{%
        \csxdef{module@names@\module@uri}{\symdecl@name,%
          \csname module@names@\module@uri\endcsname}%
      }%
    }{%
    % not compatible with circular dependencies, e.g. test/omdoc/07-modules/smstesta.tex
      \PackageWarning{stex}{symbol already defined: \symdecl@uri}{}%
    }%
    \define@in@module\symdecl@uri{\noexpand\@invoke@symbol{\symdecl@uri}}%
    \IfStrEq\symdecl@local{false}{%
      \define@in@module{#2}{\noexpand\@invoke@symbol{\symdecl@uri}}%
    }{%
      \csdef{#2}{\noexpand\@invoke@symbol{\symdecl@uri}}%
    }%
  }{%
    \PackageError{stex}{\detokenize{\symdecl} not in a module}{You need to be in a module%
    in order to declare a new symbol}
  }%
  \ex\gdef\csname\symdecl@uri\@QuestionMark notations\endcsname{}%
  \if@inimport\else\if@inabbrdef\else\if@smsmode\else%
    %\stex@debug{HERE2: \symdecl@uri}%
    \html@symdecl\symdecl@uri{$\symdecl@type$}{\csname\symdecl@uri\@QuestionMark args\endcsname}{}{#2}%
  \fi\fi\fi%
  \if@insymdef@\else\parsemodule@maybesetcodes\fi%
}
\def\html@symdecl#1#2#3#4#5{\stex@annotate@invisible{symdecl}{#1}{%
  \stex@annotate@invisible{type}{}{#2}%
  \stex@annotate@invisible{args}{#3}{}%
  \stex@annotate@invisible{definiens}{}{#4}%
  \stex@annotate@invisible{macroname}{#5}{}%
}}

\newcommand\mmt@symdecl[4][]{
  \def\symdecl@local{false}%
  \def\symdecl@args{}%
  \setkeys{symdecl}{#1}%
  \edef\mmt@symdecl@name{#3}%
  \edef\mmt@symdecl@uri{#2\@QuestionMark\mmt@symdecl@name}%
  \stex@debug{MMT-Symdecl \mmt@symdecl@uri^^fof arity \symdecl@args}%
  \ifcsvoid{stex@symbol@\mmt@symdecl@name}{%
    \ex\edef\csname stex@symbol@\mmt@symdecl@name\endcsname{\mmt@symdecl@uri}%
  }{%
    \ex\def\csname stex@symbol@\mmt@symdecl@name\endcsname{\detokenize{ambiguous}}%
  }%
  \edef\symdecl@symbolmacro{%
    \noexpand\ifcsvoid{stex@symbol@\mmt@symdecl@name}{%
      \ex\edef\ex\noexpand\csname stex@symbol@\mmt@symdecl@name\endcsname{\mmt@symdecl@uri}%
    }{%
      \ex\def\ex\noexpand\csname stex@symbol@\mmt@symdecl@name\endcsname{\detokenize{ambiguous}}%
    }%
  }%
  \ifcsvoid{symdecl@args}{\csgdef{\mmt@symdecl@uri\@QuestionMark args}{}}{%
    \IfInteger\symdecl@args{\notation@num@to@ia@\symdecl@args\csxdef{\mmt@symdecl@uri\@QuestionMark args}{\notation@curr@args}}{%
      \ex\global\ex\let\csname\mmt@symdecl@uri\@QuestionMark args\endcsname\symdecl@args%
    }%
  }%
  \ex\g@addto@macro@safe\csname module@defs@\module@uri%
    \ex\endcsname\ex{\symdecl@symbolmacro}%
  \ifcsvoid{\mmt@symdecl@uri}{%
    \ifcsvoid{module@names@\module@uri}{%
      \csxdef{module@names@\module@uri}{\mmt@symdecl@name}%
    }{%
      \csxdef{module@names@\module@uri}{\mmt@symdecl@name,%
        \csname module@names@\module@uri\endcsname}%
    }%
  }{%
  % not compatible with circular dependencies, e.g. test/omdoc/07-modules/smstesta.tex
    \PackageWarning{stex}{symbol already defined: \mmt@symdecl@uri}{}%
  }%
  \define@in@module\mmt@symdecl@uri{\noexpand\@invoke@symbol{\mmt@symdecl@uri}}%
  \IfStrEq\symdecl@local{false}{%
    \define@in@module{#4}{\noexpand\@invoke@symbol{\mmt@symdecl@uri}}%
  }{%
    \csdef{#4}{\noexpand\@invoke@symbol{\mmt@symdecl@uri}}%
  }%
  \ex\gdef\csname\mmt@symdecl@uri\@QuestionMark notations\endcsname{}%
  \parsemodule@maybesetcodes%
}
%    \end{macrocode}
% \test{
%  \begin{module}[name=foo]
%    \printltx{\symdecl{bar}} \symdecl{bar}\\
%    Yields: \present\bar
%  \end{module}
%}
% \end{macro}
%
% \subsubsection{Notations}
%
% \begin{macro}{\modules@getURIfromName}
% This macro searches for the full URI given a symbol name
% and stores it in |\notation@uri|. Used by e.g. |\notation[...]{foo}{...}| to
% figure out what symbol |foo| refers to:
%    \begin{macrocode} % TODO make this work with actual macros (it doesn't)
\edef\stex@ambiguous{\detokenize{ambiguous}}
\edef\stex@macrostring{\detokenize{macro:->\@invoke@symbol}}
\def\modules@getURIfromName#1{%
  \def\notation@uri{}%
  \edef\modules@getURI@name{#1}%
  \ifcsvoid{\modules@getURI@name}{%
    \edef\modules@temp@meaning{}%
  }{%
    \edef\modules@temp@meaning{\ex\meaning\csname\modules@getURI@name\endcsname}%
  }%
  \IfBeginWith\modules@temp@meaning\stex@macrostring{%
    % is a \@invoke@symbol macro
    \StrPosition\modules@temp@meaning\@close@brace[\stex@tempnum]%
    \StrMid\modules@temp@meaning{26}{\the\numexpr\stex@tempnum-1\@Space}[\notation@uri]%
  }{%
    % Check whether full URI or module?symbol or just name
    \StrCount\modules@getURI@name\@Colon[\isuri@number]%
    \ifnum\isuri@number>0%
      \edef\notation@uri{\modules@getURI@name}%
    \else
      \StrCount\modules@getURI@name\@QuestionMark[\isuri@number]%
      \ifnum\isuri@number=2%
        \edef\notation@uri{\modules@getURI@name}%
      \else%
        \ifnum\isuri@number=1%
          % module?name
          \StrCut\modules@getURI@name\@QuestionMark\isuri@mod\isuri@name%
          \ifcsvoid{stex@module@\isuri@mod}{%
            \PackageError{stex}{No module with name \isuri@mod\@Space loaded}{}%
          }{%
            \expandafter\ifx\csname stex@module@\isuri@mod\endcsname\stex@ambiguous%
              \PackageError{stex}{Module name \isuri@mod\@Space is ambiguous}{}%
            \else%
              \edef\notation@uri{\csname stex@module@\isuri@mod\endcsname\@URI\@QuestionMark\isuri@name}%
            \fi%
          }%
        \else%
          %name
          \ifcsvoid{stex@symbol@\modules@getURI@name}{%
            \PackageError{stex}{No symbol with name \modules@getURI@name\@Space known}{}%
          }{%
          \ifcsvoid{\module@uri\@QuestionMark\modules@getURI@name}{%
            \expandafter\ifx\csname stex@symbol@\modules@getURI@name\endcsname\stex@ambiguous%
              % Symbol name ambiguous and not in current module
              \PackageError{stex}{Symbol name, URI or macroname \detokenize{#1} found!}{}%
            \else%
              % Symbol not in current module, but unambiguous
              \edef\notation@uri{\csname stex@symbol@\modules@getURI@name\endcsname}%
            \fi%
            }{% Symbol in current module
              \edef\notation@uri{\module@uri\@QuestionMark\modules@getURI@name}%
            }%
          }%
        \fi%
      \fi%
    \fi%
  }%
}
%    \end{macrocode}
% \end{macro}
%
% \begin{macro}{\notation}
%
% Adds a new notation to a symbol |foo|, as in:
% |\notation[lang=en,arity=0,variant=op]{foo}{...}|
% |\notation[variant=bar]{foo}{...}|
% |\notation[prec=500;50x49x51]{foo}{#1 bla #2 bla #3}{arg1}{arg3}|
%
% the actual notation is ultimately stored in |\|\meta{uri}|#|\meta{variant},
% where \meta{variant} contains |arity|,|lang| and |variant| in that order.
%
%    \begin{macrocode}
\newif\if@innotation\@innotationfalse
%    \end{macrocode}
% The next method actually parses the optional arguments and stores them
% in helper macros. This method will also be used later in symbol
% invokations to construct the \meta{variant}:
%    \begin{macrocode}
\def\notation@parse@params#1#2{%
  \def\notation@curr@precs{}%
  \def\notation@curr@args{}%
  \def\notation@curr@variant{}%
  \def\notation@curr@arityvar{}%
  \def\notation@curr@provided@arity{#2}
  \def\notation@curr@lang{}%
  \def\notation@options@temp{#1}
  \notation@parse@params@%
  \ifx\notation@curr@args\@empty%
    \ifx\notation@curr@provided@arity\@empty%
      \notation@num@to@ia\notation@curr@arityvar%
    \else%
      \notation@num@to@ia\notation@curr@provided@arity%
    \fi%
  \fi%
  \StrLen\notation@curr@args[\notation@curr@arity]%
}
\def\notation@parse@params@{%
  \IfSubStr\notation@options@temp,{%
    \StrCut\notation@options@temp,\notation@option@temp\notation@options@temp%
    \notation@parse@param%
    \notation@parse@params@%
  }{\ifx\notation@options@temp\@empty\else%
    \let\notation@option@temp\notation@options@temp%
    \notation@parse@param%
  \fi}%
}

\def\notation@parse@param{%
  \path@trimstring\notation@option@temp%
  \ifx\notation@option@temp\@empty\else%
    \IfSubStr\notation@option@temp={%
      \StrCut\notation@option@temp=\notation@key\notation@value%
      \path@trimstring\notation@key%
      \path@trimstring\notation@value%
      \IfStrEq\notation@key{prec}{%
        \edef\notation@curr@precs{\notation@value}%
      }{%
      \IfStrEq\notation@key{args}{%
        \edef\notation@curr@args{\notation@value}%
      }{%
      \IfStrEq\notation@key{lang}{%
        \edef\notation@curr@lang{\notation@value}%
      }{%
      \IfStrEq\notation@key{variant}{%
        \edef\notation@curr@variant{\notation@value}%
      }{%
      \IfStrEq\notation@key{arity}{%
        \edef\notation@curr@arityvar{\notation@value}%
      }{%
      }}}}}%
    }{%
        \edef\notation@curr@variant{\notation@option@temp}%
    }%
  \fi%  
}

% converts an integer to a string of 'i's, e.g. 3 => iii,
% and stores the result in \notation@curr@args
\def\notation@num@to@ia#1{%
  \IfInteger{#1}{
    \notation@num@to@ia@#1%
  }{%
    %
  }%
}
\def\notation@num@to@ia@#1{%
  \ifnum#1>0%
    \edef\notation@curr@args{\notation@curr@args i}%
    \expandafter\notation@num@to@ia@\expandafter{\the\numexpr#1-1\@Space}%
  \fi%
}

\newcount\notation@argument@counter

% parses the notation arguments and wraps them in 
% \notation@assoc and \notation@argprec for flexary arguments and precedences
\providerobustcmd\notation[3][]{%
  \modules@getURIfromName{#2}%
  \notation@parse@params{#1}{}%
  \def\notation@temp@notation{}%
  \ex\let\ex\notation@curr@args\csname\notation@uri\@QuestionMark args\endcsname%
  \stex@debug{Notation for \notation@uri: \notation@curr@args}%
  \let\notation@curr@todo@args\notation@curr@args%
  \StrLen\notation@curr@todo@args[\notation@curr@arity]%
  \ex\renewcommand\ex\notation@temp@notation\ex[\notation@curr@arity]{#3}%
  % precedence
  \let\notation@curr@precstring\notation@curr@precs%
  \IfSubStr\notation@curr@precs;{%
    \StrCut\notation@curr@precs;\notation@curr@prec\notation@curr@precs%
    \ifx\notation@curr@prec\@empty\def\notation@curr@prec{0}\fi%
  }{%
    \ifx\notation@curr@precs\@empty%
      \ifnum\notation@curr@arity=0\relax%
        \edef\notation@curr@prec{\infprec}%
      \else%
        \def\notation@curr@prec{0}%
      \fi%
    \else%
      \edef\notation@curr@prec{\notation@curr@precs}%
      \def\notation@curr@precs{}%
    \fi%
  }%
  % arguments
  \notation@argument@counter=0%
  \def\notation@curr@extargs{}%
  \notation@do@args%
}

\edef\notation@ichar{\detokenize{i}}%
\edef\notation@achar{\detokenize{a}}%
\edef\notation@bchar{\detokenize{b}}%

% parses additional notation components for (associative) arguments
\def\notation@do@args{%
  \advance\notation@argument@counter by 1%
  \def\notation@nextarg@temp{}%
  \ifx\notation@curr@todo@args\@empty%
    \ex\notation@after%
  \else%
    % argument precedence
    \IfSubStr\notation@curr@precs{x}{%
      \StrCut\notation@curr@precs{x}\notation@curr@argprec\notation@curr@precs%
    }{%
      \edef\notation@curr@argprec{\notation@curr@precs}%
      \def\notation@curr@precs{}%
    }%
    \ifx\notation@curr@argprec\@empty%
      \let\notation@curr@argprec\notation@curr@prec%
    \fi%
    \StrChar\notation@curr@todo@args1[\notation@argchar]%
    \edef\notation@argchar{\ex\detokenize\ex{\notation@argchar}}%
    \StrGobbleLeft\notation@curr@todo@args1[\notation@curr@todo@args]%
    \ifx\notation@argchar\notation@ichar%
      % normal argument
      \edef\notation@nextarg@temp{%
        {\stex@arg{\the\notation@argument@counter}{\notation@curr@argprec}{########\the\notation@argument@counter}}%
      }%
      \ex\g@addto@macro@safe\ex\notation@curr@extargs%
        \ex{\notation@nextarg@temp}%
      \ex\ex\ex\notation@do@args%
    \else\ifx\notation@argchar\notation@bchar%
        % bound argument
        \edef\notation@nextarg@temp{%
          {\stex@arg{\the\notation@argument@counter}{\notation@curr@argprec}{########\the\notation@argument@counter}}%
        }%
        \ex\g@addto@macro@safe\ex\notation@curr@extargs%
          \ex{\notation@nextarg@temp}%
        \ex\ex\ex\ex\ex\ex\ex\notation@do@args%
      \else%
        % associative argument
        \ex\ex\ex\ex\ex\ex\ex\notation@parse@assocarg%
      \fi%
    \fi%
  \fi%
}

\def\notation@parse@assocarg#1{%
  \def\notation@parse@assocop{#1}%
  \edef\notation@nextarg@temp{%
    {\notation@assoc{\the\notation@argument@counter}{\notation@curr@argprec}{\ex\unexpanded\ex{\notation@parse@assocop}}%
      {########\the\notation@argument@counter}}%
  }%
  \ex\g@addto@macro@safe\ex\notation@curr@extargs\ex{\notation@nextarg@temp}%
  \notation@do@args%
}

\protected\def\safe@newcommand#1{%
  \ifdefined#1\ex\renewcommand\else\ex\newcommand\fi#1%
}

% finally creates the actual macros
\def\notation@after{
  % \notation@curr@precs
  % \notation@curr@args
  % \notation@curr@variant
  % \notation@curr@arity
  % \notation@curr@provided@arity
  % \notation@curr@lang
  % \notation@uri
  \def\notation@temp@fragment{}%
  \protected\def\notation@thisuri{}%
  \ifx\notation@curr@arityvar\@empty\else%
    \edef\notation@temp@fragment{arity=\notation@curr@arityvar}%
  \fi%
  \ifx\notation@curr@lang\@empty\else%
    \ifx\notation@temp@fragment\@empty%
      \edef\notation@temp@fragment{lang=\notation@curr@lang}%
    \else%
      \edef\notation@temp@fragment{\notation@temp@fragment\@Ampersand lang=\notation@curr@lang}%
    \fi%
  \fi%
  \ifx\notation@curr@variant\@empty\else%
    \ifx\notation@temp@fragment\@empty%
      \edef\notation@temp@fragment{variant=\notation@curr@variant}%
    \else%
      \edef\notation@temp@fragment{\notation@temp@fragment\@Ampersand variant=\notation@curr@variant}%
    \fi%
  \fi%
  \notation@do@unused%
  \ex\ex\ex\def\ex\ex\ex\notation@temp@notation\ex\ex\ex%
    {\ex\notation@temp@notation\notation@curr@extargs}%
  \ifnum\notation@curr@arity=0%
    \edef\notation@temp@notation{\stex@dooms{\notation@thisuri}{\notation@temp@fragment}{\notation@curr@prec}{\ex\unexpanded\ex{\notation@temp@notation}}}%
  \else%
    \IfSubStr\notation@curr@args\notation@bchar{%
      \edef\notation@temp@notation{\stex@doomb{\notation@thisuri}{\notation@temp@fragment}{\notation@curr@prec}{\ex\unexpanded\ex{\notation@temp@notation}}}%
    }{%
      \edef\notation@temp@notation{\stex@dooma{\notation@thisuri}{\notation@temp@fragment}{\notation@curr@prec}{\ex\unexpanded\ex{\notation@temp@notation}}}%
    }%
  \fi%
  \stex@debug{Notation \notation@uri: \meaning\notation@temp@notation}%
  \notation@final%
  \parsemodule@maybesetcodes%
}

\def\notation@do@unused{%
  \edef\notation@curr@notstring{\ex\detokenize\ex{\meaning\notation@temp@notation}}%
  \StrBehind\notation@curr@notstring{->}[\notation@curr@notstring]%
  %\stex@debug{HERE: Arity: \notation@curr@arity^^J    Notation: \notation@curr@notstring}%
  \notation@argument@counter=0%
  \notation@unused@loop%
}

\def\notation@unused@loop{%
  \advance\notation@argument@counter by 1\relax%
  \ifnum\notation@argument@counter>\notation@curr@arity\else%
    \IfSubStr\notation@curr@notstring{\@Fragment\the\notation@argument@counter}{%
      %\stex@debug{Argument \the\notation@argument@counter is used}%
    }{%
      \stex@debug{Argument \the\notation@argument@counter: NOT used}%
      \notation@unused@escapeargs%
      \ex\edef\ex\notation@temp@notation\notation@unused@sig{\ex\notation@temp@notation\notation@unused@args\stex@arg@unused{##\the\notation@argument@counter}}%
      \stex@debug{Updated: \meaning\notation@temp@notation}%
    }%
    \ex\notation@unused@loop%
  \fi%
}

\newcount\notation@unused@counter
\def\notation@unused@escapeargs{%
  \notation@unused@counter=0\relax%
  \def\notation@unused@args{}%
  \def\notation@unused@sig{}%
  \def\notation@unused@arg{\noexpand\notation@unused@arg}%
  \def\notation@unused@sigarg{\noexpand\notation@unused@sigarg}%
  \notation@unused@escapeargs@loop%
  \def\notation@unused@arg##1{##########1}%
  \def\notation@unused@sigarg##1{##########1}%
  \edef\notation@unused@sig{\notation@unused@sig}%
  %\stex@debug{Sig: \meaning\notation@unused@sig}%
  \ex\edef\ex\notation@temp@notation\notation@unused@sig{\ex\notation@temp@notation\notation@unused@args}%
  %\stex@debug{New meaning: \meaning\notation@temp@notation}%
}
\def\notation@unused@escapeargs@loop{%
  \advance\notation@unused@counter by 1\relax%
  \ifnum\notation@unused@counter>\notation@curr@arity\else%
    %\stex@debug{HERE: \the\notation@unused@counter}%
    \edef\notation@unused@args{\notation@unused@args{\notation@unused@arg{\the\notation@unused@counter}}}%
    \edef\notation@unused@sig{\notation@unused@sig\notation@unused@sigarg{\the\notation@unused@counter}}%
    %\stex@debug{New: \meaning\notation@unused@sig}%
    \ex\notation@unused@escapeargs@loop%
  \fi%
}

\def\notation@final{%
  \edef\notation@csname{\notation@uri\@Fragment\notation@temp@fragment}%
  \ex\xdef\csname\notation@uri\@QuestionMark notations\endcsname{\csname\notation@uri\@QuestionMark notations\endcsname,\notation@temp@fragment}%
  %\stex@debug{Defining \notation@csname of arity \notation@curr@arity}%
  \ifcsvoid{\notation@csname}{}{%
    \PackageWarning{stex}{notation already defined: \notation@csname, %
      Choose a different set of notation options (variant,lang,arity)%
    }%
    \ex\let\csname\notation@csname\endcsname\undefined%
  }%
  \ex\ex\ex\ex\ex\ex\ex\newcommand\ex\ex\ex\csname\ex\ex\ex\notation@csname%
    \ex\ex\ex\endcsname\ex\ex\ex[\ex\notation@curr@arity\ex]%
    \ex{\notation@temp@notation}%
  \edef\symdecl@temps{%
    \noexpand\safe@newcommand\ex\noexpand\csname\notation@csname\endcsname[\notation@curr@arity]%
  }%
  \ex\g@addto@macro@safe\csname module@defs@\module@uri\ex\endcsname\ex{\symdecl@temps}%
  \ex\g@addto@macro@safe\csname module@defs@\module@uri\ex\endcsname\ex{\ex{\notation@temp@notation}}%
  \ifcsvoid{\notation@uri\@Fragment}{%
    \ex\let\csname\notation@uri\@Fragment\ex\endcsname\csname\notation@csname\endcsname%
    \ex\g@addto@macro@safe\csname module@defs@\module@uri\ex\endcsname\ex{%
      \ex\let\csname\notation@uri\@Fragment\ex\endcsname\csname\notation@csname\endcsname%
    }%
  }{}%
  \@innotationfalse%
  \if@inimport\else\if@htmlmode%
    \let\notation@simarg@args\notation@curr@args%
    \notation@argument@counter=0%
    \def\notation@simargs{}%
    \notation@simulate@arguments%
    \html@notation\notation@uri\notation@temp@fragment\notation@curr@args\notation@curr@precstring%
      {$\displaystyle\csname\notation@csname\ex\endcsname\notation@simargs$}%
  \fi\fi%
}
\def\notation@simulate@arguments{%
  \ifx\notation@simarg@args\@empty\else%
    \advance\notation@argument@counter by 1%
    \IfBeginWith\notation@simarg@args{i}{%
      \protected@edef\notation@simargs{\notation@simargs{\notation@symprec{\infprec}{\noexpand\textrm{\@Fragment\the\notation@argument@counter}}}}%
    }{%
      \IfBeginWith\notation@simarg@args{b}{%
        \protected@edef\notation@simargs{\notation@simargs{\notation@symprec{\infprec}{\noexpand\textrm{\@Fragment\the\notation@argument@counter}}}}%
      }{%
        \protected@edef\notation@simargs{\notation@simargs{\noexpand\textrm{\@Fragment\@Fragment\the\notation@argument@counter a},\noexpand\textrm{\@Fragment\@Fragment\the\notation@argument@counter b}}}%
      }%
    }%
    \StrGobbleLeft\notation@simarg@args1[\notation@simarg@args]%
    \notation@simulate@arguments%
  \fi%
}
% URI, fragment, arity, notation
\def\html@notation#1#2#3#4#5{%
  \begingroup%
  \def\notation@thisuri{#1}%
  \stex@annotate@invisible{notation}{#1}{%
    \stex@annotate@invisible{notationfragment}{#2}{}%
    \stex@annotate@invisible{args}{#3}{}%
    \stex@annotate@invisible{precedence}{#4}{}%
    \stex@annotate@invisible{notationcomp}{}{#5}%
  }\endgroup%
}

\protected\def\instantiates#1#2{%
  \modules@getURIfromName{#2}%
  \let\instantiates@domuri\notation@uri%
  \modules@getURIfromName{#1}%
  \let\instantiates@coduri\notation@uri%
  \stex@debug{Instantiation: \instantiates@coduri = \instantiates@domuri }%
  \stex@debug{inherit notations: \csname\instantiates@domuri\@QuestionMark notations\endcsname}%
  \ex\let\ex\instantiates@notations\csname\instantiates@domuri\@QuestionMark notations\endcsname%
  \ex\xdef\csname\instantiates@coduri\@QuestionMark notations\endcsname{\csname\instantiates@coduri\@QuestionMark notations\endcsname,\instantiates@notations}%
  \@for\@I:=\instantiates@notations%
  \do{%
    \stex@debug{Inherit \@I^^J\ex\meaning\csname\instantiates@domuri\@Fragment\@I\endcsname}%
    \ex\let\csname\instantiates@coduri\@Fragment\@I\ex\endcsname%
      \csname\instantiates@domuri\@Fragment\@I\endcsname%
  }%
  \parsemodule@maybesetcodes%
  \addto@thismodule{\instantiates{#1}{#2}}%
}

\protected\def\setnotation#1#2{%
  \modules@getURIfromName{#1}%
  \notation@construct@fragment{#2}%
  \stex@debug{Defining default notation for \notation@uri
    as \notation@temp@fragment: \ex\meaning\csname\notation@uri\@Fragment\notation@temp@fragment\endcsname}%
  \ex\let\csname\notation@uri\@Fragment\ex\endcsname%
    \csname\notation@uri\@Fragment\notation@temp@fragment\endcsname%
  \parsemodule@maybesetcodes%
  \addto@thismodule{\setnotation{#1}{#2}}%
}
%    \end{macrocode}
% \end{macro}
%
% 
% The following macros take care of precedences, parentheses/bracketing,
% associative (flexary) arguments etc. in presentation:
%    \begin{macrocode}
\protected\def\notation@assoc#1#2#3#4{% function, argv
  \let\@tmpop=\relax% do not print the function the first time round
  \@for\@I:=#4\do{\@tmpop% print the function
    % write the i-th argument
    \stex@arg{#1}{#2}{\@I}%
    \def\@tmpop{#3}%
  }%
}%

\def\notation@lparen{(}
\def\notation@rparen{)}
\def\infprec{1000000}
\def\neginfprec{-\infprec}

\newcount\notation@downprec
\notation@downprec=\neginfprec

% patching displaymode
\newif\if@displaymode\@displaymodefalse
\ex\everydisplay\ex{\the\everydisplay\@displaymodetrue}
%\let\old@displaystyle\displaystyle
%\protected\def\displaystyle{\old@displaystyle\@displaymodetrue}

\protected\def\dobrackets#1{% avoiding groups at all costs to ensure \parray still works!
  \def\notation@innertmp{#1}%
  \if@displaymode%
    \ex\ex\ex\left\ex\ex\ex\notation@lparen%
    \ex\notation@resetbrackets\ex\notation@innertmp%
    \ex\right\notation@rparen%
  \else%
    \ex\ex\ex\notation@lparen%
    \ex\notation@resetbrackets\ex\notation@innertmp%
    \notation@rparen%
  \fi%
}

\protected\def\withbrackets#1#2#3{%
  \edef\notation@lparen{#1}%
  \edef\notation@rparen{#2}%
  #3%
  \notation@resetbrackets%
}

\protected\def\notation@resetbrackets{%
  \def\notation@lparen{(}%
  \def\notation@rparen{)}%
}

\protected\def\stex@dooms#1#2#3#4{%
  \if@innotation%
    \notation@symprec{#3}{#4}%
  \else%
   \@innotationtrue%
    \html@oms{#1}{#2}{\notation@symprec{#3}{#4}}%
    \@innotationfalse%
  \fi%
}

\protected\def\stex@doomb#1#2#3#4{%
  \if@innotation%
    \notation@symprec{#3}{#4}%
  \else%
    \@innotationtrue%
    \html@ombind{#1}{#2}{\notation@symprec{#3}{#4}}%
    \@innotationfalse%
  \fi%
}

\protected\def\stex@dooma#1#2#3#4{%
  \if@innotation%
    \notation@symprec{#3}{#4}%
  \else%
    \@innotationtrue%
    \html@oma{#1}{#2}{\notation@symprec{#3}{#4}}%
    \@innotationfalse%
  \fi%
}

% for LaTeXML Bindings
\protected\def\html@oms#1#2#3{%
  \stex@annotate{OMID}{#1\@Fragment#2}{#3}%
}

\protected\def\html@oma#1#2#3{%
  \edef\html@oma@uri{%
    \ifcsname#1\@QuestionMark args\endcsname%
      #1\@Fragment\csname#1\@QuestionMark args\endcsname\@Fragment#2%
    \else#1\@Fragment\@Fragment#2\fi%
    }%
  \stex@annotate{OMA}{\html@oma@uri}{#3}%
}

\protected\def\html@ombind#1#2#3{%
  \edef\html@oma@uri{%
    \ifcsname#1\@QuestionMark args\endcsname%
      #1\@Fragment\csname#1\@QuestionMark args\endcsname\@Fragment#2%
    \else#1\@Fragment\@Fragment#2\fi%
  }%
  \stex@annotate{OMBIND}{\html@oma@uri}{#3}%
}

\protected\def\notation@symprec#1#2{%
  %\stex@debug{Precedence: #1 > \number\notation@downprec}%
  \ifnum#1>\notation@downprec\relax%
    \notation@resetbrackets#2%
  \else%
    \ifnum\notation@downprec=\neginfprec\relax%
      \notation@resetbrackets#2%
    \else
      \if@inparray@
        \notation@resetbrackets#2
      \else\dobrackets{#2}%
      \fi%
    \fi%
  \fi%
}

\newif\if@inparray@\@inparray@false


\protected\def\stex@arg#1#2#3{%
  \@innotationfalse%
  \html@arg{#1}{\notation@argprec{#2}{#3}}%
  \@innotationtrue%
}

\protected\def\stex@arg@unused#1{%
    \stex@invisible{#1}%
}

% for LaTeXML Bindings
\def\html@arg#1#2{%
  \stex@annotate{arg}{#1}{#2}%
}

\def\notation@argprec#1#2{%
  \def\notation@innertmp{#2}
  \edef\notation@downprec@temp{\number#1}%
  \notation@downprec=\ex\notation@downprec@temp%
  \ex\relax\ex\notation@innertmp%
  \ex\notation@downprec\ex=\number\notation@downprec\relax%
}
%    \end{macrocode}
% Macros for introducing MMT/OMDoc primitives manually
%    \begin{macrocode}
\protected\def\stex@oms#1#2{\modules@getURIfromName{#1}\html@oms{\notation@uri}{}{#2}}
\protected\def\stex@oma#1#2{\modules@getURIfromName{#1}\html@oma{\notation@uri}{}{#2}}
\protected\def\stex@ombind#1#2{\modules@getURIfromName{#1}\html@ombind{\notation@uri}{}{#2}}
\protected\def\stex@rule#1#2{%
  \stex@annotate@invisible{mmtrule}{#1}{%
    \notation@argument@counter=0%
    \@for\@I:=#2\do{%
      \advance\notation@argument@counter by 1%
      \stex@annotate{arg}{\the\notation@argument@counter}{\ensuremath{\@I}}%
    }%
  }%
}

\newenvironment{symboldoc}[1]{%
  \modules@getURIfromName{#1}%
  \begin{stex@annotateenv}{symboldoc}{\notation@uri}%
    \def\symboldoc@lang{en}%
    \ifcsvoid{module@lang}{%
      \ifcsvoid{main@language}{}{%
        \ifcsvoid{stex@babel@short\languagename}{}{%
          \edef\symboldoc@lang{\csname stex@babel@short\languagename\endcsname}%
        }%
      }%
    }{%
      \let\symboldoc@lang\module@lang%
    }%
    \stex@annotate{language}\symboldoc@lang{}%
  }{\end{stex@annotateenv}}

%    \end{macrocode}
%
% \begin{macro}{\@invoke@symbol}
% after |\symdecl{foo}|, |\foo| expands to |\@invoke@symbol{<uri>}|:
%    \begin{macrocode}
\protected\def\@invoke@symbol#1{%
  \def\notation@thisuri{#1}%
  \ifmmode%
    \def\@invoke@symbol@first{#1}%
    \let\invoke@symbol@next\invoke@symbol@math%
  \else%
    \def\invoke@symbol@next{\invoke@symbol@text{#1}}%
  \fi%
  \invoke@symbol@next%
}
%    \end{macrocode}
%
% takes care of the optional notation-option-argument, and either invokes |\@invoke@symbol@math| for
% symbolic presentation or |\@invoke@symbol@text| for verbalization (TODO)
%    \begin{macrocode}
\def\notation@construct@fragment#1{%
  \notation@parse@params{#1}{}%
  \def\notation@temp@fragment{}%
  \ifx\notation@curr@arityvar\@empty\else%
    \edef\notation@temp@fragment{arity=\notation@curr@arity}%
  \fi%
  \ifx\notation@curr@lang\@empty\else%
    \ifx\notation@temp@fragment\@empty%
      \edef\notation@temp@fragment{lang=\notation@curr@lang}%
    \else%
      \edef\notation@temp@fragment{\notation@temp@fragment\@Ampersand lang=\notation@curr@lang}%
    \fi%
  \fi%
  \ifx\notation@curr@variant\@empty\else%
    \ifx\notation@temp@fragment\@empty%
      \edef\notation@temp@fragment{variant=\notation@curr@variant}%
    \else%
      \edef\notation@temp@fragment{\notation@temp@fragment\@Ampersand variant=\notation@curr@variant}%
    \fi%
  \fi%
}

\newcommand\invoke@symbol@math[1][]{%
  \notation@construct@fragment{#1}%
  \csname\@invoke@symbol@first\@Fragment\notation@temp@fragment\endcsname%
}
%    \end{macrocode}
%
% \end{macro}
%
% TODO: To set notational options (globally or locally) generically:
%
%    \begin{macrocode}
\def\setstexlang#1{%
  \def\stex@lang{#1}%
}%
\setstexlang{en}
\def\setstexvariant#1#2{%
  % TODO
}
\def\setstexvariants#1{%
  \def\stex@variants{#1}%
}
%    \end{macrocode}
% \test{
%\def\mathhub@archive@ns{http\@Colon\@Slash\@Slash mathhub.info\@Slash test}
%\def\mathhub@archive@id{Others\@Slash stextests}
%\def\mathhub@archive@dir{\@Slash home\@Slash jazzpirate\@Slash work\@Slash LaTeX\@Slash Others\@Slash stextests}
%\@setcurrentreposinfo
%\begin{module}[name=FooBar]
%    \printltx{\symdecl[args=a]{plus}}\symdecl[args=a]{plus}\\
%    \printltx{\symdecl[args=a]{times}}\symdecl[args=a]{times}\\
%    \printltx{\symdecl{vara}}\symdecl{vara}\\
%    \printltx{\symdecl{varb}}\symdecl{varb}\\
%    \printltx{\symdecl{varc}}\symdecl{varc}\\
%    \printltx{\symdecl{vard}}\symdecl{vard}\\
%    \printltx{\symdecl{vare}}\symdecl{vare}\\
%    \printltx{\notation{vara}{a}}\notation{vara}{a}\\
%    \printltx{\notation{varb}{b}}\notation{varb}{b}\\
%    \printltx{\notation{varc}{c}}\notation{varc}{c}\\
%    \printltx{\notation{vard}{d}}\notation{vard}{d}\\
%    \printltx{\notation{vare}{e}}\notation{vare}{e}\\
%    \printltx{\notation[prec=500;500]{plus}{\withbrackets\langle\rangle{##1}}{+}}
%    \notation[prec=500;500]{plus}{\withbrackets\langle\rangle{#1}}{+}\\
%    \printltx{\notation[prec=600;600]{times}{##1}{\cdot}}
%    \notation[prec=600;600]{times}{#1}{\cdot}\\\ \\
%    \printltx{
%      $\times{\frac\vara\varb,\plus{\frac\vara{\frac\vara\varb},\times{\varc,\plus{\vard,\vare,2}}}}$}:\\
%    $\times{\frac\vara\varb,\plus{\frac\vara{\frac\vara\varb},\times{\varc,\plus{\vard,\vare,2}}}}$\\\ \\
%    \printltx{
%      \[\times{\frac\vara\varb,\plus{\frac\vara{\frac\vara\varb},\times{\varc,\plus{\vard,\vare,2}}}}\]}:\\
%    \[\times{\frac\vara\varb,\plus{\frac\vara{\frac\vara\varb},\times{\varc,\plus{\vard,\vare,2}}}}\]
%  \end{module}
%}
%
% \begin{macro}{\abbrdef}
%   The |\abbrdef| macro is a variant of |\symdecl| that does the same on the {\LaTeX}
%   level, and adds a definiens on the OMDoc level.
%    \begin{macrocode}
\newif\if@inabbrdef\@inabbrdeffalse
\def\abbrdef@definiens{}
\newcommand\abbrdef[3][]{%
  \@inabbrdeftrue\symdecl[#1]{#2}%
  \@inabbrdeffalse%
  \ex\let\ex\abbrdef@args\csname\symdecl@uri\@QuestionMark args\endcsname%
  \StrLen\abbrdef@args[\abbrdef@arity]
  \ex\renewcommand\ex\abbrdef@definiens\ex[\abbrdef@arity]{\unexpanded{#3}}%
  \if@inimport\else\if@smsmode\else\if@htmlmode%
    \let\notation@simarg@args\abbrdef@args%
    \notation@argument@counter=0%
    \def\notation@simargs{}%
    \notation@simulate@arguments%
    \html@symdecl\symdecl@uri{$\symdecl@type$}{\csname\symdecl@uri\@QuestionMark args\endcsname}%
      {\[\ex\abbrdef@definiens\notation@simargs\]}{#2}%
  \fi\fi\fi%
}
%    \end{macrocode}
% \test{
%   \printltx{\symdecl{foo}}\\
%   \symdecl{foo}
%   \printltx{\notation{foo}{\psi}}\\
%   \notation{foo}{\psi}\\
%   \printltx{$\foo$}\\
%   \printltx{\abbrdef{lftype}{\stex@oms{http://cds.omdoc.org/urtheories?Typed?type}{}}}\\
%   \abbrdef{lftype}{\stex@oms{http://cds.omdoc.org/urtheories?Typed?type}{}}
%   \printltx{\notation{lftype}{\noexpand\mathtt{type}}}\\
%   \notation{lftype}{\noexpand\mathtt{type}}
%   \printltx{$\lftype$}\\
%   $\lftype$
%}
% \end{macro}
%
%
%
% \subsection{Verbalizations}
%
%
%    \begin{macrocode}
\newif\if@inoms
\def\invoke@symbol@text#1{%
  \edef\invoke@symbol@uri{#1}%
  \def\invoke@symbol@return{}%
  \notation@argument@counter=0%
  \edef\invoke@symbol@arity{\csname #1\@QuestionMark args\endcsname}%
  \ifx\invoke@symbol@arity\@empty\@inomstrue\else\@inomsfalse\fi%
  \invoke@symbol@text@args%
}

\edef\notation@Xchar{\detokenize{X}}%

\protected\def\opref#1{%
  \modules@getURIfromName{#1}%
  \let\invoke@symbol@uri\notation@uri%
  \def\invoke@symbol@return{}%
  \notation@argument@counter=0%
  \def\invoke@symbol@arity{}%
  \@inomstrue%
  \invoke@symbol@text@args%
}

\def\invoke@symbol@text@args{%
  \advance\notation@argument@counter by 1%
  \edef\notation@charnum{\the\notation@argument@counter}%
  \StrChar\invoke@symbol@arity{\the\notation@argument@counter}[\invoke@symbol@nextchar]%
  \ifx\invoke@symbol@nextchar\notation@Xchar%
    \ex\invoke@symbol@text@args%
  \else%
    \ifx\invoke@symbol@nextchar\@empty%
      \let\invoke@symbol@nextstep\invoke@symbol@text@finally%
      \ex\ex\ex\invoke@symbol@maybesqbracket%
    \else%
      \let\invoke@symbol@nextstep\invoke@symbol@normalarg%
      \ex\ex\ex\invoke@symbol@maybestarI%
    \fi%
  \fi%
}

\def\invoke@symbol@maybestarI{%
  \@ifnextchar*{%
    \invoke@symbol@maybestarIII%
  }{%
    \invoke@symbol@maybesqbracket%
  }%
}

\def\invoke@symbol@maybestarIII*{%
  \@ifnextchar[{%
    \invoke@symbol@switchnum%
  }{%
    \invoke@symbol@invisible%
  }%
}

\def\invoke@symbol@maybesqbracket{%
  \@ifnextchar[{\invoke@symbol@verbcomp}{\invoke@symbol@nextstep}%
}

\def\invoke@symbol@verbcomp[#1]{%
  \ex\def\ex\invoke@symbol@return\ex{\invoke@symbol@return #1}%
  \invoke@symbol@nextstep%
}

\def\invoke@symbol@invisible#1{% TODO a-args
  \edef\invoke@symbol@frame{\noexpand\stex@annotate@invisible{arg}{\notation@charnum}}%
  \ex\ex\ex\def\ex\ex\ex\invoke@symbol@return\ex\ex\ex{\ex\invoke@symbol@return\invoke@symbol@frame{#1}}%
  \invoke@symbol@text@args%
}

\def\invoke@symbol@normalarg#1{% TODO a-args
  \edef\invoke@symbol@frame{\noexpand\stex@annotate{arg}{\notation@charnum}}%
  \ex\ex\ex\def\ex\ex\ex\invoke@symbol@return\ex\ex\ex{\ex\invoke@symbol@return\invoke@symbol@frame{#1}}%
  \invoke@symbol@text@args%
}

\def\invoke@symbol@switchnum[#1]{%
  \advance\notation@argument@counter by -1%
  \edef\notation@charnum{#1}%
  \edef\notation@charnum@mone{\the\numexpr\notation@charnum-1\relax}%
  %\StrChar\invoke@symbol@arity\notation@charnum[\invoke@symbol@nextchar]%
%  \ifx\invoke@symbol@nextchar\notation@ichar%
    \StrLeft\invoke@symbol@arity\notation@charnum@mone[\invoke@symbol@newarityLeft]%
    \StrGobbleLeft\invoke@symbol@arity\notation@charnum[\invoke@symbol@newarity]%
    \edef\invoke@symbol@arity{\invoke@symbol@newarityLeft\notation@Xchar\invoke@symbol@newarity}%
%  \else% TODO
%  \fi%
  \invoke@symbol@maybestarII%
}

\def\invoke@symbol@maybestarII{%
  \@ifnextchar*{%
    \invoke@symbol@invisible%
  }{%
    \invoke@symbol@normalarg%
  }%
}

\def\invoke@symbol@text@finally{%
  %\stex@debug{HERE! \meaning\invoke@symbol@return}%
  \if@inoms\html@oms{\invoke@symbol@uri}{}{\invoke@symbol@return}%
  \else\html@oma{\invoke@symbol@uri}{}{\invoke@symbol@return}%
  \fi%
}
%    \end{macrocode}
%
%\test{\begin{module}[name=FooBarVerbs]
% \printltx{\symdecl[args=ii]{plus}}\symdecl[args=ii]{plus}\\
% \printltx{\symdecl{someprime}}\symdecl{someprime}\\
% \printltx{\plus[The sum of ]{\someprime[$p$]}[ and ]{$2$}}: ``\plus[The sum of ]{\someprime[$p$]}[ and ]{$2$}''\\
% \opref{plus}[plus] and $\opref{plus}[+]$ and $+$.
% \end{module}
%}
%
%
%
%
























%
% \subsection{Term References}
%
% \begin{macro}{\ifhref}
%    \begin{macrocode}
\newif\ifhref\hreffalse%
\AtBeginDocument{%
  \@ifpackageloaded{hyperref}{%
    \hreftrue%
  }{%
    \hreffalse%
  }%
}
%    \end{macrocode}
% \end{macro}
%
% \begin{macro}{\termref@maketarget}
% This macro creates a hypertarget |sref@|\meta{symbol URI}|@target| and
% defines |\sref@|\meta{symbol URI}|#1| to create a hyperlink
% to here on the text |#1|.
%    \begin{macrocode}
\newbox\stex@targetbox
\def\termref@maketarget#1#2{%
  % #1: symbol URI
  % #2: text
  %\stex@debug{Here: #1 <> #2}%
  \ifhref\if@smsmode\else%
    \hypertarget{sref@#1@target}{#2}%
  \fi\fi%
  %\stex@debug{Here!}%
  \expandafter\edef\csname sref@#1\endcsname##1{%
    \ifhref\if@smsmode\else\noexpand\hyperlink{sref@#1@target}{##1}\fi\fi%
  }%
}
%    \end{macrocode}
% \end{macro}
%
% \begin{macro}{\@termref}
%    \begin{macrocode}
\def\@termref#1#2{%
  % #1: symbol URI
  % #2: text
  \ifcsvoid{#1}{%
    \StrCut[2]{#1}\@QuestionMark\termref@mod\termref@name%
    \ifcsvoid{\termref@mod}{%
      \PackageError{stex}{Term reference: Module with URI \termref@mod\ not found}{}%
    }{%
      \PackageError{stex}{Term reference: Module \termref@mod\ exists, but %
        contains no symbol with name \termref@name.%
      }{}%
    }%
  }{%
    \ifcsvoid{sref@#1}{%
      #2% TODO: No reference point exists!
    }{%
      \csname sref@#1\endcsname{#2}%
    }%
  }%
}
%    \end{macrocode}
% \end{macro}
%
% \begin{macro}{\tref}
%    \begin{macrocode}

\def\@capitalize#1{\uppercase{#1}}%
\newrobustcmd\capitalize[1]{\expandafter\@capitalize #1}%

\newcommand\tref[2][]{%
  \edef\tref@name{#1}%
  \expandafter\modules@getURIfromName\expandafter{\tref@name}%
  \expandafter\@termref\expandafter{\notation@uri}{#2}%
}
\def\trefs#1{%
  \modules@getURIfromName{#1}%
  % TODO
}
\def\Tref#1{%
  \modules@getURIfromName{#1}%
  % TODO
}
\def\Trefs#1{%
  \modules@getURIfromName{#1}%
  % TODO
}
%    \end{macrocode}
% \end{macro}
%
% \begin{macro}{\defi}
%    \begin{macrocode}
\addmetakey{defi}{name}
\def\@definiendum#1#2{%
  \parsemodule@maybesetcodes%
  %\stex@debug{Here: #1 | #2}%
  \termref@maketarget{#1}{#2}%\termref@maketarget{#1}{\defemph{#2}}%
}

\newcommand\defi[2][]{%
  \metasetkeys{defi}{#1}%
  \ifx\defi@name\@empty%
    \symdecl@constructname{#2}%
    \let\defi@name\symdecl@name%
    \let\defi@verbalization\symdecl@verbalization%
  \else%
    \edef\defi@verbalization{#2}%
  \fi%
  \ifcsvoid{\module@uri\@QuestionMark\defi@name}{%
    \symdecl\defi@name%
  }{\edef\symdecl@uri{\module@uri\@QuestionMark\defi@name}}%
  \@definiendum\symdecl@uri\defi@verbalization%
}
\def\Defi#1{%
  \symdecl{#1}%
  \@definiendum\symdecl@uri{\capitalize\symdecl@verbalization}%
}
\def\defis#1{%
  \symdecl{#1}%
  \@definiendum\symdecl@uri{\symdecl@verbalization s}%
}
\def\Defis#1{%
  \symdecl{#1}%
  \@definiendum\symdecl@uri{\capitalize\symdecl@verbalization s}%
}
%    \end{macrocode}
% \end{macro}
%
%
%
%
%
%
%
%
%
%
%
%
%
%
%
%
%
%
%
%
%
%
%
%
%
%
%
%
%
%
%
%
%
%
%
%
%
%
% \subsection{sref}
%
% We find out whether the |hyperref| package is loaded, since we may want to use it for
% cross-references, for which we set up some internal macros that gracefully degrade if
% |hyperref| is not loaded. 
% 
% \begin{macro}{\sref@*@ifh}
%    \begin{macrocode}
\newif\ifhref\hreffalse%
\AtBeginDocument{%
  \@ifpackageloaded{hyperref}{%
    \hreftrue%
  }{%
    \hreffalse%
  }%
}%
\newcommand\sref@href@ifh[2]{%
  \ifhref%
    \href{#1}{#2}%
  \else%
    #2%
  \fi%
}%
\newcommand\sref@hlink@ifh[2]{%
  \ifhref%
    \hyperlink{#1}{#2}%
  \else%
    #2%
  \fi%
}%
\newcommand\sref@target@ifh[2]{%
  \ifhref%
    \hypertarget{#1}{#2}%
  \else%
    #2%
  \fi%
}%
%    \end{macrocode}
% \end{macro}
% 
% Then we provide some macros for \sTeX-specific crossreferencing 
%
% \begin{macro}{\sref@target}
%   The next macro uses this and makes an target from the current |sref@id| declared by
%   a |id| key.
%    \begin{macrocode}
\def\sref@target{%
  \ifx\sref@id\@empty%
    \relax%
  \else%
    \edef\@target{sref@\ifcsundef{sref@part}{}{\sref@part @}\sref@id @target}%
    \sref@target@ifh\@target{}%
  \fi%
}%
%    \end{macrocode}
% \end{macro}
%
%
% \begin{macro}{\srefaddidkey}
%   |\srefaddidkey[|\meta{keyval}|]{|\meta{group}|}| extends the metadata keys of the
%   group \meta{group} with an |id| key. In the optional key/value pairs in \meta{keyval}
%   the |prefix| key can be used to specify a prefix. Note that the |id| key defined by
%   |\srefaddidkey[|\meta{keyval}|]{|\meta{group}|}| not only defines |\sref@id|, which is
%   used for referencing by the |sref| package, but also |\|\meta{group}|@id|, which is
%   used for showing metadata via the |showmeta| option of the |metakeys| package.
%    \begin{macrocode}
\addmetakey{srefaddidkey}{prefix}
\newcommand\srefaddidkey[2][]{%
  \metasetkeys{srefaddidkey}{#1}%
  \@metakeys@ext@clear@keys{#2}{sref@id}{}% id cannot have a default
  \metakeys@ext@clear@keys{#2}{id}{}%
  \metakeys@ext@showkeys{#2}{id}%
  \define@key{#2}{id}{%
    \edef\sref@id{\srefaddidkey@prefix ##1}%
    %\expandafter\edef\csname #2@id\endcsname{\srefaddidkey@prefix ##1}%
    \csedef{#2@id}{\srefaddidkey@prefix ##1}%
  }%
}%
%    \end{macrocode}
% \end{macro}
%
% \begin{macro}{\@sref@def}
%   This macro stores the value of its last argument in a custom macro for reference.
%    \begin{macrocode}
\newcommand\@sref@def[3]{\csgdef{sref@#1@#2}{#3}}
%    \end{macrocode}
% \end{macro}
%
% The next step is to set up a file to which the references are written, this is normally
% the |.aux| file, but if the |extref| option is set, we have to use an |.ref| file. 
%    \begin{macrocode}
\ifextrefs%
  \newwrite\refs@file%
\else%
  \def\refs@file{\@auxout}%
\fi%
%    \end{macrocode}
%
% \begin{macro}{\sref@def}
%    This macro writes an |\@sref@def| command to the current aux file and also executes it. 
%    \begin{macrocode}
\newcommand\sref@def[3]{%
  \protected@write\refs@file{}{\string\@sref@def{#1}{#2}{#3}}%
}%
%    \end{macrocode}
% \end{macro}
%
% \begin{macro}{\sref@label}
%   The |\sref@label| macro writes a label definition to the auxfile. 
%    \begin{macrocode}
\newcommand\sref@label[2]{%
  \sref@def{\ifcsundef{sref@part}{}{\sref@part @}#2}{page}{\thepage}%
  \sref@def{\ifcsundef{sref@part}{}{\sref@part @}#2}{label}{#1}%
}%
%    \end{macrocode}
% \end{macro}
%
% \begin{macro}{\sreflabel}
%   The |\sreflabel| macro is a semantic version of |\label|, it combines the categorization
%   given in the first argument with {\LaTeX}'s |\@currentlabel|.
%    \begin{macrocode}
\newcommand\sreflabel[2]{\sref@label{#1 \@currentlabel}{#2}}
%    \end{macrocode}
% \end{macro}
%
% \begin{macro}{\sref@label@id}
%   The |\sref@label@id| writes a label definition for the current |\sref@id| if it is
%   defined.
%    \begin{macrocode}
\def\sref@id{} % make sure that defined
\newcommand\sref@label@id[1]{%
  \ifx\sref@id\@empty%
    \relax%
  \else%
    \sref@label{#1}{\sref@id}%
  \fi%
}%
%    \end{macrocode}
% \end{macro}
%
% \begin{macro}{\sref@label@id@arg}
%   The |\sref@label@id@arg| writes a label definition for the second argument if it is
%   defined.
%    \begin{macrocode}
\newcommand\sref@label@id@arg[2]{%
  \def\@@id{#2}
  \ifx\@@id\@empty%
    \relax%
  \else%
    \sref@label{#1}{\@@id}%
  \fi%
}%
%    \end{macrocode}
% \end{macro}
%
%
%
% \subsection{smultiling}
%
% \begin{environment}{modsig}
%   The |modsig| environment is just a layer over the |module| environment. We also
%   redefine macros that may occur in module signatures so that they do not create
%   markup. Finally, we set the flag |\mod@|\meta{mod}|@multiling| to |true|.
%    \begin{macrocode}
\newenvironment{modsig}[2][]{\def\@test{#1}%
\ifx\@test\@empty\begin{module}[name=#2]\else\begin{module}[name=#2,#1]\fi%
\expandafter\gdef\csname mod@#2@multiling\endcsname{true}%
%\ignorespacesandpars
}
{\end{module}%\ignorespacesandpars
}
%    \end{macrocode}
% \end{environment}
%
%
% \subsection{smglom}
%
% \begin{macro}{\gimport}
% Just a shortcut, we have a starred and unstarred version, the first one is conservative.
% For example, if we execute:\newline
% \begin{center}
% |\gimport[smglom/numberfields]{naturalnumbers}|\newline
% \end{center}
% First we are redirected to |\@gimport@nostar|, we store the
% |smglom/numberfields|\meta{the repo's path} in |\@test|, then store
% |\mh@currentrepos|\meta{current directory} in |\mh@repos|. If no repo's path is offered,
% that means the module to import is under the same directory, so we let |mhrepos=\mh@repos|
% and pass bunch of parameters to |\importmhmodule|, which is defined in |module.sty|. If
% there's a repo's path, then we let |mhrepos=|\meta{the repo's path}. Finally we use
% |\mhcurrentrepos|(defined in |module.sty|) to change the |\mh@currentrepos|.
%    \begin{macrocode}
\def\gimport{\@ifstar\@gimport@star\@gimport@nostar}%
\newrobustcmd\@gimport@star[2][]{\def\@test{#1}%
\edef\mh@@repos{\mh@currentrepos}%
\ifx\@test\@empty%
\importmhmodule[conservative,mhrepos=\mh@@repos,path=#2]{#2}%
\else\importmhmodule[conservative,mhrepos=#1,path=#2]{#2}\fi%
\mathhub@setcurrentreposinfo{\mh@@repos}%
%\ignorespacesandpars
\parsemodule@maybesetcodes}
\newrobustcmd\@gimport@nostar[2][]{\def\@test{#1}%
\edef\mh@@repos{\mh@currentrepos}%
\ifx\@test\@empty%
\importmhmodule[mhrepos=\mh@@repos,path=#2]{#2}%
\else\importmhmodule[mhrepos=#1,path=#2]{#2}\fi%
\mathhub@setcurrentreposinfo{\mh@@repos}%
%\ignorespacesandpars
\parsemodule@maybesetcodes}
%    \end{macrocode}
% \end{macro}
%
% \subsection{mathhub}
%
% \begin{macro}{\libinput}
%   the |\libinput| macro inputs from the |lib| directory of the MathHub repository and
%   then the |meta-inf/lib| repository of the group, if they exist. Since in practice
%   nested libinputs may occur, we make sure that we stash the old values of |\mh@inffile|
%   and |\mh@libfile| and restore them at the end.
%    \begin{macrocode}
\def\modules@@first#1/#2;{#1}
\newcommand\libinput[1]{%
\stex@debug{Libinput current repo: \meaning\mh@currentrepos}%
\ifcsvoid{mh@currentrepos}{%
  \PackageError{stex}{current MathHub repository not found}{}}%
  {}
\edef\@mh@group{\expandafter\modules@@first\mh@currentrepos;}
\let\orig@inffile\mh@inffile\let\orig@libfile\mh@libfile
\def\mh@inffile{\MathHub{\@mh@group/meta-inf/lib/#1}}
\def\mh@libfile{\MathHub{\mh@currentrepos/lib/#1}}%
\IfFileExists\mh@inffile{\stexinput\mh@inffile}{}%
\IfFileExists\mh@inffile{}{\IfFileExists\mh@libfile{}{%
  {\PackageError{stex}
    {Library file missing; cannot input #1.tex\MessageBreak%
    Both \mh@libfile.tex\MessageBreak and \mh@inffile.tex\MessageBreak%
    do not exist}%
  {Check whether the file name is correct}}}}
\IfFileExists\mh@libfile{\stexinput\mh@libfile\relax}{}
\let\mh@inffile\orig@inffile\let\mh@libfile\orig@libfile}
%    \end{macrocode}
% \end{macro}
%
% \subsection{omdoc/omgroup}
%
%    \begin{macrocode}
\newcount\section@level

\section@level=2
\ifdefstring{\omdoc@sty@class}{book}{\section@level=0}{}
\ifdefstring{\omdoc@sty@class}{report}{\section@level=0}{}
\ifdefstring{\omdoc@sty@topsect}{part}{\section@level=0}{}
\ifdefstring{\omdoc@sty@topsect}{chapter}{\section@level=1}{}
%    \end{macrocode}
%
% \begin{macro}{\omgroup@nonum}
%   convenience macro: |\omgroup@nonum{|\meta{level}|}{|\meta{title}|}| makes an unnumbered
%   sectioning with title \meta{title} at level \meta{level}.
%    \begin{macrocode}
\newcommand\omgroup@nonum[2]{%
\ifx\hyper@anchor\@undefined\else\phantomsection\fi%
\addcontentsline{toc}{#1}{#2}\@nameuse{#1}*{#2}}
%    \end{macrocode}
% \end{macro}
%
% \begin{macro}{\omgroup@num}
%   convenience macro: |\omgroup@nonum{|\meta{level}|}{|\meta{title}|}| makes numbered
%   sectioning with title \meta{title} at level \meta{level}. We have to check the |short|
%   key was given in the |omgroup| environment and -- if it is use it. But how to do that
%   depends on whether the |rdfmeta| package has been loaded. In the end we call
%   |\sref@label@id| to enable crossreferencing.
%    \begin{macrocode}
\newcommand\omgroup@num[2]{%
\edef\@@ID{\sref@id}
\ifx\omgroup@short\@empty% no short title
\@nameuse{#1}{#2}%
\else% we have a short title
\@ifundefined{rdfmeta@sectioning}%
  {\@nameuse{#1}[\omgroup@short]{#2}}%
  {\@nameuse{rdfmeta@#1@old}[\omgroup@short]{#2}}%
\fi%
\sref@label@id@arg{\omdoc@sect@name~\@nameuse{the#1}}\@@ID}
%    \end{macrocode}
% \end{macro}
%
% \begin{environment}{omgroup}
%    \begin{macrocode}
\def\@true{true}
\def\@false{false}
\srefaddidkey{omgroup}
\addmetakey{omgroup}{date}
\addmetakey{omgroup}{creators}
\addmetakey{omgroup}{contributors}
\addmetakey{omgroup}{srccite}
\addmetakey{omgroup}{type}
\addmetakey*{omgroup}{short}
\addmetakey*{omgroup}{display}
\addmetakey[false]{omgroup}{loadmodules}[true]
%    \end{macrocode}
% we define a switch for numbering lines and a hook for the beginning of groups: The
% \DescribeMacro{\at@begin@omgroup}|\at@begin@omgroup| macro allows customization. It is
% run at the beginning of the |omgroup|, i.e. after the section heading.
%    \begin{macrocode}
\newif\if@mainmatter\@mainmattertrue
\newcommand\at@begin@omgroup[3][]{}
%    \end{macrocode}
%
% Then we define a helper macro that takes care of the sectioning magic. It comes with its
% own key/value interface for customization.
%
%    \begin{macrocode}
\addmetakey{omdoc@sect}{name}
\addmetakey[false]{omdoc@sect}{clear}[true]
\addmetakey{omdoc@sect}{ref}
\addmetakey[false]{omdoc@sect}{num}[true]
\newcommand\omdoc@sectioning[3][]{\metasetkeys{omdoc@sect}{#1}%
\ifx\omdoc@sect@clear\@true\cleardoublepage\fi%
\if@mainmatter% numbering not overridden by frontmatter, etc.
\ifx\omdoc@sect@num\@true\omgroup@num{#2}{#3}\else\omgroup@nonum{#2}{#3}\fi%
\def\current@section@level{\omdoc@sect@name}%
\else\omgroup@nonum{#2}{#3}%
\fi}% if@mainmatter
%    \end{macrocode}
% and another one, if redefines the |\addtocontentsline| macro of {\LaTeX} to import the
% respective macros. It takes as an argument a list of module names.
%    \begin{macrocode}
\newcommand\omgroup@redefine@addtocontents[1]{%
%\edef\@@import{#1}%
%\@for\@I:=\@@import\do{%
%\edef\@path{\csname module@\@I  @path\endcsname}%
%\@ifundefined{tf@toc}\relax%
%     {\protected@write\tf@toc{}{\string\@requiremodules{\@path}}}}
%\ifx\hyper@anchor\@undefined% hyperref.sty loaded?
%\def\addcontentsline##1##2##3{%
%\addtocontents{##1}{\protect\contentsline{##2}{\string\withusedmodules{#1}{##3}}{\thepage}}}
%\else% hyperref.sty not loaded
%\def\addcontentsline##1##2##3{%
%\addtocontents{##1}{\protect\contentsline{##2}{\string\withusedmodules{#1}{##3}}{\thepage}{\@currentHref}}}%
%\fi
}% hypreref.sty loaded?
%    \end{macrocode}
% now the |omgroup| environment itself. This takes care of the table of contents via the
% helper macro above and then selects the appropriate sectioning command from
% |article.cls|. It also registeres the current level of omgroups in the |\omgroup@level|
% counter. 
%    \begin{macrocode}
\newcount\omgroup@level 
\newenvironment{omgroup}[2][]% keys, title
{\metasetkeys{omgroup}{#1}\sref@target%
\advance\omgroup@level by 1\relax%
%    \end{macrocode}
% If the |loadmodules| key is set on |\begin{omgroup}|, we redefine the |\addcontetsline|
%   macro that determines how the sectioning commands below construct the entries for the
%   table of contents.
%    \begin{macrocode}
\ifx\omgroup@loadmodules\@true%
\omgroup@redefine@addtocontents{\@ifundefined{module@id}\used@modules%
{\@ifundefined{module@\module@id @path}{\used@modules}\module@id}}\fi%
%    \end{macrocode}
% now we only need to construct the right sectioning depending on the value of
% |\section@level|.
%    \begin{macrocode}
\advance\section@level by 1\relax%
\ifcase\section@level%
\or\omdoc@sectioning[name=\omdoc@part@kw,clear,num]{part}{#2}%
\or\omdoc@sectioning[name=\omdoc@chapter@kw,clear,num]{chapter}{#2}%
\or\omdoc@sectioning[name=\omdoc@section@kw,num]{section}{#2}%
\or\omdoc@sectioning[name=\omdoc@subsection@kw,num]{subsection}{#2}%
\or\omdoc@sectioning[name=\omdoc@subsubsection@kw,num]{subsubsection}{#2}%
\or\omdoc@sectioning[name=\omdoc@paragraph@kw,ref=this \omdoc@paragraph@kw]{paragraph}{#2}%
\or\omdoc@sectioning[name=\omdoc@subparagraph@kw,ref=this \omdoc@subparagraph@kw]{paragraph}{#2}%
\fi% \ifcase
\at@begin@omgroup[#1]\section@level{#2}}% for customization
{\advance\section@level by -1\advance\omgroup@level by -1}
%    \end{macrocode}
% \end{environment}
%
% and finally, we localize the sections
%    \begin{macrocode}
\newcommand\omdoc@part@kw{Part}
\newcommand\omdoc@chapter@kw{Chapter}
\newcommand\omdoc@section@kw{Section}
\newcommand\omdoc@subsection@kw{Subsection}
\newcommand\omdoc@subsubsection@kw{Subsubsection}
\newcommand\omdoc@paragraph@kw{paragraph}
\newcommand\omdoc@subparagraph@kw{subparagraph}
%    \end{macrocode}
%
% \begin{macro}{\setSGvar}
%   set a global variable
%    \begin{macrocode}
\newcommand\setSGvar[1]{\@namedef{sTeX@Gvar@#1}}
%    \end{macrocode}
% \end{macro}
%
% \begin{macro}{\useSGvar}
%   use a global variable
%    \begin{macrocode}
\newrobustcmd\useSGvar[1]{%
  \@ifundefined{sTeX@Gvar@#1}
  {\PackageError{omdoc}
    {The sTeX Global variable #1 is undefined}
    {set it with \protect\setSGvar}}
\@nameuse{sTeX@Gvar@#1}}
%    \end{macrocode}
% \end{macro}
%
% \begin{environment}{blindomgroup}
% \begin{macrocode}
\newcommand\at@begin@blindomgroup[1]{}
\newenvironment{blindomgroup}
{\advance\section@level by 1\at@begin@blindomgroup\setion@level}
{\advance\section@level by -1}
%    \end{macrocode}
% \end{environment}
%
%
% \subsection{omtext}
%
% \subsubsection{Mathematical Text}
%
%   We define the actions that are undertaken, when the keys are encountered. The first
%   set just records metadata; this is very simple via the |\addmetakey|
%   infrastructure~\ctancite{Kohlhase:metakeys}. Note that we allow math in the |title|
%   field, so we do not declare it to be |Semiverbatim| (indeed not at all, which allows
%   it by default).
%
%    \begin{macrocode}
\srefaddidkey{omtext}
\addmetakey[]{omtext}{functions}
\addmetakey*{omtext}{display}
\addmetakey{omtext}{for}
\addmetakey{omtext}{from}
\addmetakey{omtext}{type}
\addmetakey*{omtext}{title}
\addmetakey*{omtext}{start}
\addmetakey{omtext}{theory}
\addmetakey{omtext}{continues}
\addmetakey{omtext}{verbalizes}
\addmetakey{omtext}{subject}
%    \end{macrocode}
%
% \begin{macro}{\st@flow}
% We define this macro, so that we can test whether the |display| key has the value |flow|
%    \begin{macrocode}
\def\st@flow{flow}
%    \end{macrocode}
% \end{macro}
%
% We define a switch that allows us to see whether we are inside an |omtext| environment
% or a statement. It will be used to give better error messages for inline statements.
%
%    \begin{macrocode}
\newif\if@in@omtext\@in@omtextfalse
%    \end{macrocode}
%
% \begin{environment}{omtext}
%   The |omtext| environment can have a title, which is used in a similar way. We redefine
%   the |\lec| macro so the trailing |\par| does not get into the way. 
%    \begin{macrocode}
\def\omtext@pre@skip{\smallskip}
\def\omtext@post@skip{}
\newenvironment{omtext}[1][]{\@in@omtexttrue%
  \bgroup\metasetkeys{omtext}{#1}\sref@label@id{this paragraph}%
  \def\lec##1{\@lec{##1}}%
  \omtext@pre@skip\par\noindent%
  \ifx\omtext@title\@empty%
    \ifx\omtext@start\@empty\else%
      \ifx\omtext@display\st@flow\omtext@start\else\stDMemph{\omtext@start}\fi\enspace%
    \fi% end omtext@start empty
  \else\stDMemph{\omtext@title}:\enspace%
    \ifx\omtext@start\@empty\else\omtext@start\enspace\fi%
  \fi% end omtext@title empty
  %\ignorespacesandpars
  }
{\egroup\omtext@post@skip\@in@omtextfalse%\ignorespacesandpars
}
%    \end{macrocode}
% \end{environment}
%
% \subsubsection{Phrase-level Markup}
% 
% \begin{macro}{\phrase}
%    For the moment, we do disregard the most of the keys
%    \begin{macrocode}
\srefaddidkey{phrase}
\addmetakey{phrase}{style}
\addmetakey{phrase}{class}
\addmetakey{phrase}{index}
\addmetakey{phrase}{verbalizes}
\addmetakey{phrase}{type}
\addmetakey{phrase}{only}
\newcommand\phrase[2][]{\metasetkeys{phrase}{#1}%
\ifx\prhase@only\@empty\only<\phrase@only>{#2}\else #2\fi}
%    \end{macrocode}
% \end{macro}
%
% \begin{macro}{\coref*}
%    \begin{macrocode}
\providecommand\textsubscript[1]{\ensuremath{_{#1}}}
\newcommand\corefs[2]{#1\textsubscript{#2}}
\newcommand\coreft[2]{#1\textsuperscript{#2}}
%    \end{macrocode}
% \end{macro}
%
% \begin{macro}{\n*lex}
%    \begin{macrocode}
\newcommand\nlex[1]{\green{\sl{#1}}}
\newcommand\nlcex[1]{*\green{\sl{#1}}}
%    \end{macrocode}
% \end{macro}
% 
% \begin{macro}{sinlinequote}
%    \begin{macrocode}
\def\@sinlinequote#1{``{\sl{#1}}''}
\def\@@sinlinequote#1#2{\@sinlinequote{#2}~#1}
\newcommand\sinlinequote[2][]
{\def\@opt{#1}\ifx\@opt\@empty\@sinlinequote{#2}\else\@@sinlinequote\@opt{#2}\fi}
%    \end{macrocode}
% \end{macro}
%
% \subsubsection{Declarations (under development)}
% 
%   The declaration macros are still under development (i.e. the macros) are still under
%   development and may change at any time. Currently they are completely empty.
% 
% \begin{macrocode}
\newcommand\vdec[2][]{#2}
\newcommand\vrest[2][]{#2}
\newcommand\vcond[2][]{#2}
%    \end{macrocode}
%
% \begin{macro}{\strucdec}
%   \ednote{document above}
%    \begin{macrocode}
\newcommand\strucdec[2][]{#2}
%    \end{macrocode}
% \end{macro}
% 
% \begin{macro}{\impdec}
%   \ednote{document above}
%    \begin{macrocode}
\newcommand\impdec[2][]{#2}
%    \end{macrocode}
% \end{macro}
%
% \subsubsection{Block-Level Markup}
%
% \begin{environment}{sblockquote}
%    \begin{macrocode}
\def\begin@sblockquote{\begin{quote}\sl}
\def\end@sblockquote{\end{quote}}
\def\begin@@sblockquote#1{\begin@sblockquote}
\def\end@@sblockquote#1{\def\@@lec##1{\textrm{##1}}\@lec{#1}\end@sblockquote}
\newenvironment{sblockquote}[1][]
  {\def\@opt{#1}\ifx\@opt\@empty\begin@sblockquote\else\begin@@sblockquote\@opt\fi}
  {\ifx\@opt\@empty\end@sblockquote\else\end@@sblockquote\@opt\fi}
%    \end{macrocode}
% \end{environment}
%
% \begin{environment}{sboxquote}
%    \begin{macrocode}
\newenvironment{sboxquote}[1][]
{\def\@@src{#1}\begin{mdframed}[leftmargin=.5cm,rightmargin=.5cm]}
{\@lec{\textrm\@@src}\end{mdframed}}
%    \end{macrocode}
% \end{environment}
% 
% The line end comment macro makes sure that it will not be forced on the next line unless
% necessary.
% \begin{macro}{\lec}
%   The actual appearance of the line end comment is determined by the |\@@lec| macro,
%   which can be customized in the document class. The basic one here is provided so that
%   it is not missing.
%    \begin{macrocode}
\providecommand{\@@lec}[1]{(#1)}
\def\@lec#1{\strut\hfil\strut\null\nobreak\hfill\@@lec{#1}}
\def\lec#1{\@lec{#1}\par}
%    \end{macrocode}
% \end{macro}
% 
% \subsubsection{Index Markup}
%
% \begin{macro}{\omdoc@index*}
%   These are the main internal indexing commands -- dividing them into four macros is awful,
%   but I did not get list processing running. It makes sure that the modules necessary
%   for interpreting the math in the index entries are loaded. If the |loadmodules| key is
%   given, we import the module we are in otherwise all the currently imported modules. We
%   do not have to require the module files, since the index is a the end of the
%   document. If the |at| key is given, then we use that for sorting in the index.
%    \begin{macrocode}
\addmetakey{omdoc@index}{at}
\addmetakey[false]{omdoc@index}{loadmodules}[true]
\newcommand\omdoc@indexi[2][]{\ifindex%
\metasetkeys{omdoc@index}{#1}%
\@bsphack\begingroup\@sanitize%
\protected@write\@indexfile{}{\string\indexentry%
{\ifx\omdoc@index@at\@empty\else\omdoc@index@at @\fi%
\ifx\omdoc@index@loadmodules\@true%
\string\withusedmodules{\@ifundefined{module@id}\used@modules\module@id}{#2}%
\else #2\fi% loadmodules
}{\thepage}}%
\endgroup\@esphack\fi}%ifindex
\newcommand\omdoc@indexii[3][]{\ifindex%
\metasetkeys{omdoc@index}{#1}%
\@bsphack\begingroup\@sanitize%
\protected@write\@indexfile{}{\string\indexentry%
{\ifx\omdoc@index@at\@empty\else\omdoc@index@at @\fi%
\ifx\omdoc@index@loadmodules\@true%
\string\withusedmodules{\@ifundefined{module@id}\used@modules\module@id}{#2}!%
\string\withusedmodules{\@ifundefined{module@id}\used@modules\module@id}{#3}%
\else #2!#3\fi% loadmodules
}{\thepage}}%
\endgroup\@esphack\fi}%ifindex
\newcommand\omdoc@indexiii[4][]{\ifindex%
\metasetkeys{omdoc@index}{#1}%
\@bsphack\begingroup\@sanitize%
\protected@write\@indexfile{}{\string\indexentry%
{\ifx\omdoc@index@at\@empty\else\omdoc@index@at @\fi%
\ifx\omdoc@index@loadmodules\@true%
\string\withusedmodules{\@ifundefined{module@id}\used@modules\module@id}{#2}!%
\string\withusedmodules{\@ifundefined{module@id}\used@modules\module@id}{#3}!%
\string\withusedmodules{\@ifundefined{module@id}\used@modules\module@id}{#4}%
\else #2!#3!#4\fi% loadmodules
}{\thepage}}%
\endgroup\@esphack\fi}%ifindex
\newcommand\omdoc@indexiv[5][]{\ifindex%
\metasetkeys{omdoc@index}{#1}%
\@bsphack\begingroup\@sanitize%
\protected@write\@indexfile{}{\string\indexentry%
{\ifx\omdoc@index@at\@empty\else\omdoc@index@at @\fi%
\ifx\omdoc@index@loadmodules\@true%
\string\withusedmodules{\@ifundefined{module@id}\used@modules\module@id}{#2}!%
\string\withusedmodules{\@ifundefined{module@id}\used@modules\module@id}{#3}!%
\string\withusedmodules{\@ifundefined{module@id}\used@modules\module@id}{#4}%
\string\withusedmodules{\@ifundefined{module@id}\used@modules\module@id}{#5}%
\else #2!#3!#4!#5\fi% loadmodules
}{\thepage}}%
\endgroup\@esphack\fi}%ifindex
%    \end{macrocode}
% \end{macro}
% 
% Now, we make two interface macros that make use of this: 
%
% \begin{macro}{\*indi*}
%    \begin{macrocode}
\newcommand\aindi[3][]{{#2}\omdoc@indexi[#1]{#3}}
\newcommand\indi[2][]{{#2}\omdoc@indexi[#1]{#2}}
\newcommand\indis[2][]{{#2}\omdoc@indexi[#1]{#2s}}
\newcommand\Indi[2][]{{\captitalize{#2}}\omdoc@indexi[#1]{#2}}
\newcommand\Indis[2][]{{\capitalize{#2}}\omdoc@indexi[#1]{#2s}}

\newcommand\@indii[3][]{\omdoc@indexii[#1]{#2}{#3}\omdoc@indexii[#1]{#3}{#2}}
\newcommand\aindii[4][]{#2\@indii[#1]{#3}{#4}}
\newcommand\indii[3][]{{#2 #3}\@indii[#1]{#2}{#3}}
\newcommand\indiis[3][]{{#2 #3s}\@indii[#1]{#2}{#3}}
\newcommand\Indii[3][]{{\captitalize{#2 #3}}\@indii[#1]{#2}{#3}}
\newcommand\Indiis[3][]{{\capitalize{#2 #3}}\@indii[#1]{#2}{#3}}

\newcommand\@indiii[4][]{\omdoc@indexiii[#1]{#2}{#3}{#4}\omdoc@indexii[#1]{#3}{#2 (#4)}}
\newcommand\aindiii[5][]{{#2}\@indiii[#1]{#3}{#4}{#5}}
\newcommand\indiii[4][]{{#2 #3 #4}\@indiii[#1]{#2}{#3}{#4}}
\newcommand\indiiis[4][]{{#2 #3 #4s}\@indiii[#1]{#2}{#3}{#4}}
\newcommand\Indiii[4][]{\captitalize{#2 #3 #4}\@indiii[#1]{#2}{#3}{#4}}
\newcommand\Indiiis[4][]{\capitalize{#2 #3 #4s}\@indiii[#1]{#2}{#3}{#4}}

\newcommand\@indiv[5][]{\omdoc@indexiv[#1]{#2}{#3}{#4}{#5}}
\newcommand\aindiv[6][]{#2\@indiv[#1]{#3}{#4}{#5}{#6}}
\newcommand\indiv[5][]{{#2 #3 #4 #5}\@indiv[#1]{#2}{#3}{#4}{#5}}
\newcommand\indivs[5][]{{#2 #3 #4 #5s}\@indiv[#1]{#2}{#3}{#4}{#5}}
\newcommand\Indiv[5][]{\capitalize{#2 #3 #4 #5s}\@indiv[#1]{#2}{#3}{#4}{#5}}
\newcommand\Indivs[5][]{\capitalize{#2 #3 #4 #5s}\@indiv[#1]{#2}{#3}{#4}{#5}}
%    \end{macrocode}
% \end{macro}
% 
% \subsubsection{Miscellaneous}
%   Some shortcuts that use math symbols but are not mathematical at all; in particular,
%   they should not be translated by {\latexml}.
%    \begin{macrocode}
\newcommand\hateq{\ensuremath{\widehat=}\xspace}
\newcommand\hatequiv{\ensuremath{\widehat\equiv}\xspace}
\@ifundefined{ergo}%
{\newcommand\ergo{\ensuremath{\leadsto}\xspace}}%
{\renewcommand\ergo{\ensuremath{\leadsto}\xspace}}%
\newcommand{\reflect@squig}[2]{\reflectbox{$\m@th#1\rightsquigarrow$}}%
\newcommand\ogre{\ensuremath{\mathrel{\mathpalette\reflect@squig\relax}}\xspace}%
\newcommand\notergo{\ensuremath{\not\leadsto}}
\newcommand\notogre{\ensuremath{\not\mathrel{\mathpalette\reflect@squig\relax}}\xspace}%
%    \end{macrocode}
%
%
% \subsubsection{Deprecated Functionality}\label{sec:deprecated}
%
% In this section we centralize old interfaces that are only partially supported any
% more. 
% \begin{macro}{\*def*}
%    \begin{macrocode}
\newcommand\indextoo[2][]{\indi[#1]{#2}%
\PackageWarning{omtext}{\protect\indextoo\space is deprecated, use \protect\indi\space instead}}
\newcommand\indexalt[2][]{\aindi[#1]{#2}%
\PackageWarning{omtext}{\protect\indextoo\space is deprecated, use \protect\aindi\space instead}}
\newcommand\twintoo[3][]{\indii[#1]{#2}{#3}%
\PackageWarning{omtext}{\protect\twintoo\space is deprecated, use \protect\indii\space instead}}
\newcommand\twinalt[3][]{\aindii[#1]{#2}{#3}%
\PackageWarning{omtext}{\protect\twinalt\space is deprecated, use \protect\aindii\space instead}}
\newcommand\atwintoo[4][]{\indiii[#1]{#2}{#3}{#4}%
\PackageWarning{omtext}{\protect\atwintoo\space is deprecated, use \protect\indiii\space instead}}
\newcommand\atwinalt[4][]{\aindii[#1]{#2}{#3}{#4}%
\PackageWarning{omtext}{\protect\atwinalt\space is deprecated, use \protect\aindiii\space instead}}
%    \end{macrocode}
% \end{macro}
%
% \begin{macro}{\my*graphics}
%    \begin{macrocode}
\newcommand\mygraphics[2][]{\includegraphics[#1]{#2}%
  \PackageWarning{omtext}{\protect\mygraphics\space is deprecated, use \protect\includegraphics\space instead}}
\newcommand\mycgraphics[2][]{\begin{center}\mygraphics[#1]{#2}\end{center}%
  \PackageWarning{omtext}{\protect\mycgraphics\space is deprecated, use \protect\includegraphics\space instead}}
\newcommand\mybgraphics[2][]{\fbox{\mygraphics[#1]{#2}}%
  \PackageWarning{omtext}{\protect\mybgraphics\space is deprecated, use \protect\includegraphics\space instead}}
\newcommand\mycbgraphics[2][]{\begin{center}\fbox{\mygraphics[#1]{#2}}\end{center}%
  \PackageWarning{omtext}{\protect\mycbgraphics\space is deprecated, use \protect\includegraphics\space instead}}
%    \end{macrocode}
% \end{macro}
%
%
%
%
%
%
%
%
%
%
%
%
%
%
%
%
%
%
%
%
%
%
%
%
%
%
%
%
%
%
%
%
% \section{Things to deprecate}
%
% Module options:
%    \begin{macrocode}
\addmetakey*{module}{id} % TODO: deprecate properly
\addmetakey*{module}{load}
\addmetakey*{module}{path}
\addmetakey*{module}{dir}
\addmetakey*{module}{align}[WithTheModuleOfTheSameName]
\addmetakey*{module}{noalign}[true]

\newif\if@insymdef@\@insymdef@false
%    \end{macrocode}
% \begin{macro}{symdef:keys}
%   The optional argument local specifies the scope of the function to be defined. If
%   local is not present as an optional argument then |\symdef| assumes the scope of the
%   function is global and it will include it in the pool of macros of the current
%   module. Otherwise, if local is present then the function will be defined only locally
%   and it will not be added to the current module (i.e. we cannot inherit a local
%   function).  Note, the optional key local does not need a value: we write
%   |\symdef[local]{somefunction}[0]{some expansion}|. The other keys are not used in the
%   {\LaTeX} part. 
%    \begin{macrocode}
%\srefaddidkey{symdef}% what does this do?
\define@key{symdef}{noverb}[all]{}%
\define@key{symdef}{align}[WithTheSymbolOfTheSameName]{}%
\define@key{symdef}{specializes}{}%
\addmetakey*{symdef}{noalign}[true]
\define@key{symdef}{primary}[true]{}%
\define@key{symdef}{assocarg}{}%
\define@key{symdef}{bvars}{}%
\define@key{symdef}{bargs}{}%
\addmetakey{symdef}{lang}%
\addmetakey{symdef}{prec}%
\addmetakey{symdef}{arity}%
\addmetakey{symdef}{variant}%
\addmetakey{symdef}{ns}%
\addmetakey{symdef}{args}%
\addmetakey{symdef}{name}%
\addmetakey*{symdef}{title}%
\addmetakey*{symdef}{description}%
\addmetakey{symdef}{subject}%
\addmetakey*{symdef}{display}%
\addmetakey*{symdef}{gfc}%

\define@key{symdef}{type}{\def\symdef@type{#1}}%
\define@key{symdef}{args}{\def\symdef@args{#1}}%
\define@key{symdef}{local}[true]{\def\symdef@local{#1}}%

%    \end{macrocode}
% \end{macro}
%
% \ednote{MK@MK: we need to document the binder keys above.}
% \begin{macro}{\symdef}
%    The the |\symdef|, and |\@symdef| macros just handle optional arguments.
%    \begin{macrocode}
\def\symdef{\@ifnextchar[{\@@symdef}{\@@symdef[]}}%
%    \end{macrocode}
% \end{macro}
% \begin{macro}{\@@symdef}
%    now comes the real meat: the |\@@symdef| macro does two things, it adds the macro
%    definition to the macro definition pool of the current module and also provides it.
%    \begin{macrocode}
\def\@@symdef[#1]#2{%
  \@insymdef@true%
  \def\symdef@name{}%
  \def\symdef@type{}%
  \def\symdef@args{}%
  \def\symdef@local{}%
  \metasetkeys{symdef}{#1}%
  \edef\symdef@tmp@optpars{[%
    \ifcsvoid{symdef@name}{}{name=\symdef@name,}%
    \ifcsvoid{symdef@type}{}{type={\noexpand\symdef@type},}%
    \ifcsvoid{symdef@args}{}{args={\symdef@args},}%
    \ifcsvoid{symdef@local}{}{local={\symdef@local}}%
  ]}%
  \ex\symdecl\symdef@tmp@optpars{#2}%
  \@insymdef@false%
  \notation[args=\symdef@args,%
    prec=\symdef@prec,%
    lang=\symdef@lang,%
    arity=\symdef@arity,%
    variant=\symdef@variant%
  ]{#2}%
}% mod@show
\def\symdef@type{Symbol}%
\providecommand{\stDMemph}[1]{\textbf{#1}}
%    \end{macrocode}
% \end{macro}
%
% \begin{macro}{\symvariant}
%   |\symvariant{|\meta{sym}|}[|\meta{args}|]{|\meta{var}|}{|\meta{cseq}|}| just extends
%   the internal macro |\modules@|\meta{sym}|@pres@| defined by
%   |\symdef{|\meta{sym}|}[|\meta{args}|]{|\ldots|}| with a variant
%   |\modules@|\meta{sym}|@pres@|\meta{var} which expands to \meta{cseq}. Recall that this
%   is called by the macro |\|\meta{sym}|[|\meta{var}|]| induced by the
%   |\symdef|.
%    \begin{macrocode}
\def\symvariant#1{%
  \@ifnextchar[{\@symvariant{#1}}{\@symvariant{#1}[0]}%
  }%
\def\@symvariant#1[#2]#3#4{%
  \notation[#3]{#1}[#2]{#4}%
%\ignorespacesandpars
}%
%    \end{macrocode}
% \end{macro}
%
%
% \begin{macro}{\@sym*}
%   has a starred form for primary symbols. The key/value interface has no effect on the
%   {\LaTeX} side. We read the to check whether only allowed ones are used.
%    \begin{macrocode}
\newif\if@importing\@importingfalse
\define@key{symi}{noverb}[all]{}%
\define@key{symi}{align}[WithTheSymbolOfTheSameName]{}%
\define@key{symi}{specializes}{}%
\define@key{symi}{gfc}{}%
\define@key{symi}{noalign}[true]{}%
\newcommand\symi{\@ifstar\@symi@star\@symi}
\newcommand\@symi[2][]{\metasetkeys{symi}{#1}%
  \parsemodule@maybesetcodes\if@importing\else\par\noindent Symbol: \textsf{#2}\fi%\ignorespacesandpars
  }
\newcommand\@symi@star[2][]{\metasetkeys{symi}{#1}%
  \parsemodule@maybesetcodes\if@importing\else\par\noindent Primary Symbol: \textsf{#2}\fi%\ignorespacesandpars
  }
\newcommand\symii{\@ifstar\@symii@star\@symii}
\newcommand\@symii[3][]{\metasetkeys{symi}{#1}%
  \parsemodule@maybesetcodes\if@importing\else\par\noindent Symbol: \textsf{#2-#3}\fi%\ignorespacesandpars
  }
\newcommand\@symii@star[3][]{\metasetkeys{symi}{#1}%
  \parsemodule@maybesetcodes\if@importing\else\par\noindent Primary Symbol: \textsf{#2-#3}\fi%\ignorespacesandpars
  }
\newcommand\symiii{\@ifstar\@symiii@star\@symiii}
\newcommand\@symiii[4][]{\metasetkeys{symi}{#1}%
  \parsemodule@maybesetcodes\if@importing\else\par\noindent Symbol: \textsf{#2-#3-#4}\fi%\ignorespacesandpars
  }
\newcommand\@symiii@star[4][]{\metasetkeys{symi}{#1}%
  \parsemodule@maybesetcodes\if@importing\else\par\noindent Primary Symbol: \textsf{#2-#3-#4}\fi%\ignorespacesandpars
  }
\newcommand\symiv{\@ifstar\@symiv@star\@symiv}
\newcommand\@symiv[5][]{\metasetkeys{symi}{#1}%
  \parsemodule@maybesetcodes\if@importing\else\par\noindent Symbol: \textsf{#2-#3-#4-#5}\fi%\ignorespacesandpars
  }
\newcommand\@symiv@star[5][]{\metasetkeys{symi}{#1}%
  \parsemodule@maybesetcodes\if@importing\else\par\noindent Primary Symbol: \textsf{#2-#3-#4-#5}\fi%\ignorespacesandpars
  }
%    \end{macrocode}
% \end{macro}
%
% \begin{macro}{\importmhmodule}
%   The |\importmhmodule[|\meta{key=value list}|]{module}| saves the current value of
%   |\mh@currentrepos| in a local macro |\mh@@repos|, resets |\mh@currentrepos| to the new
%   value if one is given in the optional argument, and after importing resets
%   |\mh@currentrepos| to the old value in |\mh@@repos|. We do all the |\ifx| comparison
%   with an |\expandafter|, since the values may be passed on from other key
%   bindings. Parameters will be passed to |\importmodule|.
%    \begin{macrocode}
%\srefaddidkey{importmhmodule}%
\addmetakey{importmhmodule}{mhrepos}%
\addmetakey{importmhmodule}{path}%
\addmetakey{importmhmodule}{ext}% why does this exist?
\addmetakey{importmhmodule}{dir}%
\addmetakey[false]{importmhmodule}{conservative}[true]%
\newcommand\importmhmodule[2][]{%
  \parsemodule@maybesetcodes
  \metasetkeys{importmhmodule}{#1}%
  \ifx\importmhmodule@dir\@empty%
    \edef\@path{\importmhmodule@path}%
  \else\edef\@path{\importmhmodule@dir/#2}\fi%
  \ifx\@path\@empty% if module name is not set
    \@importmodule[]{#2}{export}%
  \else%
    \edef\mh@@repos{\mh@currentrepos}% remember so that we can reset it.
    \ifx\importmhmodule@mhrepos\@empty% if in the same repos
      \relax% no need to change mh@currentrepos, i.e, current directory.
    \else%
      \mathhub@setcurrentreposinfo\importmhmodule@mhrepos% change it. 
      \addto@thismodulex{\noexpand\mathhub@setcurrentreposinfo{\importmhmodule@mhrepos}}%
    \fi%
    \@importmodule[\MathHub{\mh@currentrepos/source/\@path}]{#2}{export}%
    \mathhub@setcurrentreposinfo\mh@@repos% after importing, reset to old value
    \addto@thismodulex{\noexpand\mathhub@setcurrentreposinfo{\mh@@repos}}%
  \fi%
  %\ignorespacesandpars%
}
%    \end{macrocode}
% \end{macro}
%
% \begin{macro}{\usemhmodule}
%    \begin{macrocode}
\addmetakey{importmhmodule}{load}
\addmetakey{importmhmodule}{id}
\addmetakey{importmhmodule}{dir}
\addmetakey{importmhmodule}{mhrepos}

\addmetakey{importmodule}{load}
\addmetakey{importmodule}{id}

\newcommand\usemhmodule[2][]{%
\metasetkeys{importmhmodule}{#1}%
\ifx\importmhmodule@dir\@empty%
\edef\@path{\importmhmodule@path}%
\else\edef\@path{\importmhmodule@dir/#2}\fi%
\ifx\@path\@empty%
\usemodule[id=\importmhmodule@id]{#2}%
\else%
\edef\mh@@repos{\mh@currentrepos}%
\ifx\importmhmodule@mhrepos\@empty%
\else\mathhub@setcurrentreposinfo{\importmhmodule@mhrepos}\fi%
\usemodule{\@path\@QuestionMark#2}%
%\usemodule[load=\MathHub{\mh@currentrepos/source/\@path},
%                        id=\importmhmodule@id]{#2}%
\mathhub@setcurrentreposinfo\mh@@repos%
\fi%
%\ignorespacesandpars
}
%    \end{macrocode}
% \end{macro}
%
% \begin{macro}{\mhinputref}
%    \begin{macrocode}
\newcommand\mhinputref[2][]{%
  \edef\mhinputref@first{#1}%
  \ifx\mhinputref@first\@empty%
    \inputref{#2}%
  \else%
    \inputref[mhrepos=\mhinputref@first]{#2}%
  \fi%
}
%    \end{macrocode}
% \end{macro}
%
% \begin{macro}{\trefi*}
%    \begin{macrocode}
\newcommand\trefi[2][]{%
  \edef\trefi@mod{#1}%
  \ifx\trefi@mod\@empty\tref{#2}\else\tref{#1\@QuestionMark#2}\fi%
}
\newcommand\trefii[3][]{%
  \edef\trefi@mod{#1}%
  \ifx\trefi@mod\@empty\tref{#2-#3}\else\tref{#1\@QuestionMark#2-#3}\fi%
}
%    \end{macrocode}
% \end{macro}
% \begin{macro}{\defi*}
%    \begin{macrocode}
\def\defii#1#2{\defi{#1!#2}}
\def\Defii#1#2{\Defi{#1!#2}}
\def\defiis#1#2{\defis{#1!#2}}
\def\Defiis#1#2{\Defis{#1!#2}}
\def\defiii#1#2#3{\defi{#1!#2!#3}}
\def\Defiii#1#2#3{\Defi{#1!#2!#3}}
\def\defiiis#1#2#3{\defis{#1!#2!#3}}
\def\Defiiis#1#2#3{\Defis{#1!#2!#3}}
\def\defiv#1#2#3#4{\defi{#1!#2!#3!#4}}
\def\Defiv#1#2#3#4{\Defi{#1!#2!#3!#4}}
\def\defivs#1#2#3#4{\defis{#1!#2!#3!#4}}
\def\Defivs#1#2#3#4{\Defis{#1!#2!#3!#4}}
\def\adefi#1#2{\defi[name=#2]{#1}}
\def\adefii#1#2#3{\defi[name=#2-#3]{#1}}
\def\adefiii#1#2#3#4{\defi[name=#2-#3-#4]{#1}}
\def\adefiv#1#2#3#4#5{\defi[name=#2-#3-#4-#5]{#1}}
%    \end{macrocode}
% \end{macro}
%
%\testwrite{^^J\end{document}}
%\immediate\closeout\testoutfile
% \endinput