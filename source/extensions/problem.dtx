% \iffalse meta-comment
% An Infrastructure for Problems 
% Copyright (c) 2019 Michael Kohlhase, all rights reserved
%               this file is released under the
%               LaTeX Project Public License (LPPL)
% The original of this file is in the public repository at 
% http://github.com/sLaTeX/sTeX/
% \fi
% 
% \iffalse
%
%<*driver>
\def\bibfolder#1{../../lib/bib/#1}
\RequirePackage{paralist}
\ifcsname stexdocpath\endcsname\else\def\stexdocpath{.}\fi
\documentclass[full]{l3doc}
%\RequirePackage{document-structure}
\usepackage[hyperref=auto,style=alphabetic]{biblatex}
\usepackage[mathhub=\stexdocpath/mh,usesms]{stex}
\usepackage{stex-highlighting,stexthm}

\newif\ifhadtitle\hadtitlefalse

\def\fileversion{3.3.0}
\def\filedate{\today}
\def\stexdoctitle#1#2{\title{#1\thanks{Version {\fileversion} (last revised {\filedate})} }\def\thispkg{#2}}

\author{Michael Kohlhase, Dennis Müller\\
 	FAU Erlangen-Nürnberg\\
 	\url{http://kwarc.info/}
}

\def\stexmaketitle{\ifhadtitle\else\hadtitletrue\maketitle\fi}

\def\docmodule{
\begin{document}
  \EnableManual
  \DisableImplementation
  \DocInput{\jobname.dtx}
  \EnableImplementation
  \DisableDocumentation
  \DisableManual
  \DocInputAgain
  \clearpage
  \PrintIndex
\end{document}
}

\ExplSyntaxOn
  \bool_new:N \g_stexdoc_typeset_manual_bool
  \NewDocumentCommand \EnableManual {}{
    \bool_gset_true:N \g_stexdoc_typeset_manual_bool
  }
  \NewDocumentCommand \DisableManual {}{
    \bool_gset_false:N \g_stexdoc_typeset_manual_bool
  }
  \NewDocumentEnvironment {stexmanual} {} {
    \bool_if:NTF \g_stexdoc_typeset_manual_bool
      {\bool_set_false:N \l__codedoc_in_implementation_bool}
      {\comment}
  }{
    \bool_if:NF \g_stexdoc_typeset_manual_bool {\endcomment}
  }
\ExplSyntaxOff

%\usepackage{makeidx}
%\makeindex

%\usepackage{document-structure}
\ExplSyntaxOn
\int_new:N \l_stex_docheader_sect

\cs_new_protected:Nn \stexdoc_do_section:n {
  \int_case:nnF \l_stex_docheader_sect {
    {0}{\cs_if_exist:NTF \part {\part{#1}}{
      \int_incr:N \l_stex_docheader_sect
      \stexdoc_do_section:
    }}
    {1}{\cs_if_exist:NTF \chapter {\chapter{#1}}{
      \int_incr:N \l_stex_docheader_sect
      \stexdoc_do_section:
    }}
    {2}{\section{#1}}
    {3}{\subsection{#1}}
    {4}{\subsubsection{#1}}
  }{\paragraph{#1}}
  \int_incr:N \l_stex_docheader_sect
}
%\int_incr:N \l_stex_docheader_sect
\NewDocumentEnvironment{sfragment}{m}{
  \stexdoc_do_section:n{#1}
}{}

\cs_set_nopar:Nn \_stexdoc_do_cs:Nn {
  \stex_debug:nn{here}{\tl_to_str:n{#1},~#2}
  \cs_if_exist:cTF{s\tl_to_str:n{#2}}{
    \cs_if_exist:cTF{s\tl_to_str:n{#2}name}{
    \symref{#2-sym}{#1{#2}}
    }{#1{#2}}
  }{
    #1{#2}
  }
}
\let\my_old_cs\cs
\protected\def\cs#1{
  \_stexdoc_do_cs:Nn \my_old_cs{#1}
}
\let\my_old_cmd\cmd
\protected\def\cmd#1{
  \_stexdoc_do_cs:Nn \my_old_cmd{#1}
}

\ExplSyntaxOff

\mhinput[sTeX/Documentation]{../lib/examples.tex}
\RequirePackage[hints,solutions,notes]{problem}

\begin{document}
  \DocInput{\jobname.dtx}
\end{document}
%</driver>
% \fi
% 
% \GetFileInfo{problem.sty}
% 
% \MakeShortVerb{\|}
%
% \title{\texttt{problem.sty}: An Infrastructure for formatting Problems\thanks{Version {\fileversion} (last revised
%        {\filedate})}}
%    \author{Michael Kohlhase\\
%            FAU Erlangen-N\"urnberg\\
%            \url{http://kwarc.info/kohlhase}}
% \maketitle
%
% \ifinfulldoc\else
% \begin{documentation}
% \begin{abstract}
% This is the documentation for the \pkg{problem} package.
% For a more high-level introduction, 
%  see \href{\basedocurl/manual.pdf}{the \sTeX Manual} or the
%  \href{\basedocurl/stex.pdf}{full \sTeX documentation}.
% 
%  The |problem| package supplies an infrastructure that allows specify problems (for
%  homework assignments, exams, or quizzes) and to reuse them efficiently in multiple
%  environments.
% \end{abstract}
%
% \tableofcontents
% \begin{sfragment}{User Interface}
% \begin{sfragment}[id=sec:intro]{Introduction}

The |problem| package supplies an infrastructure that allows specify problem.  Problems
are text fragments that come with auxiliary functions: hints, notes, and
solutions\footnote{for the moment multiple choice problems are not supported, but may
  well be in a future version}. Furthermore, we can specify how long the solution to a
given problem is estimated to take and how many points will be awarded for a perfect
solution.

Finally, the |problem| package facilitates the management of problems in small files,
so that problems can be re-used in multiple environment. 

\begin{sfragment}{Package Options}
  \begin{function}{solutions,notes,hints,gnotes,pts,min,boxed,test}
    The |problem| package takes the options |solutions| (should solutions be output?),
    |notes| (should the problem notes be presented?), |hints| (do we give the hints?),
    |gnotes| (do we show grading notes?), |pts| (do we display the points awarded for
    solving the problem?), |min| (do we display the estimated minutes for problem
    soling). If theses are specified, then the corresponding auxiliary parts of the
    problems are output, otherwise, they remain invisible.

    The |boxed| option specifies that problems should be formatted in framed boxes so that
    they are more visible in the text. Finally, the |test| option signifies that we are in
    a test situation, so this option does not show the solutions (of course), but leaves
    space for the students to solve them.
  \end{function}
\end{sfragment}

\begin{sfragment}[id=sec:user:probsols]{Problems and Solutions}


\begin{environment}{problem}
  The main environment provided by the |problem| package is (surprise surprise) the
  |problem| environment. It is used to mark up problems and exercises. The environment
  takes an optional KeyVal argument with the keys |id| as an identifier that can be
  reference later, |pts| for the points to be gained from this exercise in homework or
  quiz situations, |min| for the estimated minutes needed to solve the problem, and
  finally |title| for an informative title of the problem.
\end{environment}

\begin{latexcode}
\usepackage[solutions,hints,pts,min]{problem}
\begin{document}
  \begin{sproblem}[id=elefants,pts=10,min=2,title=Fitting Elefants,name=elefants]
    How many Elefants can you fit into a Volkswagen beetle?
\begin{hint}
  Think positively, this is simple!
\end{hint}
\begin{exnote}
  Justify your answer
\end{exnote}
\begin{solution}[for=elefants,height=3cm]
  Four, two in the front seats, and two in the back.
\begin{gnote}
  if they do not give the justification deduct 5 pts
\end{gnote}
\end{solution}
  \end{sproblem}
\end{document}
\end{latexcode}

The result of formatting this problem is
\begin{minipage}{.9\linewidth}
\begin{sproblem}[id=elefants.prob,title=Fitting Elefants,name=elefants2]
  How many Elefants can you fit into a Volkswagen beetle?
\begin{hint}
  Think positively, this is simple!
\end{hint}
\begin{exnote}
  Justify your answer
\end{exnote}
\smallskip\noindent\hrule\textbf{Solution:}
  Four, two in the front seats, and two in the back.
\hrule
\end{sproblem}
\end{minipage}

\begin{environment}{solution}
  The |solution| environment can be to specify a solution to a problem. If the package
  option |solutions| is set or |\solutionstrue| is set in the text, then the solution will
  be presented in the output. The |solution| environment takes an optional KeyVal argument
  with the keys |id| for an identifier that can be reference |for| to specify which
  problem this is a solution for, and |height| that allows to specify the amount of space
  to be left in test situations (i.e. if the |test| option is set in the |\usepackage|
  statement).
\end{environment}

\begin{environment}{hint,exnote,gnote}
  The |hint| and |exnote| environments can be used in a |problem| environment to give
  hints and to make notes that elaborate certain aspects of the problem.  The |gnote|
  (grading notes) environment can be used to document situtations that may arise in
  grading.
\end{environment}

\begin{function}{\startsolutions,\stopsolutions}
  Sometimes we would like to locally override the |solutions| option we have given to the
  package. To turn on solutions we use the |\startsolutions|, to turn them off,
  |\stopsolutions|. These two can be used at any point in the documents.
\end{function}

\begin{function}{\ifsolutions}
  Also, sometimes, we want content (e.g. in an exam with master solutions) conditional on
  whether solutions are shown. This can be done with the |\ifsolutions| conditional.
\end{function}
\end{sfragment}

\begin{sfragment}[id=sec:user:mcq]{Multiple Choice Blocks}

\begin{environment}{mcb}
  Multiple choice blocks can be formatted using the |mcb| environment, in which single
  choices are marked up with |\mcc| macro.
\end{environment}

\begin{function}{\mcc}
  |\mcc[|\meta{keyvals}|]{|\meta{text}|}| takes an optional key/value argument
  \meta{keyvals} for choice metadata and a required argument \meta{text} for the proposed
  answer text. The following keys are supported
  \begin{itemize}
  \item |T| for true answers, |F| for false ones,
  \item |Ttext| the verdict for true answers, |Ftext| for false ones, and
  \item |feedback| for a short feedback text given to the student.
  \end{itemize}
\end{function}

\begin{latexcode}
\begin{sproblem}[title=Functions,name=functions1]
  What is the keyword to introduce a function definition in python?
  \begin{mcb}
    \mcc[T]{def}
    \mcc[F,feedback=that is for C and C++]{function}
    \mcc[F,feedback=that is for Standard ML]{fun}
    \mcc[F,Ftext=Nooooooooo,feedback=that is for Java]{public static void}
  \end{mcb}
\end{sproblem}
\end{latexcode}
This results in 

\begin{sproblem}[title=Functions,name=functions2]
  What is the keyword to introduce a function definition in python?
  \begin{mcb}
    \mcc[T]{def}
    \mcc[F,feedback=that is for C and C++]{function}
    \mcc[F,feedback=that is for Standard ML]{fun}
    \mcc[F,Ftext=Nooooooooo,feedback=that is for Java]{public static void}
  \end{mcb}
\end{sproblem}
\solutionstrue\hrule
\begin{sproblem}[title=Functions,name=functions3]
  What is the keyword to introduce a function definition in python?
  \begin{mcb}
    \mcc[T]{def}
    \mcc[F,feedback=that is for C and C++]{function}
    \mcc[F,feedback=that is for Standard ML]{fun}
    \mcc[F,Ftext=Nooooooooo,feedback=that is for Java]{public static void}
  \end{mcb}
\end{sproblem}
\end{sfragment}

\begin{sfragment}[id=sec:user:includeproblem]{Including Problems}
  
\begin{function}{\includeproblem}
  The |\includeproblem| macro can be used to include a problem from another file. It takes
  an optional KeyVal argument and a second argument which is a path to the file containing
  the problem (the macro assumes that there is only one problem in the include file). The
  keys |title|, |min|, and |pts| specify the problem title, the estimated minutes for
  solving the problem and the points to be gained, and their values (if given) overwrite
  the ones specified in the |problem| environment in the included file.
\end{function}
\end{sfragment}

\begin{sfragment}[id=sec:user:reporting]{Reporting Metadata}
The sum of the points and estimated minutes (that we specified in the |pts| and |min|
keys to the |problem| environment or the |\includeproblem| macro) to the log file and
the screen after each run. This is useful in preparing exams, where we want to make sure
that the students can indeed solve the problems in an allotted time period.

The |\min| and |\pts| macros allow to specify (i.e. to print to the margin) the
distribution of time and reward to parts of a problem, if the |pts| and |pts| package
options are set. This allows to give students hints about the estimated time and the
points to be awarded.
\end{sfragment}

\begin{sfragment}[id=sec:limitations]{Limitations}

In this section we document known limitations. If you want to help alleviate them,
please feel free to contact the package author. Some of them are currently discussed in
the \sTeX GitHub repository~\cite{sTeX:github:on}. 
\begin{enumerate}
\item none reported yet
\end{enumerate}
\end{sfragment}
\end{sfragment}

%%% Local Variables:
%%% mode: latex
%%% TeX-master: "../stex-manual"
%%% End:

% \end{sfragment}
% \end{documentation}
% \fi
%
% \begin{implementation}
%
% \section{The Implementation}\label{sec:implementation}
% 
% \subsection{Package Options}\label{sec:impl:options}
% 
% The first step is to declare (a few) package options that handle whether certain
% information is printed or not. They all come with their own conditionals that are set by
% the options.
%
%    \begin{macrocode}
%<*package>
%<@@=problems>
\ProvidesExplPackage{problem}{2022/05/24}{3.1.0}{Semantic Markup for Problems}
\RequirePackage{l3keys2e,stex}

\keys_define:nn { problem / pkg }{
  notes     .default:n    = { true },
  notes     .bool_set:N   = \c_@@_notes_bool,
  gnotes    .default:n    = { true },
  gnotes    .bool_set:N   = \c_@@_gnotes_bool,
  hints     .default:n    = { true },
  hints     .bool_set:N   = \c_@@_hints_bool,
  solutions .default:n    = { true },
  solutions .bool_set:N   = \c_@@_solutions_bool,
  pts       .default:n    = { true },
  pts       .bool_set:N   = \c_@@_pts_bool,
  min       .default:n    = { true },
  min       .bool_set:N   = \c_@@_min_bool,
  boxed     .default:n    = { true },
  boxed     .bool_set:N   = \c_@@_boxed_bool,
  unknown   .code:n       = {}
}
\newif\ifsolutions

\ProcessKeysOptions{ problem / pkg }
\bool_if:NTF \c_@@_solutions_bool {
  \solutionstrue
}{
  \solutionsfalse
}
%    \end{macrocode}
%
% Then we make sure that the necessary packages are loaded (in the right versions).
%    \begin{macrocode}
\RequirePackage{comment}
%    \end{macrocode}
%
% The next package relies on the \LaTeX3 kernel, which \latexml only partially
% supports. As it is purely presentational, we only load it when the |boxed| option is
% given and we run {\latexml}.
%
%    \begin{macrocode}
\bool_if:NT \c_@@_boxed_bool { \RequirePackage{mdframed} }
%    \end{macrocode}
%
% \begin{macro}{\prob@*@kw}
%   For multilinguality, we define internal macros for keywords that can be specialized in
%   |*.ldf| files. 
%    \begin{macrocode}
\def\prob@problem@kw{Problem}
\def\prob@solution@kw{Solution}
\def\prob@hint@kw{Hint}
\def\prob@note@kw{Note}
\def\prob@gnote@kw{Grading}
\def\prob@pt@kw{pt}
\def\prob@min@kw{min}
%    \end{macrocode}
% \end{macro}
% 
% For the other languages, we set up triggers
%    \begin{macrocode}
\AddToHook{begindocument}{
  \ltx@ifpackageloaded{babel}{
      \makeatletter
      \clist_set:Nx \l_tmpa_clist {\bbl@loaded}
      \clist_if_in:NnT \l_tmpa_clist {ngerman}{
        \input{problem-ngerman.ldf}
      }
      \clist_if_in:NnT \l_tmpa_clist {finnish}{
        \input{problem-finnish.ldf}
      }
      \clist_if_in:NnT \l_tmpa_clist {french}{
        \input{problem-french.ldf}
      }
      \clist_if_in:NnT \l_tmpa_clist {russian}{
        \input{problem-russian.ldf}
      }
      \makeatother
  }{}
}
%    \end{macrocode}
%
% \subsection{Problems and Solutions}\label{sec:impl:probsols}
% 
% We now prepare the KeyVal support for problems. The key macros just set appropriate
% internal macros.
%
%    \begin{macrocode}
\keys_define:nn{ problem / problem }{
  id      .str_set_x:N  = \l_@@_prob_id_str,
  pts     .tl_set:N     = \l_@@_prob_pts_tl,
  min     .tl_set:N     = \l_@@_prob_min_tl,
  title   .tl_set:N     = \l_@@_prob_title_tl,
  type    .tl_set:N     = \l_@@_prob_type_tl,
  imports .tl_set:N     = \l_@@_prob_imports_tl,
  name    .str_set_x:N  = \l_@@_prob_name_str,
  refnum  .int_set:N    = \l_@@_prob_refnum_int
}
\cs_new_protected:Nn \_@@_prob_args:n {
  \str_clear:N \l_@@_prob_id_str
  \str_clear:N \l_@@_prob_name_str
  \tl_clear:N \l_@@_prob_pts_tl
  \tl_clear:N \l_@@_prob_min_tl
  \tl_clear:N \l_@@_prob_title_tl
  \tl_clear:N \l_@@_prob_type_tl
  \tl_clear:N \l_@@_prob_imports_tl
  \int_zero_new:N \l_@@_prob_refnum_int
  \keys_set:nn { problem / problem }{ #1 }
  \int_compare:nNnT \l_@@_prob_refnum_int = 0 {
    \let\l_@@_prob_refnum_int\undefined
  }
}
%    \end{macrocode}
%
% Then we set up a counter for problems.
% \begin{macro}{\numberproblemsin}
%    \begin{macrocode}
\newcounter{problem}[section]
\newcommand\numberproblemsin[1]{\@addtoreset{problem}{#1}}
%    \end{macrocode}
% \end{macro}
% 
% \begin{macro}{\prob@label}
%   We provide the macro |\prob@label| to redefine later to get context involved. 
%    \begin{macrocode}
\newcommand\prob@label[1]{\thesection.#1} 
%    \end{macrocode}
% \end{macro}
% 
% \begin{macro}{\prob@number}
%   We consolidate the problem number into a reusable internal macro
%    \begin{macrocode}
\newcommand\prob@number{
  \int_if_exist:NTF \l_@@_inclprob_refnum_int {
    \prob@label{\int_use:N \l_@@_inclprob_refnum_int }
  }{
    \int_if_exist:NTF \l_@@_prob_refnum_int {
      \prob@label{\int_use:N \l_@@_prob_refnum_int }
    }{
        \prob@label\theproblem
    }
  }
}
%    \end{macrocode}
% \end{macro}
% 
% \begin{macro}{\prob@title}
%   We consolidate the problem title into a reusable internal macro as well. |\prob@title|
%   takes three arguments the first is the fallback when no title is given at all, the
%   second and third go around the title, if one is given.
%    \begin{macrocode}
\newcommand\prob@title[3]{%
  \tl_if_exist:NTF \l_@@_inclprob_title_tl {
    #2 \l_@@_inclprob_title_tl #3
  }{
    \tl_if_exist:NTF \l_@@_prob_title_tl {
      #2 \l_@@_prob_title_tl #3
    }{
      #1
    }
  }
}
%    \end{macrocode}
% \end{macro}
% 
% With these the problem header is a one-liner
%
% \begin{macro}{\prob@heading}
%   We consolidate the problem header line into a separate internal macro that can be
%   reused in various settings. 
%    \begin{macrocode}
\def\prob@heading{
  {\prob@problem@kw}\ \prob@number\prob@title{~}{~(}{)\strut}
  %\sref@label@id{\prob@problem@kw~\prob@number}{}
}
%    \end{macrocode}
% \end{macro}
% 
% With this in place, we can now define the |problem| environment. It comes in two shapes,
% depending on whether we are in boxed mode or not. In both cases we increment the problem
% number and output the points and minutes (depending) on whether the respective options
% are set.
% \begin{environment}{sproblem}
%    \begin{macrocode}
\newenvironment{sproblem}[1][]{
  \_@@_prob_args:n{#1}%\sref@target%
  \@in@omtexttrue% we are in a statement (for inline definitions)
  \stepcounter{problem}\record@problem
  \def\current@section@level{\prob@problem@kw}

  \str_if_empty:NT \l_@@_prob_name_str {
    \seq_get_right:NN \g_stex_currentfile_seq \l_tmpa_str
    \seq_set_split:NnV \l_tmpa_seq . \l_tmpa_str
    \seq_get_left:NN \l_tmpa_seq \l_@@_prob_name_str
  }

  \stex_if_do_html:T{
    \tl_if_empty:NF \l_@@_prob_title_tl {
      \exp_args:No \stex_document_title:n \l_@@_prob_title_tl
    }
  }
  
  \exp_args:Nno\stex_module_setup:nn{type=problem}\l_@@_prob_name_str

  \stex_reactivate_macro:N \STEXexport
  \stex_reactivate_macro:N \importmodule
  \stex_reactivate_macro:N \symdecl
  \stex_reactivate_macro:N \notation
  \stex_reactivate_macro:N \symdef

  \stex_if_do_html:T{
    \begin{stex_annotate_env} {problem} {
      \l_stex_module_ns_str ? \l_stex_module_name_str
    }

    \stex_annotate_invisible:nnn{header}{} {
      \stex_annotate:nnn{language}{ \l_stex_module_lang_str }{}
      \stex_annotate:nnn{signature}{ \l_stex_module_sig_str }{}
      \str_if_eq:VnF \l_stex_module_meta_str {NONE} {
        \stex_annotate:nnn{metatheory}{ \l_stex_module_meta_str }{}
      }
    }
  }

  \stex_csl_to_imports:No \importmodule \l_@@_prob_imports_tl


  \tl_if_exist:NTF \l_@@_inclprob_type_tl {
    \tl_set_eq:NN \sproblemtype \l_@@_inclprob_type_tl
  }{
    \tl_set_eq:NN \sproblemtype \l_@@_prob_type_tl
  }
  \str_if_exist:NTF \l_@@_inclprob_id_str {
    \str_set_eq:NN \sproblemid \l_@@_inclprob_id_str
  }{
    \str_set_eq:NN \sproblemid \l_@@_prob_id_str
  }


  \stex_if_smsmode:F {
    \clist_set:No \l_tmpa_clist \sproblemtype
    \tl_clear:N \l_tmpa_tl
    \clist_map_inline:Nn \l_tmpa_clist {
      \tl_if_exist:cT {_@@_sproblem_##1_start:}{
        \tl_set:Nn \l_tmpa_tl {\use:c{_@@_sproblem_##1_start:}}
      }
    }
    \tl_if_empty:NTF \l_tmpa_tl {
      \_@@_sproblem_start:
    }{
      \l_tmpa_tl
    }
  }
  \stex_ref_new_doc_target:n \sproblemid
  \stex_smsmode_do:
}{
  \__stex_modules_end_module:
  \stex_if_smsmode:F{
    \clist_set:No \l_tmpa_clist \sproblemtype
    \tl_clear:N \l_tmpa_tl
    \clist_map_inline:Nn \l_tmpa_clist {
      \tl_if_exist:cT {_@@_sproblem_##1_end:}{
        \tl_set:Nn \l_tmpa_tl {\use:c{_@@_sproblem_##1_end:}}
      }
    }
    \tl_if_empty:NTF \l_tmpa_tl {
      \_@@_sproblem_end:
    }{
      \l_tmpa_tl
    }
  }
  \stex_if_do_html:T{
    \end{stex_annotate_env}
  }

  \smallskip
}

\seq_put_right:Nx\g_stex_smsmode_allowedenvs_seq{\tl_to_str:n{sproblem}}



\cs_new_protected:Nn \_@@_sproblem_start: {
  \par\noindent\textbf\prob@heading\show@pts\show@min\\\ignorespacesandpars
}
\cs_new_protected:Nn \_@@_sproblem_end: {\par\smallskip}

\newcommand\stexpatchproblem[3][] {
    \str_set:Nx \l_tmpa_str{ #1 }
    \str_if_empty:NTF \l_tmpa_str {
      \tl_set:Nn \_@@_sproblem_start: { #2 }
      \tl_set:Nn \_@@_sproblem_end: { #3 }
    }{
      \exp_after:wN \tl_set:Nn \csname _@@_sproblem_#1_start:\endcsname{ #2 }
      \exp_after:wN \tl_set:Nn \csname _@@_sproblem_#1_end:\endcsname{ #3 }
    }
}


\bool_if:NT \c_@@_boxed_bool {
  \surroundwithmdframed{problem}
}
%    \end{macrocode}
% \end{environment}
% 
% \begin{macro}{\record@problem}
%   This macro records information about the problems in the |*.aux| file. 
%    \begin{macrocode}
\def\record@problem{
  \protected@write\@auxout{}
  {
    \string\@problem{\prob@number}
    {
      \tl_if_exist:NTF \l_@@_inclprob_pts_tl {
        \l_@@_inclprob_pts_tl
      }{
        \l_@@_prob_pts_tl
      }
    }%
    {
      \tl_if_exist:NTF \l_@@_inclprob_min_tl {
        \l_@@_inclprob_min_tl
      }{
        \l_@@_prob_min_tl
      }
    }
  }
}
%    \end{macrocode}
% \end{macro}
%
% \begin{macro}{\@problem}
%   This macro acts on a problem's record in the |*.aux| file. It does not have any
%   functionality here, but can be redefined elsewhere (e.g. in the |assignment|
%   package). 
%    \begin{macrocode}
\def\@problem#1#2#3{}
%    \end{macrocode}
% \end{macro}
% 
% The \DescribeEnv{solution}|solution| environment is similar to the |problem|
% environment, only that it is independent of the boxed mode. It also has it's own keys
% that we need to define first.
% 
%    \begin{macrocode}
\keys_define:nn { problem / solution }{
  id            .str_set_x:N  = \l_@@_solution_id_str ,
  for           .tl_set:N     = \l_@@_solution_for_tl ,
  height        .dim_set:N    = \l_@@_solution_height_dim ,
  creators      .clist_set:N  = \l_@@_solution_creators_clist ,
  contributors  .clist_set:N  = \l_@@_solution_contributors_clist ,
  srccite       .tl_set:N     = \l_@@_solution_srccite_tl
}
\cs_new_protected:Nn \_@@_solution_args:n {
  \str_clear:N \l_@@_solution_id_str
  \tl_clear:N \l_@@_solution_for_tl
  \tl_clear:N \l_@@_solution_srccite_tl
  \clist_clear:N \l_@@_solution_creators_clist
  \clist_clear:N \l_@@_solution_contributors_clist
  \dim_zero:N \l_@@_solution_height_dim
  \keys_set:nn { problem / solution }{ #1 }
}
%    \end{macrocode}
% the next step is to define a helper macro that does what is needed to start a solution. 
%    \begin{macrocode}
\newcommand\@startsolution[1][]{
  \_@@_solution_args:n { #1 }
  \@in@omtexttrue% we are in a statement.
  \bool_if:NF \c_@@_boxed_bool { \hrule }
  \smallskip\noindent
  {\textbf\prob@solution@kw :\enspace}
  \begin{small}
  \def\current@section@level{\prob@solution@kw}
  \ignorespacesandpars
}
%    \end{macrocode}
%
% \begin{macro}{\startsolutions}
% for the |\startsolutions| macro we use the |\specialcomment| macro from the |comment|
% package. Note that we use the |\@startsolution| macro in the start codes, that parses
% the optional argument. 
%    \begin{macrocode}
\box_new:N \l_@@_solution_box
\newenvironment{solution}[1][]{
  \stex_html_backend:TF{
    \stex_if_do_html:T{
      \begin{stex_annotate_env}{solution}{}
    }
  }{
    \setbox\l_@@_solution_box\vbox\bgroup
      \par\smallskip\hrule\smallskip
      \noindent\textbf{Solution:}~
  }
}{
  \stex_html_backend:TF{
    \stex_if_do_html:T{
      \end{stex_annotate_env}
    }
  }{
    \smallskip\hrule
    \egroup
    \bool_if:NT \c_@@_solutions_bool {
      \box\l_@@_solution_box
    }
  }
}

\newcommand\startsolutions{
  \bool_set_true:N \c_@@_solutions_bool
%  \specialcomment{solution}{\@startsolution}{
%    \bool_if:NF \c_@@_boxed_bool {
%      \hrule\medskip
%    }
%    \end{small}%
%  }
%  \bool_if:NT \c_@@_boxed_bool {
%    \surroundwithmdframed{solution}
%  }
}
%    \end{macrocode}
% \end{macro}
% 
% \begin{macro}{\stopsolutions}
%    \begin{macrocode}
\newcommand\stopsolutions{\bool_set_false:N \c_@@_solutions_bool}%\excludecomment{solution}}
%    \end{macrocode}
% \end{macro}
% 
% so it only remains to start/stop solutions depending on what option was specified.
%
%    \begin{macrocode}
\ifsolutions
  \startsolutions
\else
  \stopsolutions
\fi
%    \end{macrocode}
%
% \begin{environment}{exnote}
%    \begin{macrocode}
\bool_if:NTF \c_@@_notes_bool {
  \newenvironment{exnote}[1][]{
    \par\smallskip\hrule\smallskip
    \noindent\textbf{\prob@note@kw :~ }\small
  }{
    \smallskip\hrule
  }
}{
  \excludecomment{exnote}
}
%    \end{macrocode}
% \end{environment}
% 
% \begin{environment}{hint}
%    \begin{macrocode}
\bool_if:NTF \c_@@_notes_bool {
  \newenvironment{hint}[1][]{
    \par\smallskip\hrule\smallskip
    \noindent\textbf{\prob@hint@kw :~ }\small
  }{
    \smallskip\hrule
  }
  \newenvironment{exhint}[1][]{
    \par\smallskip\hrule\smallskip
    \noindent\textbf{\prob@hint@kw :~ }\small
  }{
    \smallskip\hrule
  }
}{
  \excludecomment{hint}
  \excludecomment{exhint}
}
%    \end{macrocode}
% \end{environment}
% 
% \begin{environment}{gnote}
%    \begin{macrocode}
\bool_if:NTF \c_@@_notes_bool {
  \newenvironment{gnote}[1][]{
    \par\smallskip\hrule\smallskip
    \noindent\textbf{\prob@gnote@kw :~ }\small
  }{
    \smallskip\hrule
  }
}{
  \excludecomment{gnote}
}
%    \end{macrocode}
% \end{environment}
% 
% \subsection{Multiple Choice Blocks}\label{sec:impl:mcq}
%
% \begin{environment}{mcb}
%   \ednote{MK: maybe import something better here from a dedicated MC package}
%    \begin{macrocode}
\newenvironment{mcb}{
  \begin{enumerate}
}{
  \end{enumerate}
}
%    \end{macrocode}
% \end{environment}
% we define the keys for the |mcc| macro
%    \begin{macrocode} 
\cs_new_protected:Nn \_@@_do_yes_param:Nn {
  \exp_args:Nx \str_if_eq:nnTF { \str_lowercase:n{ #2 } }{ yes }{
    \bool_set_true:N #1
  }{
    \bool_set_false:N #1
  }
}
\keys_define:nn { problem / mcc }{
  id        .str_set_x:N  = \l_@@_mcc_id_str ,
  feedback  .tl_set:N     = \l_@@_mcc_feedback_tl ,
  T         .default:n    = { false } ,
  T         .bool_set:N   = \l_@@_mcc_t_bool ,
  F         .default:n    = { false } ,
  F         .bool_set:N   = \l_@@_mcc_f_bool ,
  Ttext     .tl_set:N     = \l_@@_mcc_Ttext_str ,
  Ftext     .tl_set:N     = \l_@@_mcc_Ftext_str
}
\cs_new_protected:Nn \l_@@_mcc_args:n {
  \str_clear:N \l_@@_mcc_id_str
  \tl_clear:N \l_@@_mcc_feedback_tl
  \bool_set_false:N \l_@@_mcc_t_bool
  \bool_set_false:N \l_@@_mcc_f_bool
  \tl_clear:N \l_@@_mcc_Ttext_tl
  \tl_clear:N \l_@@_mcc_Ftext_tl
  \str_clear:N \l_@@_mcc_id_str
  \keys_set:nn { problem / mcc }{ #1 }
}
%    \end{macrocode}
%
% \begin{macro}{\mcc}
%    \begin{macrocode}
\def\mccTrueText{\textbf{(true)~}}
\def\mccFalseText{\textbf{(false)~}}
\newcommand\mcc[2][]{
  \l_@@_mcc_args:n{ #1 }
  \item[$\Box$] #2
  \ifsolutions
    \\
    \bool_if:NT \l_@@_mcc_t_bool {
      \tl_if_empty:NTF\l_@@_mcc_Ttext_tl\mccTrueText\l_@@_mcc_Ttext_tl
    }
    \bool_if:NT \l_@@_mcc_f_bool {
      \tl_if_empty:NTF\l_@@_mcc_Ttext_tl\mccFalseText\l_@@_mcc_Ftext_tl
    }
    \tl_if_empty:NF \l_@@_mcc_feedback_tl {
      \emph{(\l_@@_mcc_feedback_tl)}
    }
  \fi
} %solutions
%    \end{macrocode}
% \end{macro}
% 
% \subsection{Including Problems}\label{sec:impl:includeproblem}
%
% \begin{macro}{\includeproblem}
%   The |\includeproblem| command is essentially a glorified |\input| statement, it sets
%   some internal macros first that overwrite the local points. Importantly, it resets the
%   |inclprob| keys after the input. 
%    \begin{macrocode}

\keys_define:nn{ problem / inclproblem }{
  id      .str_set_x:N  = \l_@@_inclprob_id_str,
  pts     .tl_set:N     = \l_@@_inclprob_pts_tl,
  min     .tl_set:N     = \l_@@_inclprob_min_tl,
  title   .tl_set:N     = \l_@@_inclprob_title_tl,
  refnum  .int_set:N    = \l_@@_inclprob_refnum_int,
  type    .tl_set:N     = \l_@@_inclprob_type_tl,
  mhrepos .str_set_x:N  = \l_@@_inclprob_mhrepos_str
}
\cs_new_protected:Nn \_@@_inclprob_args:n {
  \str_clear:N \l_@@_prob_id_str
  \tl_clear:N \l_@@_inclprob_pts_tl
  \tl_clear:N \l_@@_inclprob_min_tl
  \tl_clear:N \l_@@_inclprob_title_tl
  \tl_clear:N \l_@@_inclprob_type_tl
  \int_zero_new:N \l_@@_inclprob_refnum_int
  \str_clear:N \l_@@_inclprob_mhrepos_str
  \keys_set:nn { problem / inclproblem }{ #1 }
  \tl_if_empty:NT \l_@@_inclprob_pts_tl {
    \let\l_@@_inclprob_pts_tl\undefined
  }
  \tl_if_empty:NT \l_@@_inclprob_min_tl {
    \let\l_@@_inclprob_min_tl\undefined
  }
  \tl_if_empty:NT \l_@@_inclprob_title_tl {
    \let\l_@@_inclprob_title_tl\undefined
  }
  \tl_if_empty:NT \l_@@_inclprob_type_tl {
    \let\l_@@_inclprob_type_tl\undefined
  }
  \int_compare:nNnT \l_@@_inclprob_refnum_int = 0 {
    \let\l_@@_inclprob_refnum_int\undefined
  }
}

\cs_new_protected:Nn \_@@_inclprob_clear: {
  \let\l_@@_inclprob_id_str\undefined
  \let\l_@@_inclprob_pts_tl\undefined
  \let\l_@@_inclprob_min_tl\undefined
  \let\l_@@_inclprob_title_tl\undefined
  \let\l_@@_inclprob_type_tl\undefined
  \let\l_@@_inclprob_refnum_int\undefined
  \let\l_@@_inclprob_mhrepos_str\undefined
}
\_@@_inclprob_clear:

\newcommand\includeproblem[2][]{
  \_@@_inclprob_args:n{ #1 }
  \exp_args:No \stex_in_repository:nn\l_@@_inclprob_mhrepos_str{
    \stex_html_backend:TF {
      \str_clear:N \l_tmpa_str
      \prop_get:NnNF \l_stex_current_repository_prop { narr } \l_tmpa_str {
        \prop_get:NnNF \l_stex_current_repository_prop { ns } \l_tmpa_str {}
      }
      \stex_annotate_invisible:nnn{includeproblem}{
        \l_tmpa_str / #2
      }{}
    }{
      \begingroup
        \inputreftrue
        \tl_if_empty:nTF{ ##1 }{
          \input{#2}
        }{
          \input{ \c_stex_mathhub_str / ##1 / source / #2 }
        }
      \endgroup
    }
  }
  \_@@_inclprob_clear:
}
%    \end{macrocode}
% \end{macro}
%
% \subsection{Reporting Metadata}
%
% For messages it is OK to have them in English as the whole documentation is, and we can
% therefore assume authors can deal with it. 
%
%    \begin{macrocode}
\AddToHook{enddocument}{
  \bool_if:NT \c_@@_pts_bool {
    \message{Total:~\arabic{pts}~points}
  }
  \bool_if:NT \c_@@_min_bool {
    \message{Total:~\arabic{min}~minutes}
  }
}
%    \end{macrocode}
%
% The margin pars are reader-visible, so we need to translate
% 
%    \begin{macrocode}
\def\pts#1{
  \bool_if:NT \c_@@_pts_bool {
    \marginpar{#1~\prob@pt@kw}
  }
}
\def\min#1{
  \bool_if:NT \c_@@_min_bool {
    \marginpar{#1~\prob@min@kw}
  }
}
%    \end{macrocode}
%
% \begin{macro}{\show@pts}
%   The |\show@pts| shows the points: if no points are given from the outside and also no
%   points are given locally do nothing, else show and add. If there are outside points
%   then we show them in the margin.
%    \begin{macrocode}
\newcounter{pts}
\def\show@pts{
  \tl_if_exist:NTF \l_@@_inclprob_pts_tl {
    \bool_if:NT \c_@@_pts_bool {
      \marginpar{\l_@@_inclprob_pts_tl\ \prob@pt@kw\smallskip}
      \addtocounter{pts}{\l_@@_inclprob_pts_tl}
    }
  }{
    \tl_if_exist:NT \l_@@_prob_pts_tl {
      \bool_if:NT \c_@@_pts_bool {
        \tl_if_empty:NT\l_@@_prob_pts_tl{
          \tl_set:Nn \l_@@_prob_pts_tl {0}
        }
        \marginpar{\l_@@_prob_pts_tl\ \prob@pt@kw\smallskip}
        \addtocounter{pts}{\l_@@_prob_pts_tl}
      }
    }
  }
}
%    \end{macrocode}
% \end{macro}
% and now the same for the minutes
% \begin{macro}{\show@min}
%    \begin{macrocode}
\newcounter{min}
\def\show@min{
  \tl_if_exist:NTF \l_@@_inclprob_min_tl {
    \bool_if:NT \c_@@_min_bool {
      \marginpar{\l_@@_inclprob_pts_tl\ min}
      \addtocounter{min}{\l_@@_inclprob_min_tl}
    }
  }{
    \tl_if_exist:NT \l_@@_prob_min_tl {
      \bool_if:NT \c_@@_min_bool {
        \tl_if_empty:NT\l_@@_prob_min_tl{
          \tl_set:Nn \l_@@_prob_min_tl {0}
        }
        \marginpar{\l_@@_prob_min_tl\ min}
        \addtocounter{min}{\l_@@_prob_min_tl}
      }
    }
  }
}
%</package>
%    \end{macrocode}
% \end{macro}
% \end{implementation}
% \ifinfulldoc\else\printbibliography\fi
\endinput
% LocalWords:  GPL structuresharing STR dtx pts keyval xcomment CPERL DefKeyVal iffalse
%%% Local Variables: 
%%% mode: doctex
%%% TeX-master: t
%%% End: 
% \fi
% LocalWords:  RequirePackage Semiverbatim DefEnvironment OptionalKeyVals soln texttt baz
% LocalWords:  exnote DefConstructor inclprob NeedsTeXFormat omd.sty textbackslash exfig
%  LocalWords:  stopsolution fileversion filedate maketitle setcounter tocdepth newpage
%  LocalWords:  tableofcontents showmeta showmeta solutionstrue usepackage minipage hrule
%  LocalWords:  linewidth elefants.prob Elefants smallskip noindent textbf startsolutions
%  LocalWords:  startsolutions stopsolutions stopsolutions includeproblem includeproblem
%  LocalWords:  textsf HorIacJuc cscpnrr11 includemhproblem includemhproblem importmodule
%  LocalWords:  importmhmodule foobar ldots latexml mhcurrentrepos mh-variants mh-variant
%  LocalWords:  compactenum langle rangle langle rangle ltxml metakeys newif ifexnotes rm
%  LocalWords:  exnotesfalse exnotestrue ifhints hintsfalse hintstrue ifsolutions ifpts
%  LocalWords:  solutionsfalse ptsfalse ptstrue ifmin minfalse mintrue ifboxed boxedfalse
%  LocalWords:  boxedtrue sref mdframed marginpar prob srefaddidkey addmetakey refnum kv
%  LocalWords:  newcounter ifx thesection theproblem hfill newenvironment metasetkeys ltx
%  LocalWords:  stepcounter currentsectionlevel xspace ignorespaces surroundwithmdframed
%  LocalWords:  omdoc autoopen autoclose solvedinminutes kvi qw vals newcommand exhint
%  LocalWords:  specialcomment excludecomment mhrepos xref marginpar addtocounter doctex
%  LocalWords:  mh@currentrepos endinput

