% \iffalse meta-comment
% An Infrastructure for Semantic Macros and Module Scoping
% Copyright (c) 2019 Michael Kohlhase, all rights reserved
%                this file is released under the
%                LaTeX Project Public License (LPPL)
% 
% The original of this file is in the public repository at 
% http://github.com/sLaTeX/sTeX/
%
% TODO update copyright  
%
%<*driver>
\def\bibfolder#1{../../lib/bib/#1}
\RequirePackage{paralist}
\ifcsname stexdocpath\endcsname\else\def\stexdocpath{.}\fi
\documentclass[full]{l3doc}
%\RequirePackage{document-structure}
\usepackage[hyperref=auto,style=alphabetic]{biblatex}
\usepackage[mathhub=\stexdocpath/mh,usesms]{stex}
\usepackage{stex-highlighting,stexthm}

\newif\ifhadtitle\hadtitlefalse

\def\fileversion{3.3.0}
\def\filedate{\today}
\def\stexdoctitle#1#2{\title{#1\thanks{Version {\fileversion} (last revised {\filedate})} }\def\thispkg{#2}}

\author{Michael Kohlhase, Dennis Müller\\
 	FAU Erlangen-Nürnberg\\
 	\url{http://kwarc.info/}
}

\def\stexmaketitle{\ifhadtitle\else\hadtitletrue\maketitle\fi}

\def\docmodule{
\begin{document}
  \EnableManual
  \DisableImplementation
  \DocInput{\jobname.dtx}
  \EnableImplementation
  \DisableDocumentation
  \DisableManual
  \DocInputAgain
  \clearpage
  \PrintIndex
\end{document}
}

\ExplSyntaxOn
  \bool_new:N \g_stexdoc_typeset_manual_bool
  \NewDocumentCommand \EnableManual {}{
    \bool_gset_true:N \g_stexdoc_typeset_manual_bool
  }
  \NewDocumentCommand \DisableManual {}{
    \bool_gset_false:N \g_stexdoc_typeset_manual_bool
  }
  \NewDocumentEnvironment {stexmanual} {} {
    \bool_if:NTF \g_stexdoc_typeset_manual_bool
      {\bool_set_false:N \l__codedoc_in_implementation_bool}
      {\comment}
  }{
    \bool_if:NF \g_stexdoc_typeset_manual_bool {\endcomment}
  }
\ExplSyntaxOff

%\usepackage{makeidx}
%\makeindex

%\usepackage{document-structure}
\ExplSyntaxOn
\int_new:N \l_stex_docheader_sect

\cs_new_protected:Nn \stexdoc_do_section:n {
  \int_case:nnF \l_stex_docheader_sect {
    {0}{\cs_if_exist:NTF \part {\part{#1}}{
      \int_incr:N \l_stex_docheader_sect
      \stexdoc_do_section:
    }}
    {1}{\cs_if_exist:NTF \chapter {\chapter{#1}}{
      \int_incr:N \l_stex_docheader_sect
      \stexdoc_do_section:
    }}
    {2}{\section{#1}}
    {3}{\subsection{#1}}
    {4}{\subsubsection{#1}}
  }{\paragraph{#1}}
  \int_incr:N \l_stex_docheader_sect
}
%\int_incr:N \l_stex_docheader_sect
\NewDocumentEnvironment{sfragment}{m}{
  \stexdoc_do_section:n{#1}
}{}

\cs_set_nopar:Nn \_stexdoc_do_cs:Nn {
  \stex_debug:nn{here}{\tl_to_str:n{#1},~#2}
  \cs_if_exist:cTF{s\tl_to_str:n{#2}}{
    \cs_if_exist:cTF{s\tl_to_str:n{#2}name}{
    \symref{#2-sym}{#1{#2}}
    }{#1{#2}}
  }{
    #1{#2}
  }
}
\let\my_old_cs\cs
\protected\def\cs#1{
  \_stexdoc_do_cs:Nn \my_old_cs{#1}
}
\let\my_old_cmd\cmd
\protected\def\cmd#1{
  \_stexdoc_do_cs:Nn \my_old_cmd{#1}
}

\ExplSyntaxOff

\mhinput[sTeX/Documentation]{../lib/examples.tex}

\begin{document}
  \DocInput{\jobname.dtx}
\end{document}
%</driver>
% \fi
%
% \title{Tikzinput: Treating TIKZ code as images
% 	\thanks{Version {\fileversion} (last revised {\filedate})} 
% }
%
% \author{Michael Kohlhase, Dennis Müller\\
% 	FAU Erlangen-Nürnberg\\
% 	\url{http://kwarc.info/}
% }
%
% \maketitle
%
% \ifinfulldoc\else
% \begin{abstract}
%   This is the documentation for the \pkg{tikzinput} package.  For a more high-level
%   introduction, see \href{\basedocurl/manual.pdf}{the \sTeX Manual} or the
%   \href{\basedocurl/stex.pdf}{full \sTeX documentation}.
% 
%   In some situations it is more efficient externalize the TIKZ pictures into separate
%   (standalone) files, to let {\LaTeX} handle the TIKZ pictures to generate an image, and
%   just load it via the usual {\LaTeX} graphics packages. The |tikzinput| package
%   supports this workflow, and allows to switch back to native TIKZ via a package option.
% \end{abstract}
%
% \tableofcontents
% 
% \textcolor{red}{TODO: tikzinput documentation}
% \fi
%
% \begin{documentation}\label{pkg:tikzinput:doc}
%
%
% \section{Macros and Environments}\label{pkg:tikzinput:doc:macros}
%
% \end{documentation}
%
% \begin{implementation}\label{pkg:tikzinput:impl}
%
% \section{Tikzinput Implementation}
%
%    \begin{macrocode}
%<@@=tikzinput>
%<*package>

%%%%%%%%%%%%%   tikzinput.dtx   %%%%%%%%%%%%%

%    \end{macrocode}
%
%    \begin{macrocode}
\ProvidesExplPackage{tikzinput}{2022/05/24}{3.1.0}{tikzinput package}
\RequirePackage{l3keys2e}

\keys_define:nn { tikzinput } {
  image   .bool_set:N   = \c_tikzinput_image_bool,
  image   .default:n    = false ,
  unknown   .code:n       = {}
}

\ProcessKeysOptions { tikzinput }

\bool_if:NTF \c_tikzinput_image_bool {
  \RequirePackage{graphicx}

  \providecommand\usetikzlibrary[]{}
  \newcommand\tikzinput[2][]{\includegraphics[#1]{#2}}
}{
  \RequirePackage{tikz}
  \RequirePackage{standalone}

  \newcommand \tikzinput [2] [] {
    \setkeys{Gin}{#1}
    \ifx \Gin@ewidth \Gin@exclamation
      \ifx \Gin@eheight \Gin@exclamation
        \input { #2 }
      \else
        \resizebox{!}{ \Gin@eheight }{
          \input { #2 }
        }
      \fi
    \else
      \ifx \Gin@eheight \Gin@exclamation
        \resizebox{ \Gin@ewidth }{!}{
          \input { #2 }
        }
      \else
        \resizebox{ \Gin@ewidth }{ \Gin@eheight }{
          \input { #2 }
        }
      \fi
    \fi
  }
}

\newcommand \ctikzinput [2] [] {
  \begin{center}
    \tikzinput [#1] {#2}
  \end{center}
}

\@ifpackageloaded{stex}{
  \RequirePackage{stex-tikzinput}
}{}

%    \end{macrocode}
%
%
%    \begin{macrocode}
%</package>
%<*stex>
%    \end{macrocode}
%
%    \begin{macrocode}
\ProvidesExplPackage{stex-tikzinput}{2022/05/24}{3.1.0}{stex-tikzinput}
\RequirePackage{stex}
\RequirePackage{tikzinput}

\newcommand\mhtikzinput[2][]{%
  \def\Gin@mhrepos{}\setkeys{Gin}{#1}%
  \stex_in_repository:nn\Gin@mhrepos{
    \tikzinput[#1]{\mhpath{##1}{#2}}
  }
}
\newcommand\cmhtikzinput[2][]{\begin{center}\mhtikzinput[#1]{#2}\end{center}}

\cs_new_protected:Nn \_@@_usetikzlibrary:nn {
  \pgfkeys@spdef\pgf@temp{#1}
  \expandafter\ifx\csname tikz@library@\pgf@temp @loaded\endcsname\relax%
  \expandafter\global\expandafter\let\csname tikz@library@\pgf@temp @loaded\endcsname=\pgfutil@empty%
  \expandafter\edef\csname tikz@library@#1@atcode\endcsname{\the\catcode`\@}
  \expandafter\edef\csname tikz@library@#1@barcode\endcsname{\the\catcode`\|}
  \expandafter\edef\csname tikz@library@#1@dollarcode\endcsname{\the\catcode`\$}
  \catcode`\@=11
  \catcode`\|=12
  \catcode`\$=3
  \pgfutil@InputIfFileExists{#2}{}{}
  \catcode`\@=\csname tikz@library@#1@atcode\endcsname
  \catcode`\|=\csname tikz@library@#1@barcode\endcsname
  \catcode`\$=\csname tikz@library@#1@dollarcode\endcsname
}


\newcommand\libusetikzlibrary[1]{
  \prop_if_exist:NF \l_stex_current_repository_prop {
    \msg_error:nnn{stex}{error/notinarchive}\libusetikzlibrary
  } 
  \prop_get:NnNF \l_stex_current_repository_prop {id} \l_tmpa_str {
    \msg_error:nnn{stex}{error/notinarchive}\libusetikzlibrary
  }
  \seq_clear:N \l_@@_libinput_files_seq
  \seq_set_eq:NN \l_tmpa_seq \c_stex_mathhub_seq
  \seq_set_split:NnV \l_tmpb_seq / \l_tmpa_str

  \bool_while_do:nn { ! \seq_if_empty_p:N \l_tmpb_seq }{
    \str_set:Nx \l_tmpa_str {\stex_path_to_string:N \l_tmpa_seq / meta-inf / lib / tikzlibrary #1 .code.tex}
    \IfFileExists{ \l_tmpa_str }{
      \seq_put_right:No \l_@@_libinput_files_seq \l_tmpa_str
    }{}
    \seq_pop_left:NN \l_tmpb_seq \l_tmpa_str
    \seq_put_right:No \l_tmpa_seq \l_tmpa_str
  }

  \str_set:Nx \l_tmpa_str {\stex_path_to_string:N \l_tmpa_seq / lib / tikzlibrary #1 .code.tex}
  \IfFileExists{ \l_tmpa_str }{
    \seq_put_right:No \l_@@_libinput_files_seq \l_tmpa_str
  }{}

  \seq_if_empty:NTF \l_@@_libinput_files_seq {
    \msg_error:nnxx{stex}{error/nofile}{\exp_not:N\libusetikzlibrary}{tikzlibrary #1 .code.tex}
  }{
    \int_compare:nNnTF {\seq_count:N \l_@@_libinput_files_seq} = 1 {
      \seq_map_inline:Nn \l_@@_libinput_files_seq {
        \_@@_usetikzlibrary:nn{#1}{ ##1 }
      }
    }{
      \msg_error:nnxx{stex}{error/twofiles}{\exp_not:N\libusetikzlibrary}{tikzlibrary #1 .code.tex}
    }
  }
}
%    \end{macrocode}
%
%    \begin{macrocode}
%</stex>
%    \end{macrocode}
%
% \end{implementation}
% \ifinfulldoc\else\printbibliography\fi
%
% \PrintIndex

%  LocalWords:  bibfolder jobname.dtx tikzinput.dtx usetikzlibrary Gin@ewidth Gin@eheight
%  LocalWords:  resizebox ctikzinput mhtikzinput Gin@mhrepos mhpath
