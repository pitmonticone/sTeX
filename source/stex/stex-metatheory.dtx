% \iffalse meta-comment
% An Infrastructure for Semantic Macros and Module Scoping
% Copyright (c) 2019 Michael Kohlhase, all rights reserved
%                this file is released under the
%                LaTeX Project Public License (LPPL)
% 
% The original of this file is in the public repository at 
% http://github.com/sLaTeX/sTeX/
%
% TODO update copyright  
%
%<*driver>
\def\libfolder#1{../../lib/#1}
\RequirePackage{paralist}
\ifcsname stexdocpath\endcsname\else\def\stexdocpath{.}\fi
\documentclass[full]{l3doc}
%\RequirePackage{document-structure}
\usepackage[hyperref=auto,style=alphabetic]{biblatex}
\usepackage[mathhub=\stexdocpath/mh,usesms]{stex}
\usepackage{stex-highlighting,stexthm}

\newif\ifhadtitle\hadtitlefalse

\def\fileversion{3.3.0}
\def\filedate{\today}
\def\stexdoctitle#1#2{\title{#1\thanks{Version {\fileversion} (last revised {\filedate})} }\def\thispkg{#2}}

\author{Michael Kohlhase, Dennis Müller\\
 	FAU Erlangen-Nürnberg\\
 	\url{http://kwarc.info/}
}

\def\stexmaketitle{\ifhadtitle\else\hadtitletrue\maketitle\fi}

\def\docmodule{
\begin{document}
  \EnableManual
  \DisableImplementation
  \DocInput{\jobname.dtx}
  \EnableImplementation
  \DisableDocumentation
  \DisableManual
  \DocInputAgain
  \clearpage
  \PrintIndex
\end{document}
}

\ExplSyntaxOn
  \bool_new:N \g_stexdoc_typeset_manual_bool
  \NewDocumentCommand \EnableManual {}{
    \bool_gset_true:N \g_stexdoc_typeset_manual_bool
  }
  \NewDocumentCommand \DisableManual {}{
    \bool_gset_false:N \g_stexdoc_typeset_manual_bool
  }
  \NewDocumentEnvironment {stexmanual} {} {
    \bool_if:NTF \g_stexdoc_typeset_manual_bool
      {\bool_set_false:N \l__codedoc_in_implementation_bool}
      {\comment}
  }{
    \bool_if:NF \g_stexdoc_typeset_manual_bool {\endcomment}
  }
\ExplSyntaxOff

%\usepackage{makeidx}
%\makeindex

%\usepackage{document-structure}
\ExplSyntaxOn
\int_new:N \l_stex_docheader_sect

\cs_new_protected:Nn \stexdoc_do_section:n {
  \int_case:nnF \l_stex_docheader_sect {
    {0}{\cs_if_exist:NTF \part {\part{#1}}{
      \int_incr:N \l_stex_docheader_sect
      \stexdoc_do_section:
    }}
    {1}{\cs_if_exist:NTF \chapter {\chapter{#1}}{
      \int_incr:N \l_stex_docheader_sect
      \stexdoc_do_section:
    }}
    {2}{\section{#1}}
    {3}{\subsection{#1}}
    {4}{\subsubsection{#1}}
  }{\paragraph{#1}}
  \int_incr:N \l_stex_docheader_sect
}
%\int_incr:N \l_stex_docheader_sect
\NewDocumentEnvironment{sfragment}{m}{
  \stexdoc_do_section:n{#1}
}{}

\cs_set_nopar:Nn \_stexdoc_do_cs:Nn {
  \stex_debug:nn{here}{\tl_to_str:n{#1},~#2}
  \cs_if_exist:cTF{s\tl_to_str:n{#2}}{
    \cs_if_exist:cTF{s\tl_to_str:n{#2}name}{
    \symref{#2-sym}{#1{#2}}
    }{#1{#2}}
  }{
    #1{#2}
  }
}
\let\my_old_cs\cs
\protected\def\cs#1{
  \_stexdoc_do_cs:Nn \my_old_cs{#1}
}
\let\my_old_cmd\cmd
\protected\def\cmd#1{
  \_stexdoc_do_cs:Nn \my_old_cmd{#1}
}

\ExplSyntaxOff

\mhinput[sTeX/Documentation]{../lib/examples.tex}

\begin{document}
  \DocInput{\jobname.dtx}
\end{document}
%</driver>
% \fi
%
% \title{ \sTeX-Metatheory
% 	\thanks{Version {\fileversion} (last revised {\filedate})} 
% }
%
% \author{Michael Kohlhase, Dennis Müller\\
% 	FAU Erlangen-Nürnberg\\
% 	\url{http://kwarc.info/}
% }
%
% \maketitle
%
%\ifinfulldoc\else
% This is the documentation for the \pkg{stex-metatheory} package.
% For a more high-level introduction, 
%  see \href{\basedocurl/manual.pdf}{the \sTeX Manual} or the
% \href{\basedocurl/stex.pdf}{full \sTeX documentation}.
%
% The \pkg{stex-metatheory} package contains \sTeX symbols so ubiquitous, that it is
virtually impossible to describe any flexiformal content without them, or that are
required to annotate even the most primitive symbols with meaningful
(foundation-independent) ``type''-annotations, or required for basic structuring
principles (theorems, definitions). As such, it serves as the default meta theory for any
\sTeX module.

We can also see the \pkg{stex-metatheory} as a foundation of mathematics in the sense of
\cite{rabe:future:15}, albeit an informal one (the ones discussed there are all formal
foundations). The state of the \pkg{stex-metatheory} is necessarily incomplete, and will
stay so for a long while: It arises as a collection of empirically useful symbols that are
collected as more and more mathematics are encoded in \sTeX and are classified as
foundational.

Formal foundations should ideally instantiate these symbols with their formal counterparts,
e.g. |isa| corresponds to a typing operation in typed setting, or the $\in$-operator in
set-theoretic contexts; |bind| corresponds to a universal quantifier in ($n$th-order)
logic, or a $\Pi$ in dependent type theories.

We make this theory part of the \sTeX collection rather than encoding it in \sTeX
itself\ednote{MK: why? continue}

%%% Local Variables:
%%% mode: latex
%%% TeX-master: "../stex-manual"
%%% End:

%  LocalWords:  stex-metatheory th-order

% \fi
%
% \begin{documentation}\label{pkg:metatheory:doc}
%
% \section{Symbols}\label{pkg:metatheory:symbols}
%
% \end{documentation}
%
% \begin{implementation}\label{pkg:metatheory:impl}
%
% \section{\sTeX-Metatheory Implementation}
%
%    \begin{macrocode}
%<*package>
%<@@=stex_modules>

%%%%%%%%%%%%%   metatheory.dtx   %%%%%%%%%%%%%

\str_const:Nn \c_stex_metatheory_ns_str {http://mathhub.info/sTeX/meta}
\begingroup
\stex_module_setup:nn{
  ns=\c_stex_metatheory_ns_str,
  meta=NONE
}{Metatheory}
\stex_reactivate_macro:N \symdecl
\stex_reactivate_macro:N \notation
\stex_reactivate_macro:N \symdef
\ExplSyntaxOff
\csname stex_suppress_html:n\endcsname{
  % is-a (a:A, a \in A, a is an A, etc.)
  \symdecl{isa}[args=ai]
  \notation{isa}[typed,op=:]{#1 \comp{:} #2}{##1 \comp, ##2}
  \notation{isa}[in]{#1 \comp\in #2}{##1 \comp, ##2}
  \notation{isa}[pred]{#2\comp(#1 \comp)}{##1 \comp, ##2}

  % bind (\forall, \Pi, \lambda etc.)
  \symdecl{bind}[args=Bi,assoc=pre]
  \notation{bind}[depfun,prec=nobrackets,op={(\cdot)\;\to\;\cdot}]{\comp( #1 \comp{)\;\to\;} #2}{##1 \comp, ##2}
  \notation{bind}[forall]{\comp\forall #1.\;#2}{##1 \comp, ##2}
  \notation{bind}[Pi]{\comp\prod_{#1}#2}{##1 \comp, ##2}

  % implicit bind
  \symdecl{implicitbind}[args=Bi,assoc=pre]
  \notation{implicitbind}[braces,prec=nobrackets,op={\{\cdot\}_I\;\cdot}]{\comp\{ #1 \comp{\}_I\;} #2}{##1 \comp, ##2}
  \notation{implicitbind}[depfun,prec=nobrackets]{\comp( #1 \comp{)\;\to_I\;} #2}{##1 \comp, ##2}
  \notation{implicitbind}[Pi]{\comp\prod^I_{#1}#2}{##1\comp,##2}

  % dummy variable
  \symdecl{dummyvar}
  \notation{dummyvar}[underscore]{\comp\_}
  \notation{dummyvar}[dot]{\comp\cdot}
  \notation{dummyvar}[dash]{\comp{{\rm --}}}

  %fromto (function space, Hom-set, implication etc.)
  \symdecl{fromto}[args=ai]
  \notation{fromto}[xarrow]{#1 \comp\to #2}{##1 \comp\times ##2}
  \notation{fromto}[arrow]{#1 \comp\to #2}{##1 \comp\to ##2}

  % mapto (lambda etc.)
  %\symdecl{mapto}[args=Bi]
  %\notation{mapto}[mapsto]{#1 \comp\mapsto #2}{#1 \comp, #2}
  %\notation{mapto}[lambda]{\comp\lambda #1 \comp.\; #2}{#1 \comp, #2}
  %\notation{mapto}[lambdau]{\comp\lambda_{#1} \comp.\; #2}{#1 \comp, #2}

  % function/operator application
  \symdecl{apply}[args=ia]
  \notation{apply}[prec=0;0x\infprec,parens,op=\cdot(\cdot)]{#1 \comp( #2 \comp)}{##1 \comp, ##2}
  \notation{apply}[prec=0;0x\infprec,lambda]{#1 \; #2 }{##1 \; ##2}

  % collection of propositions/booleans/truth values
  \symdecl{prop}[name=proposition]
  \notation{prop}[prop]{\comp{{\rm prop}}}
  \notation{prop}[BOOL]{\comp{{\rm BOOL}}}

  \symdecl{judgmentholds}[args=1]
  \notation{judgmentholds}[vdash,op=\vdash]{\comp\vdash\; #1}

  % sequences
  \symdecl{seqtype}[args=1]
  \notation{seqtype}[kleene]{#1^{\comp\ast}}

  \symdecl{seqexpr}[args=a]
  \notation{seqexpr}[angle,prec=nobrackets]{\comp\langle #1\comp\rangle}{##1\comp,##2}

  \symdef{seqmap}[args=abi,setlike]{\comp\{#3 \comp| #2\comp\in \dobrackets{#1} \comp\}}{##1 \comp, ##2}
  \symdef{seqprepend}[args=ia]{#1 \comp{::} #2}{##1 \comp, ##2}
  \symdef{seqappend}[args=ai]{#1 \comp{::} #2}{##1 \comp, ##2}
  \symdef{seqfoldleft}[args=iabbi]{ \comp{foldl}\dobrackets{#1,#2}\dobrackets{#3\comp,#4\comp\mapsto#5}}{##1 \comp, ##2}
  \symdef{seqfoldright}[args=iabbi,op=foldr]{ \comp{foldr}\dobrackets{#1,#2}\dobrackets{#3\comp,#4\comp\mapsto#5}}{##1 \comp, ##2}
  \symdef{seqhead}[args=a]{\comp{head}\dobrackets{#1}}{##1 \comp, ##2}
  \symdef{seqtail}[args=a]{\comp{tail}\dobrackets{#1}}{##1 \comp, ##2}
  \symdef{seqlast}[args=a]{\comp{last}\dobrackets{#1}}{##1 \comp, ##2}
  \symdef{seqinit}[args=a]{\comp{tail}\dobrackets{#1}}{##1 \comp, ##2}

  \symdef{sequence-index}[args=2,li,prec=nobrackets]{{#1}_{#2}}
  \notation{sequence-index}[ui,prec=nobrackets]{{#1}^{#2}}

  \symdef{aseqdots}[args=a,prec=nobrackets]{#1\comp{,\ellipses}}{##1\comp,##2}
  \symdef{aseqfromto}[args=ai,prec=nobrackets]{#1\comp{,\ellipses,}#2}{##1\comp,##2}
  \symdef{aseqfromtovia}[args=aii,prec=nobrackets]{#1\comp{,\ellipses,}#2\comp{,\ellipses,}#3}{##1\comp,##2}

  % nat literals
  \symdef{natliteral}{\comp{\mathtt{Ord}}}

  % letin (``let'', local definitions, variable substitution)
  \symdecl{letin}[args=bii]
  \notation{letin}[let]{\comp{{\rm let}}\;#1\comp{=}#2\;\comp{{\rm in}}\;#3}
  \notation{letin}[subst]{#3 \comp[ #1 \comp/ #2 \comp]}
  \notation{letin}[frac]{#3 \comp[ \frac{#2}{#1} \comp]}

  % structures
  \symdecl*{module-type}[args=1]
  \notation{module-type}{\comp{\mathtt{MOD}} #1}
  \symdecl{mathstruct}[name=mathematical-structure,args=a] % TODO
  \notation{mathstruct}[angle,prec=nobrackets]{\comp\langle #1 \comp\rangle}{##1 \comp, ##2}

  % objects
  \symdecl{object}
  \notation{object}{\comp{\mathtt{OBJECT}}}

}

% The following are abbreviations in the sTeX corpus that are left over from earlier
% developments. They will eventually be phased out. 

  \ExplSyntaxOn
  \stex_add_to_current_module:n{
    \def\nappli#1#2#3#4{\apply{#1}{\naseqli{#2}{#3}{#4}}}
    \def\nappui#1#2#3#4{\apply{#1}{\nasequi{#2}{#3}{#4}}}
    \def\livar{\csname sequence-index\endcsname[li]}
    \def\uivar{\csname sequence-index\endcsname[ui]}
    \def\naseqli#1#2#3{\aseqfromto{\livar{#1}{#2}}{\livar{#1}{#3}}}
    \def\nasequi#1#2#3{\aseqfromto{\uivar{#1}{#2}}{\uivar{#1}{#3}}}
  }
\_@@_end_module:
\endgroup
%    \end{macrocode}
%
%
%    \begin{macrocode}
%</package>
%    \end{macrocode}
%
% \end{implementation}
%
% \PrintIndex
% \ifinfulldoc\else\printbibliography\fi
% \endinput
% Local Variables:
% mode: doctex
% TeX-master: t
% End:
