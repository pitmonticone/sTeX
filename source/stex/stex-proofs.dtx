% \iffalse meta-comment 
% An Infrastructure for Structural Markup for Proofs
% Copyright (c) 2019 Michael Kohlhase, all rights reserved
%                this file is released under the
%                LaTeX Project Public License (LPPL)
% 
% The original of this file is in the public repository at 
% http://github.com/sLaTeX/sTeX/
%
% TODO update copyright  
%
%<*driver>
\def\libfolder#1{../../lib/#1}
\RequirePackage{paralist}
\ifcsname stexdocpath\endcsname\else\def\stexdocpath{.}\fi
\documentclass[full]{l3doc}
%\RequirePackage{document-structure}
\usepackage[hyperref=auto,style=alphabetic]{biblatex}
\usepackage[mathhub=\stexdocpath/mh,usesms]{stex}
\usepackage{stex-highlighting,stexthm}

\newif\ifhadtitle\hadtitlefalse

\def\fileversion{3.3.0}
\def\filedate{\today}
\def\stexdoctitle#1#2{\title{#1\thanks{Version {\fileversion} (last revised {\filedate})} }\def\thispkg{#2}}

\author{Michael Kohlhase, Dennis Müller\\
 	FAU Erlangen-Nürnberg\\
 	\url{http://kwarc.info/}
}

\def\stexmaketitle{\ifhadtitle\else\hadtitletrue\maketitle\fi}

\def\docmodule{
\begin{document}
  \EnableManual
  \DisableImplementation
  \DocInput{\jobname.dtx}
  \EnableImplementation
  \DisableDocumentation
  \DisableManual
  \DocInputAgain
  \clearpage
  \PrintIndex
\end{document}
}

\ExplSyntaxOn
  \bool_new:N \g_stexdoc_typeset_manual_bool
  \NewDocumentCommand \EnableManual {}{
    \bool_gset_true:N \g_stexdoc_typeset_manual_bool
  }
  \NewDocumentCommand \DisableManual {}{
    \bool_gset_false:N \g_stexdoc_typeset_manual_bool
  }
  \NewDocumentEnvironment {stexmanual} {} {
    \bool_if:NTF \g_stexdoc_typeset_manual_bool
      {\bool_set_false:N \l__codedoc_in_implementation_bool}
      {\comment}
  }{
    \bool_if:NF \g_stexdoc_typeset_manual_bool {\endcomment}
  }
\ExplSyntaxOff

%\usepackage{makeidx}
%\makeindex

%\usepackage{document-structure}
\ExplSyntaxOn
\int_new:N \l_stex_docheader_sect

\cs_new_protected:Nn \stexdoc_do_section:n {
  \int_case:nnF \l_stex_docheader_sect {
    {0}{\cs_if_exist:NTF \part {\part{#1}}{
      \int_incr:N \l_stex_docheader_sect
      \stexdoc_do_section:
    }}
    {1}{\cs_if_exist:NTF \chapter {\chapter{#1}}{
      \int_incr:N \l_stex_docheader_sect
      \stexdoc_do_section:
    }}
    {2}{\section{#1}}
    {3}{\subsection{#1}}
    {4}{\subsubsection{#1}}
  }{\paragraph{#1}}
  \int_incr:N \l_stex_docheader_sect
}
%\int_incr:N \l_stex_docheader_sect
\NewDocumentEnvironment{sfragment}{m}{
  \stexdoc_do_section:n{#1}
}{}

\cs_set_nopar:Nn \_stexdoc_do_cs:Nn {
  \stex_debug:nn{here}{\tl_to_str:n{#1},~#2}
  \cs_if_exist:cTF{s\tl_to_str:n{#2}}{
    \cs_if_exist:cTF{s\tl_to_str:n{#2}name}{
    \symref{#2-sym}{#1{#2}}
    }{#1{#2}}
  }{
    #1{#2}
  }
}
\let\my_old_cs\cs
\protected\def\cs#1{
  \_stexdoc_do_cs:Nn \my_old_cs{#1}
}
\let\my_old_cmd\cmd
\protected\def\cmd#1{
  \_stexdoc_do_cs:Nn \my_old_cmd{#1}
}

\ExplSyntaxOff

\mhinput[sTeX/Documentation]{../lib/examples.tex}

\begin{document}
  \DocInput{\jobname.dtx}
\end{document}
%</driver>
% \fi
%
% \GetFileInfo{sproof.sty}
% 
% \title{\sTeX-Proofs: Structural Markup for Proofs\thanks{Version {\fileversion} (last revised {\filedate})}
% }
%
% \author{Michael Kohlhase, Dennis Müller\\
% 	FAU Erlangen-Nürnberg\\
% 	\url{http://kwarc.info/}
%c }
%
% \maketitle
%
%\ifinfulldoc\else
% \begin{abstract}
%   This is the documentation for the \pkg{stex-proofs} package.
%
% The |sproof| package is part of the {\stex} collection, a version of {\TeX/\LaTeX} that
% allows to markup {\TeX/\LaTeX} documents semantically without leaving the document
% format, essentially turning {\TeX/\LaTeX} into a document format for mathematical
% knowledge management (MKM).
%
% The \pkg{stex-proofs} package supplies macros and environment that allow to annotate the
% structure of mathematical proofs in {\stex} files. This structure can be used by MKM
% systems for added-value services, either directly from the \sTeX sources, or after
% translation.
%
% For a more high-level introduction, see \href{\basedocurl/manual.pdf}{the \sTeX Manual}
% or the \href{\basedocurl/stex.pdf}{full \sTeX documentation}.
% \end{abstract}
% 
% \tableofcontents
%
% \begin{sfragment}{Proofs}

The \pkg{stex-proof} package supplies macros and environment that allow to annotate the
structure of mathematical proofs in \sTeX document. This structure can be used by MKM
systems for added-value services, either directly from the \sTeX sources, or after
translation.

Its central component is the |sproof|-environment, whose body consists
of:
\begin{itemize}
  \item \emph{subproofs} via the |subproof|-environment,
  \item \emph{proof steps} via the |\spfstep|, |\eqstep|
    |\assumption|, and |\conclude| macros, and
  \item \emph{comments}, via normal text without special markup.
\end{itemize}

|sproof|, |subproof| and the various proof step macros take the following
optional arguments:
\begin{itemize}
  \item[|id|] (\meta{string}) for referencing,
  \item[|method|] (\meta{string}) the proof method (e.g. contradiction, induction,...)
  \item[|term|] (\meta{token list}) the (ideally semantically-marked up)
    proposition that is derived/proven by this proof/subproof/proof step.
\end{itemize}
Additionally, they take one mandatory argument for the document text
to be annotated, or (in the case of the environments)
as an introductory description of the proof itself. Since
the latter often contains the |term| to be derived
as text, alternatively to providing it as 
an optional argument, the mandatory argument can use the
|\yield|-macro to mark it up in the text.

The |sproof| and |subproof| environments additionally take
two optional arguments:
\begin{itemize}
  \item[|for|] the symbol identifier/name corresponding to the |sassertion|
    to be proven. This too subsumes |\yield| and the |term|-argument.
  \item[|hide|] In the pdf, this only shows the mandatory argument text
    and hides the body of the environment. In the HTML (as served by \mmt),
    the bodies of all |proof| and |subproof| environments are \emph{collapsible},
    and |hide| collapses the body by default.
\end{itemize}

\begin{smodule}{sproofs}
  \symdef{coprime}[args=2]{#1 \comp\bot #2}
  \symdef{realroot}[args=1]{\sqrt{#1}}
  \symdecl{irrational}[args=1]
  \symdecl{rational}[args=1]
  \symdef{eq}[args=2]{#1 \comp= #2}
  \symdef{ratfrac}[args=2,prec=1]{\frac{#1}{#2}}
  \symdef{intpow}[args=2,prec=-1]{{#1}^{#2}}
  \symdef{ratpow}[args=2]{{#1}^{#2}}
  \symdef{inset}[args=ai]{#1\comp\in#2}{##1\comp,##2}
  \symdef{Int}{\comp{\mathbb Z}}
  \symdef{PosInt}{\comp{{\mathbb Z}^+}}
  \symdef{inttimes}[args=2]{#1 #2}
  \symdef{divides}[args=2]{#1\comp\|#2}

  \vardef{vara}{\comp a}
  \vardef{varb}{\comp b}
  \vardef{varc}{\comp c}

  \def\contradiction{}

\begin{latexcode}
  \begin{sassertion}[type=theorem,name=sqrt2irr]
    \conclusion{\irrational{$\arg{\realroot{2}}$ is \comp{irrational}}}.
\end{sassertion}

\begin{sproof}[for=sqrt2irr,method=contradiction]{By contradiction}
    \assumption{Assume \yield{\rational{$\arg{\realroot{2}}$ is 
      \comp{rational}}}}
    \begin{subproof}[method=straightforward]{Then 
        \yield{$\eq{\ratfrac{\intpow{\vara}{2}}{\intpow{\varb}2}}{2}$ 
        for some $\inset{\vara,\varb}\PosInt$ with 
        \coprime{$\arg{\vara},\arg{\varb}$ \comp{coprime}}}}
        \assumption{By assumption, \yield{there are 
        $\inset{\vara,\varb}\PosInt $ with 
        $\realroot{2}=\ratfrac{\vara}{\varb}$}}
        \spfstep{wlog, we can assume \coprime{$\arg{\vara},\arg{\varb}$ 
        to be \comp{coprime}}}
            % a comment:
            If not, reduce the fraction until numerator and denominator 
            are coprime, and let the resulting components be 
            $\vara $ and $\varb $
        \spfstep{Then \yield{$\eq{\intpow{\ratfrac{\vara}{\varb}}2}2$}}
        \eqstep{\ratfrac{\intpow{\vara}2}{\intpow{\varb}2}}
    \end{subproof}
    \begin{subproof}[term=\divides{2}{\vara},method=straightforward]{
        Then $\vara $ is even}
        \spfstep{Multiplying the equation by $\intpow{\varb}2$ yields 
        $\yield{\eq{\intpow{\vara}2}{\inttimes{2}{\intpow{\varb}2}}}$}
        \spfstep[term=\divides{2}{\intpow{\vara}2}]{Hence 
        $\intpow{\vara}2$ is even}
        \conclude[term=\divides{2}{\vara}]{Hence $\vara $ is even as well}
        % another comment:
        Hint: Think about the prime factorizations of $\vara $ and 
        $\intpow{\vara}2$
    \end{subproof}
    \begin{subproof}[term=\divides{2}{\varb},method=straightforward,]{
        Then $\varb $ is also even}
        \spfstep{Since $\vara $ is even, we have \yield{some $\varc $ 
          such that $\eq{\inttimes{2}{\varc}}{\vara}$}}
        \spfstep{Plugging into the above, we get 
          \yield{$\eq{\intpow{\inttimes{2}{\vara}}2}
            {\inttimes{2}{\intpow{\varb}2}}$}}
        \eqstep{\inttimes{4}{\intpow{\vara}2}}
        \spfstep{Dividing both sides by $2$ yields 
          \yield{$\eq{\intpow{\varb}2}{\inttimes{2}{\intpow{\vara}2}}$}}
        \spfstep[term=\divides{2}{\intpow{\varb}2}]{Hence 
          $\intpow{\varb}2$ is even}
        \conclude[term=\divides{2}{\varb}]{Hence $\varb $ is even}
        % one more comment:
        By the same argument as above
    \end{subproof}
    \conclude[term=\contradiction]{Contradiction to $\vara,\varb $ being 
    \symname{coprime}.}
\end{sproof}
\end{latexcode}

which will produce:

\begin{mdframed}
\begin{sassertion}[type=theorem,name=sqrt2irr]
  \conclusion{\irrational{$\arg{\realroot2}$ is \comp{irrational}}}.
\end{sassertion}

\begin{sproof}[for=sqrt2irr,method=contradiction]{By contradiction}
  \assumption{Assume \yield{\rational{$\arg{\realroot2}$ is \comp{rational}}}}
  \begin{subproof}[method=straightforward]%
      {Then \yield{$\eq{\ratfrac{\intpow \vara2}{\intpow \varb2}}{2}$ for some $\inset{\vara,\varb}\PosInt$ with \coprime{$\arg \vara,\arg \varb$ \comp{coprime}}}}
      \assumption{By assumption, \yield{there are $\inset{\vara,\varb}\PosInt$ with $\realroot2=\ratfrac\vara\varb$}}
      \spfstep{wlog, we can assume \coprime{$\arg{\vara},\arg{\varb}$ to be \comp{coprime}}}
          If not, reduce the fraction until numerator and denominator are 
          coprime, and let the resulting components be $\vara$ and $\varb$
      \spfstep{Then \yield{$\eq{\intpow{\ratfrac\vara\varb}2}2$}}
      \eqstep{\ratfrac{\intpow\vara2}{\intpow\varb2}}
  \end{subproof}
  \begin{subproof}[term=\divides{2}{\vara},method=straightforward]{Then $\vara$ is even}
      \spfstep{Multiplying the equation by $\intpow\varb2$ yields $\yield{\eq{\intpow\vara2}{\inttimes2{\intpow\varb2}}}$}
      \spfstep[term=\divides{2}{\intpow\vara2}]{Hence $\intpow\vara2$ is even}
      \conclude[term=\divides{2}{\vara}]{Hence $\vara$ is even as well}
      Hint: Think about the prime factorizations of $\vara$ and $\intpow\vara2$
  \end{subproof}
  \begin{subproof}[term=\divides{2}{\varb},method=straightforward]{Then $\varb$ is also even}
      \spfstep{Since $\vara$ is even, we have \yield{some $\varc$ such that $\eq{\inttimes2\varc}{\vara}$}}
      \spfstep{Plugging into the above, we get \yield{$\eq{\intpow{\inttimes2\vara}2}{\inttimes2{\intpow\varb2}}$}}
      \eqstep{\inttimes4{\intpow\vara2}}
      \spfstep{Dividing both sides by $2$ yields \yield{$\eq{\intpow\varb2}{\inttimes2{\intpow\vara2}}$}}
      \spfstep[term=\divides{2}{\intpow\varb2}]{Hence $\intpow\varb2$ is even}
      \conclude[term=\divides{2}{\varb}]{Hence $\varb$ is even}
      By the same argument as above
  \end{subproof}
  \conclude[term=\contradiction]{Contradiction to $\vara,\varb$ being \symname{coprime}.}
\end{sproof}
\end{mdframed}

If we mark all subproofs with |hide|, we will obtain the following
instead:

\begin{mdframed}
  \begin{sassertion}[type=theorem,name=sqrt2irr]
    \conclusion{\irrational{$\arg{\realroot2}$ is \comp{irrational}}}.
  \end{sassertion}
  
  \begin{sproof}[for=sqrt2irr,method=contradiction]{By contradiction}
    \assumption{Assume \yield{\rational{$\arg{\realroot2}$ is \comp{rational}}}}
    \begin{subproof}[hide,method=straightforward]%
        {Then \yield{$\eq{\ratfrac{\intpow \vara2}{\intpow \varb2}}{2}$ for some $\inset{\vara,\varb}\PosInt$ with \coprime{$\arg \vara,\arg \varb$ \comp{coprime}}}}
        \assumption{By assumption, \yield{there are $\inset{\vara,\varb}\PosInt$ with $\realroot2=\ratfrac\vara\varb$}}
        \spfstep{wlog, we can assume \coprime{$\arg{\vara},\arg{\varb}$ to be \comp{coprime}}}
            If not, reduce the fraction until numerator and denominator are 
            coprime, and let the resulting components be $\vara$ and $\varb$
        \spfstep{Then \yield{$\eq{\intpow{\ratfrac\vara\varb}2}2$}}
        \eqstep{\ratfrac{\intpow\vara2}{\intpow\varb2}}
    \end{subproof}
    \begin{subproof}[hide,term=\divides{2}{\vara},method=straightforward]{Then $\vara$ is even}
        \spfstep{Multiplying the equation by $\intpow\varb2$ yields $\yield{\eq{\intpow\vara2}{\inttimes2{\intpow\varb2}}}$}
        \spfstep[term=\divides{2}{\intpow\vara2}]{Hence $\intpow\vara2$ is even}
        \conclude[term=\divides{2}{\vara}]{Hence $\vara$ is even as well}
        Hint: Think about the prime factorizations of $\vara$ and $\intpow\vara2$
    \end{subproof}
    \begin{subproof}[hide,term=\divides{2}{\varb},method=straightforward,]{Then $\varb$ is also even}
        \spfstep{Since $\vara$ is even, we have \yield{some $\varc$ such that $\eq{\inttimes2\varc}{\vara}$}}
        \spfstep{Plugging into the above, we get \yield{$\eq{\intpow{\inttimes2\vara}2}{\inttimes2{\intpow\varb2}}$}}
        \eqstep{\inttimes4{\intpow\vara2}}
        \spfstep{Dividing both sides by $2$ yields \yield{$\eq{\intpow\varb2}{\inttimes2{\intpow\vara2}}$}}
        \spfstep[term=\divides{2}{\intpow\varb2}]{Hence $\intpow\varb2$ is even}
        \conclude[term=\divides{2}{\varb}]{Hence $\varb$ is even}
        By the same argument as above
    \end{subproof}
    \conclude[term=\contradiction]{Contradiction to $\vara,\varb$ being \symname{coprime}.}
  \end{sproof}
  \end{mdframed}

  However, the hidden subproofs will still be shown in the HTML,
  only in an expandable section which is collapsed by default.


The above style of writing proofs is usually called \emph{structured proofs}.
They have a huge advantage over the traditional purely prosaic style,
in that (as the name suggests) the actual \emph{structure} of the proof
is made explicit, which almost always makes it considerably more 
comprehensible. We, among many others, encourage the general use of 
structured proofs.

Alas, most proofs are not written in this style, and we would
do users a disservice by insisting on this style. For that reason,
the |spfblock| environment turns all subproofs and proof step
macros into presentationally neutral \emph{inline} annotations,
as in the induction step of the following example:

\begin{latexcode}
\begin{sproof}[id=simple-proof,method=induction]
   {We prove that $\sum_{i=1}^n{2i-1}=n^{2}$ by induction over $n$}
  For the induction we have to consider three cases: % <- a comment
   \begin{subproof}{$n=1$}
    \spfstep*{then we compute $1=1^2$}
   \end{subproof}
   \begin{subproof}{$n=2$}
        This case is not really necessary, but we do it for the
        fun of it (and to get more intuition).
      \spfstep*{We compute $1+3=2^{2}=4$.}
   \end{subproof}
   \begin{subproof}{$n>1$}\begin{spfblock}
      \assumption[id=ind-hyp]{
        Now, we assume that the assertion is true for a certain $k\geq 1$,
        i.e. \yield{$\sum_{i=1}^k{(2i-1)}=k^{2}$}.
      }
      
        We have to show that we can derive the assertion for $n=k+1$ from
        this assumption, i.e. $\sum_{i=1}^{k+1}{(2i-1)}=(k+1)^{2}$.

      \spfstep{
        We obtain $\yield{\sum_{i=1}^{k+1}{2i-1}=
          \sum_{i=1}^k{2i-1}+2(k+1)-1}$
        \spfjust{by \splitsum{\comp{splitting the sum}
        \arg*{$\sum_{i=1}^{k+1}{(2i-1)}=(k+1)^{2}$}}}.
      }
      \spfstep{
        Thus we have $\yield{\sum_{i=1}^{k+1}{(2i-1)}=k^2+2k+1}$
        \spfjust{by \symname{induction-hypothesis}}.
      }
      \conclude{
        We can \spfjust{\simplification{\comp{simplify} the right-hand side
        \arg*{k^2+2k+1}}} to
        ${k+1}^2$, which proves the assertion.
      }
   \end{spfblock}\end{subproof}
    \conclude{
      We have considered all the cases, so we have proven the assertion.
    }
\end{sproof}
\end{latexcode}


\symdecl{splitsum}[args=1]
\symdecl{inducthypothesis}[name=induction-hypothesis]
\symdecl{simplification}[args=1]

This yields the following result: 

\begin{mdframed}
  \begin{sproof}[id=simple-proof,method=induction]
    {We prove that $\sum_{i=1}^n{2i-1}=n^{2}$ by induction over $n$}
   For the induction we have to consider three cases: % <- a comment
    \begin{subproof}{$n=1$}
     \spfstep*{then we compute $1=1^2$}
    \end{subproof}
    \begin{subproof}{$n=2$}
         This case is not really necessary, but we do it for the
         fun of it (and to get more intuition).
       \spfstep*{We compute $1+3=2^{2}=4$.}
    \end{subproof}
    \begin{subproof}{$n>1$}\begin{spfblock}
       \assumption[id=ind-hyp]{
         Now, we assume that the assertion is true for a certain $k\geq 1$,
         i.e. \yield{$\sum_{i=1}^k{(2i-1)}=k^{2}$}.
       }
       
         We have to show that we can derive the assertion for $n=k+1$ from
         this assumption, i.e. $\sum_{i=1}^{k+1}{(2i-1)}=(k+1)^{2}$.
 
       \spfstep{
         We obtain $\yield{\sum_{i=1}^{k+1}{2i-1}=\sum_{i=1}^k{2i-1}+2(k+1)-1}$
         \spfjust{by \splitsum{\comp{splitting the sum}\arg*{$\sum_{i=1}^{k+1}{(2i-1)}=(k+1)^{2}$}}}.
       }
       \spfstep{
         Thus we have $\yield{\sum_{i=1}^{k+1}{(2i-1)}=k^2+2k+1}$
         \spfjust{by \symname{induction-hypothesis}}.
       }
       \conclude{
         We can \spfjust{\simplification{\comp{simplify} the right-hand side
         \arg*{k^2+2k+1}}} to
         ${k+1}^2$, which proves the assertion.
       }
    \end{spfblock}\end{subproof}
     \conclude{
       We have considered all the cases, so we have proven the assertion.
     }
 \end{sproof}
\end{mdframed}

\end{smodule}

\begin{environment}{sproof}
  The |sproof| environment is the main container for proofs. It takes an optional |KeyVal|
  argument that allows to specify the |id| (identifier) and |for| (for which assertion is
  this a proof) keys. The regular argument of the |proof| environment contains an
  introductory comment, that may be used to announce the proof style. The |proof|
  environment contains a sequence of |spfstep|, |spfcomment|, and |spfcases| environments
  that are used to markup the proof steps.
\end{environment}
  
\begin{function}{\spfidea}
  The |\spfidea| macro allows to give a one-paragraph description of the proof idea.
\end{function}

\begin{function}{\spfsketch}
  For one-line proof sketches, we use the |\spfsketch| macro, which takes the same
  optional argument as |sproof| and another one: a natural language text that sketches
  the proof.
\end{function}

\begin{function}{\spfstep}
  Regular proof steps are marked up with the |\spfstep| macro, which takes an optional
  |KeyVal| argument for annotations. A proof step usually contains a local assertion
  (the text of the step) together with some kind of evidence that this can be derived
  from already established assertions.
\end{function}

\begin{function}{\yield}
  See above
\end{function}

\begin{function}{\spfjust}
  This evidence is marked up with the |\spfjust| macro in the \pkg{stex-proofs}
  package. This environment totally invisible to the formatted result; it wraps the text
  in the proof step that corresponds to the evidence (ideally, a
  semantically marked-up term).
\end{function}

\begin{function}{\assumption}
  The |\assumption| macro allows to mark up a (justified) assumption.
\end{function}

\begin{function}{\justarg}
\end{function}

\begin{environment}{subproof}
  The |subproof| environment is used to mark up a subproof. This environment takes an
  optional |KeyVal| argument for semantic annotations and a second argument that allows
  to specify an introductory comment (just like in the |proof| environment). The
  |method| key can be used to give the name of the proof method
  executed to make this subproof.
\end{environment}

\begin{function}{\sproofend}
  Traditionally, the end of a mathematical proof is marked with a little box at the end of
  the last line of the proof (if there is space and on the end of the next line if there
  isn't), like so:\sproofend

  The \pkg{stex-proofs} package provides the |\sproofend| macro for this.
\end{function}
  
\begin{variable}{\sProofEndSymbol}
  If a different symbol for the proof end is to be used (e.g. {\sl{q.e.d}}), then this can
  be obtained by specifying it using the |\sProofEndSymbol| configuration macro (e.g. by
  specifying |\sProofEndSymbol{q.e.d}|).
\end{variable}
  
Some of the proof structuring macros above will insert proof end symbols for sub-proofs,
in most cases, this is desirable to make the proof structure explicit, but sometimes this
wastes space (especially, if a proof ends in a case analysis which will supply its own
proof end marker). To suppress it locally, just set |proofend={}| in them or use use
|\sProofEndSymbol{}|.
\end{sfragment}

%%% Local Variables:
%%% mode: latex
%%% TeX-master: "../stex-manual"
%%% End:

%  LocalWords:  hypothesis,id geq splitit arith:split-sum byindhyp rhs proofend eqstep
%  LocalWords:  sproofs coprime ratfrac 2,prec intpow ratpow inttimes varb varc sqrt2irr
%  LocalWords:  theorem,name sqrt2irr,method wlog hide,method hide,term spfblock splitsum
%  LocalWords:  simple-proof,method inducthypothesis

% \fi
%
% \begin{documentation}
% \changes{v0.9}{2022/02/14}{Moved over from the \pkg{sproofs} package}
% 
%
%
% \end{documentation}
%
% \begin{implementation}
%
% \section{The Implementation} 
% 
%    \begin{macrocode}
%<*package>
%<@@=stex_sproof>

%%%%%%%%%%%%%   sproof.dtx   %%%%%%%%%%%%%

%    \end{macrocode}
%
%
% \subsection{Proofs}\label{sec:impl:proofs}
% 
% We first define some keys for the |proof| environment.
%    \begin{macrocode}
\keys_define:nn { stex / spf } {
  id          .str_set_x:N  = \spfid,
  for         .clist_set:N  = \l_@@_spf_for_clist ,
  from        .tl_set:N     = \l_@@_spf_from_tl ,
  proofend    .tl_set:N     = \l_@@_spf_proofend_tl,
  type        .str_set_x:N  = \spftype,
  title       .tl_set:N     = \spftitle,
  continues   .tl_set:N     = \l_@@_spf_continues_tl,
  functions   .tl_set:N     = \l_@@_spf_functions_tl,
  method      .tl_set:N     = \l_@@_spf_method_tl
}
\cs_new_protected:Nn \_@@_spf_args:n {
	\str_clear:N \spfid
	\tl_clear:N \l_@@_spf_for_tl
	\tl_clear:N \l_@@_spf_from_tl
	\tl_set:Nn \l_@@_spf_proofend_tl {\sproof@box}
	\str_clear:N \spftype
	\tl_clear:N \spftitle
	\tl_clear:N \l_@@_spf_continues_tl
	\tl_clear:N \l_@@_spf_functions_tl
	\tl_clear:N \l_@@_spf_method_tl
  \bool_set_false:N \l_@@_inc_counter_bool
	\keys_set:nn { stex / spf }{ #1 }
}
%    \end{macrocode}
%
% \begin{macro}{\c_@@_flow_str}
% We define this macro, so that we can test whether the |display| key has the value |flow|
%    \begin{macrocode}
\str_set:Nn\c_@@_flow_str{inline}
%    \end{macrocode}
% \end{macro}
%
% For proofs, we will have to have deeply nested structures of enumerated list-like
% environments. However, {\LaTeX} only allows |enumerate| environments up to nesting depth
% 4 and general list environments up to listing depth 6. This is not enough for us.
% Therefore we have decided to go along the route proposed by Leslie Lamport to use a
% single top-level list with dotted sequences of numbers to identify the position in the
% proof tree. Unfortunately, we could not use his |pf.sty| package directly, since it does
% not do automatic numbering, and we have to add keyword arguments all over the place, to
% accomodate semantic information.
%
%    \begin{macrocode}
\intarray_new:Nn\l_@@_counter_intarray{50}
\cs_new_protected:Npn \sproofnumber {
  \int_set:Nn \l_tmpa_int {1}
  \bool_while_do:nn {
    \int_compare_p:nNn {
      \intarray_item:Nn \l_@@_counter_intarray \l_tmpa_int
    } > 0
  }{
    \intarray_item:Nn \l_@@_counter_intarray \l_tmpa_int .
    \int_incr:N \l_tmpa_int
  }
}
\cs_new_protected:Npn \_@@_inc_counter: {
  \int_set:Nn \l_tmpa_int {1}
  \bool_while_do:nn {
    \int_compare_p:nNn {
      \intarray_item:Nn \l_@@_counter_intarray \l_tmpa_int
    } > 0
  }{
    \int_incr:N \l_tmpa_int
  }
  \int_compare:nNnF \l_tmpa_int = 1 {
    \int_decr:N \l_tmpa_int
  }
  \intarray_gset:Nnn \l_@@_counter_intarray \l_tmpa_int {
    \intarray_item:Nn \l_@@_counter_intarray \l_tmpa_int + 1
  }
}

\cs_new_protected:Npn \_@@_add_counter: {
  \int_set:Nn \l_tmpa_int {1}
  \bool_while_do:nn {
    \int_compare_p:nNn {
      \intarray_item:Nn \l_@@_counter_intarray \l_tmpa_int
    } > 0
  }{
    \int_incr:N \l_tmpa_int
  }
  \intarray_gset:Nnn \l_@@_counter_intarray \l_tmpa_int { 1 }
}

\cs_new_protected:Npn \_@@_remove_counter: {
  \int_set:Nn \l_tmpa_int {1}
  \bool_while_do:nn {
    \int_compare_p:nNn {
      \intarray_item:Nn \l_@@_counter_intarray \l_tmpa_int
    } > 0
  }{
    \int_incr:N \l_tmpa_int
  }
  \int_decr:N \l_tmpa_int
  \intarray_gset:Nnn \l_@@_counter_intarray \l_tmpa_int { 0 }
}
%    \end{macrocode}
%
%\begin{macro}{\sproofend}
%    This macro places a little box at the end of the line if there is space, or at the
%    end of the next line if there isn't
%    \begin{macrocode}
\def\sproof@box{
  \hbox{\vrule\vbox{\hrule width 6 pt\vskip 6pt\hrule}\vrule}
}
\def\sproofend{
  \tl_if_empty:NF \l_@@_spf_proofend_tl {
    \hfil\null\nobreak\hfill\l_@@_spf_proofend_tl\par\smallskip
  }
}
%    \end{macrocode}
% \end{macro}
%
% \begin{macro}{spf@*@kw}
%    \begin{macrocode}
\def\spf@proofsketch@kw{Proof~Sketch}
\def\spf@proof@kw{Proof}
\def\spf@step@kw{Step}
%    \end{macrocode}
% \end{macro}
%
% For the other languages, we set up triggers
%    \begin{macrocode}
\AddToHook{begindocument}{
  \ltx@ifpackageloaded{babel}{
    \makeatletter
    \clist_set:Nx \l_tmpa_clist {\bbl@loaded}
    \clist_if_in:NnT \l_tmpa_clist {ngerman}{
      \input{sproof-ngerman.ldf}
    }
    \clist_if_in:NnT \l_tmpa_clist {finnish}{
      \input{sproof-finnish.ldf}
    }
    \clist_if_in:NnT \l_tmpa_clist {french}{
      \input{sproof-french.ldf}
    }
    \clist_if_in:NnT \l_tmpa_clist {russian}{
      \input{sproof-russian.ldf}
    }
    \makeatother
  }{}
}
%    \end{macrocode}
%
% \begin{macro}{spfsketch}
%    \begin{macrocode}
\newcommand\spfsketch[2][]{
  \begingroup
  \let \premise \stex_proof_premise:
  \_@@_spf_args:n{#1}
  \stex_if_smsmode:TF {
    \str_if_empty:NF \spfid {
      \stex_ref_new_doc_target:n \spfid
    }
  }{
    \seq_clear:N \l_tmpa_seq
    \clist_map_inline:Nn \l_@@_spf_for_clist {
      \tl_if_empty:nF{ ##1 }{
        \stex_get_symbol:n { ##1 }
        \exp_args:NNo \seq_put_right:Nn \l_tmpa_seq {
          \l_stex_get_symbol_uri_str
        }
      }
    }
    \exp_args:Nnx
    \stex_annotate:nnn{proofsketch}{\seq_use:Nn \l_tmpa_seq {,}}{
      \str_if_empty:NF \spftype {
        \stex_annotate_invisible:nnn{type}{\spftype}{}
      }
      \clist_set:No \l_tmpa_clist \spftype
      \tl_set:Nn \l_tmpa_tl {
        \titleemph{
          \tl_if_empty:NTF \spftitle {
            \spf@proofsketch@kw
          }{
            \spftitle
          }
        }:~
      }
      \clist_map_inline:Nn \l_tmpa_clist {
        \exp_args:No \str_if_eq:nnT \c_@@_flow_str {##1} {
          \tl_clear:N \l_tmpa_tl
        }
      }
      \str_if_empty:NF \spfid {
        \stex_ref_new_doc_target:n \spfid
      }
      \l_tmpa_tl #2 \sproofend
    }
  }
  \endgroup
  \stex_smsmode_do:
}

%    \end{macrocode}
% \end{macro}
%
% \begin{macro}{spfeq}
%   This is very similar to |\spfsketch|, but uses a computation array\ednote{This should
%   really be more like a tabular with an ensuremath in it. or invoke text on the last
%   column}\ednote{document above}
%    \begin{macrocode}
\newenvironment{spfeq}[2][]{
  \_@@_spf_args:n{#1}
  \let \premise \stex_proof_premise:
  \stex_if_smsmode:TF {
    \str_if_empty:NF \spfid {
      \stex_ref_new_doc_target:n \spfid
    }
  }{
    \seq_clear:N \l_tmpa_seq
    \clist_map_inline:Nn \l_@@_spf_for_clist {
      \tl_if_empty:nF{ ##1 }{
        \stex_get_symbol:n { ##1 }
        \exp_args:NNo \seq_put_right:Nn \l_tmpa_seq {
          \l_stex_get_symbol_uri_str
        }
      }
    }
    \exp_args:Nnnx
    \begin{stex_annotate_env}{spfeq}{\seq_use:Nn \l_tmpa_seq {,}}
    \str_if_empty:NF \spftype {
      \stex_annotate_invisible:nnn{type}{\spftype}{}
    }

    \clist_set:No \l_tmpa_clist \spftype
    \tl_clear:N \l_tmpa_tl
    \clist_map_inline:Nn \l_tmpa_clist {
      \tl_if_exist:cT {_@@_spfeq_##1_start:}{
        \tl_set:Nn \l_tmpa_tl {\use:c{_@@_spfeq_##1_start:}}
      }
      \exp_args:No \str_if_eq:nnT \c_@@_flow_str {##1} {
        \tl_set:Nn \l_tmpa_tl {\use:n{}}
      }
    }
    \tl_if_empty:NTF \l_tmpa_tl {
      \_@@_spfeq_start:
    }{
      \l_tmpa_tl
    }{~#2}
    \str_if_empty:NF \spfid {
      \stex_ref_new_doc_target:n \spfid
    }
    \begin{displaymath}\begin{array}{rcll}
  }
  \stex_smsmode_do:
}{
  \stex_if_smsmode:F {
    \end{array}\end{displaymath}
    \clist_set:No \l_tmpa_clist \spftype
    \tl_clear:N \l_tmpa_tl
    \clist_map_inline:Nn \l_tmpa_clist {
      \tl_if_exist:cT {_@@_spfeq_##1_end:}{
        \tl_set:Nn \l_tmpa_tl {\use:c{_@@_spfeq_##1_end:}}
      }
    }
    \tl_if_empty:NTF \l_tmpa_tl {
      \_@@_spfeq_end:
    }{
      \l_tmpa_tl
    }
    \end{stex_annotate_env}
  }
}

\cs_new_protected:Nn \_@@_spfeq_start: {
  \titleemph{
    \tl_if_empty:NTF \spftitle {
      \spf@proof@kw
    }{
      \spftitle
    }
  }:
}
\cs_new_protected:Nn \_@@_spfeq_end: {\sproofend}

\newcommand\stexpatchspfeq[3][] {
    \str_set:Nx \l_tmpa_str{ #1 }
    \str_if_empty:NTF \l_tmpa_str {
      \tl_set:Nn \_@@_spfeq_start: { #2 }
      \tl_set:Nn \_@@_spfeq_end: { #3 }
    }{
      \exp_after:wN \tl_set:Nn \csname _@@_spfeq_#1_start:\endcsname{ #2 }
      \exp_after:wN \tl_set:Nn \csname _@@_spfeq_#1_end:\endcsname{ #3 }
    }
}

%    \end{macrocode}
% \end{macro}
%
% \begin{environment}{sproof}
%    In this environment, we initialize the proof depth counter |\count10| to 10, and set
%    up the description environment that will take the proof steps. At the end of the
%    proof, we position the proof end into the last line.
%    \begin{macrocode}
\newenvironment{sproof}[2][]{
  \let \premise \stex_proof_premise:
  \intarray_gzero:N \l_@@_counter_intarray
  \intarray_gset:Nnn \l_@@_counter_intarray 1 1
  \_@@_spf_args:n{#1}
  \stex_if_smsmode:TF {
    \str_if_empty:NF \spfid {
      \stex_ref_new_doc_target:n \spfid
    }
  }{
    \seq_clear:N \l_tmpa_seq
    \clist_map_inline:Nn \l_@@_spf_for_clist {
      \tl_if_empty:nF{ ##1 }{
        \stex_get_symbol:n { ##1 }
        \exp_args:NNo \seq_put_right:Nn \l_tmpa_seq {
          \l_stex_get_symbol_uri_str
        }
      }
    }
    \exp_args:Nnnx
    \begin{stex_annotate_env}{sproof}{\seq_use:Nn \l_tmpa_seq {,}}
    \str_if_empty:NF \spftype {
      \stex_annotate_invisible:nnn{type}{\spftype}{}
    }

    \clist_set:No \l_tmpa_clist \spftype
    \tl_clear:N \l_tmpa_tl
    \clist_map_inline:Nn \l_tmpa_clist {
      \tl_if_exist:cT {_@@_sproof_##1_start:}{
        \tl_set:Nn \l_tmpa_tl {\use:c{_@@_sproof_##1_start:}}
      }
      \exp_args:No \str_if_eq:nnT \c_@@_flow_str {##1} {
        \tl_set:Nn \l_tmpa_tl {\use:n{}}
      }
    }
    \tl_if_empty:NTF \l_tmpa_tl {
      \_@@_sproof_start:
    }{
      \l_tmpa_tl
    }{~#2}
    \str_if_empty:NF \spfid {
      \stex_ref_new_doc_target:n \spfid
    }
    \begin{description}
  }
  \stex_smsmode_do:
}{
  \stex_if_smsmode:F{
    \end{description}
    \clist_set:No \l_tmpa_clist \spftype
    \tl_clear:N \l_tmpa_tl
    \clist_map_inline:Nn \l_tmpa_clist {
      \tl_if_exist:cT {_@@_sproof_##1_end:}{
        \tl_set:Nn \l_tmpa_tl {\use:c{_@@_sproof_##1_end:}}
      }
    }
    \tl_if_empty:NTF \l_tmpa_tl {
      \_@@_sproof_end:
    }{
      \l_tmpa_tl
    }
    \end{stex_annotate_env}
  }
}

\cs_new_protected:Nn \_@@_sproof_start: {
  \par\noindent\titleemph{
    \tl_if_empty:NTF \spftype {
      \spf@proof@kw
    }{
      \spftype
    }
  }:
}
\cs_new_protected:Nn \_@@_sproof_end: {\sproofend}

\newcommand\stexpatchproof[3][] {
  \str_set:Nx \l_tmpa_str{ #1 }
  \str_if_empty:NTF \l_tmpa_str {
    \tl_set:Nn \_@@_sproof_start: { #2 }
    \tl_set:Nn \_@@_sproof_end: { #3 }
  }{
    \exp_after:wN \tl_set:Nn \csname _@@_sproof_#1_start:\endcsname{ #2 }
    \exp_after:wN \tl_set:Nn \csname _@@_sproof_#1_end:\endcsname{ #3 }
  }
}
%    \end{macrocode}
% \end{environment}
% 
% \begin{macro}{\spfidea}
%    \begin{macrocode}
\newcommand\spfidea[2][]{
  \_@@_spf_args:n{#1}
  \titleemph{
    \tl_if_empty:NTF \spftype {Proof~Idea}{
      \spftype
    }:
  }~#2
  \sproofend
}
%    \end{macrocode}
% \end{macro}
%
% The next two environments (proof steps) and comments, are mostly semantical, they take
% |KeyVal| arguments that specify their semantic role. In draft mode, they read these
% values and show them. If the surrounding proof had |display=flow|, then no new |\item| is
% generated, otherwise it is. In any case, the proof step number (at the current level) is
% incremented.
% \begin{environment}{spfstep}
%    \begin{macrocode}
\newenvironment{spfstep}[1][]{
  \_@@_spf_args:n{#1}
  \stex_if_smsmode:TF {
    \str_if_empty:NF \spfid {
      \stex_ref_new_doc_target:n \spfid
    }
  }{
    \@in@omtexttrue
    \clist_set:No \l_tmpa_clist \spftype
    \tl_set:Nn \l_tmpa_tl {
      \item[\sproofnumber]
      \bool_set_true:N \l_@@_inc_counter_bool
    }
    \clist_map_inline:Nn \l_tmpa_clist {
      \exp_args:No \str_if_eq:nnT \c_@@_flow_str {##1} {
        \tl_clear:N \l_tmpa_tl
      }
    }
    \l_tmpa_tl
    \seq_clear:N \l_tmpa_seq
    \clist_map_inline:Nn \l_@@_spf_for_clist {
      \tl_if_empty:nF{ ##1 }{
        \stex_get_symbol:n { ##1 }
        \exp_args:NNo \seq_put_right:Nn \l_tmpa_seq {
          \l_stex_get_symbol_uri_str
        }
      }
    }
    \exp_args:Nnnx
    \begin{stex_annotate_env}{spfstep}{\seq_use:Nn \l_tmpa_seq {,}}
    \str_if_empty:NF \spftype {
      \stex_annotate_invisible:nnn{type}{\spftype}{}
    }
    \tl_if_empty:NF \spftitle {
      {(\titleemph{\spftitle})\enspace}
    }
    \str_if_empty:NF \spfid {
      \stex_ref_new_doc_target:n \spfid
    }
  }
  \stex_smsmode_do:
  \ignorespacesandpars
}{
  \bool_if:NT \l_@@_inc_counter_bool {
    \_@@_inc_counter:
  }
  \stex_if_smsmode:F {
    \end{stex_annotate_env}
  }
}
%    \end{macrocode}
% \end{environment}
%
% \begin{environment}{spfcomment}
%    \begin{macrocode}
\newenvironment{spfcomment}[1][]{
  \_@@_spf_args:n{#1}
  \clist_set:No \l_tmpa_clist \spftype
  \tl_set:Nn \l_tmpa_tl {
    \item[\sproofnumber]
    \bool_set_true:N \l_@@_inc_counter_bool
  }
  \clist_map_inline:Nn \l_tmpa_clist {
    \exp_args:No \str_if_eq:nnT \c_@@_flow_str {##1} {
      \tl_clear:N \l_tmpa_tl
    }
  }
  \l_tmpa_tl
}{
  \bool_if:NT \l_@@_inc_counter_bool {
    \_@@_inc_counter:
  }
}
%    \end{macrocode}
% \end{environment}
%
% The next two environments also take a |KeyVal| argument, but also a regular one, which
% contains a start text. Both environments start a new numbered proof level.
%
% \begin{environment}{subproof}
%   In the |subproof| environment, a new (lower-level) proproofof environment is started.
%    \begin{macrocode}
\newenvironment{subproof}[2][]{
  \_@@_spf_args:n{#1}
  \stex_if_smsmode:TF{
    \str_if_empty:NF \spfid {
      \stex_ref_new_doc_target:n \spfid
    }
  }{
    \seq_clear:N \l_tmpa_seq
    \clist_map_inline:Nn \l_@@_spf_for_clist {
      \tl_if_empty:nF{ ##1 }{
        \stex_get_symbol:n { ##1 }
        \exp_args:NNo \seq_put_right:Nn \l_tmpa_seq {
          \l_stex_get_symbol_uri_str
        }
      }
    }
    \exp_args:Nnnx
    \begin{stex_annotate_env}{subproof}{\seq_use:Nn \l_tmpa_seq {,}}
    \str_if_empty:NF \spftype {
      \stex_annotate_invisible:nnn{type}{\spftype}{}
    }

    \clist_set:No \l_tmpa_clist \spftype
    \tl_set:Nn \l_tmpa_tl {
      \item[\sproofnumber]
      \bool_set_true:N \l_@@_inc_counter_bool
    }
    \clist_map_inline:Nn \l_tmpa_clist {
      \exp_args:No \str_if_eq:nnT \c_@@_flow_str {##1} {
        \tl_clear:N \l_tmpa_tl
      }
    }
    \l_tmpa_tl
    \tl_if_empty:NF \spftitle {
      {(\titleemph{\spftitle})\enspace}
    }
    {~#2}
    \str_if_empty:NF \spfid {
      \stex_ref_new_doc_target:n \spfid
    }
  }
  \_@@_add_counter:
  \stex_smsmode_do:
}{
  \_@@_remove_counter:
  \bool_if:NT \l_@@_inc_counter_bool {
    \_@@_inc_counter:
  }
  \stex_if_smsmode:F{
    \end{stex_annotate_env}
  }
}
%    \end{macrocode}
% \end{environment}
%
% \begin{environment}{spfcases}
%   In the |pfcases| environment, the start text is displayed as the first comment of the
%   proof.
%    \begin{macrocode}
\newenvironment{spfcases}[2][]{
  \tl_if_empty:nTF{#1}{
    \begin{subproof}[method=by-cases]{#2}
  }{
    \begin{subproof}[#1,method=by-cases]{#2}
  }
}{
  \end{subproof}
}
%    \end{macrocode}
% \end{environment}
%
% \begin{environment}{spfcase}
%    In the |pfcase| environment, the start text is displayed specification of the case
%    after the |\item|
%    \begin{macrocode}
\newenvironment{spfcase}[2][]{
  \_@@_spf_args:n{#1}
  \stex_if_smsmode:TF {
    \str_if_empty:NF \spfid {
      \stex_ref_new_doc_target:n \spfid
    }
  }{
    \seq_clear:N \l_tmpa_seq
    \clist_map_inline:Nn \l_@@_spf_for_clist {
      \tl_if_empty:nF{ ##1 }{
        \stex_get_symbol:n { ##1 }
        \exp_args:NNo \seq_put_right:Nn \l_tmpa_seq {
          \l_stex_get_symbol_uri_str
        }
      }
    }
    \exp_args:Nnnx
    \begin{stex_annotate_env}{spfcase}{\seq_use:Nn \l_tmpa_seq {,}}
    \str_if_empty:NF \spftype {
      \stex_annotate_invisible:nnn{type}{\spftype}{}
    }
    \clist_set:No \l_tmpa_clist \spftype
    \tl_set:Nn \l_tmpa_tl {
      \item[\sproofnumber]
      \bool_set_true:N \l_@@_inc_counter_bool
    }
    \clist_map_inline:Nn \l_tmpa_clist {
      \exp_args:No \str_if_eq:nnT \c_@@_flow_str {##1} {
        \tl_clear:N \l_tmpa_tl
      }
    }
    \l_tmpa_tl
    \tl_if_empty:nF{#2}{
      \titleemph{#2}:~
    }
  }
  \_@@_add_counter:
  \stex_smsmode_do:
}{
  \_@@_remove_counter:
  \bool_if:NT \l_@@_inc_counter_bool {
    \_@@_inc_counter:
  }
  \stex_if_smsmode:F{
    \clist_set:No \l_tmpa_clist \spftype
    \tl_set:Nn \l_tmpa_tl{\sproofend}
    \clist_map_inline:Nn \l_tmpa_clist {
      \exp_args:No \str_if_eq:nnT \c_@@_flow_str {##1} {
        \tl_clear:N \l_tmpa_tl
      }
    }
    \l_tmpa_tl
    \end{stex_annotate_env}
  }
}
%    \end{macrocode}
% \end{environment}
%
% \begin{environment}{spfcase}
%    similar to |spfcase|, takes a third argument. 
%    \begin{macrocode}
\newcommand\spfcasesketch[3][]{
  \begin{spfcase}[#1]{#2}#3\end{spfcase}
}
%    \end{macrocode}
% \end{environment}
%
% \subsection{Justifications}
%
% We define the actions that are undertaken, when the keys for justifications are
% encountered. Here this is very simple, we just define an internal macro with the value,
% so that we can use it later.
%    \begin{macrocode}
\keys_define:nn { stex / just }{
  id        .str_set_x:N  = \l_@@_just_id_str,
  method    .tl_set:N     = \l_@@_just_method_tl,
  premises  .tl_set:N     = \l_@@_just_premises_tl,
  args      .tl_set:N     = \l_@@_just_args_tl
}
%    \end{macrocode}
%
% The next three environments and macros are purely semantic, so we ignore the keyval
% arguments for now and only display the content.\ednote{need to do something about the
% premise in draft mode.}
%
% \begin{macro}{\spfjust}
%    \begin{macrocode}
\newcommand\spfjust[1][]{}
%    \end{macrocode}
% \end{macro}
%
% \begin{macro}{\premise}
%    \begin{macrocode}
\newcommand\stex_proof_premise:[2][]{#2}
%    \end{macrocode}
% \end{macro}
%
% \begin{macro}{\justarg}
% the |\justarg| macro is purely semantic, so we ignore the keyval arguments for now and
% only display the content.
%    \begin{macrocode}
\newcommand\justarg[2][]{#2}
%</package>
%    \end{macrocode}
% \end{macro}
% \end{implementation}
% \Finale
% \ifinfulldoc\else\printbibliography\fi
\endinput

%%% Local Variables: 
%%% mode: doctex
%%% TeX-master: t
%%% End: 
