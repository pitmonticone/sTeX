% \iffalse meta-comment 
% An Infrastructure for Structural Markup for Proofs
% Copyright (c) 2019 Michael Kohlhase, all rights reserved
%               this file is released under the
%               LaTeX Project Public License (LPPL)
%
% The original of this file is in the public repository at 
% http://github.com/sLaTeX/sTeX/
% \fi
% 
% \iffalse
%<*driver>
\def\bibfolder{../../lib/bib}
\RequirePackage{paralist}
\documentclass[full,kernel]{l3doc}
\usepackage[dvipsnames]{xcolor}
\usepackage[utf8]{inputenc}
\usepackage[T1]{fontenc}
%\usepackage{document-structure}
\usepackage[showmods,debug=all,lang={de,en}, mathhub=./tests]{stex}
\usepackage{url,array,float,textcomp}
\usepackage[show]{ed}
\usepackage[hyperref=auto,style=alphabetic]{biblatex}
\addbibresource{\bibfolder/kwarcpubs.bib}
\addbibresource{\bibfolder/extpubs.bib}
\addbibresource{\bibfolder/kwarccrossrefs.bib}
\addbibresource{\bibfolder/extcrossrefs.bib}
\usepackage{amssymb}
\usepackage{amsfonts}
\usepackage{xspace}
\usepackage{hyperref}

\usepackage{morewrites}


\makeindex
\floatstyle{boxed}
\newfloat{exfig}{thp}{lop}
\floatname{exfig}{Example}

\usepackage{stex-tests}

\MakeShortVerb{\|}

\def\scsys#1{{{\sc #1}}\index{#1@{\sc #1}}\xspace}
\def\mmt{\textsc{Mmt}\xspace}
\def\xml{\scsys{Xml}}
\def\mathml{\scsys{MathML}}
\def\omdoc{\scsys{OMDoc}}
\def\openmath{\scsys{OpenMath}}
\def\latexml{\scsys{LaTeXML}}
\def\perl{\scsys{Perl}}
\def\cmathml{Content-{\sc MathML}\index{Content {\sc MathML}}\index{MathML@{\sc MathML}!content}}
\def\activemath{\scsys{ActiveMath}}
\def\twin#1#2{\index{#1!#2}\index{#2!#1}}
\def\twintoo#1#2{{#1 #2}\twin{#1}{#2}}
\def\atwin#1#2#3{\index{#1!#2!#3}\index{#3!#2 (#1)}}
\def\atwintoo#1#2#3{{#1 #2 #3}\atwin{#1}{#2}{#3}}
\def\cT{\mathcal{T}}\def\cD{\mathcal{D}}

\def\fileversion{3.0}
\def\filedate{\today}

\RequirePackage{pdfcomment}

\ExplSyntaxOn\makeatletter
\cs_set_protected:Npn \@comp #1 #2 {
  \pdftooltip {
    \textcolor{blue}{#1}
  } { #2 }
}

\cs_set_protected:Npn \@defemph #1 #2 {
  \pdftooltip { 
    \textbf{\textcolor{magenta}{#1}}
  } { #2 }
}
\makeatother\ExplSyntaxOff

\begin{document}
  \DocInput{\jobname.dtx}
\end{document}
%</driver>
% \fi
% 
% \iffalse\CheckSum{341}\fi
%
% ^^A \changes{v0.9}{2005/06/14}{First Version with Documentation}
% 
% \GetFileInfo{sproof.sty}
% 
% \title{\sTeX-Proofs: Structural Markup for Proofs\thanks{Version {\fileversion} (last revised
%        {\filedate})}}
%    \author{Michael Kohlhase, Dennis Müller\\
%            FAU Erlangen-N\"urnberg\\
%            \url{http://kwarc.info/kohlhase}}
% \maketitle
%
%\ifinfulldoc\else
% This is the documentation for the \pkg{stex-proofs} package.
% For a more high-level introduction, 
%  see \href{\basedocurl/manual.pdf}{the \sTeX Manual} or the
% \href{\basedocurl/stex.pdf}{full \sTeX documentation}.
%
% \textcolor{red}{TODO: sproofs documentation}
% \fi
%
% \begin{documentation}
%
%   The |sproof| package is part of the {\stex} collection, a version of {\TeX/\LaTeX} that
%   allows to markup {\TeX/\LaTeX} documents semantically without leaving the document
%   format, essentially turning {\TeX/\LaTeX} into a document format for mathematical
%   knowledge management (MKM).
%
%   This package supplies macros and environment that allow to annotate the structure of
%   mathematical proofs in {\stex} files. This structure can be used by MKM systems for
%   added-value services, either directly from the \sTeX sources, or after translation.
%
% \newpage\tableofcontents\newpage
%
%\section{Introduction}\label{sec:sproof}
%
% The |sproof| ({\twintoo{semantic}{proofs}}) package supplies macros and environment that
% allow to annotate the structure of mathematical proofs in {\stex} files. This structure
% can be used by MKM systems for added-value services, either directly from the \sTeX
% sources, or after translation. Even though it is part of the {\stex} collection, it can
% be used independently, like it's sister package |statements|.
%
% {\stex} is a version of {\TeX/\LaTeX} that allows to markup {\TeX/\LaTeX} documents
% semantically without leaving the document format, essentially turning {\TeX/\LaTeX} into
% a document format for mathematical knowledge management (MKM).
%
% \begin{exfig}\scriptsize
% \begin{verbatim}
% \begin{sproof}[id=simple-proof]
%    {We prove that $\sum_{i=1}^n{2i-1}=n^{2}$ by induction over $n$}
%   \begin{spfcases}{For the induction we have to consider the following cases:}
%    \begin{spfcase}{$n=1$}
%     \begin{spfstep}[type=inline] then we compute $1=1^2$\end{spfstep}
%    \end{spfcase}
%    \begin{spfcase}{$n=2$}
%       \begin{sproofcomment}[type=inline]
%         This case is not really necessary, but we do it for the
%         fun of it (and to get more intuition).
%       \end{sproofcomment}
%       \begin{spfstep}[type=inline] We compute $1+3=2^{2}=4$.\end{spfstep}
%    \end{spfcase}
%    \begin{spfcase}{$n>1$}
%       \begin{spfstep}[type=assumption,id=ind-hyp]
%         Now, we assume that the assertion is true for a certain $k\geq 1$,
%         i.e. $\sum_{i=1}^k{(2i-1)}=k^{2}$.
%       \end{spfstep}
%       \begin{sproofcomment}
%         We have to show that we can derive the assertion for $n=k+1$ from
%         this assumption, i.e. $\sum_{i=1}^{k+1}{(2i-1)}=(k+1)^{2}$.
%       \end{sproofcomment}
%       \begin{spfstep}
%         We obtain $\sum_{i=1}^{k+1}{2i-1}=\sum_{i=1}^k{2i-1}+2(k+1)-1$
%         \begin{justification}[method=arith:split-sum]
%           by splitting the sum.
%         \end{justification}
%       \end{spfstep}
%       \begin{spfstep}
%         Thus we have $\sum_{i=1}^{k+1}{(2i-1)}=k^2+2k+1$
%         \begin{justification}[method=fertilize]
%           by inductive hypothesis.
%         \end{justification}
%       \end{spfstep}
%       \begin{spfstep}[type=conclusion]
%         We can \begin{justification}[method=simplify]simplify\end{justification} 
%         the right-hand side to ${k+1}^2$, which proves the assertion.
%       \end{spfstep}
%    \end{spfcase}
%     \begin{spfstep}[type=conclusion]
%       We have considered all the cases, so we have proven the assertion.
%     \end{spfstep}
%   \end{spfcases}
% \end{sproof}
% \end{verbatim}
% \vspace*{-.5cm}
% \caption{A very explicit proof, marked up semantically}\label{fig:proof:src}
% \end{exfig}
%
% We will go over the general intuition by way of our running example (see
% Figure~\ref{fig:proof:src} for the source and Figure~\ref{fig:proof:result} for the
% formatted result).\ednote{talk a bit more about proofs and their structure,... maybe
%   copy from OMDoc spec. }
%
% \section{The User Interface}
% 
% \subsection{Package Options}\label{sec:user:options}
% 
% The |sproof| package takes a single option: \DescribeMacro{showmeta}|showmeta|. If
% this is set, then the metadata keys are shown (see~\cite{Kohlhase:metakeys} for details
% and customization options).
%
% \subsection{Proofs and Proof steps}
%
% \DescribeEnv{sproof} The |proof| environment is the main container for proofs. It takes
% an optional |KeyVal| argument that allows to specify the |id| (identifier) and |for|
% (for which assertion is this a proof) keys. The regular argument of the |proof|
% environment contains an introductory comment, that may be used to announce the proof
% style. The |proof| environment contains a sequence of |\step|, |proofcomment|, and
% |pfcases| environments that are used to markup the proof steps. The |proof| environment
% has a variant |Proof|, which does not use the proof end marker. This is convenient, if a
% proof ends in a case distinction, which brings it's own proof end marker with it.
% \DescribeEnv{sProof} The |Proof| environment is a variant of |proof| that does not mark
% the end of a proof with a little box; presumably, since one of the subproofs already has
% one and then a box supplied by the outer proof would generate an otherwise empty line.
% \DescribeMacro{\spfidea} The |\spfidea| macro allows to give a one-paragraph
% description of the proof idea.
% 
% For one-line proof sketches, we use the \DescribeMacro{spfsketch}|\spfsketch| macro,
% which takes the |KeyVal| argument as |sproof| and another one: a natural language text
% that sketches the proof.
%
% \DescribeEnv{spfstep} Regular proof steps are marked up with the |step| environment, which
% takes an optional |KeyVal| argument for annotations. A proof step usually contains a
% local assertion (the text of the step) together with some kind of evidence that this can
% be derived from already established assertions.
%
% Note that both |\premise| and |\justarg| can be used with an empty second argument to
% mark up premises and arguments that are not explicitly mentioned in the text.
%
% \begin{exfig}
% \begin{sproof}[id=simple-proof]
%   {We prove that $\sum_{i=1}^n{2i-1}=n^{2}$ by induction over $n$}
%   \begin{spfcases}{For the induction we have to consider the following cases:}
%     \begin{spfcase}{$n=1$}
%       \begin{spfstep}[type=inline] then we compute $1=1^2$\end{spfstep}
%     \end{spfcase}
%     \begin{spfcase}{$n=2$}
%       \begin{sproofcomment}[type=inline]
%          This case is not really necessary, but we do it for the fun
%          of it (and to get more intuition).
%       \end{sproofcomment}
%       \begin{spfstep}[type=inline]
%          We compute $1+3=2^{2}=4$
%       \end{spfstep}
%     \end{spfcase}
%     \begin{spfcase}{$n>1$}
%       \begin{spfstep}[type=hypothesis,id=ind-hyp]
%         Now, we assume that the assertion is true for a certain $k\geq 1$, i.e.
%         $\sum_{i=1}^k{(2i-1)}=k^{2}$.
%       \end{spfstep}
%       \begin{sproofcomment}
%         We have to show that we can derive the assertion for $n=k+1$ from this
%         assumption, i.e.  $\sum_{i=1}^{k+1}{(2i-1)}=(k+1)^{2}$.
%       \end{sproofcomment}
%       \begin{spfstep}[id=splitit]
%         We obtain $\sum_{i=1}^{k+1}{(2i-1)}=\sum_{i=1}^k{(2i-1)}+2(k+1)-1$
%        \begin{justification}[method=arith:split-sum]
%          by splitting the sum
%        \end{justification}
%      \end{spfstep}
%      \begin{spfstep}[id=byindhyp]
%        Thus we have $\sum_{i=1}^{k+1}{(2i-1)}=k^2+2k+1$
%        \begin{justification}[method=fertilize]
%          by \premise[ind-hyp]{inductive hypothesis}.
%        \end{justification}
%      \end{spfstep}
%      \begin{spfstep}[type=conclusion]
%        We can \begin{justification}[method=simplify-eq]
%          simplify the {\justarg[rhs]{right-hand side}}
%        \end{justification} to $(k+1)^2$, which proves the assertion.
%      \end{spfstep}
%    \end{spfcase}
%    \begin{spfstep}[type=conclusion]
%      We have considered all the cases, so we have proven the assertion.
%    \end{spfstep}
%   \end{spfcases}
% \end{sproof}
% \caption{The formatted result of the proof in Figure~\ref{fig:proof:src}}\label{fig:proof:result}
% \end{exfig}
%
% \subsection{Justifications}
% 
% \DescribeEnv{justification} This evidence is marked up with the |justification|
% environment in the |sproof| package. This environment totally invisible to the formatted
% result; it wraps the text in the proof step that corresponds to the evidence. The
% environment takes an optional |KeyVal| argument, which can have the |method| key, whose
% value is the name of a proof method (this will only need to mean something to the
% application that consumes the semantic annotations). Furthermore, the justification can
% contain ``premises'' (specifications to assertions that were used justify the step) and
% ``arguments'' (other information taken into account by the proof method).
%
% \DescribeMacro{\premise} The |\premise| macro allows to mark up part of the text as
% reference to an assertion that is used in the argumentation. In the example in
% Figure~\ref{fig:proof:src} we have used the |\premise| macro to identify the inductive
% hypothesis.
%
% \DescribeMacro{\justarg} The |\justarg| macro is very similar to |\premise| with the
% difference that it is used to mark up arguments to the proof method. Therefore the
% content of the first argument is interpreted as a mathematical object rather than as an
% identifier as in the case of |\premise|. In our example, we specified that the
% simplification should take place on the right hand side of the equation. Other examples
% include proof methods that instantiate. Here we would indicate the substituted object in
% a |\justarg| macro.
%
% \subsection{Proof Structure}
% 
% \DescribeEnv{subproof} The |pfcases| environment is used to mark up a subproof. This
% environment takes an optional |KeyVal| argument for semantic annotations and a second
% argument that allows to specify an introductory comment (just like in the |proof|
% environment). The \DescribeMacro{method}|method| key can be used to give the name of the
% proof method executed to make this subproof.
% 
% \DescribeEnv{spfcases} The |pfcases| environment is used to mark up a proof by
% cases. Technically it is a variant of the |subproof| where the |method| is
% |by-cases|. Its contents are |spfcase| environments that mark up the cases one by one.
%
% \DescribeEnv{spfcase} The content of a |pfcases| environment are a sequence of case
% proofs marked up in the |pfcase| environment, which takes an optional |KeyVal| argument
% for semantic annotations. The second argument is used to specify the the description of
% the case under consideration. The content of a |pfcase| environment is the same as that
% of a |proof|, i.e. |step|s, |proofcomment|s, and |pfcases|
% environments. \DescribeMacro{\spfcasesketch}|\spfcasesketch| is a variant of the |spfcase|
% environment that takes the same arguments, but instead of the |spfstep|s in the body
% uses a third argument for a proof sketch.
%
% \DescribeEnv{sproofcomment} The |proofcomment| environment is much like a |step|, only
% that it does not have an object-level assertion of its own. Rather than asserting some
% fact that is relevant for the proof, it is used to explain where the proof is going,
% what we are attempting to to, or what we have achieved so far. As such, it cannot be the
% target of a |\premise|.
% 
% \subsection{Proof End Markers}
% 
% Traditionally, the end of a mathematical proof is marked with a little box at the end of
% the last line of the proof (if there is space and on the end of the next line if there
% isn't), like so:\sproofend
% 
% The |sproof| package provides the \DescribeMacro{\sproofend}|\sproofend| macro for
% this. If a different symbol for the proof end is to be used (e.g. {\sl{q.e.d}}), then
% this can be obtained by specifying it using the
% \DescribeMacro{\sProofEndSymbol}|\sProofEndSymbol| configuration macro (e.g. by specifying
% |\sProofEndSymbol{q.e.d}|).
% 
% Some of the proof structuring macros above will insert proof end symbols for sub-proofs,
% in most cases, this is desirable to make the proof structure explicit, but sometimes
% this wastes space (especially, if a proof ends in a case analysis which will supply its
% own proof end marker). To suppress it locally, just set |proofend={}| in them or use use
% |\sProofEndSymbol{}|.
% 
% \subsection{Configuration of the Presentation}\label{sec:user:conf}
%
% Finally, we provide configuration hooks in Figure~\ref{fig:hooks} for the keywords in
% proofs. These are mainly intended for package authors building on |statements|, e.g. for
% multi-language support.\ednote{we might want to develop an extension
% \texttt{sproof-babel} in the future.}.
%\begin{figure}[ht]\centering
% \begin{tabular}{|lll|}\hline
% Environment & configuration macro & value\\\hline\hline
% \texttt{sproof} & \texttt{\textbackslash spf@proof@kw} & \makeatletter\spf@proof@kw\\\hline
% \texttt{sketchproof} & \texttt{\textbackslash spf@sketchproof@kw} & \makeatletter\spf@proofsketch@kw\\\hline
% \end{tabular}
% \caption{Configuration Hooks for Semantic Proof Markup}\label{fig:hooks}
% \end{figure}
% The proof step labels can be customized via the
% \DescribeMacro{\pstlabelstyle}|\pstlabelstyle| macro: |\pstlabelstyle{|\meta{style}|}|
% sets the style; see Figure~\ref{fig:pstlabel} for an overview of styles. Package writers
% can add additional styles by adding a macro |\pst@make@label@|\meta{style} that takes
% two arguments: a comma-separated list of ordinals that make up the prefix and the current
% ordinal. Note that comma-separated lists can be conveniently iterated over by the
% {\LaTeX} |\@for|\ldots|:=|\ldots|\do{|\ldots|}| macro; see Figure~\ref{fig:pstlabel} for
% examples.
%
% \section{Limitations}\label{sec:limitations}
% 
% In this section we document known limitations. If you want to help alleviate them,
% please feel free to contact the package author. Some of them are currently discussed in
% the \sTeX issue tracker at \cite{sTeX:github:on}.
% \begin{enumerate}
% \item The numbering scheme of proofs cannot be changed. It is more geared for teaching
%   proof structures (the author's main use case) and not for writing papers.\lec{reported
%   by Tobias Pfeiffer (fixed)}
% \item currently proof steps are formatted by the {\LaTeX} |description| environment. We
%   would like to configure this, e.g. to use the |inparaenum| environment for more
%   condensed proofs. I am just not sure what the best user interface would be I can
%   imagine redefining an internal environment |spf@proofstep@list| or adding a key
%   |prooflistenv| to the |proof| environment that allows to specify the environment
%   directly. Maybe we should do both. 
% \end{enumerate}
%
% \end{documentation}
%
% \begin{implementation}
%
% \section{The Implementation} 
% 
% \subsection{Package Options}\label{sec:impl:options}
%
% We declare some switches which will modify the behavior according to the package
% options. Generally, an option |xxx| will just set the appropriate switches to true
% (otherwise they stay false).\ednote{need an implementation for {\latexml}}
%
%    \begin{macrocode}
%<*package>
%<@@=stex_sproof>

%%%%%%%%%%%%%   sproof.dtx   %%%%%%%%%%%%%

%    \end{macrocode}
%
%
% \subsection{Proofs}\label{sec:impl:proofs}
% 
% We first define some keys for the |proof| environment.
%    \begin{macrocode}
\keys_define:nn { stex / spf } {
  id          .str_set_x:N  = \spfid,
  for         .clist_set:N  = \l_@@_spf_for_clist ,
  from        .tl_set:N     = \l_@@_spf_from_tl ,
  proofend    .tl_set:N     = \l_@@_spf_proofend_tl,
  type        .str_set_x:N  = \spftype,
  title       .tl_set:N     = \spftitle,
  continues   .tl_set:N     = \l_@@_spf_continues_tl,
  functions   .tl_set:N     = \l_@@_spf_functions_tl,
  method      .tl_set:N     = \l_@@_spf_method_tl
}
\cs_new_protected:Nn \_@@_spf_args:n {
	\str_clear:N \spfid
	\tl_clear:N \l_@@_spf_for_tl
	\tl_clear:N \l_@@_spf_from_tl
	\tl_set:Nn \l_@@_spf_proofend_tl {\sproof@box}
	\str_clear:N \spftype
	\tl_clear:N \spftitle
	\tl_clear:N \l_@@_spf_continues_tl
	\tl_clear:N \l_@@_spf_functions_tl
	\tl_clear:N \l_@@_spf_method_tl
  \bool_set_false:N \l_@@_inc_counter_bool
	\keys_set:nn { stex / spf }{ #1 }
}
%    \end{macrocode}
%
% \begin{macro}{\c_@@_flow_str}
% We define this macro, so that we can test whether the |display| key has the value |flow|
%    \begin{macrocode}
\str_set:Nn\c_@@_flow_str{inline}
%    \end{macrocode}
% \end{macro}
%
% For proofs, we will have to have deeply nested structures of enumerated list-like
% environments. However, {\LaTeX} only allows |enumerate| environments up to nesting depth
% 4 and general list environments up to listing depth 6. This is not enough for us.
% Therefore we have decided to go along the route proposed by Leslie Lamport to use a
% single top-level list with dotted sequences of numbers to identify the position in the
% proof tree. Unfortunately, we could not use his |pf.sty| package directly, since it does
% not do automatic numbering, and we have to add keyword arguments all over the place, to
% accomodate semantic information.
%
% \begin{environment}{pst@with@label}
%   This environment manages\footnote{This gets the labeling right but only works 8 levels
%   deep} the path labeling of the proof steps in the description environment of the
%   outermost |proof| environment. The argument is the label prefix up to now; which we
%   cache in |\pst@label| (we need evaluate it first, since are in the right place
%   now!). Then we increment the proof depth which is stored in |\count10| (lower counters
%   are used by {\TeX} for page numbering) and initialize the next level counter
%   |\count\count10| with 1. In the end call for this environment, we just decrease the
%   proof depth counter by 1 again.
%    \begin{macrocode}
\intarray_new:Nn\l_@@_counter_intarray{50}
\cs_new_protected:Npn \sproofnumber {
  \int_set:Nn \l_tmpa_int {1}
  \bool_while_do:nn {
    \int_compare_p:nNn {
      \intarray_item:Nn \l_@@_counter_intarray \l_tmpa_int
    } > 0
  }{
    \intarray_item:Nn \l_@@_counter_intarray \l_tmpa_int .
    \int_incr:N \l_tmpa_int
  }
}
\cs_new_protected:Npn \_@@_inc_counter: {
  \int_set:Nn \l_tmpa_int {1}
  \bool_while_do:nn {
    \int_compare_p:nNn {
      \intarray_item:Nn \l_@@_counter_intarray \l_tmpa_int
    } > 0
  }{
    \int_incr:N \l_tmpa_int
  }
  \int_compare:nNnF \l_tmpa_int = 1 {
    \int_decr:N \l_tmpa_int
  }
  \intarray_gset:Nnn \l_@@_counter_intarray \l_tmpa_int {
    \intarray_item:Nn \l_@@_counter_intarray \l_tmpa_int + 1
  }
}

\cs_new_protected:Npn \_@@_add_counter: {
  \int_set:Nn \l_tmpa_int {1}
  \bool_while_do:nn {
    \int_compare_p:nNn {
      \intarray_item:Nn \l_@@_counter_intarray \l_tmpa_int
    } > 0
  }{
    \int_incr:N \l_tmpa_int
  }
  \intarray_gset:Nnn \l_@@_counter_intarray \l_tmpa_int { 1 }
}

\cs_new_protected:Npn \_@@_remove_counter: {
  \int_set:Nn \l_tmpa_int {1}
  \bool_while_do:nn {
    \int_compare_p:nNn {
      \intarray_item:Nn \l_@@_counter_intarray \l_tmpa_int
    } > 0
  }{
    \int_incr:N \l_tmpa_int
  }
  \int_decr:N \l_tmpa_int
  \intarray_gset:Nnn \l_@@_counter_intarray \l_tmpa_int { 0 }
}
%    \end{macrocode}
% \end{environment}
%
%
%\begin{macro}{\sproofend}
%    This macro places a little box at the end of the line if there is space, or at the
%    end of the next line if there isn't
%    \begin{macrocode}
\def\sproof@box{
  \hbox{\vrule\vbox{\hrule width 6 pt\vskip 6pt\hrule}\vrule}
}
\def\sproofend{
  \tl_if_empty:NF \l_@@_spf_proofend_tl {
    \hfil\null\nobreak\hfill\l_@@_spf_proofend_tl\par\smallskip
  }
}
%    \end{macrocode}
% \end{macro}
%
% \begin{macro}{spf@*@kw}
%    \begin{macrocode}
\def\spf@proofsketch@kw{Proof~Sketch}
\def\spf@proof@kw{Proof}
\def\spf@step@kw{Step}
%    \end{macrocode}
% \end{macro}
%
% For the other languages, we set up triggers
%    \begin{macrocode}
\AddToHook{begindocument}{
  \ltx@ifpackageloaded{babel}{
    \makeatletter
    \clist_set:Nx \l_tmpa_clist {\bbl@loaded}
    \clist_if_in:NnT \l_tmpa_clist {ngerman}{
      \input{sproof-ngerman.ldf}
    }
    \clist_if_in:NnT \l_tmpa_clist {finnish}{
      \input{sproof-finnish.ldf}
    }
    \clist_if_in:NnT \l_tmpa_clist {french}{
      \input{sproof-french.ldf}
    }
    \clist_if_in:NnT \l_tmpa_clist {russian}{
      \input{sproof-russian.ldf}
    }
    \makeatother
  }{}
}
%    \end{macrocode}
%
% \begin{macro}{spfsketch}
%    \begin{macrocode}
\newcommand\spfsketch[2][]{
  \begingroup
  \let \premise \stex_proof_premise:
  \_@@_spf_args:n{#1}
  \stex_if_smsmode:TF {
    \str_if_empty:NF \spfid {
      \stex_ref_new_doc_target:n \spfid
    }
  }{
    \seq_clear:N \l_tmpa_seq
    \clist_map_inline:Nn \l_@@_spf_for_clist {
      \tl_if_empty:nF{ ##1 }{
        \stex_get_symbol:n { ##1 }
        \exp_args:NNo \seq_put_right:Nn \l_tmpa_seq {
          \l_stex_get_symbol_uri_str
        }
      }
    }
    \exp_args:Nnx
    \stex_annotate:nnn{proofsketch}{\seq_use:Nn \l_tmpa_seq {,}}{
      \str_if_empty:NF \spftype {
        \stex_annotate_invisible:nnn{type}{\spftype}{}
      }
      \clist_set:No \l_tmpa_clist \spftype
      \tl_set:Nn \l_tmpa_tl {
        \titleemph{
          \tl_if_empty:NTF \spftitle {
            \spf@proofsketch@kw
          }{
            \spftitle
          }
        }:~
      }
      \clist_map_inline:Nn \l_tmpa_clist {
        \exp_args:No \str_if_eq:nnT \c_@@_flow_str {##1} {
          \tl_clear:N \l_tmpa_tl
        }
      }
      \str_if_empty:NF \spfid {
        \stex_ref_new_doc_target:n \spfid
      }
      \l_tmpa_tl #2 \sproofend
    }
  }
  \endgroup
  \stex_smsmode_do:
}

%    \end{macrocode}
% \end{macro}
%
% \begin{macro}{spfeq}
%   This is very similar to |\spfsketch|, but uses a computation array\ednote{This should
%   really be more like a tabular with an ensuremath in it. or invoke text on the last
%   column}\ednote{document above}
%    \begin{macrocode}
\newenvironment{spfeq}[2][]{
  \_@@_spf_args:n{#1}
  \let \premise \stex_proof_premise:
  \stex_if_smsmode:TF {
    \str_if_empty:NF \spfid {
      \stex_ref_new_doc_target:n \spfid
    }
  }{
    \seq_clear:N \l_tmpa_seq
    \clist_map_inline:Nn \l_@@_spf_for_clist {
      \tl_if_empty:nF{ ##1 }{
        \stex_get_symbol:n { ##1 }
        \exp_args:NNo \seq_put_right:Nn \l_tmpa_seq {
          \l_stex_get_symbol_uri_str
        }
      }
    }
    \exp_args:Nnnx
    \begin{stex_annotate_env}{spfeq}{\seq_use:Nn \l_tmpa_seq {,}}
    \str_if_empty:NF \spftype {
      \stex_annotate_invisible:nnn{type}{\spftype}{}
    }

    \clist_set:No \l_tmpa_clist \spftype
    \tl_clear:N \l_tmpa_tl
    \clist_map_inline:Nn \l_tmpa_clist {
      \tl_if_exist:cT {_@@_spfeq_##1_start:}{
        \tl_set:Nn \l_tmpa_tl {\use:c{_@@_spfeq_##1_start:}}
      }
      \exp_args:No \str_if_eq:nnT \c_@@_flow_str {##1} {
        \tl_set:Nn \l_tmpa_tl {\use:n{}}
      }
    }
    \tl_if_empty:NTF \l_tmpa_tl {
      \_@@_spfeq_start:
    }{
      \l_tmpa_tl
    }{~#2}
    \str_if_empty:NF \spfid {
      \stex_ref_new_doc_target:n \spfid
    }
    \begin{displaymath}\begin{array}{rcll}
  }
  \stex_smsmode_do:
}{
  \stex_if_smsmode:F {
    \end{array}\end{displaymath}
    \clist_set:No \l_tmpa_clist \spftype
    \tl_clear:N \l_tmpa_tl
    \clist_map_inline:Nn \l_tmpa_clist {
      \tl_if_exist:cT {_@@_spfeq_##1_end:}{
        \tl_set:Nn \l_tmpa_tl {\use:c{_@@_spfeq_##1_end:}}
      }
    }
    \tl_if_empty:NTF \l_tmpa_tl {
      \_@@_spfeq_end:
    }{
      \l_tmpa_tl
    }
    \end{stex_annotate_env}
  }
}

\cs_new_protected:Nn \_@@_spfeq_start: {
  \titleemph{
    \tl_if_empty:NTF \spftitle {
      \spf@proof@kw
    }{
      \spftitle
    }
  }:
}
\cs_new_protected:Nn \_@@_spfeq_end: {\sproofend}

\newcommand\stexpatchspfeq[3][] {
    \str_set:Nx \l_tmpa_str{ #1 }
    \str_if_empty:NTF \l_tmpa_str {
      \tl_set:Nn \_@@_spfeq_start: { #2 }
      \tl_set:Nn \_@@_spfeq_end: { #3 }
    }{
      \exp_after:wN \tl_set:Nn \csname _@@_spfeq_#1_start:\endcsname{ #2 }
      \exp_after:wN \tl_set:Nn \csname _@@_spfeq_#1_end:\endcsname{ #3 }
    }
}

%    \end{macrocode}
% \end{macro}
%
% \begin{environment}{sproof}
%    In this environment, we initialize the proof depth counter |\count10| to 10, and set
%    up the description environment that will take the proof steps. At the end of the
%    proof, we position the proof end into the last line.
%    \begin{macrocode}
\newenvironment{sproof}[2][]{
  \let \premise \stex_proof_premise:
  \intarray_gzero:N \l_@@_counter_intarray
  \intarray_gset:Nnn \l_@@_counter_intarray 1 1
  \_@@_spf_args:n{#1}
  \stex_if_smsmode:TF {
    \str_if_empty:NF \spfid {
      \stex_ref_new_doc_target:n \spfid
    }
  }{
    \seq_clear:N \l_tmpa_seq
    \clist_map_inline:Nn \l_@@_spf_for_clist {
      \tl_if_empty:nF{ ##1 }{
        \stex_get_symbol:n { ##1 }
        \exp_args:NNo \seq_put_right:Nn \l_tmpa_seq {
          \l_stex_get_symbol_uri_str
        }
      }
    }
    \exp_args:Nnnx
    \begin{stex_annotate_env}{sproof}{\seq_use:Nn \l_tmpa_seq {,}}
    \str_if_empty:NF \spftype {
      \stex_annotate_invisible:nnn{type}{\spftype}{}
    }

    \clist_set:No \l_tmpa_clist \spftype
    \tl_clear:N \l_tmpa_tl
    \clist_map_inline:Nn \l_tmpa_clist {
      \tl_if_exist:cT {_@@_sproof_##1_start:}{
        \tl_set:Nn \l_tmpa_tl {\use:c{_@@_sproof_##1_start:}}
      }
      \exp_args:No \str_if_eq:nnT \c_@@_flow_str {##1} {
        \tl_set:Nn \l_tmpa_tl {\use:n{}}
      }
    }
    \tl_if_empty:NTF \l_tmpa_tl {
      \_@@_sproof_start:
    }{
      \l_tmpa_tl
    }{~#2}
    \str_if_empty:NF \spfid {
      \stex_ref_new_doc_target:n \spfid
    }
    \begin{description}
  }
  \stex_smsmode_do:
}{
  \stex_if_smsmode:F{
    \end{description}
    \clist_set:No \l_tmpa_clist \spftype
    \tl_clear:N \l_tmpa_tl
    \clist_map_inline:Nn \l_tmpa_clist {
      \tl_if_exist:cT {_@@_sproof_##1_end:}{
        \tl_set:Nn \l_tmpa_tl {\use:c{_@@_sproof_##1_end:}}
      }
    }
    \tl_if_empty:NTF \l_tmpa_tl {
      \_@@_sproof_end:
    }{
      \l_tmpa_tl
    }
    \end{stex_annotate_env}
  }
}

\cs_new_protected:Nn \_@@_sproof_start: {
  \par\noindent\titleemph{
    \tl_if_empty:NTF \spftype {
      \spf@proof@kw
    }{
      \spftype
    }
  }:
}
\cs_new_protected:Nn \_@@_sproof_end: {\sproofend}

\newcommand\stexpatchsproof[3][] {
  \str_set:Nx \l_tmpa_str{ #1 }
  \str_if_empty:NTF \l_tmpa_str {
    \tl_set:Nn \_@@_sproof_start: { #2 }
    \tl_set:Nn \_@@_sproof_end: { #3 }
  }{
    \exp_after:wN \tl_set:Nn \csname _@@_sproof_#1_start:\endcsname{ #2 }
    \exp_after:wN \tl_set:Nn \csname _@@_sproof_#1_end:\endcsname{ #3 }
  }
}
%    \end{macrocode}
% \end{environment}
% 
% \begin{macro}{\spfidea}
%    \begin{macrocode}
\newcommand\spfidea[2][]{
  \_@@_spf_args:n{#1}
  \titleemph{
    \tl_if_empty:NTF \spftype {Proof~Idea}{
      \spftype
    }:
  }~#2
  \sproofend
}
%    \end{macrocode}
% \end{macro}
%
% The next two environments (proof steps) and comments, are mostly semantical, they take
% |KeyVal| arguments that specify their semantic role. In draft mode, they read these
% values and show them. If the surrounding proof had |display=flow|, then no new |\item| is
% generated, otherwise it is. In any case, the proof step number (at the current level) is
% incremented.
% \begin{environment}{spfstep}
%    \begin{macrocode}
\newenvironment{spfstep}[1][]{
  \_@@_spf_args:n{#1}
  \stex_if_smsmode:TF {
    \str_if_empty:NF \spfid {
      \stex_ref_new_doc_target:n \spfid
    }
  }{
    \@in@omtexttrue
    \seq_clear:N \l_tmpa_seq
    \clist_map_inline:Nn \l_@@_spf_for_clist {
      \tl_if_empty:nF{ ##1 }{
        \stex_get_symbol:n { ##1 }
        \exp_args:NNo \seq_put_right:Nn \l_tmpa_seq {
          \l_stex_get_symbol_uri_str
        }
      }
    }
    \exp_args:Nnnx
    \begin{stex_annotate_env}{spfstep}{\seq_use:Nn \l_tmpa_seq {,}}
    \str_if_empty:NF \spftype {
      \stex_annotate_invisible:nnn{type}{\spftype}{}
    }
    \clist_set:No \l_tmpa_clist \spftype
    \tl_set:Nn \l_tmpa_tl {
      \item[\sproofnumber]
      \bool_set_true:N \l_@@_inc_counter_bool
    }
    \clist_map_inline:Nn \l_tmpa_clist {
      \exp_args:No \str_if_eq:nnT \c_@@_flow_str {##1} {
        \tl_clear:N \l_tmpa_tl
      }
    }
    \l_tmpa_tl
    \tl_if_empty:NF \spftitle {
      {(\titleemph{\spftitle})\enspace}
    }
    \str_if_empty:NF \spfid {
      \stex_ref_new_doc_target:n \spfid
    }
  }
  \stex_smsmode_do:
  \ignorespacesandpars
}{
  \bool_if:NT \l_@@_inc_counter_bool {
    \_@@_inc_counter:
  }
  \stex_if_smsmode:F {
    \end{stex_annotate_env}
  }
}
%    \end{macrocode}
% \end{environment}
%
% \begin{environment}{sproofcomment}
%    \begin{macrocode}
\newenvironment{sproofcomment}[1][]{
  \_@@_spf_args:n{#1}
  \clist_set:No \l_tmpa_clist \spftype
  \tl_set:Nn \l_tmpa_tl {
    \item[\sproofnumber]
    \bool_set_true:N \l_@@_inc_counter_bool
  }
  \clist_map_inline:Nn \l_tmpa_clist {
    \exp_args:No \str_if_eq:nnT \c_@@_flow_str {##1} {
      \tl_clear:N \l_tmpa_tl
    }
  }
  \l_tmpa_tl
}{
  \bool_if:NT \l_@@_inc_counter_bool {
    \_@@_inc_counter:
  }
}
%    \end{macrocode}
% \end{environment}
%
% The next two environments also take a |KeyVal| argument, but also a regular one, which
% contains a start text. Both environments start a new numbered proof level.
%
% \begin{environment}{subproof}
%   In the |subproof| environment, a new (lower-level) proproofof environment is started.
%    \begin{macrocode}
\newenvironment{subproof}[2][]{
  \_@@_spf_args:n{#1}
  \stex_if_smsmode:TF{
    \str_if_empty:NF \spfid {
      \stex_ref_new_doc_target:n \spfid
    }
  }{
    \seq_clear:N \l_tmpa_seq
    \clist_map_inline:Nn \l_@@_spf_for_clist {
      \tl_if_empty:nF{ ##1 }{
        \stex_get_symbol:n { ##1 }
        \exp_args:NNo \seq_put_right:Nn \l_tmpa_seq {
          \l_stex_get_symbol_uri_str
        }
      }
    }
    \exp_args:Nnnx
    \begin{stex_annotate_env}{subproof}{\seq_use:Nn \l_tmpa_seq {,}}
    \str_if_empty:NF \spftype {
      \stex_annotate_invisible:nnn{type}{\spftype}{}
    }

    \clist_set:No \l_tmpa_clist \spftype
    \tl_set:Nn \l_tmpa_tl {
      \item[\sproofnumber]
      \bool_set_true:N \l_@@_inc_counter_bool
    }
    \clist_map_inline:Nn \l_tmpa_clist {
      \exp_args:No \str_if_eq:nnT \c_@@_flow_str {##1} {
        \tl_clear:N \l_tmpa_tl
      }
    }
    \l_tmpa_tl
    \tl_if_empty:NF \spftitle {
      {(\titleemph{\spftitle})\enspace}
    }
    {~#2}
    \str_if_empty:NF \spfid {
      \stex_ref_new_doc_target:n \spfid
    }
  }
  \_@@_add_counter:
  \stex_smsmode_do:
}{
  \_@@_remove_counter:
  \bool_if:NT \l_@@_inc_counter_bool {
    \_@@_inc_counter:
  }
  \stex_if_smsmode:F{
    \end{stex_annotate_env}
  }
}
%    \end{macrocode}
% \end{environment}
%
% \begin{environment}{spfcases}
%   In the |pfcases| environment, the start text is displayed as the first comment of the
%   proof.
%    \begin{macrocode}
\newenvironment{spfcases}[2][]{
  \tl_if_empty:nTF{#1}{
    \begin{subproof}[method=by-cases]{#2}
  }{
    \begin{subproof}[#1,method=by-cases]{#2}
  }
}{
  \end{subproof}
}
%    \end{macrocode}
% \end{environment}
%
% \begin{environment}{spfcase}
%    In the |pfcase| environment, the start text is displayed specification of the case
%    after the |\item|
%    \begin{macrocode}
\newenvironment{spfcase}[2][]{
  \_@@_spf_args:n{#1}
  \stex_if_smsmode:TF {
    \str_if_empty:NF \spfid {
      \stex_ref_new_doc_target:n \spfid
    }
  }{
    \seq_clear:N \l_tmpa_seq
    \clist_map_inline:Nn \l_@@_spf_for_clist {
      \tl_if_empty:nF{ ##1 }{
        \stex_get_symbol:n { ##1 }
        \exp_args:NNo \seq_put_right:Nn \l_tmpa_seq {
          \l_stex_get_symbol_uri_str
        }
      }
    }
    \exp_args:Nnnx
    \begin{stex_annotate_env}{spfcase}{\seq_use:Nn \l_tmpa_seq {,}}
    \str_if_empty:NF \spftype {
      \stex_annotate_invisible:nnn{type}{\spftype}{}
    }
    \clist_set:No \l_tmpa_clist \spftype
    \tl_set:Nn \l_tmpa_tl {
      \item[\sproofnumber]
      \bool_set_true:N \l_@@_inc_counter_bool
    }
    \clist_map_inline:Nn \l_tmpa_clist {
      \exp_args:No \str_if_eq:nnT \c_@@_flow_str {##1} {
        \tl_clear:N \l_tmpa_tl
      }
    }
    \l_tmpa_tl
    \tl_if_empty:nF{#2}{
      \titleemph{#2}:~
    }
  }
  \_@@_add_counter:
  \stex_smsmode_do:
}{
  \_@@_remove_counter:
  \bool_if:NT \l_@@_inc_counter_bool {
    \_@@_inc_counter:
  }
  \stex_if_smsmode:F{
    \clist_set:No \l_tmpa_clist \spftype
    \tl_set:Nn \l_tmpa_tl{\sproofend}
    \clist_map_inline:Nn \l_tmpa_clist {
      \exp_args:No \str_if_eq:nnT \c_@@_flow_str {##1} {
        \tl_clear:N \l_tmpa_tl
      }
    }
    \l_tmpa_tl
    \end{stex_annotate_env}
  }
}
%    \end{macrocode}
% \end{environment}
%
% \begin{environment}{spfcase}
%    similar to |spfcase|, takes a third argument. 
%    \begin{macrocode}
\newcommand\spfcasesketch[3][]{
  \begin{spfcase}[#1]{#2}#3\end{spfcase}
}
%    \end{macrocode}
% \end{environment}
%
% \subsection{Justifications}
%
% We define the actions that are undertaken, when the keys for justifications are
% encountered. Here this is very simple, we just define an internal macro with the value,
% so that we can use it later.
%    \begin{macrocode}
\keys_define:nn { stex / just }{
  id        .str_set_x:N  = \l_@@_just_id_str,
  method    .tl_set:N     = \l_@@_just_method_tl,
  premises  .tl_set:N     = \l_@@_just_premises_tl,
  args      .tl_set:N     = \l_@@_just_args_tl
}
%    \end{macrocode}
%
% The next three environments and macros are purely semantic, so we ignore the keyval
% arguments for now and only display the content.\ednote{need to do something about the
% premise in draft mode.}
%
% \begin{environment}{justification}
%    \begin{macrocode}
\newenvironment{justification}[1][]{}{}
%    \end{macrocode}
% \end{environment}
%
% \begin{macro}{\premise}
%    \begin{macrocode}
\newcommand\stex_proof_premise:[2][]{#2}
%    \end{macrocode}
% \end{macro}
%
% \begin{macro}{\justarg}
% the |\justarg| macro is purely semantic, so we ignore the keyval arguments for now and
% only display the content.
%    \begin{macrocode}
\newcommand\justarg[2][]{#2}
%</package>
%    \end{macrocode}
% \end{macro}
% \end{implementation}
% \Finale
\endinput
%%% Local Variables: 
%%% mode: doctex
%%% TeX-master: t
%%% End: 
% LocalWords:  GPL structuresharing STR sproof dtx CPERL keyval methodfalse env
% LocalWords:  methodtrue envtrue medhodtrue DefKeyVal Semiverbatim omdoc args
% LocalWords:  DefEnvironment OptionalKeyVals KeyVal omtext DefConstructor str
% LocalWords:  proofidea KeyVal pfstep DefCMPEnvironment KeyVal proofcomment eq
% LocalWords:  KeyVal pfcases KeyVal pfcase KeyVal extractBodyText unlist elsif
% LocalWords:  foreach getBody toString str str str LookupValue LastSeenCMP Thu
% LocalWords:  appendText getValue undef openElement closeElement DefMacro omd
% LocalWords:  afterClose nodeType childNodes firstCMP localname hasChildNodes
% LocalWords:  firstChild textContent removeChild iffalse kohlhase sref scsys
% LocalWords:  sproofs.sty sc sc mathml openmath latexml cmathml activemath geq
% LocalWords:  twintoo atwin atwintoo texttt fileversion maketitle stex newpage
% LocalWords:  tableofcontents newpage exfig scriptsize vspace ednote spfidea
% LocalWords:  spfidea spfsketch spfsketch spfstep justarg spfcases spfcase rhs
% LocalWords:  sproofcomment ind-hyp splitit arith byindhyp sproofend proofend
% LocalWords:  printbibliography textsf langle textsf langle ltxml ctancite spf
% LocalWords:  srefaddidkey pf.sty newenvironment hbox vrule vbox ifx showmeta
% LocalWords:  hrule vskip hrule vrule hfil nobreak hfill smallskip newcommand
% LocalWords:  stDMemph newcount endsproof xref doctex showmeta hline lec ldots
% LocalWords:  textbackslash makeatletter sketchproof compactenum tracissue
% LocalWords:  metakeys addmetakey metasetkeys stylable pstlabelstyle pstlabel
% LocalWords:  pstlabelstyle pstlabelstyle ldots ldots ensuremath inparaenum
% LocalWords:  nameuse prooflistenv spfcasesketch spfcasesketch spfeq rcll
% LocalWords:  displaymath noindent ignorespaces
