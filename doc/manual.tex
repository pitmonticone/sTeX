\makeatletter
\ifcsname if@infulldoc\endcsname\else
    \expandafter\newif\csname if@infulldoc\endcsname\@infulldocfalse
\fi
\makeatother

\csname if@infulldoc\endcsname\else

\def\bibfolder{../lib/bib}

\RequirePackage{paralist}
\documentclass[full,kernel]{l3doc}
\usepackage[dvipsnames]{xcolor}
\usepackage[utf8]{inputenc}
\usepackage[T1]{fontenc}
%\usepackage{document-structure}
\usepackage[showmods,debug=all,lang={de,en}, mathhub=./tests]{stex}
\usepackage{url,array,float,textcomp}
\usepackage[show]{ed}
\usepackage[hyperref=auto,style=alphabetic]{biblatex}
\addbibresource{\bibfolder/kwarcpubs.bib}
\addbibresource{\bibfolder/extpubs.bib}
\addbibresource{\bibfolder/kwarccrossrefs.bib}
\addbibresource{\bibfolder/extcrossrefs.bib}
\usepackage{amssymb}
\usepackage{amsfonts}
\usepackage{xspace}
\usepackage{hyperref}

\usepackage{morewrites}


\makeindex
\floatstyle{boxed}
\newfloat{exfig}{thp}{lop}
\floatname{exfig}{Example}

\usepackage{stex-tests}

\MakeShortVerb{\|}

\def\scsys#1{{{\sc #1}}\index{#1@{\sc #1}}\xspace}
\def\mmt{\textsc{Mmt}\xspace}
\def\xml{\scsys{Xml}}
\def\mathml{\scsys{MathML}}
\def\omdoc{\scsys{OMDoc}}
\def\openmath{\scsys{OpenMath}}
\def\latexml{\scsys{LaTeXML}}
\def\perl{\scsys{Perl}}
\def\cmathml{Content-{\sc MathML}\index{Content {\sc MathML}}\index{MathML@{\sc MathML}!content}}
\def\activemath{\scsys{ActiveMath}}
\def\twin#1#2{\index{#1!#2}\index{#2!#1}}
\def\twintoo#1#2{{#1 #2}\twin{#1}{#2}}
\def\atwin#1#2#3{\index{#1!#2!#3}\index{#3!#2 (#1)}}
\def\atwintoo#1#2#3{{#1 #2 #3}\atwin{#1}{#2}{#3}}
\def\cT{\mathcal{T}}\def\cD{\mathcal{D}}

\def\fileversion{3.0}
\def\filedate{\today}

\RequirePackage{pdfcomment}

\ExplSyntaxOn\makeatletter
\cs_set_protected:Npn \@comp #1 #2 {
  \pdftooltip {
    \textcolor{blue}{#1}
  } { #2 }
}

\cs_set_protected:Npn \@defemph #1 #2 {
  \pdftooltip { 
    \textbf{\textcolor{magenta}{#1}}
  } { #2 }
}
\makeatother\ExplSyntaxOff

\csname bool_set_true:N\expandafter\endcsname\csname stex_dtx_tests_bool\endcsname

\begin{document}
  \csname if@infulldoc\endcsname\else
	\title{
		The {\stex{3}} Manual
		\thanks{Version {\fileversion} (last revised {\filedate})}
 	}
	\author{Michael Kohlhase, Dennis Müller\\
		FAU Erlangen-Nürnberg\\
		\url{http://kwarc.info/}
	}
	\pagenumbering{roman}
	\maketitle
	
	\begin{abstract}
  TODO
\end{abstract}

	
	\makeatletter
		\renewcommand\part{%
    		\clearpage
  			\thispagestyle{plain}%
  			\@tempswafalse
  			\null\vfil
  			\secdef\@part\@spart%
  		}
		\newcounter{chapter}
		\numberwithin{section}{chapter}
		\renewcommand\thechapter{\@arabic\c@chapter}
		\renewcommand\thesection{\thechapter.\@arabic\c@section}
		\newcommand*\chaptermark[1]{}
		\setcounter{secnumdepth}{2}
		\newcommand\@chapapp{\chaptername}
		%\newcommand\chaptername{Chapter}
  		\def\ps@headings{%
    		\let\@oddfoot\@empty
    		\def\@oddhead{{\slshape\rightmark}\hfil\thepage}%
    		\let\@mkboth\markboth
    		\def\chaptermark##1{%
      			\markright{\MakeUppercase{%
        			\ifnum \c@secnumdepth >\m@ne
            			\@chapapp\ \thechapter. \ %
        			\fi
        		##1}}%
        	}%
        }
		\newcommand\chapter{\clearpage
			\thispagestyle{plain}%
			\global\@topnum\z@
			\@afterindentfalse
			\secdef\@chapter\@schapter%
		}
		\def\@chapter[#1]#2{\refstepcounter{chapter}%
			\typeout{\@chapapp\space\thechapter.}%
			\addcontentsline{toc}{chapter}%
				{\protect\numberline{\thechapter}#1}%
			\chaptermark{#1}%
			\addtocontents{lof}{\protect\addvspace{10\p@}}%
			\addtocontents{lot}{\protect\addvspace{10\p@}}%
			\@makechapterhead{#2}%
			\@afterheading%
		}
		\def\@makechapterhead#1{%
			\vspace*{50\p@}%
			{\parindent \z@ \raggedright \normalfont
				\huge\bfseries \@chapapp\space \thechapter
				\par\nobreak
				\vskip 20\p@
				\interlinepenalty\@M
				\Huge \bfseries #1\par\nobreak
				\vskip 40\p@
			}%
		}
\newcommand*\l@chapter[2]{%
  \ifnum \c@tocdepth >\m@ne
    \addpenalty{-\@highpenalty}%
    \vskip 1.0em \@plus\p@
    \setlength\@tempdima{1.5em}%
    \begingroup
      \parindent \z@ \rightskip \@pnumwidth
      \parfillskip -\@pnumwidth
      \leavevmode \bfseries
      \advance\leftskip\@tempdima
      \hskip -\leftskip
      #1\nobreak\hfil
      \nobreak\hb@xt@\@pnumwidth{\hss #2%
                                 \kern-\p@\kern\p@}\par
      \penalty\@highpenalty
    \endgroup
  \fi}
\renewcommand*\l@section{\@dottedtocline{1}{1.5em}{2.8em}}
\renewcommand*\l@subsection{\@dottedtocline{2}{3.8em}{3.2em}}
\renewcommand*\l@subsubsection{\@dottedtocline{3}{7.0em}{4.1em}}
\def\partname{Part}
\def\toclevel@part{-1}
\def\maketitle{\chapter{\@title}}
\let\thanks\@gobble
\let\DelayPrintIndex\PrintIndex
\let\PrintIndex\@empty
\providecommand*{\hexnum}[1]{\text{\texttt{\char`\"}#1}}
\makeatother

\clearpage

{%
  \def\\{:}% fix "newlines" in the ToC
  \tableofcontents
}

\clearpage
\pagenumbering{arabic}	
	
\fi

\chapter{Stuff}

\section{Modules}


\begin{function}{\sTeX , \stex}
  Both print this \stex logo.
\end{function}

 \subsection{Semantic Macros and Notations}

 Semantic macros invoke a formally declared symbol.

 To declare a symbol (in a module), we use \cs{symdecl},
 which takes as argument the name of the corresponding
 semantic macro, e.g. |\symdecl{foo}| introduces the macro
 \cs{foo}. Additionally, \cs{symdecl} takes several options,
 the most important one being its arity. |foo| as declared above
 yields a \emph{constant} symbol. To introduce an \emph{operator}
 which takes arguments, we have to specify which arguments it takes.

 \begin{@module}{SemanticMacrosExample}
   For example, to introduce binary multiplication,
   we can do |\symdecl[args=2]{mult}|. We can then supply
   the semantic macro with arbitrarily many notations, such as
   |\notation{mult}{#1 #2}|.
   
   \stexexample{
 \symdecl[args=2]{mult}
 \notation{mult}{#1 #2}
 $\mult{a}{b}$
}

 Since usually, a freshly introduced symbol also comes with a
 notation from the start, the \cs{symdef} command combines
 \cs{symdecl} and \cs{notation}. So instead of the above,
 we could have also written
 \begin{center} |\symdef[args=2]{mult}{#1 #2}| \end{center}

 \symdecl[args=2]{mult}
 \notation{mult}{#1 #2}

   \notation[cdot]{mult}{#1 \comp{\cdot} #2}
   \notation[times]{mult}{#1 \comp{\times} #2}
   Adding more notations like
   |\notation[cdot]{mult}{#1 \comp{\cdot} #2}| or 
   |\notation[times]{mult}{#1 \comp{\times} #2}|
   allows us to write |$\mult[cdot]{a}{b}$| and
   |$\mult[times]{a}{b}$|:
   \stexexample{
   \notation[cdot]{mult}{#1 \comp{\cdot} #2}
   \notation[times]{mult}{#1 \comp{\times} #2}
 $\mult[cdot]{a}{b}$ and $\mult[times]{a}{b}$
}
   \notation[cdot]{mult}{#1 \comp{\cdot} #2}
   \notation[times]{mult}{#1 \comp{\times} #2}

   Not using an explicit option with a semantic macro yields
   the first declared notation, unless changed\ednote{TODO}.

   Outside of math mode, or by using the starred variant
   |\foo*|, allows to provide a custom notation, where
   notational (or textual) components can be given
   explicitly in square brackets.
   \stexexample{
 $\mult*{a}[\comp{\ast}]{b}$ is the 
 \mult[\comp{product of} ]{$a$}[ \comp{and} ]{$b$}
}

   In custom mode, prefixing an argument with a star will not
   print that argument, but still export it to \omdoc:
   \stexexample{
 \mult[\comp{Multiplying}]*{$\mult{a}{b}$}[ again by ]{$b$} yields...
}
   The syntax |*[|\meta{int}|]| allows switching
   the order of arguments. For example, given a 2-ary semantic
   macro |\forevery| with exemplary notation
   |\forall #1. #2|, we can write
   \stexexample{
     \symdecl[args=2]{forevery}
     \forevery*[2]{The proposition $P$}[ \comp{holds for every} ]*[1]{$x\in A$}
}

 When using |*[|$n$|]|, after reading the provided ($n$th) argument,
  the ``argument counter'' automatically 
 continues where we left off, so the |*[1]| in the above example
 can be omitted.

   For a macro with arity $>0$, we can refer to the operator
   \emph{itself} semantically by suffixing the semantic macro
   with an exclamation point |!| in either text or math mode.
   For that reason \cs{notation} (and thus \cs{symdef}) take an
   additional optional argument |op=|, which allows to assign
   a notation for the operator itself. e.g.
   \stexexample{
     \symdef[args=2,op={+}]{add}{#1 \comp+ #2}
     The operator $\add!$ adds two elements, as in $\add ab$.
   }

  |*| is composable with |!| for custom notations, as in:

   \stexexample{
 \mult![\comp{Multiplication}] (denoted by $\mult*![\comp\cdot]$) is defined by...
}

 The macro \cs{comp} as used everywhere above is responsible
 for highlighting, linking, and tooltips, and should be wrapped
 around the notation (or text) components that should be treated
 accordingly. While it is attractive to just wrap a whole notation,
 this would also wrap around e.g. the arguments themselves, so
 instead, the user is tasked with marking the notation components
 themself.

 The precise behaviour of \cs{comp} is governed by
 the macro \cs{@comp}, which takes two arguments: The tex code
 of the text
 (unexpanded) to highlight, and the URI of the current symbol.
 \cs{@comp} can be safely redefined to customize the behaviour.


 The starred variant |\symdecl*{foo}| does not introduce a semantic
 macro, but still declares a corresponding symbol. |foo| (like
 any other symbol, for that matter) can
 then be accessed via \cs{STEXsymbol}|{foo}| or (if |foo| was declared
 in a module |Foo|) via \cs{STEXModule}|{Foo}?{foo}|.

 both \cs{STEXsymbol} and \cs{STEXModule} take any
 arbitrary ending segment of a full URI to determine
 which symbol or module is meant. e.g.
 \cs{STEXsymbol}|{Foo?foo}| is also valid, as are e.g.
 \cs{STEXModule}|{path?Foo}?{foo}| or
 \cs{STEXsymbol}|{path?Foo?foo}|

 There's also a convient shortcut \cs{symref}|{?foo}{some text}| for
 \cs{STEXsymbol}|{?foo}![some text]|.

 \end{@module}

 \subsubsection{Other Argument Types}

 So far, we have stated the arity of a semantic macro directly.
 This works if we only have ``normal'' (or more precisely: |i|-type) arguments.
  To make use of other argument types, instead of providing the arity
 numerically, we can provide it as a sequence of characters representing
 the argument types -- e.g. instead of writing |args=2|, we
 can equivalently write |args=ii|, indicating that the macro
 takes two |i|-type arguments.

 Besides |i|-type arguments, \sTeX has two other types, which we will
 discuss now.

 The first are \emph{binding} (|b|-type) arguments, representing
 variables that are \emph{bound} by the operator. This is the
 case for example in the above \cs{forevery}-macro:
 The first argument is not actually an argument that the
 |forevery| ``function'' is ``applied'' to; rather, the first argument
 is a new variable (e.g. $x$) that is \emph{bound} in the subsequent
 argument. More accurately, the macro should therefore have been
 implemented thusly:
   \begin{center}|\symdef[args=bi]{forevery}{\forall #1.\; #2}|\end{center}

 \begin{@module}{OtherArgs}
 |b|-type arguments are indistinguishable from |i|-type arguments
 within \sTeX, but are treated very differently in \omdoc and by \mmt.
 More interesting \emph{within} \sTeX are |a|-type arguments,
 which represent (associative) arguments of flexible arity, which are
 provided as comma-separated lists.
 This allows e.g. better representing the \cs{mult}-macro above:
 
   \stexexample{
 \symdef[args=a]{mult}{#1}{#1 \comp\cdot #2}
 $\mult{a,b,c,{d^e},f}$
}
 As the example above shows, notations get a little more complicated
 for associative arguments. For every |a|-type argument, the
 \cs{notation}-macro takes an additional argument that declares
 how individual entries in an |a|-type argument list are aggregated.
 The first notation argument then describes how the aggregated
 expression is combined into the full representation.

 For a more interesting example, consider a flexary operator
 for ordered sequences in ordered set, that taking 
 arguments |{a,b,c}| and |\mathbb{R}| prints
 $a \leq b \leq c\in \mathbb R$. This operator takes
 two arguments (an |a|-type argument and an |i|-type argument),
 aggregates the individuals of the associative argument using |\leq|,
 and combines the result with |\in| and the second argument thusly:

   \stexexample{
 \symdef[args=ai]{numseq}{#1 \comp\in #2}{#1 \comp\leq #2}
 $\numseq{a,b,c}{\mathbb R}$
}

 Finally, |B|-type arguments combine the functionalities of |a|
 and |b|, i.e. they represent flexary binding operator arguments.

\ednote{what about e.g. \detokenize{\int_x\int_y\int_z f dx dy dz}?}
\ednote{``decompose'' a-type arguments into fixed-arity operators?}

 \end{@module}

 \subsubsection{Precedences}

 Every notation has an (upwards) \emph{operator precedence} and
 for each argument a (downwards) \emph{argument precedence}
 used for automated bracketing. For example, a notation
 for a binary operator \cs{foo} could be declared like this:
 \begin{center} |\notation[prec=200;500x600]{foo}{#1 \comp{+} #2}| \end{center}
 assigning an operator precedence of 200, an argument precedence
 of 500 for the first argument, and an argument precedence of 600
 for the second argument.

 \sTeX insert brackets thusly: Upon encountering a semantic
 macro (such as \cs{foo}), its operator precedence (e.g. 200)
 is compared to the current downwards precedence (initially 
 \cs{neginfprec}). If the operator precedence is \emph{larger}
 than the current downwards precedence, parentheses are inserted
 around the semantic macro.

 Notations for symbols of arity 0 have a default precedence of \cs{infprec},
 i.e. by default, parentheses are never inserted around constants.
 Notations for symbols with arity $>0$ have a default operator
 precedence of $0$.
 If no argument precedences are explicitly provided, then by
 default they are equal to the operator precedence.

 Consequently, if some operator $A$ should bind stronger than
 some operator $B$, then $A$s operator precedence should be
 smaller than $B$s argument precedences.

 For example:
 \begin{@module}{NotationsEx}
 \symdecl[args=2]{plus}
 \symdecl[args=2]{times}
 \stexexample{
\notation[prec=100]{plus}{#1 \comp{+} #2}
\notation[prec=50]{times}{#1 \comp{\cdot} #2}
$\plus{a}{\times{b}{c}}$ and $\times{a}{\plus{b}{c}}$
}


 \end{@module}

 \subsection{Archives and Imports}

 \subsubsection{Namespaces}
   Ideally, \sTeX would use arbitrary URIs for modules, with no
   forced relationships between the \emph{logical} namespace
   of a module and the \emph{physical} location of the file
   declaring the module -- like \mmt does things.

   Unfortunately, \TeX\ only provides very restricted access to
   the file system, so we are forced to generate namespaces
   systematically in such a way that they reflect the physical
   location of the associated files, so that \sTeX can resolve
   them accordingly. Largely, users need not concern themselves
   with namespaces at all, but for completenesses sake, we describe
   how they are constructed:

   \begin{itemize}
     \item If \cs{begin}|{module}{Foo}| occurs in a file
       |/path/to/file/Foo[.|\meta{lang}|].tex| which does not belong
       to an archive, the namespace is |file://path/to/file|.
     \item If the same statement occurs in a file
       |/path/to/file/bar[.|\meta{lang}|].tex|, the namespace is 
       |file://path/to/file/bar|.
   \end{itemize}

   In other words: outside of archives, the namespace corresponds to
   the file URI with the filename dropped iff it is equal to the
   module name, and ignoring the (optional) language suffix^^A
   \footnote{which is internally attached to the module name instead,
   but a user need not worry about that.}.

   If the current file is in an archive, the procedure is the same
   except that the initial segment of the file path up to the archive's
   |source|-folder is replaced by the archive's namespace URI.

 \subsubsection{Paths in Import-Statements}

 Conversely, here is how namespaces/URIs and file paths are computed
 in import statements, examplary \cs{importmodule}:

 \begin{itemize}
   \item \cs{importmodule}|{Foo}| outside of an archive refers 
     to module |Foo| in the current namespace. Consequently, |Foo|
     must have been declared earlier in the same document or, if not,
     in a file |Foo[.|\meta{lang}|].tex| in the same directory.
   \item The same statement \emph{within} an archive refers to either
     the module |Foo| declared earlier in the same document, or
     otherwise to the module |Foo| in the archive's top-level namespace.
     In the latter case, is has to be declared in a file |Foo[.|\meta{lang}|].tex|
     directly in the archive's |source|-folder.
   \item Similarly, in \cs{importmodule}|{some/path?Foo}| the path
     |some/path| refers to either the sub-directory and relative 
     namespace path of the current directory and namespace outside of an archive,
     or relative to the current archive's top-level namespace and |source|-folder,
     respectively.

     The module |Foo| must either be declared in the file
     \meta{top-directory}|/some/path/Foo[.|\meta{lang}|].tex|, or in
     \meta{top-directory}|/some/path[.|\meta{lang}|].tex| (which are
     checked in that order).
   \item Similarly, \cs{importmodule}|[Some/Archive]{some/path?Foo}|
     is resolved like the previous cases, but relative to the archive
     |Some/Archive| in the mathhub-directory.
   \item Finally, \cs{importmodule}|{full://uri?Foo}| naturally refers to the
     module |Foo| in the namespace |full://uri|. Since the file this module
     is declared in can not be determined directly from the URI, the module
     must be in memory already, e.g. by being referenced earlier in the
     same document.

     Since this is less compatible with a modular development, using full
     URIs directly is discouraged.

 \end{itemize} 


	
	
\csname if@infulldoc\endcsname\else\end{document}\fi
