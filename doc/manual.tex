\makeatletter
\ifcsname if@infulldoc\endcsname\else
    \expandafter\newif\csname if@infulldoc\endcsname\@infulldocfalse
\fi
\makeatother

\csname if@infulldoc\endcsname\else

\def\bibfolder{../lib/bib}

\RequirePackage{paralist}
\documentclass[full,kernel]{l3doc}
\usepackage[dvipsnames]{xcolor}
\usepackage[utf8]{inputenc}
\usepackage[T1]{fontenc}
%\usepackage{document-structure}
\usepackage[showmods,debug=all,lang={de,en}, mathhub=./tests]{stex}
\usepackage{url,array,float,textcomp}
\usepackage[show]{ed}
\usepackage[hyperref=auto,style=alphabetic]{biblatex}
\addbibresource{\bibfolder/kwarcpubs.bib}
\addbibresource{\bibfolder/extpubs.bib}
\addbibresource{\bibfolder/kwarccrossrefs.bib}
\addbibresource{\bibfolder/extcrossrefs.bib}
\usepackage{amssymb}
\usepackage{amsfonts}
\usepackage{xspace}
\usepackage{hyperref}

\usepackage{morewrites}


\makeindex
\floatstyle{boxed}
\newfloat{exfig}{thp}{lop}
\floatname{exfig}{Example}

\usepackage{stex-tests}

\MakeShortVerb{\|}

\def\scsys#1{{{\sc #1}}\index{#1@{\sc #1}}\xspace}
\def\mmt{\textsc{Mmt}\xspace}
\def\xml{\scsys{Xml}}
\def\mathml{\scsys{MathML}}
\def\omdoc{\scsys{OMDoc}}
\def\openmath{\scsys{OpenMath}}
\def\latexml{\scsys{LaTeXML}}
\def\perl{\scsys{Perl}}
\def\cmathml{Content-{\sc MathML}\index{Content {\sc MathML}}\index{MathML@{\sc MathML}!content}}
\def\activemath{\scsys{ActiveMath}}
\def\twin#1#2{\index{#1!#2}\index{#2!#1}}
\def\twintoo#1#2{{#1 #2}\twin{#1}{#2}}
\def\atwin#1#2#3{\index{#1!#2!#3}\index{#3!#2 (#1)}}
\def\atwintoo#1#2#3{{#1 #2 #3}\atwin{#1}{#2}{#3}}
\def\cT{\mathcal{T}}\def\cD{\mathcal{D}}

\def\fileversion{3.0}
\def\filedate{\today}

\RequirePackage{pdfcomment}

\ExplSyntaxOn\makeatletter
\cs_set_protected:Npn \@comp #1 #2 {
  \pdftooltip {
    \textcolor{blue}{#1}
  } { #2 }
}

\cs_set_protected:Npn \@defemph #1 #2 {
  \pdftooltip { 
    \textbf{\textcolor{magenta}{#1}}
  } { #2 }
}
\makeatother\ExplSyntaxOff

\infulldoctrue

\csname bool_set_true:N\expandafter\endcsname\csname stex_dtx_tests_bool\endcsname

\begin{document}
  \csname if@infulldoc\endcsname\else
	\title{
		The {\stex{3}} Manual
		\thanks{Version {\fileversion} (last revised {\filedate})}
 	}
	\author{Michael Kohlhase, Dennis Müller\\
		FAU Erlangen-Nürnberg\\
		\url{http://kwarc.info/}
	}
	\pagenumbering{roman}
	\maketitle
	
	\begin{abstract}
  TODO
\end{abstract}
\bigskip

  This is the user manual for the \sTeX package and 
  associated software. It is primarily directed at end-users 
  who want to use \sTeX to author semantically
  enriched documents. For the full documentation, see
  \href{\basedocurl/stex.pdf}{the \sTeX documentation}
	
	\makeatletter
		\renewcommand\part{%
    		\clearpage
  			\thispagestyle{plain}%
  			\@tempswafalse
  			\null\vfil
  			\secdef\@part\@spart%
  		}
		\newcounter{chapter}
		\numberwithin{section}{chapter}
		\renewcommand\thechapter{\@arabic\c@chapter}
		\renewcommand\thesection{\thechapter.\@arabic\c@section}
		\newcommand*\chaptermark[1]{}
		\setcounter{secnumdepth}{2}
		\newcommand\@chapapp{\chaptername}
		%\newcommand\chaptername{Chapter}
  		\def\ps@headings{%
    		\let\@oddfoot\@empty
    		\def\@oddhead{{\slshape\rightmark}\hfil\thepage}%
    		\let\@mkboth\markboth
    		\def\chaptermark##1{%
      			\markright{\MakeUppercase{%
        			\ifnum \c@secnumdepth >\m@ne
            			\@chapapp\ \thechapter. \ %
        			\fi
        		##1}}%
        	}%
        }
		\newcommand\chapter{\clearpage
			\thispagestyle{plain}%
			\global\@topnum\z@
			\@afterindentfalse
			\secdef\@chapter\@schapter%
		}
		\def\@chapter[#1]#2{\refstepcounter{chapter}%
			\typeout{\@chapapp\space\thechapter.}%
			\addcontentsline{toc}{chapter}%
				{\protect\numberline{\thechapter}#1}%
			\chaptermark{#1}%
			\addtocontents{lof}{\protect\addvspace{10\p@}}%
			\addtocontents{lot}{\protect\addvspace{10\p@}}%
			\@makechapterhead{#2}%
			\@afterheading%
		}
		\def\@makechapterhead#1{%
			\vspace*{50\p@}%
			{\parindent \z@ \raggedright \normalfont
				\huge\bfseries \@chapapp\space \thechapter
				\par\nobreak
				\vskip 20\p@
				\interlinepenalty\@M
				\Huge \bfseries #1\par\nobreak
				\vskip 40\p@
			}%
		}
\newcommand*\l@chapter[2]{%
  \ifnum \c@tocdepth >\m@ne
    \addpenalty{-\@highpenalty}%
    \vskip 1.0em \@plus\p@
    \setlength\@tempdima{1.5em}%
    \begingroup
      \parindent \z@ \rightskip \@pnumwidth
      \parfillskip -\@pnumwidth
      \leavevmode \bfseries
      \advance\leftskip\@tempdima
      \hskip -\leftskip
      #1\nobreak\hfil
      \nobreak\hb@xt@\@pnumwidth{\hss #2%
                                 \kern-\p@\kern\p@}\par
      \penalty\@highpenalty
    \endgroup
  \fi}
\renewcommand*\l@section{\@dottedtocline{1}{1.5em}{2.8em}}
\renewcommand*\l@subsection{\@dottedtocline{2}{3.8em}{3.2em}}
\renewcommand*\l@subsubsection{\@dottedtocline{3}{7.0em}{4.1em}}
\def\partname{Part}
\def\toclevel@part{-1}
\def\maketitle{\chapter{\@title}}
\let\thanks\@gobble
\let\DelayPrintIndex\PrintIndex
\let\PrintIndex\@empty
\providecommand*{\hexnum}[1]{\text{\texttt{\char`\"}#1}}
\makeatother

\ExplSyntaxOn
\int_set:Nn \l_document_structure_section_level_int {1}
\ExplSyntaxOff

\clearpage

{%
  \def\\{:}% fix "newlines" in the ToC
  \tableofcontents
}

\clearpage
\pagenumbering{arabic}
	
\fi

\long\def\ignore#1{}

\begin{omgroup}{What is \sTeX?}
  
Formal systems for mathematics (such as interactive theorem provers)
have the potential to significantly increase both the accessibility
of published knowledge, as well as the confidence in its veracity,
by rendering the precise semantics of statements machine actionable.
This allows for a plurality of added-value services, from semantic
search up to verification and automated theorem proving.
Unfortunately, their usefulness is hidden behind severe barriers
to accessibility; primarily related to their surface languages
reminiscent of programming languages and very unlike informal
standards of presentation.

\sTeX minimizes this gap between informal and formal 
mathematics by integrating formal methods into established
and widespread authoring workflows, primarily \LaTeX, via 
non-intrusive semantic
annotations of arbitrary informal document fragments. That way
formal knowledge management services become available for informal
documents, accessible via an IDE for authors and via generated
\emph{active} documents for readers, while remaining fully compatible
with existing authoring workflows and publishing systems.

Additionally, an extensible library of reusable
document fragments is being developed, that serve as reference targets
for global disambiguation, intermediaries for content exchange
between systems and other services.

Every component of the system is designed modularly and extensibly,
and thus lay the groundwork for a potential full integration of
interactive theorem proving systems into established informal document
authoring workflows.

\paragraph{} The general \sTeX workflow combines functionalities
provided by several pieces of software:
\begin{itemize}
  \item The \sTeX package to use semantic annotations in
    {\LaTeX} documents,
  \item \RusTeX to convert |tex| sources to (semantically enriched)
    |xhtml|,
  \item The \mmt software, that extracts semantic information
    from the thus generated |xhtml| and provides semantically informed
    added value services.
\end{itemize}


% ----------------------------

\ignore{The objectives of this project will be achieved by developing a 
language and system 
that uses non-intrusive annotations
to augment informal documents with semantic information
(ranging from \textbf{fully formal} to \textbf{purely informal})
 without
impacting linguistic presentation or document layout. 
That way, the system
remains compatible with established publishing
pipelines and practices, while additionally providing flexiformal 
information that
enables formal knowledge management services, and hence produces 
\emph{rich active documents}, satisfying \textbf{R3}, \textbf{R4} and 
\textbf{R5}.
In particular, it will avoid commitment to a fixed logical foundation.
Instead, it will be designed as a modular pipeline of consecutive
and compositional
annotations, semantics extraction and translation steps, extensible
via new structuring mechanisms (\textbf{R1}), library content 
(\textbf{R2}),
NLP techniques, foundations, translation methods and 
end-user services.

Naturally, the benefits of formal knowledge management services scale 
with the amount of mathematics involved. Consequently I will primarily 
focus on those 
STEM fields in which mathematical methods are most prominently
used (e.g. mathematics, physics, computer science). Since in those fields
\LaTeX~is the most commonly used scientific writing tool, I will also
primarily focus on \LaTeX~as a development and evaluation target, but 
the system will be designed such that all components apart from
the surface language will be integrable with other writing tools 
(e.g. WYSIWYG word processors).

\paragraph{} The basic architecture of the proposed system is sketched in
\autoref{fig:architecture}.
\begin{figure}\centering
  \resizebox{0.95\textwidth}{!}{\tikzinput[]{diagram}}
  {\small (Note, that the syntax used
    in the box on the top right is prototypical and subject to change during the project.
    Details and open questions regarding the syntax are discussed here:
    \url{https://github.com/KWARC/FoMID/issues/1})}
  \caption{Basic Architecture of the Proposed System}\label{fig:architecture}
\end{figure}
A user can write their content using standard \LaTeX\ in an IDE;
ideally using semantic annotations provided by \sTeX
%and the library developed in \OBJref{smglom}
(as in the upper right of 
\autoref{fig:architecture}), but not necessarily so.

The document is converted to xhtml with \omdoc annotations
using \LaTeX ML in the background,
thus becoming actionable by the \mmt system. Both the source document
as well as the generated xhtml/\omdoc are accessible to a natural language
processing pripeline that can supply additional inferred semantic 
information or suggest annotations to the user, in the latter case 
augmenting the source document directly. This pipeline can use both 
classical NLP techniques using the GLIF system, as well as machine 
learning models such as \cite{own:fifom}.

A semiformal fragment is converted 
into an appropriate syntax tree (possibly containing opaque
informal nodes), 
thus becoming amenable
to flexiformal knowledge management services. In a consecutive step
-- if sufficiently annotated --, these are
additionally translated
to a fully formal foundation, e.g. using the techniques from 
\cite{DMueller:phd:19,own:translations}, allowing
more powerful services and conversion to established formal
systems. All three representations
are thus available from within the \mmt system for various
knowledge management services, interfaces for which can be
implemented in the IDE.

Importantly, every non-trivial arrow in the figure is 
composable and extensible -- 
translations to a foundation can be provided
by supplying an appropriate formalization and alignment-based
translations (or entirely new methods),
services can be implemented generically using the \mmt API,
NLP techniques can be implemented both inside and alongside of
GLIF, and the concrete syntax within \sTeX can be extended
by convenience macros in \LaTeX\ (enabling new
structuring mechanisms as in \textbf{R1} via 
\mmt extensions, see
\cite{MueRabRot:rslffml20}) as well as via additions to
the library, which will be extensible both from within the IDE
as well as on MathHub,
remaining backwards compatible with existing content in a surface 
language. Additionally, sufficiently disambiguated
statements can be translated to the syntax of 
external systems (such as interactive theorem prover systems
or computer algebra systems),
which can thus be integrated as additional services into the system.
}

\end{omgroup}

\begin{omgroup}{Quickstart}
	\begin{omgroup}{Setup}
		\begin{omgroup}{The \sTeX IDE}
      TODO: VSCode Plugin
    \end{omgroup}
    \begin{omgroup}{Manual Setup}
      Foregoing on the \sTeX IDE, we will need several
      pieces of software; namely:
      \begin{itemize}
        \item \textbf{The \sTeX-Package} available 
          \href{https://github.com/slatex/sTeX/blob/latex3/doc/stex.pdf}{here}%
          \ednote{For now, we require the \texttt{latex3}-branch}.
          Note, that the CTAN repository for \LaTeX{} packages
          may contain outdated versions of the \sTeX package, so
          make sure, that your |TEXMF| system variable is configured such
          that the packages available in the linked repository are prioritized
          over potential default packages that come with your \TeX{} distribution.

          %If you are only interested in using semantic macros in (ultimately)
          %|pdf|s generated by |pdflatex|, this is all you need.

        \item \textbf{The \mmt System} available
          \href{https://github.com/uniformal/MMT/tree/sTeX}{here}%
          \ednote{For now, we require the \texttt{sTeX}-branch, requiring manually
          compiling the MMT sources}. We recommend following
          the setup routine documented 
          \href{https://uniformal.github.io//doc/setup/}{here}.

          Following the setup routine (Step 3) will entail designating
          a |MathHub|-directory on your local file system, where
          the \mmt system will look for \sTeX/\mmt content archives.

        \item To make sure that \sTeX too knows where to find its
          archives, we need to set a global system variable |MATHHUB|,
          that points to your local |MathHub|-directory.
        \item \textbf{\sTeX Archives} If we only care about {\LaTeX} and generating |pdf|s, we do not
          technically need \mmt at all; however, we still need the |MATHHUB|
          system variable to be set. Furthermore, \mmt can make downloading
          content archives we might want to use significantly easier, since
          it makes sure that all dependencies of (often highly interrelated)
          \sTeX archives are cloned as well.

          Once set up, we can run |mmt| in a shell and download an archive along with
          all of its dependencies like this: |lmh install <name-of-repository>|,
          or a whole \emph{group} of archives; for example,
          |lmh install smglom| will download all smglom archives.
        \item \textbf{\RusTeX} The \mmt system will also set up \RusTeX for you,
          which is used to generate (semantically annotated)
          |xhtml| from tex sources. In lieu of using \mmt, you
          can also download and use \RusTeX directly
          \href{https://github.com/slatex/RusTeX}{here}.

      \end{itemize}
    \end{omgroup}
	\end{omgroup}
	\begin{omgroup}{A First \sTeX Document}
    Having set everything up, we can write a first
    \sTeX document. As an example, we will use the
    |smglom/calculus| and |smglom/arithmetics| archives, 
    which should be present in the designated |MathHub|-folder.

    The document we will consider is the following:
    \begin{framed}\begin{latexcode}
\documentclass{article}
\usepackage{stex}
\usepackage{xcolor}

\begin{document}
  \usemodule[smglom/calculus]{series}
  \usemodule[smglom/arithmetics]{realarith}

  The \symref{series}{series} $\infinitesum{n}{1}{
    \realdivide[frac]{1}{
      \realpower{2}{n}
    }
  }$ \symref{converges}{converges} towards $1$.
    
\end{document}
    \end{latexcode}\end{framed}

    Compiling this document with |pdflatex| should yield
    the output

    \begin{framed}
        The \textbf{series} 
        $\textcolor{blue}{\sum}_{n=1}^{\textcolor{blue}\infty} \frac{1}{2^n}$
        \textbf{converges} towards $1$.
    \end{framed}

    Note that the $\sum$ and $\infty$-symbols are highlighted in blue,
    and the words ``series'' and ``converges'' in bold.
    This signifies that these words and symbols 
    reference \sTeX \emph{symbols}
    formally declared somewhere; associating their 
    \emph{presentation} in the document with their (formal)
    definition - i.e. their semantics. The precise way
    in which they are highlighted (if at all) can of course
    be customized (see \ednote{somewhere later}).

    \textcolor{red}{TODO} explain |\usemodule|, |\symref|, semantic macros,
    notation options (|[frac]|).

    \textcolor{red}{TODO} explain |xhtml| conversion, MMT compilation
    (requires an archive...?).

	\end{omgroup}
\end{omgroup}

\begin{omgroup}{Using Semantic Macros}
	\textcolor{red}{TODO}
  % \iffalse meta-comment
% An Infrastructure for Semantic Macros and Module Scoping
% Copyright (c) 2019 Michael Kohlhase, all rights reserved
%                this file is released under the
%                LaTeX Project Public License (LPPL)
% 
% The original of this file is in the public repository at 
% http://github.com/sLaTeX/sTeX/
%
% TODO update copyright  
%
%<*driver>
\providecommand\bibfolder{../../lib/bib}
\RequirePackage{paralist}
\documentclass[full,kernel]{l3doc}
\usepackage[dvipsnames]{xcolor}
\usepackage[utf8]{inputenc}
\usepackage[T1]{fontenc}
%\usepackage{document-structure}
\usepackage[showmods,debug=all,lang={de,en}, mathhub=./tests]{stex}
\usepackage{url,array,float,textcomp}
\usepackage[show]{ed}
\usepackage[hyperref=auto,style=alphabetic]{biblatex}
\addbibresource{\bibfolder/kwarcpubs.bib}
\addbibresource{\bibfolder/extpubs.bib}
\addbibresource{\bibfolder/kwarccrossrefs.bib}
\addbibresource{\bibfolder/extcrossrefs.bib}
\usepackage{amssymb}
\usepackage{amsfonts}
\usepackage{xspace}
\usepackage{hyperref}

\usepackage{morewrites}


\makeindex
\floatstyle{boxed}
\newfloat{exfig}{thp}{lop}
\floatname{exfig}{Example}

\usepackage{stex-tests}

\MakeShortVerb{\|}

\def\scsys#1{{{\sc #1}}\index{#1@{\sc #1}}\xspace}
\def\mmt{\textsc{Mmt}\xspace}
\def\xml{\scsys{Xml}}
\def\mathml{\scsys{MathML}}
\def\omdoc{\scsys{OMDoc}}
\def\openmath{\scsys{OpenMath}}
\def\latexml{\scsys{LaTeXML}}
\def\perl{\scsys{Perl}}
\def\cmathml{Content-{\sc MathML}\index{Content {\sc MathML}}\index{MathML@{\sc MathML}!content}}
\def\activemath{\scsys{ActiveMath}}
\def\twin#1#2{\index{#1!#2}\index{#2!#1}}
\def\twintoo#1#2{{#1 #2}\twin{#1}{#2}}
\def\atwin#1#2#3{\index{#1!#2!#3}\index{#3!#2 (#1)}}
\def\atwintoo#1#2#3{{#1 #2 #3}\atwin{#1}{#2}{#3}}
\def\cT{\mathcal{T}}\def\cD{\mathcal{D}}

\def\fileversion{3.0}
\def\filedate{\today}

\RequirePackage{pdfcomment}

\ExplSyntaxOn\makeatletter
\cs_set_protected:Npn \@comp #1 #2 {
  \pdftooltip {
    \textcolor{blue}{#1}
  } { #2 }
}

\cs_set_protected:Npn \@defemph #1 #2 {
  \pdftooltip { 
    \textbf{\textcolor{magenta}{#1}}
  } { #2 }
}
\makeatother\ExplSyntaxOff

\begin{document}
  \DocInput{\jobname.dtx}
\end{document}
%</driver>
% \fi
%
% \title{ \sTeX-Terms
% 	\thanks{Version {\fileversion} (last revised {\filedate})} 
% }
%
% \author{Michael Kohlhase, Dennis Müller\\
% 	FAU Erlangen-Nürnberg\\
% 	\url{http://kwarc.info/}
% }
%
% \maketitle
%
%\ifinfulldoc\else
% This is the documentation for the \pkg{stex-terms} package.
% For a more high-level introduction, 
%  see \href{\basedocurl/manual.pdf}{the \sTeX Manual} or the
% \href{\basedocurl/stex.pdf}{full \sTeX documentation}.
%
% \textcolor{red}{TODO: terms documentation}
% \fi
%
% \begin{documentation}\label{pkg:terms:doc}
%
% Code related to symbolic expressions, typesetting notations,
% notation components, etc.
%
% \section{Macros and Environments}\label{pkg:terms:doc:macros}
%
% \begin{function}{\STEXsymbol}
%   Uses \cs{stex_get_symbol:n} to find the symbol denoted by
%   the first argument and passes the result on to
%   \cs{stex_invoke_symbol:n}
% \end{function}
%
% \begin{function}{\symref}
%   \begin{syntax} \cs{symref}\Arg{symbol}\Arg{text} \end{syntax}
%   shortcut for \cs{STEXsymbol}\Arg{symbol}|![|\meta{text}|]|
% \end{function}
%
% \begin{function}{\stex_invoke_symbol:n}
%   Executes a semantic macro. Outside of math mode or if followed by |*|,
%   it continues to \cs{stex_term_custom:nn}. In math mode,
%   it uses the default or optionally provided notation of
%   the associated symbol.
%
%   If followed by |!|, it will invoke the symbol \emph{itself}
%   rather than its application (and continue to
%   \cs{stex_term_custom:nn}), i.e. it allows to refer to
%   |\plus![addition]| as an operation, rather than
%   |\plus[addition of]{some}{terms}|.
% \end{function}
%
% \begin{function}{\_stex_term_math_oms:nnnn,\_stex_term_math_oma:nnnn,\_stex_term_math_omb:nnnn}
%   \begin{syntax} \meta{URI}\meta{fragment}\meta{precedence}\meta{body} \end{syntax}
%
% Annotates \meta{body} as an \omdoc-term (|OMID|, |OMA| or |OMBIND|, respectively) 
% with head symbol \meta{URI}, generated
% by the specific notation \meta{fragment} with (upwards) operator precedence
% \meta{precedence}. Inserts parentheses according to
% the current downwards precedence and operator precedence.
% \end{function}
%
% \begin{function}{\_stex_term_math_arg:nnn}
%   \begin{syntax} \cs{stex_term_arg:nnn}\meta{int}\meta{prec}\meta{body} \end{syntax}
% Annotates \meta{body} as the \meta{int}th argument of the current |OMA| or |OMBIND|,
% with (downwards) argument precedence \meta{prec}.
% \end{function}
%
% \begin{function}{\_stex_term_math_assoc_arg:nnnn}
%   \begin{syntax} \cs{stex_term_arg:nnn}\meta{int}\meta{prec}\meta{notation}\meta{body} \end{syntax}
% Annotates \meta{body} as the \meta{int}th (associative) \emph{sequence} argument
% (as comma-separated list of terms) of the current |OMA| or |OMBIND|,
% with (downwards) argument precedence \meta{prec} and associative
% notation \meta{notation}.
% 
% \end{function}
%
% \begin{variable}{\infprec, \neginfprec}
%   Maximal and minimal notation precedences.
% \end{variable}
%
% \begin{function}{\dobrackets}
%   \begin{syntax} \cs{dobrackets} \Arg{body} \end{syntax}
%   Puts \meta{body} in parentheses; scaled if in display mode
%   unscaled otherwise. Uses the current \sTeX brackets (by default |(| and |)|),
%   which can be changed temporarily using \cs{withbrackets}.
% \end{function}
%
% \begin{function}{\withbrackets}
%   \begin{syntax} \cs{withbrackets} \meta{left} \meta{right} \Arg{body} \end{syntax}
%   Temporarily (i.e. within \meta{body}) sets the brackets used by \sTeX for automated
%   bracketing (by default |(| and |)|) to \meta{left} and \meta{right}.
%
%   Note that \meta{left} and \meta{right} need to be allowed
%   after \cs{left} and \cs{right} in displaymode.
% \end{function}
%
% \begin{function}{\stex_term_custom:nn}
%   \begin{syntax} \cs{stex_term_custom:nn}\Arg{URI}\Arg{args}\end{syntax}
% Implements custom one-time notation.
% Invoked by \cs{stex_invoke_symbol:n} in text mode, or if
% followed by |*| in math mode, or whenever followed by |!|.
% \end{function}
%
% \begin{function}{\stex_highlight_term:nn}
%   \begin{syntax} \cs{stex_highlight_term:nn}\Arg{URI}\Arg{args}\end{syntax}
% Establishes a context for \cs{comp}. Stores the URI in a variable
% so that \cs{comp} knows which symbol governs the current notation.
% \end{function}
%
% \begin{function}{\comp, \compemph,\compemph@uri, \defemph, \defemph@uri, \symrefemph,\symrefemph@uri}
%   \begin{syntax} \cs{comp}\Arg{args}\end{syntax}
% Marks \meta{args} as a notation component of the current symbol for
% highlighting, linking, etc.
%
% The precise behavior is governed by \cs{@comp}, which takes as
% additional argument the URI of the current symbol. By default,
% \cs{@comp} adds the URI as a PDF tooltip and colors the highlighted part
% in blue.
%
% \cs{@defemph} behaves like \cs{@comp}, and can be similarly redefined,
% but marks an expression as \emph{definiendum} (used by \cs{definiendum})
% \end{function}
%
% \begin{function}{\STEXinvisible}
% Exports its argument as \omdoc (invisible), but does
% not produce PDF output. Useful e.g. for semantic macros
% that take arguments that are not part of the symbolic
% notation.
% \end{function}
%
% \begin{function}{\ellipses}
%   TODO
% \end{function}
%
% \end{documentation}
%
% \begin{implementation}\label{pkg:terms:impl}
%
% \section{\sTeX-Terms Implementation}
%
%    \begin{macrocode}
%<*package>

%%%%%%%%%%%%%   terms.dtx   %%%%%%%%%%%%%

%<@@=stex_terms>
%    \end{macrocode}
%
% Warnings and error messages
%
%    \begin{macrocode}
\msg_new:nnn{stex}{error/nonotation}{
  Symbol~#1~invoked,~but~has~no~notation#2!
}
\msg_new:nnn{stex}{error/notationarg}{
  Error~in~parsing~notation~#1
}
\msg_new:nnn{stex}{error/noop}{
  Symbol~#1~has~no~operator~notation~for~notation~#2
}
\msg_new:nnn{stex}{error/notallowed}{
  Symbol~invocation~#1~not~allowed~in~notation~component~of~#2
}

%    \end{macrocode}
% \subsection{Symbol Invocations}
%
%
% \begin{macro}{\stex_invoke_symbol:n}
%
%  Invokes a semantic macro
%
%    \begin{macrocode}
\keys_define:nn { stex / terms } {
  lang    .tl_set_x:N = \l_@@_lang_str ,
  variant .tl_set_x:N = \l_@@_variant_str ,
  unknown .code:n     = \str_set:Nx 
      \l_@@_variant_str \l_keys_key_str
}

\cs_new_protected:Nn \_@@_args:n {
  \str_clear:N \l_@@_lang_str
  \str_clear:N \l_@@_variant_str
  
  \keys_set:nn { stex / terms } { #1 }
}

\cs_new:Nn \_@@_reset:N {
  \tl_if_exist:NTF #1 {
    \def \exp_not:N #1 { \exp_args:No \exp_not:n #1 }
  }{
    \let \exp_not:N #1 \exp_not:N \undefined
  }
}

\bool_new:N \l_@@_allow_semantic_bool
\bool_set_true:N \l_@@_allow_semantic_bool

\cs_new_protected:Nn \stex_invoke_symbol:n {
  \bool_if:NTF \l_@@_allow_semantic_bool {
    \str_if_eq:eeF {
      \prop_item:cn {
        l_stex_symdecl_#1_prop
      }{ deprecate }
    }{}{
      \msg_warning:nnxx{stex}{warning/deprecated}{
        Symbol~#1
      }{
        \prop_item:cn {l_stex_symdecl_#1_prop}{ deprecate }
      }
    }
    \if_mode_math:
      \exp_after:wN \_@@_invoke_math:n
    \else:
      \exp_after:wN \_@@_invoke_text:n
    \fi: { #1 }
  }{
    \msg_error:nnxx{stex}{error/notallowed}{#1}{\l_stex_current_symbol_str}
  }
}

\cs_new_protected:Nn \_@@_invoke_text:n {
  \peek_charcode_remove:NTF ! {
    \_@@_invoke_op_custom:nn {#1}
  }{
    \_@@_invoke_custom:nn {#1}
  }
}

\cs_new_protected:Nn \_@@_invoke_math:n {
  \peek_charcode_remove:NTF ! {
    % operator
    \peek_charcode_remove:NTF * {
      % custom op
      \_@@_invoke_op_custom:nn {#1}
    }{
      % op notation
      \peek_charcode:NTF [ {
        \_@@_invoke_op_notation:nw {#1}
      }{
        \_@@_invoke_op_notation:nw {#1}[]
      }
    }
  }{
    \peek_charcode_remove:NTF * {
      \_@@_invoke_custom:nn {#1}
      % custom
    }{
      % normal
      \peek_charcode:NTF [ {
        \_@@_invoke_notation:nw {#1}
      }{
        \_@@_invoke_notation:nw {#1}[]
      }
    }
  }
}


\cs_new_protected:Nn \_@@_invoke_op_custom:nn {
  \exp_args:Nnx \use:nn {
    \str_set:Nn \l_stex_current_symbol_str { #1 }
    \bool_set_false:N \l_@@_allow_semantic_bool
    \_stex_term_oms:nnn {#1 \c_hash_str\c_hash_str}{#1}{
      \comp{ #2 }
    }
  }{
    \_@@_reset:N \l_stex_current_symbol_str
    \bool_set_true:N \l_@@_allow_semantic_bool
  }
}

\cs_new_protected:Nn \_@@_find_notation:nn {
  \str_set:Nn \l_stex_current_symbol_str { #1 }
  \_@@_args:n { #2 }
  \seq_if_empty:cTF {
    l_stex_symdecl_ #1 _notations 
  } {
    \msg_error:nnxx{stex}{error/nonotation}{#1}{s}
  } {
    \bool_lazy_all:nTF {
      {\str_if_empty_p:N \l_@@_variant_str}
      {\str_if_empty_p:N \l_@@_lang_str}
    }{
      \seq_get_left:cN {l_stex_symdecl_#1_notations}\l_@@_variant_str
    }{
      \seq_if_in:cxTF {l_stex_symdecl_#1_notations}{
        \l_@@_variant_str \c_hash_str \l_@@_lang_str
      }{
        \str_set:Nx \l_@@_variant_str { \l_@@_variant_str \c_hash_str \l_@@_lang_str }
      }{
        \msg_error:nnxx{stex}{error/nonotation}{#1}{
          ~\l_@@_variant_str \c_hash_str \l_@@_lang_str
        }
      }
    }
  }
}

\cs_new_protected:Npn \_@@_invoke_op_notation:nw #1 [#2] {
  \_@@_find_notation:nn { #1 }{ #2 }
  \bool_set_false:N \l_@@_allow_semantic_bool
  \cs_if_exist:cTF {
    stex_op_notation_ #1 \c_hash_str \l_@@_variant_str _cs
  }{
    \use:c{stex_op_notation_ #1 \c_hash_str \l_@@_variant_str _cs}
  }{
    \msg_error:nnxx{stex}{error/noop}{#1}{\l_@@_variant_str}
  }
  \bool_set_true:N \l_@@_allow_semantic_bool
}

\cs_new_protected:Npn \_@@_invoke_notation:nw #1 [#2] {
  \_@@_find_notation:nn { #1 }{ #2 }
  \cs_if_exist:cTF {
    stex_notation_ #1 \c_hash_str \l_@@_variant_str _cs
  }{
    \tl_set:Nx \stex_symbol_after_invocation_tl {
      \_@@_reset:N \stex_symbol_after_invocation_tl
      \_@@_reset:N \l_stex_current_symbol_str
      \bool_set_true:N \l_@@_allow_semantic_bool
    }
    \bool_set_false:N \l_@@_allow_semantic_bool
    \use:c{stex_notation_ #1 \c_hash_str \l_@@_variant_str _cs}
  }{
    \msg_error:nnxx{stex}{error/nonotation}{#1}{
      ~\l_@@_variant_str
    }
  }
}

\prop_new:N \l_@@_custom_args_prop

\cs_new_protected:Nn \_@@_invoke_custom:nn {
  \exp_args:Nnx \use:nn {
    \bool_set_false:N \l_@@_allow_semantic_bool
    \str_set:Nn \l_stex_current_symbol_str { #1 }
    \prop_clear:N \l_@@_custom_args_prop
    \prop_put:Nnn \l_@@_custom_args_prop {currnum} {1}
    \prop_put:Nnx \l_@@_custom_args_prop {args} {
      \prop_item:cn {
        l_stex_symdecl_\l_stex_get_symbol_uri_str _prop
      }{ args }
    }
    \tl_set:Nn \arg { \_@@_arg: }
    #2
    % TODO check that all arguments exist
  }{
    \_@@_reset:N \l_stex_current_symbol_str
    \_@@_reset:N \arg
    \_@@_reset:N \l_@@_custom_args_prop
    \bool_set_true:N \l_@@_allow_semantic_bool
  }
}

\NewDocumentCommand \_@@_arg: { s O{} m}{
  \str_if_eq:nnTF {#2}{}{
    \int_set:Nn \l_tmpa_int {\prop_item:Nn \l_@@_custom_args_prop {currnum}}
    \bool_set_true:N \l_tmpa_bool
    \bool_do_while:Nn \l_tmpa_bool {
      \exp_args:NNx \prop_if_in:NnTF \l_@@_custom_args_prop {\int_use:N \l_tmpa_int} {
        \int_incr:N \l_tmpa_int
      }{
        \bool_set_false:N \l_tmpa_bool
      }
    }
  }{
    \int_set:Nn \l_tmpa_int { #2 }
    \exp_args:NNx \prop_if_in:NnT \l_@@_custom_args_prop {\int_use:N \l_tmpa_int} {
      % TODO throw error
    }
  }
  \str_set:Nx \l_tmpa_str {\prop_item:Nn \l_@@_custom_args_prop {args} }
  \int_compare:nNnT \l_tmpa_int > {\str_count:N \l_tmpa_str} {
    % TODO throw error
  }
  \IfBooleanTF#1{
    \stex_annotate_invisible:n {
      \exp_args:No \_stex_term_arg:nn {\l_stex_current_symbol_str}{#3}
    }
  }{
    \exp_args:No \_stex_term_arg:nn {\l_stex_current_symbol_str}{#3}
  }
}


\cs_new_protected:Nn \_stex_term_arg:nn {
  \exp_args:Nnx \use:nn {
    \bool_set_true:N \l_@@_allow_semantic_bool
    \stex_annotate:nnn{ arg }{ #1 }{ #2 }
  }{
    \bool_set_false:N \l_@@_allow_semantic_bool
  }
}

\cs_new_protected:Nn \_stex_term_math_arg:nnn {
  \exp_args:Nnx \use:nn
    { \int_set:Nn \l_@@_downprec { #2 } 
        \_stex_term_arg:nn { #1 }{ #3 }
    }
    { \int_set:Nn \exp_not:N \l_@@_downprec { \int_use:N \l_@@_downprec } }
}


%    \end{macrocode}
% \end{macro}
%
% \subsection{Terms}
%
% Precedences:
% \begin{variable}{\infprec, \neginfprec, \l_@@_downprec}
%    \begin{macrocode}
\tl_const:Nx \infprec {\int_use:N \c_max_int}
\tl_const:Nx \neginfprec {-\int_use:N \c_max_int}
\int_new:N \l_@@_downprec
\int_set_eq:NN \l_@@_downprec \infprec
%    \end{macrocode}
% \end{variable}
%
% Bracketing:
%
% \begin{variable}{\l_@@_left_bracket_str, \l_@@_right_bracket_str}
%    \begin{macrocode}
\tl_set:Nn \l_@@_left_bracket_str (
\tl_set:Nn \l_@@_right_bracket_str )
%    \end{macrocode}
% \end{variable}
%
% \begin{macro}{\_@@_maybe_brackets:nn}
%
% Compares precedences and insert brackets accordingly
%
%    \begin{macrocode}
\cs_new_protected:Nn \_@@_maybe_brackets:nn {
  \bool_if:NTF \l_@@_brackets_done_bool {
    \bool_set_false:N \l_@@_brackets_done_bool
    #2
  } {
    \int_compare:nNnTF { #1 } > \l_@@_downprec {
      \bool_if:NTF \l_stex_inparray_bool { #2 }{
        \stex_debug:nn{dobrackets}{\number#1 > \number\l_@@_downprec; \detokenize{#2}}
        \dobrackets { #2 }
      }
    }{ #2 }
  }
}
%    \end{macrocode}
% \end{macro}
%
% \begin{macro}{\dobrackets}
%    \begin{macrocode}
\bool_new:N \l_@@_brackets_done_bool
%\RequirePackage{scalerel}
\cs_new_protected:Npn \dobrackets #1 {
  %\ThisStyle{\if D\m@switch
  %    \exp_args:Nnx \use:nn
  %    { \exp_after:wN \left\l_@@_left_bracket_str #1 } 
  %    { \exp_not:N\right\l_@@_right_bracket_str }
  %  \else
      \exp_args:Nnx \use:nn
      { 
        \bool_set_true:N \l_@@_brackets_done_bool
        \int_set:Nn \l_@@_downprec \infprec
        \l_@@_left_bracket_str 
        #1
      } 
      {
        \bool_set_false:N \l_@@_brackets_done_bool
        \l_@@_right_bracket_str 
        \int_set:Nn \l_@@_downprec { \int_use:N \l_@@_downprec }
      }
  %\fi}
}
%    \end{macrocode}
% \end{macro}
%
% \begin{macro}{\withbrackets}
%    \begin{macrocode}
\cs_new_protected:Npn \withbrackets #1 #2 #3 {
  \exp_args:Nnx \use:nn
  {  
    \tl_set:Nx \l_@@_left_bracket_str { #1 }
    \tl_set:Nx \l_@@_right_bracket_str { #2 }
    #3
  }
  {
    \tl_set:Nn \exp_not:N \l_@@_left_bracket_str 
      {\l_@@_left_bracket_str}
    \tl_set:Nn \exp_not:N \l_@@_right_bracket_str 
      {\l_@@_right_bracket_str}
  }
}
%    \end{macrocode}
% \end{macro}
%
% \begin{macro}{\STEXinvisible}
%    \begin{macrocode}
\cs_new_protected:Npn \STEXinvisible #1 {
  \stex_annotate_invisible:n { #1 }
}
%    \end{macrocode}
% \end{macro}
%
% \omdoc terms:
%
% \begin{macro}{\_stex_term_math_oms:nnnn}
%    \begin{macrocode}
\cs_new_protected:Nn \_stex_term_oms:nnn {
  \stex_annotate:nnn{ OMID }{ #2 }{
    \stex_highlight_term:nn { #1 } { #3 } 
  }
}

\cs_new_protected:Nn \_stex_term_math_oms:nnnn {
  \_@@_maybe_brackets:nn { #3 }{ 
    \_stex_term_oms:nnn { #1 } { #1\c_hash_str#2 } { #4 }
  }
}
%    \end{macrocode}
% \end{macro}
%
% \begin{macro}{\_stex_term_math_omv:nn}
%    \begin{macrocode}
\cs_new_protected:Nn \_stex_term_omv:nn {
  \stex_annotate:nnn{ OMID }{ #1 }{
    \stex_highlight_term:nn { #1 } { #2 } 
  }
}
%    \end{macrocode}
% \end{macro}
%
% \begin{macro}{\_stex_term_math_oma:nnnn}
%    \begin{macrocode}
\cs_new_protected:Nn \_stex_term_oma:nnn {
  \stex_annotate:nnn{ OMA }{ #2 }{
    \stex_highlight_term:nn { #1 } { #3 } 
  }
}

\cs_new_protected:Nn \_stex_term_math_oma:nnnn {
  \_@@_maybe_brackets:nn { #3 }{ 
    \_stex_term_oma:nnn { #1 } { #1\c_hash_str#2 } { #4 }
  }
}
%    \end{macrocode}
% \end{macro}
%
% \begin{macro}{\_stex_term_math_omb:nnnn}
%    \begin{macrocode}
\cs_new_protected:Nn \_stex_term_ombind:nnn {
  \stex_annotate:nnn{ OMBIND }{ #2 }{
    \stex_highlight_term:nn { #1 } { #3 }
  }
}

\cs_new_protected:Nn \_stex_term_math_omb:nnnn {
  \_@@_maybe_brackets:nn { #3 }{ 
    \_stex_term_ombind:nnn { #1 } { #1\c_hash_str#2 } { #4 }
  }
}
%    \end{macrocode}
% \end{macro}
%
% \begin{macro}{\_stex_term_math_assoc_arg:nnnn}
%    \begin{macrocode}
\cs_new_protected:Nn \_stex_term_math_assoc_arg:nnnn {
  % TODO sequences
  \clist_set:Nn \l_tmpa_clist{ #3 }
  \int_compare:nNnTF { \clist_count:N \l_tmpa_clist } < 2 {
    \tl_set:Nn \l_tmpa_tl { #3 }
  }{
    \cs_set:Npn \l_tmpa_cs ##1 ##2 { #4 }
    \clist_reverse:N \l_tmpa_clist
    \clist_pop:NN \l_tmpa_clist \l_tmpa_tl

    \clist_map_inline:Nn \l_tmpa_clist {
      \exp_args:NNNo \exp_args:NNo \tl_set:No \l_tmpa_tl {
        \exp_args:Nno 
        \l_tmpa_cs { ##1 } \l_tmpa_tl 
      }
    }
  }
  \exp_args:Nnno
   \_stex_term_math_arg:nnn{#1}{#2}\l_tmpa_tl
}
%    \end{macrocode}
% \end{macro}
%
% \begin{macro}{\stex_term_custom:nn}
%    \begin{macrocode}
\cs_new_protected:Nn \stex_term_custom:nn {
  \str_set:Nn \l_@@_custom_uri { #1 }
  \str_set:Nn \l_tmpa_str { #2 }
  \tl_clear:N \l_tmpa_tl
  \int_zero:N \l_tmpa_int
  \int_set:Nn \l_tmpb_int { \str_count:N \l_tmpa_str }
  \_@@_custom_loop:
}
%    \end{macrocode}
% \end{macro}
%
% \begin{macro}{\_@@_custom_loop:}
%    \begin{macrocode}
\cs_new_protected:Nn \_@@_custom_loop: {
  \bool_set_false:N \l_tmpa_bool
  \bool_while_do:nn {
    \str_if_eq_p:ee X {
      \str_item:Nn \l_tmpa_str { \l_tmpa_int + 1 }
    }
  }{
    \int_incr:N \l_tmpa_int
  }

  \peek_charcode:NTF [ {
    % notation/text component
    \_@@_custom_component:w
  } {
    \int_compare:nNnTF \l_tmpa_int = \l_tmpb_int {
      % all arguments read => finish
      \_@@_custom_final:
    } {
      % arguments missing
      \peek_charcode_remove:NTF * {
        % invisible, specific argument position or both
        \peek_charcode:NTF [ {
          % visible specific argument position
          \_@@_custom_arg:wn
        } {
          % invisible
          \peek_charcode_remove:NTF * {
            % invisible specific argument position
            \_@@_custom_arg_inv:wn
          } {
            % invisible next argument
            \_@@_custom_arg_inv:wn [ \l_tmpa_int + 1 ]
          }
        } 
      } {
        % next normal argument
        \_@@_custom_arg:wn [ \l_tmpa_int + 1 ]
      }
    }
  }
}
%    \end{macrocode}
% \end{macro}
%
% \begin{macro}{\_@@_custom_arg_inv:wn}
%    \begin{macrocode}
\cs_new_protected:Npn \_@@_custom_arg_inv:wn [ #1 ] #2 {
  \bool_set_true:N \l_tmpa_bool
  \_@@_custom_arg:wn [ #1 ] { #2 }
}
%    \end{macrocode}
% \end{macro}
%
% \begin{macro}{\_@@_custom_arg:wn}
%    \begin{macrocode}
\cs_new_protected:Npn \_@@_custom_arg:wn [ #1 ] #2 {
  \str_set:Nx \l_tmpb_str { 
    \str_item:Nn \l_tmpa_str { #1 }
  }
  \str_case:VnTF \l_tmpb_str {
    { X } {
      \msg_error:nnx{stex}{error/notationarg}{\l_@@_custom_uri}
    }
    { i } { \_@@_custom_set_X:n { #1 } }
    { b } { \_@@_custom_set_X:n { #1 } }
    { a } { \_@@_custom_set_X:n { #1 } } % TODO ?
    { B } { \_@@_custom_set_X:n { #1 } } % TODO ?
  }{}{
    \msg_error:nnx{stex}{error/notationarg}{\l_@@_custom_uri}
  }

  \bool_if:nTF \l_tmpa_bool {
    \tl_put_right:Nx \l_tmpa_tl {
      \stex_annotate_invisible:n {
        \_stex_term_arg:nn { \int_eval:n { #1 } } 
          \exp_not:n { { #2 } }
      }
    }
  } {
    \tl_put_right:Nx \l_tmpa_tl {
      \_stex_term_arg:nn { \int_eval:n { #1 } } 
        \exp_not:n { { #2 } }
    }
  }

  \_@@_custom_loop:
}
%    \end{macrocode}
% \end{macro}
%
% \begin{macro}{\_@@_custom_set_X:n}
%    \begin{macrocode}
\cs_new_protected:Nn \_@@_custom_set_X:n {
  \str_set:Nx \l_tmpa_str {
    \str_range:Nnn \l_tmpa_str 1 { #1 - 1 }
    X
    \str_range:Nnn \l_tmpa_str { #1 + 1 } { -1 }
  }
}
%    \end{macrocode}
% \end{macro}
%
% \begin{macro}{\_@@_custom_component:}
%    \begin{macrocode}
\cs_new_protected:Npn \_@@_custom_component:w [ #1 ] {
  \tl_put_right:Nn \l_tmpa_tl { \comp{ #1 } }
  \_@@_custom_loop:
}
%    \end{macrocode}
% \end{macro}
%
% \begin{macro}{\_@@_custom_final:}
%    \begin{macrocode}
\cs_new_protected:Nn \_@@_custom_final: {
  \int_compare:nNnTF \l_tmpb_int = 0 {
    \exp_args:Nnno \_stex_term_oms:nnn
  }{
    \str_if_in:NnTF \l_tmpa_str {b} {
      \exp_args:Nnno \_stex_term_ombind:nnn
    } {
      \exp_args:Nnno \_stex_term_oma:nnn
    }
  }
  { \l_@@_custom_uri } { \l_@@_custom_uri } { \l_tmpa_tl }
}
%    \end{macrocode}
% \end{macro}
%
% \begin{macro}{\symref,\symname}
%    \begin{macrocode}
\NewDocumentCommand \symref { m m }{
  \let\compemph_uri_prev:\compemph@uri
  \let\compemph@uri\symrefemph@uri
  \STEXsymbol{#1}![#2]
  \let\compemph@uri\compemph_uri_prev:
}

\keys_define:nn { stex / symname } {
  post    .str_set_x:N   = \l_stex_symname_post_str
}

\cs_new_protected:Nn \stex_symname_args:n {
  \str_clear:N \l_stex_symname_post_str
  \keys_set:nn { stex / symname } { #1 }
}

\NewDocumentCommand \symname { O{} m }{
  \stex_symname_args:n { #1 }
  \stex_get_symbol:n { #2 }
  \str_set:Nx \l_tmpa_str {
    \prop_item:cn { l_stex_symdecl_ \l_stex_get_symbol_uri_str _prop } { name }
  }
  \exp_args:NNno \str_replace_all:Nnn \l_tmpa_str {-} {~}
  
  \let\compemph_uri_prev:\compemph@uri
  \let\compemph@uri\symrefemph@uri
  \exp_args:NNx \use:nn
  \stex_invoke_symbol:n { { \l_stex_get_symbol_uri_str }![
    \l_tmpa_str \l_stex_symname_post_str
  ] }
  \let\compemph@uri\compemph_uri_prev:
}
%    \end{macrocode}
% \end{macro}
%
%
% \subsection{Notation Components}
%    \begin{macrocode}
%<@@=stex_notationcomps>
%    \end{macrocode}
%
%
% \begin{macro}{\stex_highlight_term:nn}
%    \begin{macrocode}

\str_new:N \l_stex_current_symbol_str
\cs_new_protected:Nn \stex_highlight_term:nn {
  \exp_args:Nnx
  \use:nn {
    \str_set:Nx \l_stex_current_symbol_str { #1 }
    #2
  } {
    \str_set:Nx \exp_not:N \l_stex_current_symbol_str
      { \l_stex_current_symbol_str }
  }
}

\cs_new_protected:Nn \stex_unhighlight_term:n {
%  \latexml_if:TF {
%    #1
%  } {
%    \rustex_if:TF {
%      #1
%    } {
      #1 %\iffalse{{\fi}} #1 {{\iffalse}}\fi
%    }
%  }
}
%    \end{macrocode}
% \end{macro}
%
% \begin{macro}{\comp,\compemph@uri,\compemph,\defemph,\defemph@uri,\symrefemph,\symrefemph@uri}
%    \begin{macrocode}
\cs_new_protected:Npn \comp #1 {
  \str_if_empty:NF \l_stex_current_symbol_str {
    \rustex_if:TF {
      \stex_annotate:nnn { comp }{ \l_stex_current_symbol_str }{ #1 }
    }{
      \exp_args:Nnx \compemph@uri { #1 } { \l_stex_current_symbol_str }
    }
  }
}

\cs_new_protected:Npn \compemph@uri #1 #2 {
    \compemph{ #1 }
}


\cs_new_protected:Npn \compemph #1 {
    #1
}

\cs_new_protected:Npn \defemph@uri #1 #2 {
    \defemph{#1}
}

\cs_new_protected:Npn \defemph #1 {
    \textbf{#1}
}

\cs_new_protected:Npn \symrefemph@uri #1 #2 {
    \symrefemph{#1}
}

\cs_new_protected:Npn \symrefemph #1 {
    \textbf{#1}
}
%    \end{macrocode}
% \end{macro}
%
%
% \begin{macro}{\ellipses}
%    \begin{macrocode}
\NewDocumentCommand \ellipses {} { \ldots }
%    \end{macrocode}
% \end{macro}
%
%
% \begin{macro}{\parray,\prmatrix,\parrayline,\parraylineh,\parraycell}
%    \begin{macrocode}
\bool_new:N \l_stex_inparray_bool
\bool_set_false:N \l_stex_inparray_bool
\NewDocumentCommand \parray { m m } {
  \begingroup 
  \bool_set_true:N \l_stex_inparray_bool
  \begin{array}{#1}
    #2
  \end{array}
  \endgroup
}

\NewDocumentCommand \prmatrix { m } {
  \begingroup 
  \bool_set_true:N \l_stex_inparray_bool
  \begin{matrix}
    #1
  \end{matrix}
  \endgroup
}

\def \maybephline {
  \bool_if:NT \l_stex_inparray_bool {\hline}
}

\def \parrayline #1 #2 {
  #1 #2 \bool_if:NT \l_stex_inparray_bool {\\}
}

\def \pmrow #1 { \parrayline{}{ #1 } }

\def \parraylineh #1 #2 {
  #1 #2 \bool_if:NT \l_stex_inparray_bool {\\\hline}
}

\def \parraycell #1 {
  #1 \bool_if:NT \l_stex_inparray_bool {&}
}
%    \end{macrocode}
% \end{macro}
%
% \subsection{Variables}
%    \begin{macrocode}
%<@@=stex_variables>
%    \end{macrocode}
%
% \begin{macro}{\stex_invoke_variable:n}
%
%  Invokes a variable
%
%    \begin{macrocode}
\cs_new_protected:Nn \stex_invoke_variable:n {
  \if_mode_math:
    \exp_after:wN \_@@_invoke_math:n
  \else:
    \exp_after:wN \_@@_invoke_text:n
  \fi: {#1}
}

\cs_new_protected:Nn \_@@_invoke_text:n {
  %TODO
}


\cs_new_protected:Nn \_@@_invoke_math:n {
  \peek_charcode_remove:NTF ! {
    \peek_charcode_remove:NTF ! {
      \peek_charcode:NTF [ {
        \_@@_invoke_op_custom:nw
      }{
        % TODO throw error
      }
    }{
      \_@@_invoke_op:n { #1 }
    }
  }{
    \peek_charcode_remove:NTF * {
      \_@@_invoke_text:n { #1 }
    }{
      \_@@_invoke_math_ii:n { #1 }
    }
  }
}

\cs_new_protected:Nn \_@@_invoke_op:n {
  \cs_if_exist:cTF {
    stex_var_op_notation_ #1 _cs
  }{
    \use:c{stex_var_op_notation_ #1  _cs }
  }{
    \msg_error:nnxx{stex}{error/noop}{variable~#1}{}
  }
}

\cs_new_protected:Npn \_@@_invoke_math_ii:n  #1 {
  \cs_if_exist:cTF {
    stex_var_notation_#1_cs
  }{
    \str_set:Nn \l_stex_current_symbol_str { #1 }
    \use:c{stex_var_notation_#1_cs}
  }{
    \msg_error:nnxx{stex}{error/nonotation}{variable~#1}{s}
  }
}
%    \end{macrocode}
% \end{macro}
%
%
%    \begin{macrocode}
%</package>
%    \end{macrocode}
%
% \end{implementation}
%
% \PrintIndex

% \endinput
% Local Variables:
% mode: doctex
% TeX-master: t
% End:

  % \iffalse meta-comment
% An Infrastructure for Semantic Macros and Module Scoping
% Copyright (c) 2019 Michael Kohlhase, all rights reserved
%                this file is released under the
%                LaTeX Project Public License (LPPL)
% 
% The original of this file is in the public repository at 
% http://github.com/sLaTeX/sTeX/
%
% TODO update copyright  
%
%<*driver>
\RequirePackage{paralist}
\documentclass[full,kernel]{l3doc}
\usepackage[dvipsnames]{xcolor}
\usepackage[utf8]{inputenc}
\usepackage[T1]{fontenc}
%\usepackage{document-structure}
\usepackage[showmods,debug=all,lang={de,en}, mathhub=./tests]{stex}
\usepackage{url,array,float,textcomp}
\usepackage[show]{ed}
\usepackage[hyperref=auto,style=alphabetic]{biblatex}
\addbibresource{\bibfolder/kwarcpubs.bib}
\addbibresource{\bibfolder/extpubs.bib}
\addbibresource{\bibfolder/kwarccrossrefs.bib}
\addbibresource{\bibfolder/extcrossrefs.bib}
\usepackage{amssymb}
\usepackage{amsfonts}
\usepackage{xspace}
\usepackage{hyperref}

\usepackage{morewrites}


\makeindex
\floatstyle{boxed}
\newfloat{exfig}{thp}{lop}
\floatname{exfig}{Example}

\usepackage{stex-tests}

\MakeShortVerb{\|}

\def\scsys#1{{{\sc #1}}\index{#1@{\sc #1}}\xspace}
\def\mmt{\textsc{Mmt}\xspace}
\def\xml{\scsys{Xml}}
\def\mathml{\scsys{MathML}}
\def\omdoc{\scsys{OMDoc}}
\def\openmath{\scsys{OpenMath}}
\def\latexml{\scsys{LaTeXML}}
\def\perl{\scsys{Perl}}
\def\cmathml{Content-{\sc MathML}\index{Content {\sc MathML}}\index{MathML@{\sc MathML}!content}}
\def\activemath{\scsys{ActiveMath}}
\def\twin#1#2{\index{#1!#2}\index{#2!#1}}
\def\twintoo#1#2{{#1 #2}\twin{#1}{#2}}
\def\atwin#1#2#3{\index{#1!#2!#3}\index{#3!#2 (#1)}}
\def\atwintoo#1#2#3{{#1 #2 #3}\atwin{#1}{#2}{#3}}
\def\cT{\mathcal{T}}\def\cD{\mathcal{D}}

\def\fileversion{3.0}
\def\filedate{\today}

\RequirePackage{pdfcomment}

\ExplSyntaxOn\makeatletter
\cs_set_protected:Npn \@comp #1 #2 {
  \pdftooltip {
    \textcolor{blue}{#1}
  } { #2 }
}

\cs_set_protected:Npn \@defemph #1 #2 {
  \pdftooltip { 
    \textbf{\textcolor{magenta}{#1}}
  } { #2 }
}
\makeatother\ExplSyntaxOff

\begin{document}
  \DocInput{\jobname.dtx}
\end{document}
%</driver>
% \fi
%
% \title{ \sTeX-References
% 	\thanks{Version {\fileversion} (last revised {\filedate})} 
% }
%
% \author{Michael Kohlhase, Dennis Müller\\
% 	FAU Erlangen-Nürnberg\\
% 	\url{http://kwarc.info/}
% }
%
% \maketitle
%
% \begin{documentation}\label{pkg:sref:doc}
%
% Code related to links and cross-references
%
% \section{Macros and Environments}\label{pkg:sref:doc:macros}
%
% \end{documentation}
%
% \begin{implementation}\label{pkg:sref:impl}
%
% \section{\sTeX-References Implementation}
%
%    \begin{macrocode}
%<*package>

%%%%%%%%%%%%%   references.dtx   %%%%%%%%%%%%%

%\RequirePackage{hyperref}
%\RequirePackage{cleveref}
%<@@=stex_refs>
%    \end{macrocode}
%
% Warnings and error messages
%
%    \begin{macrocode}

%    \end{macrocode}
%
% \subsection{Document URIs and URLs}
%
%    \begin{macrocode}
\str_new:N \l_stex_current_docns_str
\cs_new_protected:Nn \stex_get_document_uri: {
  \seq_set_eq:NN \l_tmpa_seq \g_stex_currentfile_seq
  \seq_pop_right:NN \l_tmpa_seq \l_tmpb_str
  \exp_args:NNno \seq_set_split:Nnn \l_tmpb_seq . \l_tmpb_str
  \seq_get_left:NN \l_tmpb_seq \l_tmpb_str
  \seq_put_right:No \l_tmpa_seq \l_tmpb_str

  \str_clear:N \l_tmpa_str
  \prop_get:NnNF \l_stex_current_repository_prop { narr } \l_tmpa_str {
    \prop_get:NnNF \l_stex_current_repository_prop { ns } \l_tmpa_str {}
  }

  \str_if_empty:NTF \l_tmpa_str {
    \str_set:Nx \l_stex_current_docns_str { 
      file:/\stex_path_to_string:N \l_tmpa_seq
    }
  }{
    \bool_set_true:N \l_tmpa_bool
    \bool_while_do:Nn \l_tmpa_bool {
      \seq_pop_left:NN \l_tmpa_seq \l_tmpb_str
      \exp_args:No \str_case:nnTF { \l_tmpb_str } {
        {source} { \bool_set_false:N \l_tmpa_bool }
      }{}{
        \seq_if_empty:NT \l_tmpa_seq {
          \bool_set_false:N \l_tmpa_bool
        }
      }
    }
  
    \seq_if_empty:NTF \l_tmpa_seq {
      \str_set_eq:NN \l_stex_current_docns_str \l_tmpa_str
    }{
      \str_set:Nx \l_stex_current_docns_str { 
        \l_tmpa_str/\stex_path_to_string:N \l_tmpa_seq
      }
    }
  }
}
%    \end{macrocode}
%
%
%    \begin{macrocode}
\str_new:N \l_stex_current_docurl_str
\cs_new_protected:Nn \stex_get_document_url: {
  \seq_set_eq:NN \l_tmpa_seq \g_stex_currentfile_seq
  \seq_pop_right:NN \l_tmpa_seq \l_tmpb_str
  \exp_args:NNno \seq_set_split:Nnn \l_tmpb_seq . \l_tmpb_str
  \seq_get_left:NN \l_tmpb_seq \l_tmpb_str
  \seq_put_right:No \l_tmpa_seq \l_tmpb_str

  \str_clear:N \l_tmpa_str
  \prop_get:NnNF \l_stex_current_repository_prop { docurl } \l_tmpa_str {
    \prop_get:NnNF \l_stex_current_repository_prop { narr } \l_tmpa_str {
      \prop_get:NnNF \l_stex_current_repository_prop { ns } \l_tmpa_str {}
    }
  }

  \str_if_empty:NTF \l_tmpa_str {
    \str_set:Nx \l_stex_current_docurl_str { 
      file:/\stex_path_to_string:N \l_tmpa_seq
    }
  }{
    \bool_set_true:N \l_tmpa_bool
    \bool_while_do:Nn \l_tmpa_bool {
      \seq_pop_left:NN \l_tmpa_seq \l_tmpb_str
      \exp_args:No \str_case:nnTF { \l_tmpb_str } {
        {source} { \bool_set_false:N \l_tmpa_bool }
      }{}{
        \seq_if_empty:NT \l_tmpa_seq {
          \bool_set_false:N \l_tmpa_bool
        }
      }
    }
  
    \seq_if_empty:NTF \l_tmpa_seq {
      \str_set_eq:NN \l_stex_current_docurl_str \l_tmpa_str
    }{
      \str_set:Nx \l_stex_current_docurl_str { 
        \l_tmpa_str/\stex_path_to_string:N \l_tmpa_seq
      }
    }
  }
}
%    \end{macrocode}
%
% \subsection{Setting Reference Targets}
%
%    \begin{macrocode}
\str_const:Nn \c_@@_url_str{URL}
\str_const:Nn \c_@@_ref_str{REF}
\cs_new_protected:Nn \stex_ref_new_doc_target:n {
  \stex_get_document_uri:
  \str_set:Nx \l_tmpa_str { #1 }
  \str_if_empty:NT \l_tmpa_str {
    \int_zero:N \l_tmpa_int
    \bool_set_true:N \l_tmpa_bool
    \bool_while_do:Nn \l_tmpa_bool {
      \cs_if_exist:cTF {
        sref_\l_stex_current_docns_str\c_hash_str REF_\int_use:N \l_tmpa_int _type
      }{
        \int_incr:N
      }{
        \str_set:Nx \l_tmpa_str { REF_\int_use:N \l_tmpa_int }
        \bool_set_false:N \l_tmpa_bool
      }
    }
  }
  \str_set:Nx \l_tmpa_str {
    \l_stex_current_docns_str\c_hash_str\l_tmpa_str
  }
  \stex_if_smsmode:TF {
    \stex_get_document_url:
    \str_gset_eq:cN {sref_url_\l_tmpa_str _str}\l_stex_current_docurl_str
    \str_gset_eq:cN {sref_\l_tmpa_str _type}\c_@@_url_str
  }{
    \exp_after:wN\label\exp_after:wN{sref_\l_tmpa_str}
    \str_gset:cn {sref_\l_tmpa_str _type}\c_@@_ref_str
  }
}
%    \end{macrocode}
%
%    \begin{macrocode}
\cs_new_protected:Nn \stex_ref_new_sym_target:n {
  \str_gset_eq:cN {sref_sym_#1_uri} \l_stex_current_docns_str
}
%    \end{macrocode}
%
%    \begin{macrocode}
%</package>
%    \end{macrocode}
%
% \end{implementation}
%
% \PrintIndex

\end{omgroup}

\begin{omgroup}{\sTeX Archives}
  % \iffalse meta-comment
% An Infrastructure for Semantic Macros and Module Scoping
% Copyright (c) 2019 Michael Kohlhase, all rights reserved
%                this file is released under the
%                LaTeX Project Public License (LPPL)
% 
% The original of this file is in the public repository at 
% http://github.com/sLaTeX/sTeX/
%
% TODO update copyright  
%
%<*driver>
\providecommand\bibfolder{../../lib/bib}
\RequirePackage{paralist}
\documentclass[full,kernel]{l3doc}
\usepackage[dvipsnames]{xcolor}
\usepackage[utf8]{inputenc}
\usepackage[T1]{fontenc}
%\usepackage{document-structure}
\usepackage[showmods,debug=all,lang={de,en}, mathhub=./tests]{stex}
\usepackage{url,array,float,textcomp}
\usepackage[show]{ed}
\usepackage[hyperref=auto,style=alphabetic]{biblatex}
\addbibresource{\bibfolder/kwarcpubs.bib}
\addbibresource{\bibfolder/extpubs.bib}
\addbibresource{\bibfolder/kwarccrossrefs.bib}
\addbibresource{\bibfolder/extcrossrefs.bib}
\usepackage{amssymb}
\usepackage{amsfonts}
\usepackage{xspace}
\usepackage{hyperref}

\usepackage{morewrites}


\makeindex
\floatstyle{boxed}
\newfloat{exfig}{thp}{lop}
\floatname{exfig}{Example}

\usepackage{stex-tests}

\MakeShortVerb{\|}

\def\scsys#1{{{\sc #1}}\index{#1@{\sc #1}}\xspace}
\def\mmt{\textsc{Mmt}\xspace}
\def\xml{\scsys{Xml}}
\def\mathml{\scsys{MathML}}
\def\omdoc{\scsys{OMDoc}}
\def\openmath{\scsys{OpenMath}}
\def\latexml{\scsys{LaTeXML}}
\def\perl{\scsys{Perl}}
\def\cmathml{Content-{\sc MathML}\index{Content {\sc MathML}}\index{MathML@{\sc MathML}!content}}
\def\activemath{\scsys{ActiveMath}}
\def\twin#1#2{\index{#1!#2}\index{#2!#1}}
\def\twintoo#1#2{{#1 #2}\twin{#1}{#2}}
\def\atwin#1#2#3{\index{#1!#2!#3}\index{#3!#2 (#1)}}
\def\atwintoo#1#2#3{{#1 #2 #3}\atwin{#1}{#2}{#3}}
\def\cT{\mathcal{T}}\def\cD{\mathcal{D}}

\def\fileversion{3.0}
\def\filedate{\today}

\RequirePackage{pdfcomment}

\ExplSyntaxOn\makeatletter
\cs_set_protected:Npn \@comp #1 #2 {
  \pdftooltip {
    \textcolor{blue}{#1}
  } { #2 }
}

\cs_set_protected:Npn \@defemph #1 #2 {
  \pdftooltip { 
    \textbf{\textcolor{magenta}{#1}}
  } { #2 }
}
\makeatother\ExplSyntaxOff

\begin{document}
  \DocInput{\jobname.dtx}
\end{document}
%</driver>
% \fi
%
% \title{ \sTeX-MathHub
% 	\thanks{Version {\fileversion} (last revised {\filedate})} 
% }
%
% \author{Michael Kohlhase, Dennis Müller\\
% 	FAU Erlangen-Nürnberg\\
% 	\url{http://kwarc.info/}
% }
%
% \maketitle
%
% \begin{documentation}\label{pkg:mathhub:doc}
%
% Code related to managing and using MathHub repositories,
% files, paths and related hooks and methods.
%
% \section{Macros and Environments}\label{pkg:mathhub:doc:macros}
%
% \begin{function}{\stex_kpsewhich:n}
% |\stex_kpsewhich:n| executes kpsewhich and stores the return
% in\\ |\l_stex_kpsewhich_return_str|. This does not require
% shell escaping.
% \end{function}
%
%   \subsection{Files, Paths, URIs}
%
% \begin{function}{\stex_path_from_string:Nn, 
%    \stex_path_from_string:NV,
%    \stex_path_from_string:cn,
%    \stex_path_from_string:cV
%    }
%
%   \begin{syntax} \cs{stex_path_from_string:Nn} \meta{path-variable} \Arg{string} \end{syntax}
%   turns the \meta{string} into a path by splitting it at |/|-characters
%   and stores the result in \meta{path-variable}. Also applies
%   \cs{stex_path_canonicalize:N}.
% \end{function}
%
% \begin{function}{\stex_path_to_string:NN, \stex_path_to_string:N}
%   The inverse; turns a path into a string and stores it in the second
% argument variable, or leaves it in the input stream.
% \end{function}
%
% \begin{function}{\stex_path_canonicalize:N}
%   Canonicalizes the path provided; in particular, resolves |.| and |..|
%   path segments.
% \end{function}
%
% \begin{function}[pTF]{\stex_path_if_absolute:N}
%   Checks whether the path provided is \emph{absolute}, i.e. starts
%   with an empty segment
% \end{function}
%
% \begin{variable}{\c_stex_pwd_seq, \c_stex_pwd_str, \c_stex_mainfile_seq, \c_stex_mainfile_str}
%   Store the current working directory as path-sequence and string,
%   respectively, and the (heuristically guessed) full path to the
%   main file, based on the PWD and |\jobname|.
% \end{variable}
%
% \begin{variable}{\g_stex_currentfile_seq}
%   The file being currently processed (respecting |\input| etc.)
% \end{variable}
%
% \stextest{
% \ExplSyntaxOn
% \def\cpath@print#1{
%    \stex_path_from_string:Nn \l_tmpb_seq { #1 }
%    \stex_path_to_string:NN \l_tmpb_seq \l_tmpa_str
%    \str_use:N \l_tmpa_str
%  }
% \ExplSyntaxOff
% \begin{center}
% \begin{tabular}{|l|l|l|}\hline
%   path & canonicalized path & expected\\\hline
%   aaa & \cpath@print{aaa} & aaa \\
%   ../../aaa & \cpath@print{../../aaa} &  ../../aaa\\
%   aaa/bbb & \cpath@print{aaa/bbb} & aaa/bbb \\
%   aaa/.. & \cpath@print{aaa/..} &\\
%   ../../aaa/bbb & \cpath@print{../../aaa/bbb} & ../../aaa/bbb\\
%   ../aaa/../bbb & \cpath@print{../aaa/../bbb} & ../bbb \\
%   ../aaa/bbb & \cpath@print{../aaa/bbb} &  ../aaa/bbb\\
%   aaa/bbb/../ddd & \cpath@print{aaa/bbb/../ddd} & aaa/ddd\\
%   aaa/bbb/./ddd & \cpath@print{aaa/bbb/./ddd} & aaa/bbb/ddd\\
%   ./ & \cpath@print{./} & \\
%   aaa/bbb/../.. & \cpath@print{aaa/bbb/../..} & \\\hline
% \end{tabular}
% \end{center}
% }
%
%
%  \subsection{MathHub Archives}
%
% \begin{variable}{\mathhub, \c_stex_mathhub_seq, \c_stex_mathhub_str}
% We determine the path to the local MathHub folder via one of
% three means, in order of precedence:
% \begin{enumerate}
%   \item The |mathhub| package option, or
%   \item the |\mathhub|-macro, if it has been defined before
%     the |\usepackage{stex}|-statement, or
%   \item the |MATHHUB| system variable.
% \end{enumerate}
% In all three cases, \cs{c_stex_mathhub_seq} and
% \cs{c_stex_mathhub_str} are set accordingly.
% \end{variable}
%
% \begin{variable}{\l_stex_current_repository_prop}
%   Always points to the \emph{current} MathHub repository (if
%   we currently are in one). Has the fields |id|, |ns| (namespace),
%   |narr| (narrative namespace; currently not in use) and
%   |deps| (dependencies; currently not in use).
% \end{variable}
%
% \begin{function}{\stex_set_current_repository:n}
%   Sets the current repository to the one with the provided ID.
%   calls \cs{__stex_mathhub_do_manifest:n}, so works whether this
%   repository's |MANIFEST.MF|-file has already been read or not.
% \end{function}
%
% \begin{function}{\stex_require_repository:n}
%   Calls \cs{__stex_mathhub_do_manifest:n} iff the corresponding
%   archive property list does not already exist, and
%   adds a corresponding definition to the |.sms|-file.
% \end{function}
%
% \begin{function}{\stex_in_repository:nn}
%   \begin{syntax}\cs{stex_in_repository:nn}\Arg{repository-name}\Arg{code}\end{syntax}
% Change the current repository to \Arg{repository-name} (or not, if \Arg{repository-name} is
% empty), and passes its ID on to \Arg{code} as |#1|. Switches back
% to the previous repository after executing \Arg{code}.
% \end{function}
%
% \begin{function}[EXP]{\mhpath}
%   \begin{syntax}\cs{mhpath}\Arg{archive-ID}\Arg{filename}\end{syntax}
% Expands to the full path of file \meta{filename} in repository \meta{archive-ID}.
% Does not check whether the file or the repository exist. 
% \end{function}
%
% \begin{function}{\inputref,\inputref:nn}
%   \begin{syntax}\cs{inputref}|[|\meta{archive-ID}|]|\Arg{filename}\end{syntax}
% \cs{input}s the file \meta{filename} in repository \meta{archive-ID}.
% \end{function}
%
% \begin{function}{\libinput}
%   \begin{syntax} \cs{libinput}\Arg{filename} \end{syntax}
%   Inputs \meta{filename}|.tex| from the |lib| folders in the
%   current archive and the |meta-inf|-archive of the current archive group
%   (if existent). Throws an error if no file by that name exists in
%   either folder, includes both if both exist.
% \end{function}
%
% \stextest{
%  \ExplSyntaxOn
%  \stex_require_repository:n { Foo/Bar }
%  id:~\prop_item:cn {c_stex_mathhub_Foo/Bar_manifest_prop} {id}\ \\
%  narr:~\prop_item:cn {c_stex_mathhub_Foo/Bar_manifest_prop} {narr}\ \\
%  ns:~\prop_item:cn {c_stex_mathhub_Foo/Bar_manifest_prop} {ns}\ \\
%  deps:~\prop_item:cn {c_stex_mathhub_Foo/Bar_manifest_prop} {deps}\ \\
% \stex_require_repository:n { Bar/Foo }
%  \ExplSyntaxOff
% }
%
%
%
% \end{documentation}
%
% \begin{implementation}
%
% \section{\sTeX-MathHub Implementation}\label{pkg:mathhub:doc:impl}
%
%    \begin{macrocode}
%<*package>

%%%%%%%%%%%%%   mathhub.dtx   %%%%%%%%%%%%%

%<@@=stex_path>
%    \end{macrocode}
%
% Warnings and error messages
%
%    \begin{macrocode}
\msg_new:nnn{stex}{error/norepository}{
  No~archive~#1~found~in~#2
}
\msg_new:nnn{stex}{error/notinarchive}{
  Not~currently~in~an~archive,~but~\detokenize{#1}~
  needs~one!
}
\msg_new:nnn{stex}{error/nofile}{
  \detokenize{#1}~could~not~find~file~#2
}
%    \end{macrocode}
%
% \subsubsection{Generic Path Handling}
%
% We treat paths as \LaTeX3-sequences (of the individual
% path segments, i.e. separated by a /-character) unix-style;
% i.e. a path is absolute if the sequence starts with an empty 
% entry.
%
% \begin{macro}{\stex_path_from_string:Nn, 
%    \stex_path_from_string:NV,
%    \stex_path_from_string:cn,
%    \stex_path_from_string:cV
%    }
%    \begin{macrocode}
\cs_new_protected:Nn \stex_path_from_string:Nn {
  \str_set:Nx \l_tmpa_str { #2 }
  \str_if_empty:NTF \l_tmpa_str {
    \seq_clear:N #1
  }{
    \exp_args:NNNo \seq_set_split:Nnn #1 / { \l_tmpa_str }    
    \sys_if_platform_windows:T{
      \seq_clear:N \l_tmpa_tl
      \seq_map_inline:Nn #1 {
        \seq_set_split:Nnn \l_tmpb_tl \c_backslash_str { ##1 }
        \seq_concat:NNN \l_tmpa_tl \l_tmpa_tl \l_tmpb_tl
      }
      \seq_set_eq:NN #1 \l_tmpa_tl
    }
    \stex_path_canonicalize:N #1
  }
}
\cs_generate_variant:Nn \stex_path_from_string:Nn 
  { NV, cn, cV }
%    \end{macrocode}
% \end{macro}
%
% \begin{macro}{\stex_path_to_string:NN,\stex_path_to_string:N}
%    \begin{macrocode}
\cs_new_protected:Nn \stex_path_to_string:NN {
  \exp_args:NNe \str_set:Nn #2 { \seq_use:Nn #1 / }
}

\cs_new:Nn \stex_path_to_string:N {
  \seq_use:Nn #1 /
}
%    \end{macrocode}
% \end{macro}
%
% \begin{variable}{\c_@@_dot_str,\c_@@_up_str}
%
% |.| and |..|, respectively.
%
%    \begin{macrocode}
\str_const:Nn \c_@@_dot_str {.}
\str_const:Nn \c_@@_up_str {..}
%    \end{macrocode}
% \end{variable}
%
% \begin{macro}{\stex_path_canonicalize:N}
%
%  Canonicalizes the path provided; in particular, resolves |.| and |..|
%  path segments.
%
%    \begin{macrocode}
\cs_new_protected:Nn \stex_path_canonicalize:N {
  \seq_if_empty:NF #1 {
    \seq_clear:N \l_tmpa_seq
    \seq_get_left:NN #1 \l_tmpa_tl
    \str_if_empty:NT \l_tmpa_tl {
      \seq_put_right:Nn \l_tmpa_seq {}
    }
    \seq_map_inline:Nn #1 {
      \str_set:Nn \l_tmpa_tl { ##1 }
      \str_if_eq:NNTF \l_tmpa_tl \c_@@_dot_str {} {
        \str_if_eq:NNTF \l_tmpa_tl \c_@@_up_str {
          \seq_if_empty:NTF \l_tmpa_seq {
            \exp_args:NNo \seq_put_right:Nn \l_tmpa_seq {
              \c_@@_up_str
            }
          }{
            \seq_get_right:NN \l_tmpa_seq \l_tmpa_tl
            \str_if_eq:NNTF \l_tmpa_tl \c_@@_up_str {
              \exp_args:NNo \seq_put_right:Nn \l_tmpa_seq {
                \c_@@_up_str
              }
            }{
              \seq_pop_right:NN \l_tmpa_seq \l_tmpb_tl
            }
          }
        }{
          \str_if_empty:NF \l_tmpa_tl {
            \exp_args:NNo \seq_put_right:Nn \l_tmpa_seq { \l_tmpa_tl }
          }
        }
      }
    }
    \seq_gset_eq:NN #1 \l_tmpa_seq
  }
}
%    \end{macrocode}
% \end{macro}
%
% \begin{macro}[pTF]{\stex_path_if_absolute:N}
%    \begin{macrocode}
\prg_new_conditional:Nnn \stex_path_if_absolute:N {p, T, F, TF} {
  \seq_if_empty:NTF #1 {
    \prg_return_false:
  }{
    \seq_get_left:NN #1 \l_tmpa_tl
    \str_if_empty:NTF \l_tmpa_tl {
      \prg_return_true:
    }{
      \prg_return_false:
    }
  }
}
%    \end{macrocode}
% \end{macro}
%
% \subsubsection{PWD and kpsewhich}
%
% \begin{macro}{\stex_kpsewhich:n}
%    \begin{macrocode}
\str_new:N\l_stex_kpsewhich_return_str
\cs_new_protected:Nn \stex_kpsewhich:n {
  \sys_get_shell:nnN { kpsewhich ~ #1 } { } \l_tmpa_tl
  \exp_args:NNo\str_set:Nn\l_stex_kpsewhich_return_str{\l_tmpa_tl}
  \tl_trim_spaces:N \l_stex_kpsewhich_return_str
}
%    \end{macrocode}
% \end{macro}
%
% We determine the PWD
%
% \begin{variable}{\c_stex_pwd_seq,\c_stex_pwd_str}
%    \begin{macrocode}
\sys_if_platform_windows:TF{
  \stex_kpsewhich:n{-expand-var~\c_percent_str CD\c_percent_str}
}{
  \stex_kpsewhich:n{-var-value~PWD}
}

\stex_path_from_string:Nn\c_stex_pwd_seq\l_stex_kpsewhich_return_str
\stex_path_to_string:NN\c_stex_pwd_seq\c_stex_pwd_str
\stex_debug:nn {mathhub} {PWD:~\str_use:N\c_stex_pwd_str}
%    \end{macrocode}
% \end{variable}
%
% \subsubsection{File Hooks and Tracking}
%    \begin{macrocode}
%<@@=stex_files>
%    \end{macrocode}
%
% We introduce hooks for file inputs that keep track of the
% absolute paths of files used. This will be useful to keep track
% of modules, their archives, namespaces etc.
%
% Note that the absolute paths are only accurate in |\input|-statements
% for paths relative to the PWD, so they shouldn't be relied upon
% in any other setting than for \sTeX-purposes.
%
% \begin{variable}{\g_@@_stack}
%
% keeps track of file changes
%
%    \begin{macrocode}
\seq_gclear_new:N\g_@@_stack
%    \end{macrocode}
% \end{variable}
%
% \begin{variable}{\c_stex_mainfile_seq, \c_stex_mainfile_str}
%    \begin{macrocode}
\str_set:Nx \c_stex_mainfile_str {\c_stex_pwd_str/\jobname.tex}
\stex_path_from_string:Nn \c_stex_mainfile_seq 
  \c_stex_mainfile_str
%    \end{macrocode}
% \end{variable}
%
% \begin{variable}{\g_stex_currentfile_seq}
%
% Hooks for file inputs that push/pop \cs{g_@@_stack} to update
% \cs{c_stex_mainfile_seq}.
%
%    \begin{macrocode}
\seq_gclear_new:N\g_stex_currentfile_seq
\AddToHook{file/before}{
  \stex_path_from_string:Nn\g_stex_currentfile_seq{\CurrentFilePath}
  \stex_path_if_absolute:NTF\g_stex_currentfile_seq{
    \exp_args:NNe\seq_put_right:Nn\g_stex_currentfile_seq{\CurrentFile}
  }{
    \stex_path_from_string:Nn\g_stex_currentfile_seq{
      \c_stex_pwd_str/\CurrentFilePath/\CurrentFile
    }
  }
  \seq_gset_eq:NN\g_stex_currentfile_seq\g_stex_currentfile_seq
  \exp_args:NNo\seq_gpush:Nn\g_@@_stack\g_stex_currentfile_seq
}
\AddToHook{file/after}{
  \seq_if_empty:NF\g_@@_stack{
    \seq_gpop:NN\g_@@_stack\l_tmpa_seq
  }
  \seq_if_empty:NTF\g_@@_stack{
    \seq_gset_eq:NN\g_stex_currentfile_seq\c_stex_mainfile_seq
  }{
    \seq_get:NN\g_@@_stack\l_tmpa_seq
    \seq_gset_eq:NN\g_stex_currentfile_seq\l_tmpa_seq
  }
}
%    \end{macrocode}
% \end{variable}
%
% \subsection{MathHub Repositories}
%    \begin{macrocode}
%<@@=stex_mathhub>
%    \end{macrocode}
%
% \begin{variable}{\mathhub, \c_stex_mathhub_seq, \c_stex_mathhub_str}
%    \begin{macrocode}
\str_if_empty:NTF\mathhub{
  \stex_kpsewhich:n{-var-value~MATHHUB}
  \str_set_eq:NN\c_stex_mathhub_str\l_stex_kpsewhich_return_str
  
  \str_if_empty:NTF\c_stex_mathhub_str{
    \msg_warning:nn{stex}{warning/nomathhub}
  }{
    \stex_debug:nn{mathhub} {MathHub:~\str_use:N\c_stex_mathhub_str}
    \exp_args:NNo \stex_path_from_string:Nn\c_stex_mathhub_seq\c_stex_mathhub_str
  }
}{
  \stex_path_from_string:Nn \c_stex_mathhub_seq \mathhub
  \stex_path_if_absolute:NF \c_stex_mathhub_seq {
    \exp_args:NNx \stex_path_from_string:Nn \c_stex_mathhub_seq {
      \c_stex_pwd_str/\mathhub
    }
  }
  \stex_path_to_string:NN\c_stex_mathhub_seq\c_stex_mathhub_str
  \stex_debug:nn{mathhub} {MathHub:~\str_use:N\c_stex_mathhub_str}
}
%    \end{macrocode}
% \end{variable}
%
%
% \begin{macro}{\_@@_do_manifest:n}
%    \begin{macrocode}
\cs_new_protected:Nn \_@@_do_manifest:n {
  \str_set:Nx \l_tmpa_str { #1 }
  \prop_if_exist:cF {c_stex_mathhub_#1_manifest_prop} {
    \prop_new:c { c_stex_mathhub_#1_manifest_prop }
    \seq_set_split:NnV \l_tmpa_seq / \l_tmpa_str
    \seq_concat:NNN \l_tmpa_seq \c_stex_mathhub_seq \l_tmpa_seq
    \_@@_find_manifest:N \l_tmpa_seq
    \seq_if_empty:NTF \l_@@_manifest_file_seq {
      \msg_error:nnxx{stex}{error/norepository}{#1}{
        \stex_path_to_string:N \c_stex_mathhub_str
      }
    } {
      \exp_args:No \_@@_parse_manifest:n { \l_tmpa_str }
    }
  }
}
%    \end{macrocode}
% \end{macro}
%
% \begin{variable}{\l_@@_manifest_file_seq}
%    \begin{macrocode}
\str_new:N\l_@@_manifest_file_seq
%    \end{macrocode}
% \end{variable}
%
% \begin{macro}{\_@@_find_manifest:N}
%
% Attempts to find the |MANIFEST.MF| in some file path and
% stores its path in \cs{l_@@_manifest_file_seq}:
%
%    \begin{macrocode}
\cs_new_protected:Nn \_@@_find_manifest:N {
  \seq_set_eq:NN\l_tmpa_seq #1
  \bool_set_true:N\l_tmpa_bool
  \bool_while_do:Nn \l_tmpa_bool {
    \seq_if_empty:NTF \l_tmpa_seq {
      \bool_set_false:N\l_tmpa_bool
    }{
      \file_if_exist:nTF{
        \stex_path_to_string:N\l_tmpa_seq/MANIFEST.MF
      }{
        \seq_put_right:Nn\l_tmpa_seq{MANIFEST.MF}
        \bool_set_false:N\l_tmpa_bool
      }{
        \file_if_exist:nTF{
          \stex_path_to_string:N\l_tmpa_seq/META-INF/MANIFEST.MF
        }{
          \seq_put_right:Nn\l_tmpa_seq{META-INF}
          \seq_put_right:Nn\l_tmpa_seq{MANIFEST.MF}
          \bool_set_false:N\l_tmpa_bool
        }{
          \file_if_exist:nTF{
            \stex_path_to_string:N\l_tmpa_seq/meta-inf/MANIFEST.MF
          }{
            \seq_put_right:Nn\l_tmpa_seq{meta-inf}
            \seq_put_right:Nn\l_tmpa_seq{MANIFEST.MF}
            \bool_set_false:N\l_tmpa_bool
          }{
            \seq_pop_right:NN\l_tmpa_seq\l_tmpa_tl
          }
        }
      }
    }
  }
  \seq_set_eq:NN\l_@@_manifest_file_seq\l_tmpa_seq
}
%    \end{macrocode}
% \end{macro}
%
% \begin{variable}{\c_@@_manifest_ior}
%
%   File variable used for |MANIFEST|-files
%
%    \begin{macrocode}
\ior_new:N \c_@@_manifest_ior
%    \end{macrocode}
% \end{variable}
%
% \begin{macro}{\_@@_parse_manifest:n}
%
% Stores the entries in manifest file in the
% corresponding property list:
%
%    \begin{macrocode}
\cs_new_protected:Nn \_@@_parse_manifest:n {
  \seq_set_eq:NN \l_tmpa_seq \l_@@_manifest_file_seq
  \ior_open:Nn \c_@@_manifest_ior {\stex_path_to_string:N \l_tmpa_seq}
  \ior_map_inline:Nn \c_@@_manifest_ior {
    \str_set:Nn \l_tmpa_str {##1}
    \exp_args:NNoo \seq_set_split:Nnn 
        \l_tmpb_seq \c_colon_str \l_tmpa_str
    \seq_pop_left:NNTF \l_tmpb_seq \l_tmpa_tl {
      \exp_args:NNe \str_set:Nn \l_tmpb_tl { 
        \exp_args:NNo \seq_use:Nn \l_tmpb_seq \c_colon_str 
      }
      \exp_args:No \str_case:nnTF \l_tmpa_tl {
        {id} {
          \prop_gput:cno { c_stex_mathhub_#1_manifest_prop } 
            { id } \l_tmpb_tl
        }
        {narration-base} {
          \prop_gput:cno { c_stex_mathhub_#1_manifest_prop } 
            { narr } \l_tmpb_tl
        }
        {url-base} {
          \prop_gput:cno { c_stex_mathhub_#1_manifest_prop } 
            { docurl } \l_tmpb_tl
        }
        {source-base} {
          \prop_gput:cno { c_stex_mathhub_#1_manifest_prop } 
            { ns } \l_tmpb_tl
        }
        {ns} {
          \prop_gput:cno { c_stex_mathhub_#1_manifest_prop } 
            { ns } \l_tmpb_tl
        }
        {dependencies} {
          \prop_gput:cno { c_stex_mathhub_#1_manifest_prop } 
            { deps } \l_tmpb_tl
        }
      }{}{}
    }{}
  }
  \ior_close:N \c_@@_manifest_ior
}
%    \end{macrocode}
% \end{macro}
% 
%
% \begin{macro}{\stex_set_current_repository:n}
%    \begin{macrocode}
\cs_new_protected:Nn \stex_set_current_repository:n {
  \stex_require_repository:n { #1 }
  \prop_set_eq:Nc \l_stex_current_repository_prop { 
    c_stex_mathhub_#1_manifest_prop 
  }
}
%    \end{macrocode}
% \end{macro}
%
% \begin{macro}{\stex_require_repository:n}
%    \begin{macrocode}
\cs_new_protected:Nn \stex_require_repository:n {
  \prop_if_exist:cF { c_stex_mathhub_#1_manifest_prop } {
    \stex_debug:nn{mathhub}{Opening~archive:~#1}
    \_@@_do_manifest:n { #1 }
    \exp_args:Nx \stex_add_to_sms:n {
      \prop_const_from_keyval:cn { c_stex_mathhub_#1_manifest_prop } {
        id   = \prop_item:cn { c_stex_mathhub_#1_manifest_prop } {  id  } ,
        ns   = \prop_item:cn { c_stex_mathhub_#1_manifest_prop } {  ns  } ,
        narr = \prop_item:cn { c_stex_mathhub_#1_manifest_prop } { narr } ,
        deps = \prop_item:cn { c_stex_mathhub_#1_manifest_prop } { deps }
      }
    }
  }
}
%    \end{macrocode}
% \end{macro}
%
%\begin{variable}{\l_stex_current_repository_prop}
%
% Current MathHub repository
%
%    \begin{macrocode}
\prop_new:N \l_stex_current_repository_prop

\_@@_find_manifest:N \c_stex_pwd_seq
\seq_if_empty:NTF \l_@@_manifest_file_seq {
  \stex_debug:nn{mathhub}{Not~currently~in~a~MathHub~repository}
} {
  \_@@_parse_manifest:n { main }
  \prop_get:NnN \c_stex_mathhub_main_manifest_prop {id} 
    \l_tmpa_str
  \prop_set_eq:cN { c_stex_mathhub_\l_tmpa_str _manifest_prop }
    \c_stex_mathhub_main_manifest_prop
  \exp_args:Nx \stex_set_current_repository:n { \l_tmpa_str }
  \stex_debug:nn{mathhub}{Current~repository:~
    \prop_item:Nn \l_stex_current_repository_prop {id}
  }
}
%    \end{macrocode}
% \end{variable}
%
% \begin{macro}{\stex_in_repository:nn}
% Executes the code in the second argument in the context
% of the repository whose ID is provided as the first argument.
%    \begin{macrocode}
\cs_new_protected:Nn \stex_in_repository:nn {
  \str_set:Nx \l_tmpa_str { #1 }
  \cs_set:Npn \l_tmpa_cs ##1 { #2 }
  \str_if_empty:NTF \l_tmpa_str {
    \exp_args:Ne \l_tmpa_cs{
      \prop_item:Nn \l_stex_current_repository_prop { id }
    }
  }{
    \stex_require_repository:n \l_tmpa_str
    \str_set:Nx \l_tmpa_str { #1 }
    \exp_args:Nne \use:nn {
      \stex_set_current_repository:n \l_tmpa_str
      \exp_args:Nx \l_tmpa_cs{\l_tmpa_str}
    }{
       \stex_set_current_repository:n {
        \prop_item:Nn \l_stex_current_repository_prop { id }
       }
    }
  }
}
%    \end{macrocode}
% \end{macro}
%
% \begin{macro}{\inputref,\stex_inputref:nn}
%    \begin{macrocode}
\newif \ifinputref \inputreffalse

\cs_new_protected:Nn \stex_inputref:nn {
  \stex_in_repository:nn {#1} {
    \ifinputref
      \input{ \c_stex_mathhub_str / ##1 / source / #2 }
    \else
      \inputreftrue
      \input{ \c_stex_mathhub_str / ##1 / source / #2 }
      \inputreffalse
    \fi
  }
}
\NewDocumentCommand \inputref { O{} m}{
  \stex_inputref:nn{ #1 }{ #2 }
}

\cs_new_protected:Nn \stex_mhbibresource:nn {
  \stex_in_repository:nn {#1} {
    \addbibresource{ \c_stex_mathhub_str / ##1 / #2 }
  }
}
\newcommand\addmhbibresource[2][]{
  \stex_mhbibresource:nn{ #1 }{ #2 }
}
%    \end{macrocode}
% \end{macro}
%
%
% \begin{macro}{\mhpath}
%    \begin{macrocode}
  \def \mhpath #1 #2 {
    \exp_args:Ne \str_if_eq:nnTF{#1}{}{
      \c_stex_mathhub_str / 
        \prop_item:Nn \l_stex_current_repository_prop { id }
        / source / #2
    }{
      \c_stex_mathhub_str / #1 / source / #2
    }
  }
%    \end{macrocode}
% \end{macro}
%
% \begin{macro}{\libinput}
%    \begin{macrocode}
\cs_new_protected:Npn \libinput #1 {
  \prop_get:NnNF \l_stex_current_repository_prop {id} \l_tmpa_str {
    \msg_error:nnn{stex}{error/notinarchive}\libinput
  }
  \bool_set_false:N \l_tmpa_bool
  \tl_clear:N \l_tmpa_tl
  \seq_set_eq:NN \l_tmpa_seq \c_stex_mathhub_seq
  \seq_set_split:NnV \l_tmpb_seq / \l_tmpa_str
  \seq_pop_right:NN \l_tmpb_seq \l_tmpa_str
  \seq_pop_left:NNT \l_tmpb_seq \l_tmpb_str {
    \seq_put_right:No \l_tmpa_seq \l_tmpb_str
    \IfFileExists{ \stex_path_to_string:N \l_tmpa_seq 
      / meta-inf / lib / #1.tex}{
        \bool_set_true:N \l_tmpa_bool
        \tl_put_right:Nx \l_tmpa_tl {
          \exp_not:N \input { \stex_path_to_string:N \l_tmpa_seq 
          / meta-inf / lib / #1.tex}
        }
      }{}
  }
  \IfFileExists{ \stex_path_to_string:N \l_tmpa_seq
    / \l_tmpa_str / lib / #1.tex
  }{
    \bool_set_true:N \l_tmpa_bool
    \tl_put_right:Nx \l_tmpa_tl {
      \exp_not:N \input { \stex_path_to_string:N \l_tmpa_seq 
      / \l_tmpa_str / lib / #1.tex}
    }
  }{}
  \bool_if:NF \l_tmpa_bool {
    \msg_error:nnnx{stex}{error/nofile}\libinput{#1.tex}
  }
  \l_tmpa_tl
}
%    \end{macrocode}
% \end{macro}
%
%    \begin{macrocode}
%</package>
%    \end{macrocode}
%
% \end{implementation}
%
% \PrintIndex

\end{omgroup}

\begin{omgroup}{Creating New Modules and Symbols}
	\textcolor{red}{TODO}
  % \iffalse meta-comment
% An Infrastructure for Semantic Macros and Module Scoping
% Copyright (c) 2019 Michael Kohlhase, all rights reserved
%                this file is released under the
%                LaTeX Project Public License (LPPL)
% 
% The original of this file is in the public repository at 
% http://github.com/sLaTeX/sTeX/
%
% TODO update copyright  
%
%<*driver>
\RequirePackage{paralist}
\documentclass[full,kernel]{l3doc}
\usepackage[dvipsnames]{xcolor}
\usepackage[utf8]{inputenc}
\usepackage[T1]{fontenc}
%\usepackage{document-structure}
\usepackage[showmods,debug=all,lang={de,en}, mathhub=./tests]{stex}
\usepackage{url,array,float,textcomp}
\usepackage[show]{ed}
\usepackage[hyperref=auto,style=alphabetic]{biblatex}
\addbibresource{\bibfolder/kwarcpubs.bib}
\addbibresource{\bibfolder/extpubs.bib}
\addbibresource{\bibfolder/kwarccrossrefs.bib}
\addbibresource{\bibfolder/extcrossrefs.bib}
\usepackage{amssymb}
\usepackage{amsfonts}
\usepackage{xspace}
\usepackage{hyperref}

\usepackage{morewrites}


\makeindex
\floatstyle{boxed}
\newfloat{exfig}{thp}{lop}
\floatname{exfig}{Example}

\usepackage{stex-tests}

\MakeShortVerb{\|}

\def\scsys#1{{{\sc #1}}\index{#1@{\sc #1}}\xspace}
\def\mmt{\textsc{Mmt}\xspace}
\def\xml{\scsys{Xml}}
\def\mathml{\scsys{MathML}}
\def\omdoc{\scsys{OMDoc}}
\def\openmath{\scsys{OpenMath}}
\def\latexml{\scsys{LaTeXML}}
\def\perl{\scsys{Perl}}
\def\cmathml{Content-{\sc MathML}\index{Content {\sc MathML}}\index{MathML@{\sc MathML}!content}}
\def\activemath{\scsys{ActiveMath}}
\def\twin#1#2{\index{#1!#2}\index{#2!#1}}
\def\twintoo#1#2{{#1 #2}\twin{#1}{#2}}
\def\atwin#1#2#3{\index{#1!#2!#3}\index{#3!#2 (#1)}}
\def\atwintoo#1#2#3{{#1 #2 #3}\atwin{#1}{#2}{#3}}
\def\cT{\mathcal{T}}\def\cD{\mathcal{D}}

\def\fileversion{3.0}
\def\filedate{\today}

\RequirePackage{pdfcomment}

\ExplSyntaxOn\makeatletter
\cs_set_protected:Npn \@comp #1 #2 {
  \pdftooltip {
    \textcolor{blue}{#1}
  } { #2 }
}

\cs_set_protected:Npn \@defemph #1 #2 {
  \pdftooltip { 
    \textbf{\textcolor{magenta}{#1}}
  } { #2 }
}
\makeatother\ExplSyntaxOff

\begin{document}
  \DocInput{\jobname.dtx}
\end{document}
%</driver>
% \fi
%
% \title{ \sTeX-Modules
% 	\thanks{Version {\fileversion} (last revised {\filedate})} 
% }
%
% \author{Michael Kohlhase, Dennis Müller\\
% 	FAU Erlangen-Nürnberg\\
% 	\url{http://kwarc.info/}
% }
%
% \maketitle
%
% \begin{documentation}\label{pkg:modules:doc}
%
% Code related to Modules
%
% \section{Macros and Environments}\label{pkg:modules:doc:macros}
%
% \begin{variable}{\l_stex_current_module_prop}
% All information of a module is stored as a property list. 
% \cs{l_stex_current_module_prop}
% always points to the current module (if existent).
%
% Most importantly, the |content|-field stores all the code
% to execute on activation; i.e. when this module is being included.
%
% Additionally, it stores:
% \begin{itemize}
%   \item The \emph{name} in field |name|,
%   \item the \emph{namespace} in field |ns|,
%   \item this module's \emph{language} in field |lang|,
%   \item if a language module that translates some other
%     modules, the \emph{original} module in field |sig| (for signature),
%   \item the \emph{metatheory} in field |meta|,
%   \item the URIs of all \emph{imported modules} in field |imports|,
%   \item the names of all \emph{declarations} in field |constants|,
%   \item the \emph{file} this module was declared in in field |file|,
% \end{itemize}
% \end{variable}
%
% \begin{variable}{\l_stex_all_modules_seq}
%   Stores full URIs for all
%   modules currently in scope.
% \end{variable}
%
% \begin{variable}{\g_stex_module_files_prop,\g_stex_modules_in_file_seq}
%   A property list mapping file paths to the lists of all modules
%   declared therein. \cs{g_stex_modules_in_file_seq} always points to
%   the current file(-stream - \cs{input}s are considered the same file).
% \end{variable}
%
% \begin{function}[pTF]{\stex_if_in_module:}
%   Conditional for whether we are currently in a module
% \end{function}
%
% \begin{function}[pTF]{\stex_if_module_exists:n}
%   Conditional for whether a module with the provided URI
%   is already known.
% \end{function}
%
% \begin{function}{\stex_add_to_current_module:n,\STEXexport}
%   Adds the provided tokens to the |content| field of the current
%   module.
% \end{function}
%
% \begin{function}{\stex_add_constant_to_current_module:n}
%   Adds the declaration with the provided name to the |constants|
%   field of the current module.
% \end{function}
%
% \begin{function}{\stex_add_import_to_current_module:n}
%   Adds the module with the provided full URI to the |imports|
%   field of the current module.
% \end{function}
%
% \begin{function}{\stex_modules_compute_namespace:nN}
%   \begin{syntax} \cs{stex_modules_compute_namespace:nN} 
%     \Arg{namespace} \Arg{path}
%   \end{syntax}
%   Computes the namespace for file \meta{path} in repository
%   with namespace \meta{namespace} as follows:
%
%   If the file is |.../source/sub/file.tex|
%   and the namespace |http://some.namespace/foo|, then the namespace of
%   is |http://some.namespace/foo/sub/file|.
% \end{function}
%
% \begin{function}{\stex_modules_current_namespace:}
%   Computes the current namespace
% \end{function}
%
%\stextest{
% \ExplSyntaxOn
% \stex_modules_current_namespace:
% Namespace~1:\\ \l_stex_modules_ns_str \\
% Faking~a~repository:\\
% \stex_set_current_repository:n{Foo/Bar}
% \seq_pop_right:NN \g_stex_currentfile_seq \testtemp
% \edef\testtempb{\detokenize{source}}
% \exp_args:NNo \seq_put_right:Nn \g_stex_currentfile_seq { \testtempb }
% \edef\testtempb{\detokenize{test}}
% \exp_args:NNo \seq_put_right:Nn \g_stex_currentfile_seq { \testtempb }
% \exp_args:NNo \seq_put_right:Nn \g_stex_currentfile_seq { \testtemp }
% \stex_modules_current_namespace:
% Namespace~2:\\ \l_stex_modules_ns_str
% \ExplSyntaxOff
%}
%
% \subsection{The \texttt{module}-environment}
%
% \begin{environment}{module}
%   \begin{syntax} \cs{begin}|{module}[|\meta{options}|]|\Arg{name}\end{syntax}
% 
%   Opens a new module with name \meta{name}.
%
%   TODO document options.
% \end{environment}
%
% \begin{function}{\stex_module_setup:nn}
%   \begin{syntax}\cs{stex_module_setup:nn}\Arg{params}\Arg{name}\end{syntax}
%   Sets up a new module with name \meta{name} and optional parameters
%   \meta{params}. In particular, sets
%   \cs{l_stex_current_module_prop} appropriately.
% \end{function}
%
% \begin{function}{\stex_modules_heading:}
%   Takes care of the module header, if the |showmods| package option
%   is true. This macro can be overridden for customization.
% \end{function}
%
% \begin{environment}{@module}
%   \begin{syntax} \cs{begin}|{@module}[|\meta{options}|]|\Arg{name}\end{syntax}
%
%   Core functionality of the |module|-environment without a header.
%
% \end{environment}
%
%\stextest{
% \ExplSyntaxOn
% \stex_set_current_repository:n {Foo/Bar}
% \seq_pop_right:NN \g_stex_currentfile_seq \l_tmpa_tl
% \seq_put_right:Nx \g_stex_currentfile_seq { \tl_to_str:n{tests} }
% \seq_put_right:Nx \g_stex_currentfile_seq { \tl_to_str:n{Foo} }
% \seq_put_right:Nx \g_stex_currentfile_seq { \tl_to_str:n{Bar} }
% \seq_put_right:Nx \g_stex_currentfile_seq { \tl_to_str:n{source} }
% \seq_put_right:Nx \g_stex_currentfile_seq { \tl_to_str:n{Foo.tex} }
% \begin{@module}{Foo}
%  Module~path:~ 
%  \prop_item:Nn \l_stex_current_module_prop { ns }?
%  \prop_item:Nn \l_stex_current_module_prop { name }\\
%  Language:~\prop_item:Nn \l_stex_current_module_prop { lang }\\
%  Signature:~\prop_item:Nn \l_stex_current_module_prop { sig }\\
%  Metatheory:~\prop_item:Nn \l_stex_current_module_prop { meta }\\
% \end{@module}
% \ExplSyntaxOff
%}
%
%\stextest{
% \ExplSyntaxOn
% \stex_set_current_repository:n {Foo/Bar}
% \stex_debug:nn{modules}{Test:~\stex_path_to_string:N \g_stex_currentfile_seq }
% \seq_pop_right:NN \g_stex_currentfile_seq \l_tmpa_tl
% \seq_put_right:Nx \g_stex_currentfile_seq { \tl_to_str:n{tests} }
% \seq_put_right:Nx \g_stex_currentfile_seq { \tl_to_str:n{Foo} }
% \seq_put_right:Nx \g_stex_currentfile_seq { \tl_to_str:n{Bar} }
% \seq_put_right:Nx \g_stex_currentfile_seq { \tl_to_str:n{source} }
% \seq_put_right:Nx \g_stex_currentfile_seq { \tl_to_str:n{Foo.tex} }
% \stex_debug:nn{modules}{Test:~\stex_path_to_string:N \g_stex_currentfile_seq }
% \begin{module}[title=Foo Bar]{Bar}
%  Module~path:~ 
%  \prop_item:Nn \l_stex_current_module_prop { ns }?
%  \prop_item:Nn \l_stex_current_module_prop { name }\\
%  Language:~\prop_item:Nn \l_stex_current_module_prop { lang }\\
%  Signature:~\prop_item:Nn \l_stex_current_module_prop { sig }\\
%  Metatheory:~\prop_item:Nn \l_stex_current_module_prop { meta }\\
% \end{module}
% \ExplSyntaxOff
%}
%
% \begin{function}{\STEXModule}
%   \begin{syntax} \cs{STEXModule} \Arg{fragment} \end{syntax}
%   Attempts to find a module whose URI ends with \meta{fragment}
%   in the current scope and passes the full URI on to
%   \cs{stex_invoke_module:n}.
% \end{function}
%
% \begin{function}{\stex_invoke_module:n}
%   Invoked by \cs{STEXModule}. Needs to be followed either
%   by |!|\meta{macro} or |?|\Arg{symbolname}. In the first case,
%   it stores the full URI in \meta{macro}; in the second
%   case, it invokes the symbol \meta{symbolname} in the
%   selected module.
% \end{function}
%
%\stextest{
%   \begin{module}{STEXModuleTest1}
%     \symdecl{foo}
%   \end{module}
%   \begin{module}{STEXModuleTest2}
%     \importmodule{STEXModuleTest1}
%     \symdecl{foo}
%   \end{module}
%   \begin{module}{STEXModuleTest3}
%     \importmodule{STEXModuleTest2}
%     \symdecl{foo}
%     \STEXModule{STEXModuleTest1}!\teststring
%     \teststring\\
%     \STEXModule{STEXModuleTest2}!\teststring
%     \teststring\\
%     \STEXModule{STEXModuleTest3}!\teststring
%     \teststring\\
%     \STEXModule{STEXModuleTest1}?{foo}[\comp{foo1}]\\
%     \STEXModule{STEXModuleTest2}?{foo}[\comp{foo2}]\\
%     \STEXModule{STEXModuleTest3}?{foo}[\comp{foo3}]\\
%   \end{module}
%}
%
% \begin{function}{\stex_activate_module:n}
%   Activate the module with the provided URI; i.e. executes
%   all macro code of the module's |content|-field (does
%   nothing if the module is already activated in the current
%   context) and adds the module to \cs{l_stex_all_modules_seq}.
% \end{function}
%
% \end{documentation}
%
% \begin{implementation}\label{pkg:modules:impl}
%
% \section{\sTeX-Modules Implementation}
%
%    \begin{macrocode}
%<*package>

%%%%%%%%%%%%%   modules.dtx   %%%%%%%%%%%%%

%<@@=stex_modules>
%    \end{macrocode}
%
% Warnings and error messages
%
%    \begin{macrocode}
\msg_new:nnn{stex}{error/unknownmodule}{
  No~module~#1~found
}
\msg_new:nnn{stex}{error/syntax}{
  Syntax~error:~#1
}
\msg_new:nnn{stex}{error/siglanguage}{
  Module~#1~declares~signature~#2,~but~does~not~
  declare~its~language
}
%    \end{macrocode}
%
%
% \begin{variable}{\l_stex_current_module_prop}
%  The current module:
%    \begin{macrocode}
\prop_new:N \l_stex_current_module_prop
%    \end{macrocode}
% \end{variable}
%
% \begin{variable}{\l_stex_all_modules_seq}
%   Stores all available modules
%    \begin{macrocode}
\seq_new:N \l_stex_all_modules_seq
%    \end{macrocode}
% \end{variable}
%
% \begin{variable}{\g_stex_modules_in_file_seq,\g_stex_module_files_prop}
%  All modules sorted by containing file; used e.g. in \cs{importmodule}
%    \begin{macrocode}
\seq_new:N \g_stex_modules_in_file_seq
\prop_new:N \g_stex_module_files_prop
%    \end{macrocode}
% \end{variable}
%
% \begin{macro}[pTF]{\stex_if_in_module:}
%    \begin{macrocode}
\prg_new_conditional:Nnn \stex_if_in_module: {p, T, F, TF} {
  \prop_if_empty:NTF \l_stex_current_module_prop
    \prg_return_false: \prg_return_true:
}
%    \end{macrocode}
% \end{macro}
%
% \begin{macro}[pTF]{\stex_if_module_exists:n}
%    \begin{macrocode}
\prg_new_conditional:Nnn \stex_if_module_exists:n {p, T, F, TF} {
  \prop_if_exist:cTF { c_stex_module_#1_prop }
    \prg_return_true: \prg_return_false:
}
%    \end{macrocode}
% \end{macro}
%
% \begin{macro}{\stex_add_to_current_module:n,\STEXexport}
%
% Only allowed within modules:
%
%    \begin{macrocode}
\cs_new_protected:Nn \stex_add_to_current_module:n {
  \prop_get:NnN \l_stex_current_module_prop { content } \l_tmpa_tl
  \tl_put_right:Nn \l_tmpa_tl { #1 }
  \prop_put:Nno \l_stex_current_module_prop { content } { \l_tmpa_tl }
}
\cs_new_protected:Npn \STEXexport #1 {
  #1
  \stex_add_to_current_module:n { #1 }
  \stex_smsmode_set_codes:
}
\stex_deactivate_macro:Nn \STEXexport {module~environments}
%    \end{macrocode}
% \end{macro}
%
% \begin{macro}{\stex_add_constant_to_current_module:n}
%    \begin{macrocode}
\cs_new_protected:Nn \stex_add_constant_to_current_module:n {
  \str_set:Nx \l_tmpa_str { #1 }
  \prop_get:NnN \l_stex_current_module_prop { constants } \l_tmpa_seq
  \seq_put_right:No \l_tmpa_seq { \l_tmpa_str }
  \prop_put:Nno \l_stex_current_module_prop { constants } \l_tmpa_seq
}
%    \end{macrocode}
% \end{macro}
%
% \begin{macro}{\stex_add_import_to_current_module:n}
%    \begin{macrocode}
\cs_new_protected:Nn \stex_add_import_to_current_module:n {
  \str_set:Nx \l_tmpa_str { #1 }
  \prop_get:NnN \l_stex_current_module_prop { imports } \l_tmpa_seq
  \seq_put_right:No \l_tmpa_seq { \l_tmpa_str }
  \prop_put:Nno \l_stex_current_module_prop { imports } \l_tmpa_seq
}
%    \end{macrocode}
% \end{macro}
%
% \begin{macro}{\stex_modules_compute_namespace:nN}
% Computer the appropriate namespace from the top-level namespace
% of a repository (|#1|) and a file path
% (|#2|).
%
% Stores its return values in:
% \begin{variable}{\l_stex_modules_ns_str}
%    \begin{macrocode}
\str_new:N \l_stex_modules_ns_str
%    \end{macrocode}
% \end{variable}
%
%    \begin{macrocode}
\cs_new_protected:Nn \stex_modules_compute_namespace:nN {
  \str_set:Nx \l_tmpa_str { #1 }
  \seq_set_eq:NN \l_tmpa_seq #2
  % split off file extension
  \seq_pop_right:NN \l_tmpa_seq \l_tmpb_str
  \exp_args:NNno \seq_set_split:Nnn \l_tmpb_seq . \l_tmpb_str
  \seq_get_left:NN \l_tmpb_seq \l_tmpb_str
  \seq_put_right:No \l_tmpa_seq \l_tmpb_str

  \bool_set_true:N \l_tmpa_bool
  \bool_while_do:Nn \l_tmpa_bool {
    \seq_pop_left:NN \l_tmpa_seq \l_tmpb_str
    \exp_args:No \str_case:nnTF { \l_tmpb_str } {
      {source} { \bool_set_false:N \l_tmpa_bool }
    }{}{
      \seq_if_empty:NT \l_tmpa_seq {
        \bool_set_false:N \l_tmpa_bool
      }
    }
  }

  \seq_if_empty:NTF \l_tmpa_seq {
    \str_set_eq:NN \l_stex_modules_ns_str \l_tmpa_str
  }{
    \str_set:Nx \l_stex_modules_ns_str { 
      \l_tmpa_str/\stex_path_to_string:N \l_tmpa_seq
    }
  }
}
%    \end{macrocode}
% \end{macro}
%
% \begin{macro}{\stex_modules_current_namespace:}
%
% Computes the current namespace based on the current
% MathHub repository (if existent) and the current file.
%
%    \begin{macrocode}
\cs_new_protected:Nn \stex_modules_current_namespace: {
  \prop_get:NnNTF \l_stex_current_repository_prop { ns } \l_tmpa_str {
    \stex_modules_compute_namespace:nN \l_tmpa_str \g_stex_currentfile_seq
  }{
    % split off file extension
    \seq_set_eq:NN \l_tmpa_seq \g_stex_currentfile_seq
    \seq_pop_right:NN \l_tmpa_seq \l_tmpb_str
    \exp_args:NNno \seq_set_split:Nnn \l_tmpb_seq . \l_tmpb_str
    \seq_get_left:NN \l_tmpb_seq \l_tmpb_str
    \seq_put_right:No \l_tmpa_seq \l_tmpb_str
    \str_set:Nx \l_stex_modules_ns_str { 
      file:/\stex_path_to_string:N \l_tmpa_seq
    }
  }
}
%    \end{macrocode}
% \end{macro}
%
% \subsubsection{The module environment}
%
% |module| arguments:
%
%    \begin{macrocode}
\keys_define:nn { stex / module } {
  title         .str_set_x:N  = \l_stex_module_title_str ,
  ns            .str_set_x:N  = \l_stex_module_ns_str ,
  lang          .str_set_x:N  = \l_stex_module_lang_str ,
  sig           .str_set_x:N  = \l_stex_module_sig_str ,
  creators      .str_set_x:N  = \l_stex_module_creators_str ,
  contributors  .str_set_x:N  = \l_stex_module_contributors_str ,
  meta          .str_set_x:N  = \l_stex_module_meta_str
}

\cs_new_protected:Nn \_@@_args:n {
  \str_clear:N \l_stex_module_title_str
  \str_clear:N \l_stex_module_ns_str
  \str_clear:N \l_stex_module_lang_str
  \str_clear:N \l_stex_module_sig_str
  \str_clear:N \l_stex_module_creators_str
  \str_clear:N \l_stex_module_contributors_str
  \str_clear:N \l_stex_module_meta_str
  \keys_set:nn { stex / module } { #1 }
}

% module parameters here? In the body?

%    \end{macrocode}
%
% \begin{macro}{\stex_module_setup:nn}
% Sets up a new module property list:
%    \begin{macrocode}
\cs_new_protected:Nn \stex_module_setup:nn {
  \str_set:Nx \l_stex_module_name_str { #2 }
  \_@@_args:n { #1 }
%    \end{macrocode}
%
% First, we set up the name and namespace of the module.
%
% Are we in a nested module?
%
%    \begin{macrocode}
  \stex_if_in_module:TF {
    % Nested module
    \prop_get:NnN \l_stex_current_module_prop
      { ns } \l_stex_module_ns_str
    \str_set:Nx \l_stex_module_name_str {
      \prop_item:Nn \l_stex_current_module_prop
        { name } / \l_stex_module_name_str
    }
  }{
    % not nested:
    \str_if_empty:NT \l_stex_module_ns_str {
      \stex_modules_current_namespace:
      \str_set_eq:NN \l_stex_module_ns_str \l_stex_modules_ns_str
      \exp_args:NNNo \seq_set_split:Nnn \l_tmpa_seq
          / {\l_stex_module_ns_str}
      \seq_pop_right:NN \l_tmpa_seq \l_tmpa_str
      \str_if_eq:NNT \l_tmpa_str \l_stex_module_name_str {
        \str_set:Nx \l_stex_module_ns_str {
          \stex_path_to_string:N \l_tmpa_seq
        }
      }
    }
  }
%    \end{macrocode}
%
% Next, we determine the language of the module:
%
%    \begin{macrocode}
  \str_if_empty:NT \l_stex_module_lang_str {
    \seq_get_right:NN \g_stex_currentfile_seq \l_tmpa_str
    \seq_set_split:NnV \l_tmpa_seq . \l_tmpa_str
    \seq_pop_right:NN \l_tmpa_seq \l_tmpa_str % .tex
    \seq_pop_left:NN \l_tmpa_seq \l_tmpa_str % <filename>
    \seq_if_empty:NF \l_tmpa_seq { %remaining element should be language
      \stex_debug:nn{modules} {Language~\l_stex_module_lang_str~
        inferred~from~file~name}
      \seq_pop_left:NN \l_tmpa_seq \l_stex_module_lang_str
    }
  } 

  \str_if_empty:NF \l_stex_module_lang_str {
    \prop_get:NVNTF \c_stex_languages_prop \l_stex_module_lang_str 
      \l_tmpa_str {
        \ltx@ifpackageloaded{babel}{
          \exp_args:Nx \selectlanguage { \l_tmpa_str }
        }{}
      } {
        \msg_error:nnn{stex}{error/unknownlanguage}{\l_tmpa_str}
      }
  }
%    \end{macrocode}
%
% We check if we need to extend a signature module, and set
% \cs{l_stex_current_module_prop} accordingly:
%
%    \begin{macrocode}
  \str_if_empty:NTF \l_stex_module_sig_str {
    \str_clear:N \l_tmpa_str
    \seq_clear:N \l_tmpa_seq
    \tl_clear:N \l_tmpa_tl
    \exp_args:NNx \prop_set_from_keyval:Nn \l_stex_current_module_prop {
      name      = \l_stex_module_name_str ,
      ns        = \l_stex_module_ns_str ,
      imports   = \exp_not:o { \l_tmpa_seq } ,
      constants = \exp_not:o { \l_tmpa_seq } ,
      content   = \exp_not:o { \l_tmpa_tl }  ,
      file      = \exp_not:o { \g_stex_currentfile_seq } ,
      lang      = \l_stex_module_lang_str ,
      sig       = \l_stex_module_sig_str ,
      meta      = \l_stex_module_meta_str
    }
  }{
    \str_if_empty:NT \l_stex_module_lang_str {
      \msg_error:nnnn{stex}{error/siglanguage}{
        \l_stex_module_ns_str?\l_stex_module_name_str
      }{\l_stex_module_sig_str}
    }

    \seq_set_eq:NN \l_tmpa_seq \g_stex_currentfile_seq
    \seq_pop_right:NN \l_tmpa_seq \l_tmpa_str
    \seq_set_split:NnV \l_tmpb_seq . \l_tmpa_str
    \seq_pop_right:NN \l_tmpb_seq \l_tmpa_str % .tex
    \seq_pop_left:NN \l_tmpb_seq \l_tmpa_str % <filename>
    \str_set:Nx \l_tmpa_str {
      \stex_path_to_string:N \l_tmpa_seq /
      \l_tmpa_str . \l_stex_module_sig_str .tex
    }
    \IfFileExists \l_tmpa_str {
      \exp_args:No \stex_in_smsmode:nn { \l_tmpa_str } {
        \seq_clear:N \l_stex_all_modules_seq
        \prop_clear:N \l_stex_current_module_prop
        \stex_debug:nn{modules}{Loading~signature~\l_tmpa_str}
        \input { \l_tmpa_str }
      }
    }{
      \msg_error:nnn{stex}{error/unknownmodule}{for~signature~\l_tmpa_str}
    }
    \stex_activate_module:n {
      \l_stex_module_ns_str ? \l_stex_module_name_str
    }
    \prop_set_eq:Nc \l_stex_current_module_prop {
      c_stex_module_
      \l_stex_module_ns_str ?
      \l_stex_module_name_str
      _prop
    }
  }
%    \end{macrocode}
%
% We load the metatheory:
%
%    \begin{macrocode}
  \str_if_empty:NT \l_stex_module_meta_str {
    \str_set:Nx \l_stex_module_meta_str {
      \c_stex_metatheory_ns_str ? Metatheory
    }
  }
  \str_if_eq:VnF \l_stex_module_meta_str {NONE} {
    \exp_args:Nx \stex_add_to_current_module:n { 
      \stex_activate_module:n {\l_stex_module_meta_str}
    }
    \stex_activate_module:n {\l_stex_module_meta_str}
  }
}
%    \end{macrocode}
% \end{macro}
%
% \begin{environment}{module}
%
% The |module| environment.
%
% \begin{macro}{\_@@_begin_module:nn}
%
%   implements |\begin{module}|
%
%    \begin{macrocode}
\cs_new_protected:Nn \_@@_begin_module:nn {
  \stex_reactivate_macro:N \STEXexport
  \stex_reactivate_macro:N \importmodule
  \stex_reactivate_macro:N \symdecl
  \stex_reactivate_macro:N \notation
  \stex_reactivate_macro:N \symdef
  \stex_module_setup:nn{#1}{#2}

  \stex_debug:nn{modules}{
    New~module:\\
    Namespace:~\l_stex_module_ns_str\\
    Name:~\l_stex_module_name_str\\
    Language:~\l_stex_module_lang_str\\
    Signature:~\l_stex_module_sig_str\\
    Metatheory:~\l_stex_module_meta_str\\
    File:~\stex_path_to_string:N \g_stex_currentfile_seq
  }

  \seq_put_right:Nx \l_stex_all_modules_seq {
    \l_stex_module_ns_str ? \l_stex_module_name_str
  }

  \seq_gput_right:Nx  \g_stex_modules_in_file_seq
      { \l_stex_module_ns_str ? \l_stex_module_name_str }
  
  \stex_if_smsmode:TF {
    \stex_smsmode_set_codes:
  } {
    \begin{stex_annotate_env} {theory} {
      \l_stex_module_ns_str ? \l_stex_module_name_str
    }

    \stex_annotate_invisible:nnn{header}{} {
      \stex_annotate:nnn{language}{ \l_stex_module_lang_str }{}
      \stex_annotate:nnn{signature}{ \l_stex_module_sig_str }{}
      \str_if_eq:VnF \l_stex_module_meta_str {NONE} {
        \stex_annotate:nnn{metatheory}{ \l_stex_module_meta_str }{}
      }
    }
  }
  % TODO: Inherit metatheory for nested modules?
}
\iffalse \end{stex_annotate_env} \fi %^^A make syntax highlighting work again
%    \end{macrocode}
% \end{macro}
%
% \begin{macro}{\_@@_end_module:}
%
%   implements |\end{module}|
%
%    \begin{macrocode}
\cs_new_protected:Nn \_@@_end_module: {
  \str_set:Nx \l_tmpa_str {
    c_stex_module_
    \prop_item:Nn \l_stex_current_module_prop { ns } ?
    \prop_item:Nn \l_stex_current_module_prop { name }
    _prop
  }
  %^^A \prop_new:c { \l_tmpa_str }
  \prop_gset_eq:cN { \l_tmpa_str } \l_stex_current_module_prop
  \stex_debug:nn{modules}{Closing~module~\prop_item:Nn \l_stex_current_module_prop { name }}
}
%    \end{macrocode}
% \end{macro}
%
% \begin{environment}{@module}
%
%  The core environment, with no header
%    
%    \begin{macrocode}
\iffalse \begin{stex_annotate_env} \fi %^^A make syntax highlighting work again
\NewDocumentEnvironment { @module } { O{} m } {
  \par
  \_@@_begin_module:nn{#1}{#2}
} { 
  \_@@_end_module:
  \stex_if_smsmode:TF {
    \exp_args:Nx \stex_add_to_sms:n {
      \prop_gset_from_keyval:cn {
        c_stex_module_
        \prop_item:Nn \l_stex_current_module_prop { ns } ?
        \prop_item:Nn \l_stex_current_module_prop { name }
        _prop
      } {
        name      = \prop_item:cn { \l_tmpa_str } { name } ,
        ns        = \prop_item:cn { \l_tmpa_str } { ns } ,
        imports   = \prop_item:cn { \l_tmpa_str } { imports } ,
        constants = \prop_item:cn { \l_tmpa_str } { constants } ,
        content   = \prop_item:cn { \l_tmpa_str } { content } ,
        file      = \prop_item:cn { \l_tmpa_str } { file } ,
        lang      = \prop_item:cn { \l_tmpa_str } { lang } ,
        sig       = \prop_item:cn { \l_tmpa_str } { sig } ,
        meta      = \prop_item:cn { \l_tmpa_str } { meta }
      }
    }
  }{
    \end{stex_annotate_env}
  }
}
%    \end{macrocode}
% \end{environment}
%
% \begin{macro}{\stex_modules_heading:}
%
%   Code for document headers
%
%    \begin{macrocode}
\cs_if_exist:NTF \thesection {
  \newcounter{module}[section]
}{
  \newcounter{module}
}

\bool_if:NT \c_stex_showmods_bool {
  \latexml_if:F { \RequirePackage{mdframed} }
}

\cs_new_protected:Nn \stex_modules_heading: {
  \stepcounter{module}
  \par
  \bool_if:NT \c_stex_showmods_bool {
    \noindent{\textbf{Module} ~
      \cs_if_exist:NT \thesection {\thesection.}
      \themodule ~ [\l_stex_module_name_str]
    }
    % TODO references
    % \sref@label@id{Module \thesection.\themodule [\module@name]}%
    \str_if_empty:NTF \l_stex_module_title_str {
    }{
      \quad(\l_stex_module_title_str)\hfill
    }\par
  }
  \stex_ref_new_doc_target:n \l_stex_module_name_str
}
%    \end{macrocode}
% \end{macro}
%
% Finally:
%    \begin{macrocode}
\NewDocumentEnvironment { module } { O{} m } {
  \bool_if:NT \c_stex_showmods_bool {
    \begin{mdframed}
  }
  \begin{@module}[#1]{#2}
  \stex_modules_heading:
}{
  \end{@module}
  \bool_if:NT \c_stex_showmods_bool {
    \end{mdframed}
  }
}
%    \end{macrocode}
% \end{environment}
%
% \subsubsection{Invoking modules}
%
% \begin{macro}{\STEXModule,\stex_invoke_module:n}
%    \begin{macrocode}
\NewDocumentCommand \STEXModule { m } {
  \exp_args:NNx \str_set:Nn \l_tmpa_str { #1 }
  \int_set:Nn \l_tmpa_int { \str_count:N \l_tmpa_str }
  \tl_set:Nn \l_tmpa_tl {
    \msg_error:nnn{stex}{error/unknownmodule}{#1}
  }
  \seq_map_inline:Nn \l_stex_all_modules_seq {
    \str_set:Nn \l_tmpb_str { ##1 }
    \str_if_eq:eeT { \l_tmpa_str } {
      \str_range:Nnn \l_tmpb_str { -\l_tmpa_int } { -1 }
    } {
      \seq_map_break:n {
        \tl_set:Nn \l_tmpa_tl {
          \stex_invoke_module:n { ##1 }
        }
      }
    }
  }
  \l_tmpa_tl
}

\cs_new_protected:Nn \stex_invoke_module:n {
  \stex_debug:nn{modules}{Invoking~module~#1}
  \peek_charcode_remove:NTF ! {
    \_@@_invoke_uri:nN { #1 }
  } {
    \peek_charcode_remove:NTF ? {
      \_@@_invoke_symbol:nn { #1 }
    } {
      \msg_error:nnn{stex}{error/syntax}{
        ?~or~!~expected~after~
        \c_backslash_str STEXModule{#1}
      }
    }
  }
}

\cs_new_protected:Nn \_@@_invoke_uri:nN {
  \str_set:Nn #2 { #1 }
}

\cs_new_protected:Nn \_@@_invoke_symbol:nn {
  \stex_invoke_symbol:n{#1?#2}
}
%    \end{macrocode}
% \end{macro}
%
% \begin{macro}{\stex_activate_module:n}
%    \begin{macrocode}
\cs_new_protected:Nn \stex_activate_module:n {
  \stex_debug:nn{modules}{Activating~module~#1}
  \exp_args:NNx \seq_if_in:NnF \l_stex_all_modules_seq { #1 } {
    \seq_put_right:Nx \l_stex_all_modules_seq { #1 }
    \prop_item:cn { c_stex_module_#1_prop } { content }
  }
}
%    \end{macrocode}
% \end{macro}
%
%    \begin{macrocode}
%</package>
%    \end{macrocode}
%
% \end{implementation}
%
% \PrintIndex

  % \iffalse meta-comment
% An Infrastructure for Semantic Macros and Module Scoping
% Copyright (c) 2019 Michael Kohlhase, all rights reserved
%                this file is released under the
%                LaTeX Project Public License (LPPL)
% 
% The original of this file is in the public repository at 
% http://github.com/sLaTeX/sTeX/
%
% TODO update copyright  
%
%<*driver>
\providecommand\bibfolder{../../lib/bib}
\RequirePackage{paralist}
\documentclass[full,kernel]{l3doc}
\usepackage[dvipsnames]{xcolor}
\usepackage[utf8]{inputenc}
\usepackage[T1]{fontenc}
%\usepackage{document-structure}
\usepackage[showmods,debug=all,lang={de,en}, mathhub=./tests]{stex}
\usepackage{url,array,float,textcomp}
\usepackage[show]{ed}
\usepackage[hyperref=auto,style=alphabetic]{biblatex}
\addbibresource{\bibfolder/kwarcpubs.bib}
\addbibresource{\bibfolder/extpubs.bib}
\addbibresource{\bibfolder/kwarccrossrefs.bib}
\addbibresource{\bibfolder/extcrossrefs.bib}
\usepackage{amssymb}
\usepackage{amsfonts}
\usepackage{xspace}
\usepackage{hyperref}

\usepackage{morewrites}


\makeindex
\floatstyle{boxed}
\newfloat{exfig}{thp}{lop}
\floatname{exfig}{Example}

\usepackage{stex-tests}

\MakeShortVerb{\|}

\def\scsys#1{{{\sc #1}}\index{#1@{\sc #1}}\xspace}
\def\mmt{\textsc{Mmt}\xspace}
\def\xml{\scsys{Xml}}
\def\mathml{\scsys{MathML}}
\def\omdoc{\scsys{OMDoc}}
\def\openmath{\scsys{OpenMath}}
\def\latexml{\scsys{LaTeXML}}
\def\perl{\scsys{Perl}}
\def\cmathml{Content-{\sc MathML}\index{Content {\sc MathML}}\index{MathML@{\sc MathML}!content}}
\def\activemath{\scsys{ActiveMath}}
\def\twin#1#2{\index{#1!#2}\index{#2!#1}}
\def\twintoo#1#2{{#1 #2}\twin{#1}{#2}}
\def\atwin#1#2#3{\index{#1!#2!#3}\index{#3!#2 (#1)}}
\def\atwintoo#1#2#3{{#1 #2 #3}\atwin{#1}{#2}{#3}}
\def\cT{\mathcal{T}}\def\cD{\mathcal{D}}

\def\fileversion{3.0}
\def\filedate{\today}

\RequirePackage{pdfcomment}

\ExplSyntaxOn\makeatletter
\cs_set_protected:Npn \@comp #1 #2 {
  \pdftooltip {
    \textcolor{blue}{#1}
  } { #2 }
}

\cs_set_protected:Npn \@defemph #1 #2 {
  \pdftooltip { 
    \textbf{\textcolor{magenta}{#1}}
  } { #2 }
}
\makeatother\ExplSyntaxOff

\begin{document}
  \DocInput{\jobname.dtx}
\end{document}
%</driver>
% \fi
%
% \title{ \sTeX-Symbols
% 	\thanks{Version {\fileversion} (last revised {\filedate})} 
% }
%
% \author{Michael Kohlhase, Dennis Müller\\
% 	FAU Erlangen-Nürnberg\\
% 	\url{http://kwarc.info/}
% }
%
% \maketitle
%
%\ifinfulldoc\else
% This is the documentation for the \pkg{stex-symbols} package.
% For a more high-level introduction, 
%  see \href{\basedocurl/manual.pdf}{the \sTeX Manual} or the
% \href{\basedocurl/stex.pdf}{full \sTeX documentation}.
%
% \textcolor{red}{TODO: symbols documentation}
% \fi
%
% \begin{documentation}\label{pkg:symbols:doc}
%
% Code related to symbol declarations and notations
%
% \section{Macros and Environments}\label{pkg:symbols:doc:macros}
%
% \begin{function}{\symdecl}
%   \begin{syntax} \cs{symdecl}\Arg{macroname}|[|\meta{args}|]| \end{syntax}
%   Declares a new symbol with semantic macro \cs{macroname}. Optional
%   arguments are:
%   \begin{itemize}
%     \item |name|: An (\omdoc) name. By default equal to \meta{macroname}.
%     \item |type|: An (ideally semantic) term. Not used by \sTeX, but
%         passed on to \mmt for semantic services.
%     \item |local|: A boolean (by default false). If set, this declaration
%         will not be added to the module content, i.e. importing
%         the current module will not make this declaration available.
%     \item |args|: Specifies the ``signature'' of the semantic macro.
%       Can be either an integer $0 \leq n \leq 9$, or a (more precise)
%       sequence of the following characters:
%         \begin{itemize}
%           \item[|i|] a ``normal'' argument, e.g.
%             |\symdecl{plus}[args=ii]| allows for |\plus{2}{2}|.
%           \item[|a|] an \emph{associative} argument; i.e. a sequence of
%             arbitrarily many arguments provided as a comma-separated list,
%             e.g.
%             |\symdecl{plus}[args=a]| allows for |\plus{2,2,2}|.
%           \item[|b|] a \emph{variable} argument. Is treated by \sTeX
%             like an |i|-argument, but an application is turned into
%             an |OMBind| in \omdoc, binding the provided variable
%             in the subsequent arguments of the operator; e.g.
%             |\symdecl{forall}[args=bi]| allows for |\forall{x\in\Nat}{x\geq0}|.
%         \end{itemize}
%   \end{itemize}
% \end{function}
%
% \begin{function}{\stex_symdecl_do:n}
%   Implements the core functionality of \cs{symdecl}, and is
%   called by \cs{symdecl} and \cs{symdef}.
%
%   Ultimately stores the symbol \meta{URI} in the property
%   list |\l_stex_symdecl_|\meta{URI}|_prop| with fields:
%   \begin{itemize}
%     \item |name| (string),
%     \item |module| (string),
%     \item |notations| (sequence of strings; initially empty),
%     \item |local| (boolean),
%     \item |type| (token list),
%     \item |args| (string of |i|s, |a|s and |b|s),
%     \item |arity| (integer string),
%     \item |assocs| (integer string; number of associative arguments),
%   \end{itemize}
% \end{function}
%
% \begin{variable}{\l_stex_all_symbols_seq}
%   Stores full URIs for all
%   modules currently in scope.
% \end{variable}
%
% \begin{function}{\stex_get_symbol:n}
%   Computes the full URI of a symbol from a macro argument, e.g.
%   the macro name, the macro itself, the full URI...
% \end{function}
%
% \begin{function}{\notation}
%   \begin{syntax} \cs{notation}|[|\meta{args}|]|\Arg{symbol}\Arg{notations$^+$} \end{syntax}
%   Introduces a new notation for \meta{symbol}, see \cs{stex_notation_do:nn}
% \end{function}
%
% \begin{function}{\stex_notation_do:nn}
%   \begin{syntax} \cs{stex_notation_do:nn}\Arg{URI}\Arg{notations$^+$}\end{syntax}
%
%   Implements the core functionality of \cs{notation}, and is
%   called by \cs{notation} and \cs{symdef}.
%
%   Ultimately stores the notation in the property
%   list\\ |\g_stex_notation_|\meta{URI}|#|\meta{variant}|#|^^A
%   \meta{lang}|_prop| with fields:
%   \begin{itemize}
%     \item |symbol| (URI string),
%     \item |language| (string),
%     \item |variant| (string),
%     \item |opprec| (integer string),
%     \item |argprecs| (sequence of integer strings)
%   \end{itemize}
% \end{function}
%
% \begin{function}{\symdef}
%   \begin{syntax} \cs{symdef}|[|\meta{args}|]|\Arg{symbol}\Arg{notations$^+$} \end{syntax}
%   Combines \cs{symdecl} and \cs{notation} by introducing a new
%   symbol and assigning a new notation for it.
% \end{function}
%
% \end{documentation}
%
% \begin{implementation}\label{pkg:symbols:impl}
%
% \section{\sTeX-Symbols Implementation}
%
%    \begin{macrocode}
%<*package>

%%%%%%%%%%%%%   symbols.dtx   %%%%%%%%%%%%%

%    \end{macrocode}
%
% Warnings and error messages
%
%    \begin{macrocode}
\msg_new:nnn{stex}{error/wrongargs}{
  args~value~in~symbol~declaration~for~#1~
  needs~to~be~i,~a,~b~or~B,~but~#2~given
}
%    \end{macrocode}
%
% \subsection{Symbol Declarations}
%    \begin{macrocode}
%<@@=stex_symdecl>
%    \end{macrocode}
%
% \begin{variable}{\l_stex_all_symbols_seq}
%   Stores all available symbols
%    \begin{macrocode}
\seq_new:N \l_stex_all_symbols_seq
%    \end{macrocode}
% \end{variable}
%
% \begin{macro}{\STEXsymbol}
%    \begin{macrocode}
\NewDocumentCommand \STEXsymbol { m } {
  \stex_get_symbol:n { #1 }
  \exp_args:No
  \stex_invoke_symbol:n { \l_stex_get_symbol_uri_str }
}
%    \end{macrocode}
% \end{macro}
%
% |symdecl| arguments:
%
%    \begin{macrocode}
\keys_define:nn { stex / symdecl } {
  name        .str_set_x:N  = \l_stex_symdecl_name_str ,
  local       .bool_set:N   = \l_stex_symdecl_local_bool ,
  args        .str_set_x:N  = \l_stex_symdecl_args_str ,
  type        .tl_set:N     = \l_stex_symdecl_type_tl ,
  deprecate   .str_set_x:N  = \l_stex_symdecl_deprecate_str ,
  align       .str_set:N    = \l_stex_symdecl_align_str , % TODO(?)
  gfc         .str_set:N    = \l_stex_symdecl_gfc_str , % TODO(?)
  specializes .str_set:N    = \l_stex_symdecl_specializes_str , % TODO(?)
  def         .tl_set:N     = \l_stex_symdecl_definiens_tl ,
  assoc       .choices:nn   = 
      {bin,binl,binr,pre,conj,pwconj}
      {\str_set:Nx \l_stex_symdecl_assoctype_str {\l_keys_choice_tl}}
}

\bool_new:N \l_stex_symdecl_make_macro_bool

\cs_new_protected:Nn \_@@_args:n {
  \str_clear:N \l_stex_symdecl_name_str
  \str_clear:N \l_stex_symdecl_args_str
  \str_clear:N \l_stex_symdecl_deprecate_str
  \str_clear:N \l_stex_symdecl_assoctype_str
  \bool_set_false:N \l_stex_symdecl_local_bool
  \tl_clear:N \l_stex_symdecl_type_tl
  \tl_clear:N \l_stex_symdecl_definiens_tl
  
  \keys_set:nn { stex / symdecl } { #1 }
}
%    \end{macrocode}
%
% \begin{macro}{\symdecl}
%
% Parses the optional arguments and passes them on to
% \cs{stex_symdecl_do:} (so that \cs{symdef}
% can do the same)
%
%    \begin{macrocode}

\NewDocumentCommand \symdecl { s m O{}} {
  \_@@_args:n { #3 }
  \IfBooleanTF #1 {
    \bool_set_false:N \l_stex_symdecl_make_macro_bool
  } {
    \bool_set_true:N \l_stex_symdecl_make_macro_bool
  }
  \stex_symdecl_do:n { #2 }
  \stex_smsmode_do:
}

\cs_new_protected:Nn \stex_symdecl_do:nn {
  \_@@_args:n{#1}
  \bool_set_false:N \l_stex_symdecl_make_macro_bool
  \stex_symdecl_do:n{#2}
}

\stex_deactivate_macro:Nn \symdecl {module~environments}
%    \end{macrocode}
% \end{macro}
%
%
% \begin{macro}{\stex_symdecl_do:n}
%    \begin{macrocode}
\cs_new_protected:Nn \stex_symdecl_do:n {
  \stex_if_in_module:F {
    % TODO throw error? some default namespace?
  }
  
  \str_if_empty:NT \l_stex_symdecl_name_str {
    \str_set:Nx \l_stex_symdecl_name_str { #1 }
  }

  \prop_if_exist:cT { l_stex_symdecl_ 
      \l_stex_current_module_str ?
      \l_stex_symdecl_name_str
    _prop
  }{
    % TODO throw error (beware of circular dependencies)
  }

  \prop_clear:N \l_tmpa_prop
  \prop_put:Nnx \l_tmpa_prop { module } { \l_stex_current_module_str }
  \seq_clear:N \l_tmpa_seq
  \prop_put:Nno \l_tmpa_prop { name } \l_stex_symdecl_name_str
  \prop_put:Nno \l_tmpa_prop { type } \l_stex_symdecl_type_tl

  \str_if_empty:NT \l_stex_symdecl_deprecate_str {
    \str_if_empty:NF \l_stex_module_deprecate_str {
      \str_set_eq:NN \l_stex_symdecl_deprecate_str \l_stex_module_deprecate_str
    }
  }
  \prop_put:Nno \l_tmpa_prop { deprecate } \l_stex_symdecl_deprecate_str

  \exp_args:No \stex_add_constant_to_current_module:n {
    \l_stex_symdecl_name_str
  }

  % arity/args
  \int_zero:N \l_tmpb_int

  \bool_set_true:N \l_tmpa_bool
  \str_map_inline:Nn \l_stex_symdecl_args_str {
    \token_case_meaning:NnF ##1 {
      0 {} 1 {} 2 {} 3 {} 4 {} 5 {} 6 {} 7 {} 8 {} 9 {}
      {\tl_to_str:n i} { \bool_set_false:N \l_tmpa_bool }
      {\tl_to_str:n b} { \bool_set_false:N \l_tmpa_bool }
      {\tl_to_str:n a} { 
        \bool_set_false:N \l_tmpa_bool
        \int_incr:N \l_tmpb_int
      }
      {\tl_to_str:n B} { 
        \bool_set_false:N \l_tmpa_bool
        \int_incr:N \l_tmpb_int
      }
    }{
      \msg_error:nnxx{stex}{error/wrongargs}{
        \l_stex_current_module_str ?
        \l_stex_symdecl_name_str
      }{##1}
    }
  }
  \bool_if:NTF \l_tmpa_bool {
    % possibly numeric
    \str_if_empty:NTF \l_stex_symdecl_args_str {
      \prop_put:Nnn \l_tmpa_prop { args } {}
      \prop_put:Nnn \l_tmpa_prop { arity } { 0 }
    }{
      \int_set:Nn \l_tmpa_int { \l_stex_symdecl_args_str }
      \prop_put:Nnx \l_tmpa_prop { arity } { \int_use:N \l_tmpa_int }
      \str_clear:N \l_tmpa_str
      \int_step_inline:nn \l_tmpa_int {
        \str_put_right:Nn \l_tmpa_str i
      }
      \prop_put:Nnx \l_tmpa_prop { args } { \l_tmpa_str }
    }
  } {
    \prop_put:Nnx \l_tmpa_prop { args } { \l_stex_symdecl_args_str }
    \prop_put:Nnx \l_tmpa_prop { arity }
      { \str_count:N \l_stex_symdecl_args_str }
  }
  \prop_put:Nnx \l_tmpa_prop { assocs } { \int_use:N \l_tmpb_int }
  

  % semantic macro

  \bool_if:NT \l_stex_symdecl_make_macro_bool {
    \exp_args:Nx \stex_do_up_to_module:n {
      \tl_set:cn { #1 } { \stex_invoke_symbol:n {
        \l_stex_current_module_str ? \l_stex_symdecl_name_str
      }}
    }

    \bool_if:NF \l_stex_symdecl_local_bool {
      \exp_args:Nx \stex_add_to_current_module:n {
        \tl_set:cn { #1 } { \stex_invoke_symbol:n {
          \l_stex_current_module_str ? \l_stex_symdecl_name_str
        } }
      }
    }
  }

  % add to all symbols

  \bool_if:NF \l_stex_symdecl_local_bool {
    \exp_args:Nx \stex_add_to_current_module:n {
      \seq_put_right:Nn \exp_not:N \l_stex_all_symbols_seq {
        \l_stex_current_module_str ? \l_stex_symdecl_name_str
      }
    }
%    \exp_args:Nx \stex_add_field_to_current_module:n {
%      \l_stex_current_module_str ? \l_stex_symdecl_name_str
%    }
  }

  \stex_debug:nn{symbols}{New~symbol:~
    \l_stex_current_module_str ? \l_stex_symdecl_name_str^^J
    Type:~\exp_not:o { \l_stex_symdecl_type_tl }^^J
    Args:~\prop_item:Nn \l_tmpa_prop { args }
  }

  % circular dependencies require this:

  \prop_if_exist:cF {
    l_stex_symdecl_ 
    \l_stex_current_module_str ? \l_stex_symdecl_name_str
    _prop 
  } {
    \prop_set_eq:cN {
      l_stex_symdecl_ 
      \l_stex_current_module_str ? \l_stex_symdecl_name_str
      _prop 
    } \l_tmpa_prop
  }

  \seq_clear:c {
    l_stex_symdecl_ 
    \l_stex_current_module_str ? \l_stex_symdecl_name_str
    _notations
  }

  \bool_if:NF \l_stex_symdecl_local_bool {
    \exp_args:Nx
    \stex_add_to_current_module:n {
      \seq_clear:c {
        l_stex_symdecl_ 
        \l_stex_current_module_str ? \l_stex_symdecl_name_str
        _notations
      }
      \prop_set_from_keyval:cn {
        l_stex_symdecl_ 
        \l_stex_current_module_str ? \l_stex_symdecl_name_str
        _prop 
      } {
        name      = \prop_item:Nn \l_tmpa_prop { name }       ,
        module    = \prop_item:Nn \l_tmpa_prop { module }     ,
        type      = \prop_item:Nn \l_tmpa_prop { type }       ,
        args      = \prop_item:Nn \l_tmpa_prop { args }       ,
        arity     = \prop_item:Nn \l_tmpa_prop { arity }      ,
        assocs    = \prop_item:Nn \l_tmpa_prop { assocs }
      }
    }
  }

  \stex_if_smsmode:F {
    \exp_args:Nx \stex_do_up_to_module:n {
        \seq_put_right:Nn \exp_not:N \l_stex_all_symbols_seq {
        \l_stex_current_module_str ? \l_stex_symdecl_name_str
      }
    }
    \stex_if_do_html:T {
      \stex_annotate_invisible:nnn {symdecl} {
        \l_stex_current_module_str ? \l_stex_symdecl_name_str
      } {
        \tl_if_empty:NF \l_stex_symdecl_type_tl {\stex_annotate_invisible:nnn{type}{}{$\l_stex_symdecl_type_tl$}}
        \stex_annotate_invisible:nnn{args}{}{
          \prop_item:Nn \l_tmpa_prop { args }
        }
        \stex_annotate_invisible:nnn{macroname}{#1}{}
        \tl_if_empty:NF \l_stex_symdecl_definiens_tl {
          \stex_annotate_invisible:nnn{definiens}{}
            {$\l_stex_symdecl_definiens_tl$}
        }
        \str_if_empty:NF \l_stex_symdecl_assoctype_str {
          \stex_annotate_invisible:nnn{assoctype}{\l_stex_symdecl_assoctype_str}{}
        }
      }
    }
  }
}
%    \end{macrocode}
% \end{macro}
%
% \begin{macro}{\stex_get_symbol:n}
%
%    \begin{macrocode}
\str_new:N \l_stex_get_symbol_uri_str

\cs_new_protected:Nn \stex_get_symbol:n {
  \tl_if_head_eq_catcode:nNTF { #1 } \relax {
    \_@@_get_symbol_from_cs:n { #1 }
  }{
    % argument is a string
    % is it a command name?
    \cs_if_exist:cTF { #1 }{
      \cs_set_eq:Nc \l_tmpa_tl { #1 }
      \str_set:Nx \l_tmpa_str { \cs_argument_spec:N \l_tmpa_tl }
      \str_if_empty:NTF \l_tmpa_str {
        \exp_args:Nx \cs_if_eq:NNTF {
          \tl_head:N \l_tmpa_tl
        } \stex_invoke_symbol:n {
          \exp_args:No \_@@_get_symbol_from_cs:n { \use:c { #1 } }
        }{
          \_@@_get_symbol_from_string:n { #1 }
        }
      } {
        \_@@_get_symbol_from_string:n { #1 }
      }
    }{
      % argument is not a command name
      \_@@_get_symbol_from_string:n { #1 }
      % \l_stex_all_symbols_seq
    }
  }
  \str_if_eq:eeF {
    \prop_item:cn {
      l_stex_symdecl_\l_stex_get_symbol_uri_str _prop
    }{ deprecate }
  }{}{
    \msg_warning:nnxx{stex}{warning/deprecated}{
      Symbol~\l_stex_get_symbol_uri_str
    }{
      \prop_item:cn {l_stex_symdecl_\l_stex_get_symbol_uri_str _prop}{ deprecate }
    }
  }
}

\cs_new_protected:Nn \_@@_get_symbol_from_string:n {
  \str_set:Nn \l_tmpa_str { #1 }
  \bool_set_false:N \l_tmpa_bool
  \stex_if_in_module:T {
    \exp_args:Nno \seq_if_in:cnT {c_stex_module_\l_stex_current_module_str _constants} { \l_tmpa_str } {
      \bool_set_true:N \l_tmpa_bool
      \str_set:Nx \l_stex_get_symbol_uri_str {
        \l_stex_current_module_str ? #1
      }
    }
  }
  \bool_if:NF \l_tmpa_bool {
    \tl_set:Nn \l_tmpa_tl {
      \msg_set:nnn{stex}{error/unknownsymbol}{
        No~symbol~#1~found!
      }
      \msg_error:nn{stex}{error/unknownsymbol}
    }
    \str_set:Nn \l_tmpa_str { #1 }
    \int_set:Nn \l_tmpa_int { \str_count:N \l_tmpa_str }
    \seq_map_inline:Nn \l_stex_all_symbols_seq {
      \str_set:Nn \l_tmpb_str { ##1 }
      \str_if_eq:eeT { \l_tmpa_str } {
        \str_range:Nnn \l_tmpb_str { -\l_tmpa_int } { -1 }
      } {
        \seq_map_break:n {
          \tl_set:Nn \l_tmpa_tl {
            \str_set:Nn \l_stex_get_symbol_uri_str {
              ##1
            }
          }
        }
      }
    }
    \l_tmpa_tl
  }
}

\cs_new_protected:Nn \_@@_get_symbol_from_cs:n {
  \exp_args:NNx \tl_set:Nn \l_tmpa_tl 
    { \tl_tail:N \l_tmpa_tl }
  \tl_if_single:NTF \l_tmpa_tl {
    \exp_args:No \tl_if_head_is_group:nTF \l_tmpa_tl {
      \exp_after:wN \str_set:Nn \exp_after:wN
        \l_stex_get_symbol_uri_str \l_tmpa_tl
    }{
      % TODO
      % tail is not a single group
    }
  }{
    % TODO
    % tail is not a single group
  }
}
%    \end{macrocode}
% \end{macro}
%
% \subsection{Notations}
%    \begin{macrocode}
%<@@=stex_notation>
%    \end{macrocode}
%
% |notation| arguments:
%
%    \begin{macrocode}
\keys_define:nn { stex / notation } {
  lang    .tl_set_x:N  = \l_@@_lang_str ,
  variant .tl_set_x:N  = \l_@@_variant_str ,
  prec    .str_set_x:N = \l_@@_prec_str ,
  op      .tl_set:N    = \l_@@_op_tl ,
  primary .bool_set:N  = \l_@@_primary_bool ,
  primary .default:n   = {true} ,
  unknown .code:n      = \str_set:Nx 
      \l_@@_variant_str \l_keys_key_str
}

\cs_new_protected:Nn \_stex_notation_args:n {
  \str_clear:N \l_@@_lang_str
  \str_clear:N \l_@@_variant_str
  \str_clear:N \l_@@_prec_str
  \tl_clear:N \l_@@_op_tl
  \bool_set_false:N \l_@@_primary_bool
  
  \keys_set:nn { stex / notation } { #1 }
}
%    \end{macrocode}
%
%
%
% \begin{macro}{\notation}
%    \begin{macrocode}
\NewDocumentCommand \notation { s m O{}} {
  \_stex_notation_args:n { #3 }
  \tl_clear:N \l_stex_symdecl_definiens_tl
  \stex_get_symbol:n { #2 }
  \tl_set:Nn \l_stex_notation_after_do_tl {
    \_@@_final:
    \IfBooleanTF#1{
      \stex_setnotation:n {\l_stex_get_symbol_uri_str}
    }{}
    \stex_smsmode_do:
  }
  \stex_notation_do:nnnn
    { \prop_item:cn {l_stex_symdecl_\l_stex_get_symbol_uri_str _prop } { args } }
    { \prop_item:cn { l_stex_symdecl_\l_stex_get_symbol_uri_str _prop } { arity } }
    { \l_@@_variant_str \c_hash_str \l_@@_lang_str }
}
\stex_deactivate_macro:Nn \notation {module~environments}
%    \end{macrocode}
% \end{macro}
%
% \begin{macro}{\stex_notation_do:nnnn}
%    \begin{macrocode}
\seq_new:N \l_@@_precedences_seq
\tl_new:N \l_@@_opprec_tl
\int_new:N \l_@@_currarg_int
\tl_new:N \stex_symbol_after_invokation_tl

\cs_new_protected:Nn \stex_notation_do:nnnn {
  \let\l_stex_current_symbol_str\relax
  \seq_clear:N \l_@@_precedences_seq
  \tl_clear:N \l_@@_opprec_tl
  \str_set:Nx \l_@@_args_str { #1 }
  \str_set:Nx \l_@@_arity_str { #2 }
  \str_set:Nx \_@@_suffix_str { #3 }

  % precedences
  \str_if_empty:NTF \l_@@_prec_str {
    \int_compare:nNnTF \l_@@_arity_str = 0 {
      \tl_set:No \l_@@_opprec_tl { \neginfprec }
    }{
      \tl_set:Nn \l_@@_opprec_tl { 0 }
    }
  } {
    \str_if_eq:onTF \l_@@_prec_str {nobrackets}{
      \tl_set:No \l_@@_opprec_tl { \neginfprec }
      \int_step_inline:nn { \l_@@_arity_str } {
        \exp_args:NNo
        \seq_put_right:Nn \l_@@_precedences_seq { \infprec }
      }
    }{
      \seq_set_split:NnV \l_tmpa_seq ; \l_@@_prec_str
      \seq_pop_left:NNTF \l_tmpa_seq \l_tmpa_str {
        \tl_set:No \l_@@_opprec_tl { \l_tmpa_str }
        \seq_pop_left:NNT \l_tmpa_seq \l_tmpa_str {
          \exp_args:NNNo \exp_args:NNno \seq_set_split:Nnn 
            \l_tmpa_seq {\tl_to_str:n{x} } { \l_tmpa_str }
          \seq_map_inline:Nn \l_tmpa_seq {
            \seq_put_right:Nn \l_tmpb_seq { ##1 }
          }
        }
      }{
        \int_compare:nNnTF \l_@@_arity_str = 0 {
          \tl_set:No \l_@@_opprec_tl { \infprec }
        }{
          \tl_set:No \l_@@_opprec_tl { 0 }
        }
      }
    }
  }

  \seq_set_eq:NN \l_tmpa_seq \l_@@_precedences_seq
  \int_step_inline:nn { \l_@@_arity_str } {
    \seq_pop_left:NNF \l_tmpa_seq \l_tmpb_str {
      \exp_args:NNo
      \seq_put_right:No \l_@@_precedences_seq { 
        \l_@@_opprec_tl
      }
    }
  }
  \tl_clear:N \l_stex_notation_dummyargs_tl

  \int_compare:nNnTF \l_@@_arity_str = 0 {
    \exp_args:NNe
    \cs_set:Npn \l_stex_notation_macrocode_cs {
      \_stex_term_math_oms:nnnn { \l_stex_current_symbol_str } 
        { \_@@_suffix_str }
        { \l_@@_opprec_tl } 
        { \exp_not:n { #4 } }
    }
    \l_stex_notation_after_do_tl
  }{
    \str_if_in:NnTF \l_@@_args_str b {
      \exp_args:Nne \use:nn
      {
      \cs_generate_from_arg_count:NNnn \l_stex_notation_macrocode_cs
      \cs_set:Npn \l_@@_arity_str } { {
        \_stex_term_math_omb:nnnn { \l_stex_current_symbol_str } 
          { \_@@_suffix_str }
          { \l_@@_opprec_tl } 
          { \exp_not:n { #4 } }
      }}
    }{
      \str_if_in:NnTF \l_@@_args_str B {
        \exp_args:Nne \use:nn
        {
        \cs_generate_from_arg_count:NNnn \l_stex_notation_macrocode_cs
        \cs_set:Npn \l_@@_arity_str } { {
          \_stex_term_math_omb:nnnn { \l_stex_current_symbol_str } 
            { \_@@_suffix_str }
            { \l_@@_opprec_tl } 
            { \exp_not:n { #4 } }
        } }
      }{
        \exp_args:Nne \use:nn
        {
        \cs_generate_from_arg_count:NNnn \l_stex_notation_macrocode_cs
        \cs_set:Npn \l_@@_arity_str } { {
          \_stex_term_math_oma:nnnn { \l_stex_current_symbol_str } 
            { \_@@_suffix_str }
            { \l_@@_opprec_tl } 
            { \exp_not:n { #4 } }
        } }
      }
    }

    \str_set_eq:NN \l_@@_remaining_args_str \l_@@_args_str
    \int_zero:N \l_@@_currarg_int
    \seq_set_eq:NN \l_@@_remaining_precs_seq \l_@@_precedences_seq
    \_@@_arguments:
  }
}
%    \end{macrocode}
% \end{macro}
%
% \begin{macro}{\_@@_arguments:}
%
% Takes care of annotating the arguments in a
% notation macro
%
%    \begin{macrocode}
\cs_new_protected:Nn \_@@_arguments: {
  \int_incr:N \l_@@_currarg_int
  \str_if_empty:NTF \l_@@_remaining_args_str {
    \l_stex_notation_after_do_tl
  }{
    \str_set:Nx \l_tmpa_str { \str_head:N \l_@@_remaining_args_str }
    \str_set:Nx \l_@@_remaining_args_str { \str_tail:N \l_@@_remaining_args_str }
    \str_if_eq:VnTF \l_tmpa_str a {
      \_@@_argument_assoc:n
    }{
      \str_if_eq:VnTF \l_tmpa_str B {
        \_@@_argument_assoc:n
      }{
        \seq_pop_left:NN \l_@@_remaining_precs_seq \l_tmpa_str
        \tl_put_right:Nx \l_stex_notation_dummyargs_tl {
          { \_stex_term_math_arg:nnn
            { \int_use:N \l_@@_currarg_int }
            { \l_tmpa_str }
            { ####\int_use:N \l_@@_currarg_int }
          }
        }
        \_@@_arguments:
      }
    }
  }
}
%    \end{macrocode}
% \end{macro}
%
% \begin{macro}{\_@@_argument_assoc:n}
%    \begin{macrocode}
\cs_new_protected:Nn \_@@_argument_assoc:n {

  \cs_generate_from_arg_count:NNnn \l_tmpa_cs \cs_set:Npn 
    {\l_@@_arity_str}{
    #1
  }
  \int_zero:N \l_tmpa_int
  \tl_clear:N \l_tmpa_tl
  \str_map_inline:Nn \l_@@_args_str {
    \int_incr:N \l_tmpa_int
    \tl_put_right:Nx \l_tmpa_tl {
      \str_if_eq:nnTF {##1}{a}{ {} }{
        \str_if_eq:nnTF {##1}{B}{ {} }{
          {\_stex_term_arg:nn{\int_use:N \l_tmpa_int}{################ \int_use:N \l_tmpa_int}}
        }
      }
    }
  }
  \exp_after:wN\exp_after:wN\exp_after:wN \def 
  \exp_after:wN\exp_after:wN\exp_after:wN \l_tmpa_cs 
  \exp_after:wN\exp_after:wN\exp_after:wN ## 
  \exp_after:wN\exp_after:wN\exp_after:wN 1 
  \exp_after:wN\exp_after:wN\exp_after:wN ## 
  \exp_after:wN\exp_after:wN\exp_after:wN 2 
  \exp_after:wN\exp_after:wN\exp_after:wN {
    \exp_after:wN \exp_after:wN \exp_after:wN 
    \exp_not:n \exp_after:wN \exp_after:wN \exp_after:wN {
      \exp_after:wN \l_tmpa_cs \l_tmpa_tl
    }
  }

  \seq_pop_left:NN \l_@@_remaining_precs_seq \l_tmpa_str
  \tl_put_right:Nx \l_stex_notation_dummyargs_tl { {
    \_stex_term_math_assoc_arg:nnnn
      { \int_use:N \l_@@_currarg_int }
      { \l_tmpa_str }
      { ####\int_use:N \l_@@_currarg_int }
      { \l_tmpa_cs {####1} {####2} }
  } }
  \_@@_arguments:
}
%    \end{macrocode}
% \end{macro}
%
% \begin{macro}{\_@@_final:}
%
% Called after processing all notation arguments
%
%    \begin{macrocode}
\cs_new_protected:Nn \_@@_final: {
  \exp_args:Nne \use:nn
  {
  \cs_generate_from_arg_count:cNnn {
      stex_notation_ \l_stex_get_symbol_uri_str \c_hash_str 
      \_@@_suffix_str
      _cs
    }
    \cs_set:Npn \l_@@_arity_str } { {
      \exp_after:wN \exp_after:wN \exp_after:wN
      \exp_not:n \exp_after:wN \exp_after:wN \exp_after:wN 
      { \exp_after:wN \l_stex_notation_macrocode_cs \l_stex_notation_dummyargs_tl \stex_symbol_after_invokation_tl}
  } }

  \tl_if_empty:NF \l_@@_op_tl {
    \cs_set:cpx {
      stex_op_notation_ \l_stex_get_symbol_uri_str \c_hash_str
      \_@@_suffix_str
      _cs
    } {
      \_stex_term_oms:nnn {
        \l_stex_get_symbol_uri_str \c_hash_str \_@@_suffix_str
      }{
        \l_stex_get_symbol_uri_str
      }{ \comp{ \exp_args:No \exp_not:n { \l_@@_op_tl } } }
    }
  }

  \exp_args:Ne
  \stex_add_to_current_module:n {
    \cs_generate_from_arg_count:cNnn {
      stex_notation_ \l_stex_get_symbol_uri_str \c_hash_str 
      \_@@_suffix_str
      _cs
    } \cs_set:Npn {\l_@@_arity_str} {
        \exp_after:wN \exp_after:wN \exp_after:wN
        \exp_not:n \exp_after:wN \exp_after:wN \exp_after:wN 
        { \exp_after:wN \l_stex_notation_macrocode_cs \l_stex_notation_dummyargs_tl \stex_symbol_after_invokation_tl}
    }
    \tl_if_empty:NF \l_@@_op_tl {
      \cs_set:cpn {
        stex_op_notation_\l_stex_get_symbol_uri_str \c_hash_str
        \_@@_suffix_str
        _cs
      } {
        \_stex_term_oms:nnn {
          \l_stex_get_symbol_uri_str\c_hash_str \_@@_suffix_str
        }{
          \l_stex_get_symbol_uri_str
        }{ \comp{ \exp_args:No \exp_not:n { \l_@@_op_tl } } }
      }
    }
  }
  %\exp_args:Nx
 % \stex_do_up_to_module:n {
    \seq_put_right:cx {
      l_stex_symdecl_ \l_stex_get_symbol_uri_str
      _notations
    } {
      \_@@_suffix_str
    }
 % }

  \stex_debug:nn{symbols}{
    Notation~\_@@_suffix_str
    ~for~\l_stex_get_symbol_uri_str^^J
    Operator~precedence:~\l_@@_opprec_tl^^J
    Argument~precedences:~
      \seq_use:Nn \l_@@_precedences_seq {,~}^^J
    Notation: \cs_meaning:c {
      stex_notation_ \l_stex_get_symbol_uri_str \c_hash_str 
      \_@@_suffix_str
      _cs
    }
  }
  
  \exp_args:Ne
  \stex_add_to_current_module:n {
    \seq_put_right:cn {
      l_stex_symdecl_\l_stex_get_symbol_uri_str
      _notations
    } { \_@@_suffix_str }
  }

  \stex_if_smsmode:F {

    % HTML annotations
    \stex_if_do_html:T {
      \stex_annotate_invisible:nnn { notation }
      { \l_stex_get_symbol_uri_str } {
        \stex_annotate_invisible:nnn { notationfragment }
          { \_@@_suffix_str }{}
        \stex_annotate_invisible:nnn { precedence }
          { \l_@@_prec_str }{}

        \int_zero:N \l_tmpa_int
        \str_set_eq:NN \l_@@_remaining_args_str \l_@@_args_str
        \tl_clear:N \l_tmpa_tl
        \int_step_inline:nn { \l_@@_arity_str }{
          \int_incr:N \l_tmpa_int
          \str_set:Nx \l_tmpb_str { \str_head:N \l_@@_remaining_args_str }
          \str_set:Nx \l_@@_remaining_args_str { \str_tail:N \l_@@_remaining_args_str }
          \str_if_eq:VnTF \l_tmpb_str a {
            \tl_set:Nx \l_tmpa_tl { \l_tmpa_tl { 
              \c_hash_str \c_hash_str \int_use:N \l_tmpa_int a ,
              \c_hash_str \c_hash_str \int_use:N \l_tmpa_int b
            } }
          }{
            \str_if_eq:VnTF \l_tmpb_str B {
              \tl_set:Nx \l_tmpa_tl { \l_tmpa_tl { 
                \c_hash_str \c_hash_str \int_use:N \l_tmpa_int a ,
                \c_hash_str \c_hash_str \int_use:N \l_tmpa_int b
              } }
            }{
              \tl_set:Nx \l_tmpa_tl { \l_tmpa_tl { 
                \c_hash_str \c_hash_str \int_use:N \l_tmpa_int
              } }
            }
          }
        }
        \stex_annotate_invisible:nnn { notationcomp }{}{
          \str_set:Nx \l_stex_current_symbol_str {\l_stex_get_symbol_uri_str }
          $ \exp_args:Nno \use:nn { \use:c {
            stex_notation_ \l_stex_current_symbol_str
            \c_hash_str \_@@_suffix_str _cs
          } } { \l_tmpa_tl } $
        }
      }
    }
  }
}
%    \end{macrocode}
% \end{macro}
%
% \begin{macro}{\setnotation}
%    \begin{macrocode}
\keys_define:nn { stex / setnotation } {
  lang    .tl_set_x:N  = \l_@@_lang_str ,
  variant .tl_set_x:N  = \l_@@_variant_str ,
  unknown .code:n      = \str_set:Nx 
      \l_@@_variant_str \l_keys_key_str
}

\cs_new_protected:Nn \_stex_setnotation_args:n {
  \str_clear:N \l_@@_lang_str
  \str_clear:N \l_@@_variant_str
  \keys_set:nn { stex / setnotation } { #1 }
}

\cs_new_protected:Nn \stex_setnotation:n {
  \exp_args:Nnx \seq_if_in:cnTF { l_stex_symdecl_#1 _notations }
    { \l_@@_variant_str \c_hash_str \l_@@_lang_str }{
      \exp_args:Nnx \seq_remove_all:cn { l_stex_symdecl_#1 _notations }
        { \l_@@_variant_str \c_hash_str \l_@@_lang_str }
      \exp_args:Nnx \seq_remove_all:cn { l_stex_symdecl_#1 _notations }
        { \c_hash_str }
      \exp_args:Nnx \seq_put_left:cn { l_stex_symdecl_#1 _notations }
        { \l_@@_variant_str \c_hash_str \l_@@_lang_str }
      \exp_args:Nx \stex_add_to_current_module:n {
        \exp_args:Nnx \seq_remove_all:cn { l_stex_symdecl_#1 _notations }
          { \l_@@_variant_str \c_hash_str \l_@@_lang_str }
        \exp_args:Nnx \seq_put_left:cn { l_stex_symdecl_#1 _notations }
          { \l_@@_variant_str \c_hash_str \l_@@_lang_str }
        \exp_args:Nnx \seq_remove_all:cn { l_stex_symdecl_#1 _notations }
          { \c_hash_str }
      }
      \stex_debug:nn {notations}{
        Setting~default~notation~
        {\l_@@_variant_str \c_hash_str \l_@@_lang_str}~for~
        #1 \\
        \expandafter\meaning\csname
        l_stex_symdecl_#1 _notations\endcsname
      }
    }{
      % todo throw error
    }
}

\NewDocumentCommand \setnotation {m m} {
  \stex_get_symbol:n { #1 }
  \_stex_setnotation_args:n { #2 }
  \stex_setnotation:n{\l_stex_get_symbol_uri_str}
  \stex_smsmode_do:
}

\cs_new_protected:Nn \stex_copy_notations:nn {
  \stex_debug:nn {notations}{
    Copying~notations~from~#2~to~#1\\
    \seq_use:cn{l_stex_symdecl_#2_notations}{,~}
  }
  \tl_clear:N \l_tmpa_tl
  \int_step_inline:nn { \prop_item:cn {l_stex_symdecl_#2_prop}{ arity } } {
    \tl_put_right:Nn \l_tmpa_tl { {## ##1} }
  }
  \seq_map_inline:cn {l_stex_symdecl_#2_notations}{
    \cs_set_eq:Nc \l_tmpa_cs { stex_notation_ #2 \c_hash_str ##1 _cs }
    \edef \l_tmpa_tl {
      \exp_after:wN\exp_after:wN\exp_after:wN \exp_not:n 
      \exp_after:wN\exp_after:wN\exp_after:wN {
        \exp_after:wN \l_tmpa_cs \l_tmpa_tl
      }
    }
    \exp_args:Nx
    \stex_do_up_to_module:n {
      \seq_put_right:cn{l_stex_symdecl_#1_notations}{##1}
      \cs_generate_from_arg_count:cNnn {
        stex_notation_ #1 \c_hash_str ##1 _cs
      } \cs_set:Npn { \prop_item:cn {l_stex_symdecl_#2_prop}{ arity } }{
        \exp_after:wN\exp_not:n\exp_after:wN{\l_tmpa_tl}
      }
    }
  }
}

\NewDocumentCommand \copynotation {m m} {
  \stex_get_symbol:n { #1 }
  \str_set_eq:NN \l_tmpa_str \l_stex_get_symbol_uri_str
  \stex_get_symbol:n { #2 }
  \exp_args:Noo
  \stex_copy_notations:nn \l_tmpa_str \l_stex_get_symbol_uri_str
  \exp_args:Nx \stex_add_import_to_current_module:n{
    \stex_copy_notations:nn {\l_tmpa_str} {\l_stex_get_symbol_uri_str}
  }
  \stex_smsmode_do:
}

%    \end{macrocode}
% \end{macro}
%
% \begin{macro}{\symdef}
%    \begin{macrocode}
\keys_define:nn { stex / symdef } {
  name    .str_set_x:N = \l_stex_symdecl_name_str ,
  local   .bool_set:N  = \l_stex_symdecl_local_bool ,
  args    .str_set_x:N = \l_stex_symdecl_args_str ,
  type    .tl_set:N    = \l_stex_symdecl_type_tl ,
  def     .tl_set:N    = \l_stex_symdecl_definiens_tl ,
  op      .tl_set:N    = \l_@@_op_tl ,
  lang    .str_set_x:N = \l_@@_lang_str ,
  variant .str_set_x:N = \l_@@_variant_str ,
  prec    .str_set_x:N = \l_@@_prec_str ,
  assoc   .choices:nn  = 
      {bin,binl,binr,pre,conj,pwconj}
      {\str_set:Nx \l_stex_symdecl_assoctype_str {\l_keys_choice_tl}},
  unknown .code:n      = \str_set:Nx 
      \l_@@_variant_str \l_keys_key_str
}

\cs_new_protected:Nn \_@@_symdef_args:n {
  \str_clear:N \l_stex_symdecl_name_str
  \str_clear:N \l_stex_symdecl_args_str
  \str_clear:N \l_stex_symdecl_assoctype_str
  \bool_set_false:N \l_stex_symdecl_local_bool
  \tl_clear:N \l_stex_symdecl_type_tl
  \tl_clear:N \l_stex_symdecl_definiens_tl
  \str_clear:N \l_@@_lang_str
  \str_clear:N \l_@@_variant_str
  \str_clear:N \l_@@_prec_str
  \tl_clear:N \l_@@_op_tl
  
  \keys_set:nn { stex / symdef } { #1 }
}

\NewDocumentCommand \symdef { m O{} } {
  \_@@_symdef_args:n { #2 }
  \bool_set_true:N \l_stex_symdecl_make_macro_bool
  \stex_symdecl_do:n { #1 }
  \tl_set:Nn \l_stex_notation_after_do_tl {
    \_@@_final:
    \stex_smsmode_do:
  }
  \str_set:Nx \l_stex_get_symbol_uri_str {
    \l_stex_current_module_str ? \l_stex_symdecl_name_str
  }
  \exp_args:Nx \stex_notation_do:nnnn
    { \prop_item:cn {l_stex_symdecl_\l_stex_get_symbol_uri_str _prop } { args } }
    { \prop_item:cn { l_stex_symdecl_\l_stex_get_symbol_uri_str _prop } { arity } }
    { \l_@@_variant_str \c_hash_str \l_@@_lang_str }
}
\stex_deactivate_macro:Nn \symdef {module~environments}
%    \end{macrocode}
% \end{macro}
%
% \subsection{Variables}
%
%    \begin{macrocode}
%<@@=stex_variables>

\keys_define:nn { stex / vardef } {
  name    .str_set_x:N  = \l_@@_name_str ,
  args    .str_set_x:N  = \l_@@_args_str ,
  type    .tl_set:N     = \l_@@_type_tl ,
  def     .tl_set:N     = \l_@@_def_tl ,
  op      .tl_set:N     = \l_@@_op_tl ,
  prec    .str_set_x:N  = \l_@@_prec_str ,
  assoc   .choices:nn   = 
      {bin,binl,binr,pre,conj,pwconj}
      {\str_set:Nx \l_@@_assoctype_str {\l_keys_choice_tl}},
  bind    .choices:nn   =
      {forall,exists}
      {\str_set:Nx \l_@@_bind_str {\l_keys_choice_tl}}
}

\cs_new_protected:Nn \_@@_args:n {
  \str_clear:N \l_@@_name_str
  \str_clear:N \l_@@_args_str
  \str_clear:N \l_@@_prec_str
  \str_clear:N \l_@@_assoctype_str
  \str_clear:N \l_@@_bind_str
  \tl_clear:N \l_@@_type_tl
  \tl_clear:N \l_@@_def_tl
  \tl_clear:N \l_@@_op_tl

  \keys_set:nn { stex / vardef } { #1 }
}

\NewDocumentCommand \_@@_do_simple:nnn { m O{}} {
  \_@@_args:n {#2}
  \str_if_empty:NT \l_@@_name_str {
    \str_set:Nx \l_@@_name_str { #1 }
  }
  \prop_clear:N \l_tmpa_prop
  \prop_put:Nno \l_tmpa_prop { name } \l_@@_name_str
  
  \int_zero:N \l_tmpb_int
  \bool_set_true:N \l_tmpa_bool
  \str_map_inline:Nn \l_@@_args_str {
    \token_case_meaning:NnF ##1 {
      0 {} 1 {} 2 {} 3 {} 4 {} 5 {} 6 {} 7 {} 8 {} 9 {}
      {\tl_to_str:n i} { \bool_set_false:N \l_tmpa_bool }
      {\tl_to_str:n b} { \bool_set_false:N \l_tmpa_bool }
      {\tl_to_str:n a} { 
        \bool_set_false:N \l_tmpa_bool
        \int_incr:N \l_tmpb_int
      }
      {\tl_to_str:n B} { 
        \bool_set_false:N \l_tmpa_bool
        \int_incr:N \l_tmpb_int
      }
    }{
      \msg_error:nnxx{stex}{error/wrongargs}{
        variable~\l_@@_name_str
      }{##1}
    }
  }
  \bool_if:NTF \l_tmpa_bool {
    % possibly numeric
    \str_if_empty:NTF \l_@@_args_str {
      \prop_put:Nnn \l_tmpa_prop { args } {}
      \prop_put:Nnn \l_tmpa_prop { arity } { 0 }
    }{
      \int_set:Nn \l_tmpa_int { \l_@@_args_str }
      \prop_put:Nnx \l_tmpa_prop { arity } { \int_use:N \l_tmpa_int }
      \str_clear:N \l_tmpa_str
      \int_step_inline:nn \l_tmpa_int {
        \str_put_right:Nn \l_tmpa_str i
      }
      \str_set_eq:NN \l_@@_args_str \l_tmpa_str
      \prop_put:Nnx \l_tmpa_prop { args } { \l_@@_args_str }
    }
  } {
    \prop_put:Nnx \l_tmpa_prop { args } { \l_@@_args_str }
    \prop_put:Nnx \l_tmpa_prop { arity }
      { \str_count:N \l_@@_args_str }
  }
  \prop_put:Nnx \l_tmpa_prop { assocs } { \int_use:N \l_tmpb_int }
  \tl_set:cx { #1 }{ \stex_invoke_variable:n { \l_@@_name_str } }

  \prop_set_eq:cN { l_stex_variable_\l_@@_name_str _prop} \l_tmpa_prop

  \tl_if_empty:NF \l_@@_op_tl {
    \cs_set:cpx {
      stex_var_op_notation_ \l_@@_name_str _cs
    } {
      \_stex_term_omv:nn {
        var://\l_@@_name_str
      }{ \comp{ \exp_args:No \exp_not:n { \l_@@_op_tl } } }
    }
  }

  \tl_set:Nn \l_stex_notation_after_do_tl {
    \exp_args:Nne \use:nn {
      \cs_generate_from_arg_count:cNnn { stex_var_notation_\l_@@_name_str _cs }
        \cs_set:Npn { \prop_item:Nn \l_tmpa_prop { arity } }
    } {{
      \exp_after:wN \exp_after:wN \exp_after:wN
      \exp_not:n \exp_after:wN \exp_after:wN \exp_after:wN 
      { \exp_after:wN \l_stex_notation_macrocode_cs \l_stex_notation_dummyargs_tl \stex_symbol_after_invokation_tl}
    }}
    \stex_if_do_html:T {
      \stex_annotate_invisible:nnn {vardecl}{\l_@@_name_str}{
        \stex_annotate_invisible:nnn { precedence }
          { \l_@@_prec_str }{}
        \tl_if_empty:NF \l_@@_type_tl {\stex_annotate_invisible:nnn{type}{}{$\l_@@_type_tl$}}
        \stex_annotate_invisible:nnn{args}{}{ \l_@@_args_str }
        \stex_annotate_invisible:nnn{macroname}{#1}{}
        \tl_if_empty:NF \l_@@_def_tl {
          \stex_annotate_invisible:nnn{definiens}{}
            {$\l_@@_def_tl$}
        }
        \str_if_empty:NF \l_@@_assoctype_str {
          \stex_annotate_invisible:nnn{assoctype}{\l_@@_assoctype_str}{}
        }
        \int_zero:N \l_tmpa_int
        \str_set_eq:NN \l_@@_remaining_args_str \l_@@_args_str
        \tl_clear:N \l_tmpa_tl
        \int_step_inline:nn { \prop_item:Nn \l_tmpa_prop { arity } }{
          \int_incr:N \l_tmpa_int
          \str_set:Nx \l_tmpb_str { \str_head:N \l_@@_remaining_args_str }
          \str_set:Nx \l_@@_remaining_args_str { \str_tail:N \l_@@_remaining_args_str }
          \str_if_eq:VnTF \l_tmpb_str a {
            \tl_set:Nx \l_tmpa_tl { \l_tmpa_tl { 
              \c_hash_str \c_hash_str \int_use:N \l_tmpa_int a ,
              \c_hash_str \c_hash_str \int_use:N \l_tmpa_int b
            } }
          }{
            \str_if_eq:VnTF \l_tmpb_str B {
              \tl_set:Nx \l_tmpa_tl { \l_tmpa_tl { 
                \c_hash_str \c_hash_str \int_use:N \l_tmpa_int a ,
                \c_hash_str \c_hash_str \int_use:N \l_tmpa_int b
              } }
            }{
              \tl_set:Nx \l_tmpa_tl { \l_tmpa_tl { 
                \c_hash_str \c_hash_str \int_use:N \l_tmpa_int
              } }
            }
          }
        }
        \stex_annotate_invisible:nnn { notationcomp }{}{
          \str_set:Nx \l_stex_current_symbol_str {var://\l_@@_name_str }
          $ \exp_args:Nno \use:nn { \use:c {
            stex_var_notation_\l_@@_name_str _cs
          } } { \l_tmpa_tl } $
        }
      }
    }
  }

  \stex_notation_do:nnnn { \l_@@_args_str } { \prop_item:Nn \l_tmpa_prop { arity } } {}
}

\cs_new:Nn \_@@_reset:N {
  \tl_if_exist:NTF #1 {
    \def \exp_not:N #1 { \exp_args:No \exp_not:n #1 }
  }{
    \let \exp_not:N #1 \exp_not:N \undefined
  }
}

\NewDocumentCommand \_@@_do_complex:nn { m m }{
  \clist_set:Nx \l_@@_names { \tl_to_str:n {#1} }
  \exp_args:Nnx \use:nn {
    % TODO
    \stex_annotate_invisible:nnn {vardecls}{\clist_use:Nn\l_@@_names,}{
      #2
    }
  }{
    \_@@_reset:N \varnot
    \_@@_reset:N \vartype
    \_@@_reset:N \vardef
  }
}

\NewDocumentCommand \vardef { s } {
  \IfBooleanTF#1 {
    \_@@_do_complex:nn
  }{
    \_@@_do_simple:nnn
  }
}

\NewDocumentCommand \svar { O{} m }{
  \tl_if_empty:nTF {#1}{
    \str_set:Nn \l_tmpa_str { #2 }
  }{
    \str_set:Nn \l_tmpa_str { #1 }
  }
  \_stex_term_omv:nn {
        var://\l_tmpa_str
    }{ \comp{ #2 } }
}

%    \end{macrocode}
%
%    \begin{macrocode}
%</package>
%    \end{macrocode}
%
% \end{implementation}
%
% \PrintIndex

% \endinput
% Local Variables:
% mode: doctex
% TeX-master: t
% End:

  % \iffalse meta-comment
% An Infrastructure for Semantic Macros and Module Scoping
% Copyright (c) 2019 Michael Kohlhase, all rights reserved
%                this file is released under the
%                LaTeX Project Public License (LPPL)
% 
% The original of this file is in the public repository at 
% http://github.com/sLaTeX/sTeX/
%
% TODO update copyright  
%
%<*driver>
\providecommand\bibfolder{../../lib/bib}
\RequirePackage{paralist}
\documentclass[full,kernel]{l3doc}
\usepackage[dvipsnames]{xcolor}
\usepackage[utf8]{inputenc}
\usepackage[T1]{fontenc}
%\usepackage{document-structure}
\usepackage[showmods,debug=all,lang={de,en}, mathhub=./tests]{stex}
\usepackage{url,array,float,textcomp}
\usepackage[show]{ed}
\usepackage[hyperref=auto,style=alphabetic]{biblatex}
\addbibresource{\bibfolder/kwarcpubs.bib}
\addbibresource{\bibfolder/extpubs.bib}
\addbibresource{\bibfolder/kwarccrossrefs.bib}
\addbibresource{\bibfolder/extcrossrefs.bib}
\usepackage{amssymb}
\usepackage{amsfonts}
\usepackage{xspace}
\usepackage{hyperref}

\usepackage{morewrites}


\makeindex
\floatstyle{boxed}
\newfloat{exfig}{thp}{lop}
\floatname{exfig}{Example}

\usepackage{stex-tests}

\MakeShortVerb{\|}

\def\scsys#1{{{\sc #1}}\index{#1@{\sc #1}}\xspace}
\def\mmt{\textsc{Mmt}\xspace}
\def\xml{\scsys{Xml}}
\def\mathml{\scsys{MathML}}
\def\omdoc{\scsys{OMDoc}}
\def\openmath{\scsys{OpenMath}}
\def\latexml{\scsys{LaTeXML}}
\def\perl{\scsys{Perl}}
\def\cmathml{Content-{\sc MathML}\index{Content {\sc MathML}}\index{MathML@{\sc MathML}!content}}
\def\activemath{\scsys{ActiveMath}}
\def\twin#1#2{\index{#1!#2}\index{#2!#1}}
\def\twintoo#1#2{{#1 #2}\twin{#1}{#2}}
\def\atwin#1#2#3{\index{#1!#2!#3}\index{#3!#2 (#1)}}
\def\atwintoo#1#2#3{{#1 #2 #3}\atwin{#1}{#2}{#3}}
\def\cT{\mathcal{T}}\def\cD{\mathcal{D}}

\def\fileversion{3.0}
\def\filedate{\today}

\RequirePackage{pdfcomment}

\ExplSyntaxOn\makeatletter
\cs_set_protected:Npn \@comp #1 #2 {
  \pdftooltip {
    \textcolor{blue}{#1}
  } { #2 }
}

\cs_set_protected:Npn \@defemph #1 #2 {
  \pdftooltip { 
    \textbf{\textcolor{magenta}{#1}}
  } { #2 }
}
\makeatother\ExplSyntaxOff

\begin{document}
  \DocInput{\jobname.dtx}
\end{document}
%</driver>
% \fi
%
% \title{ \sTeX-Module Inheritance
% 	\thanks{Version {\fileversion} (last revised {\filedate})} 
% }
%
% \author{Michael Kohlhase, Dennis Müller\\
% 	FAU Erlangen-Nürnberg\\
% 	\url{http://kwarc.info/}
% }
%
% \maketitle
%
% \begin{documentation}\label{pkg:inheritance:doc}
%
% Code related to Module Inheritance, in particular \emph{sms mode}.
%
% \section{Macros and Environments}\label{pkg:inheritance:doc:macros}
%
%    \subsection{SMS Mode}
% ``SMS Mode'' is used when loading modules from external tex files.
% It deactivates any output and ignores all \TeX\ commands
% not explicitly allowed via the following lists:
%
% \begin{variable}{\g_stex_smsmode_allowedmacros_tl}
%   Macros that are executed as is; i.e. with the category code scheme
%   used in SMS mode.
% \end{variable}
%
% \begin{variable}{\g_stex_smsmode_allowedmacros_escape_tl}
%   Macros that are executed with the category codes restored.
%
%   Importantly, these macros need to call \cs{stex_smsmode_set_codes:}
%   after reading all arguments. Note, that 
%   \cs{stex_smsmode_set_codes:} takes
%   care of checking whether we are in SMS mode in the first place, 
%   so calling this function eagerly is unproblematic.
% \end{variable}
%
% \begin{variable}{\g_stex_smsmode_allowedenvs_seq}
%   The names of environments that should be allowed in SMS mode.
%   The corresponding \cs{begin}-statements are treated like
%   the macros in \cs{g_stex_smsmode_allowedmacros_escape_tl}, so
%   \cs{stex_smsmode_set_codes:} should be called at the end of the
%   \cs{begin}-code. Since \cs{end}-statements take no arguments anyway,
%   those are called with the SMS mode category code scheme active.
% \end{variable}
%
% \begin{function}[pTF]{\stex_if_smsmode:}
%   Tests whether SMS mode is currently active.
% \end{function}
%
% \begin{function}{\stex_smsmode_set_codes:}
%   Sets the current category code scheme to that of the SMS mode, if
%   SMS mode is currently active and if necessary.
%
%   This method should be called at the end of every macro or 
%   \cs{begin} environment code that are allowed in SMS mode.
%
% \end{function}
%
% \begin{function}{\stex_in_smsmode:nn}
%   \begin{syntax} \cs{stex_in_smsmode:nn} \Arg{name} \Arg{code} \end{syntax}
%   Executes \meta{code} in SMS mode. \meta{name} can be arbitrary,
%   but should be distinct, since it allows for nesting 
%   \cs{stex_in_smsmode:nn} without spuriously terminating SMS mode.
% \end{function}
%
%\stextest{
% \immediate\openout\testfile=./tests/sometest.tex
% \immediate\write\testfile{\detokenize{\this is \a test}^^J}
% \immediate\write\testfile{\detokenize{this \is a \test}}
% \immediate\closeout\testfile
% \ExplSyntaxOn
% \stex_in_smsmode:nn { foo } { 
%   \this is \a test

this \is a \test 
 
% }
% \ExplSyntaxOff
%}
%
%
%    \subsection{Imports and Inheritance}
%
% \begin{function}{\importmodule}
%   \begin{syntax} \cs{importmodule}|[|\meta{archive-ID}|]|\Arg{module-path} \end{syntax}
%   Imports a module by reading it from a file and ``activating'' it.
%   \sTeX determines the module and its containing file by passing its
%   arguments on to \cs{stex_import_module_path:nn}.
%   
% \end{function}
%
%\stextest{
%   \begin{module}{Foo}
%      \symdecl[name=foo, args=3]{bar}
%      \symdecl[args=bai]{foobar}
%      Meaning:~\present\bar\\
%   \end{module}
%      Meaning:~\present\bar\\
%   \begin{module}{Importtest}
%     \importmodule{Foo}
%      Meaning:~\present\bar\\
%   \end{module}
%   \begin{module}{Importtest2}
%     \importmodule{Importtest}
%      Meaning:~\present\bar\\
%   \end{module}
%}
%
% \begin{function}{\usemodule}
%   \begin{syntax} \cs{importmodule}|[|\meta{archive-ID}|]|\Arg{module-path} \end{syntax}
%   Like \cs{importmodule}, but does not export its contents;
%   i.e. including the current module will not activate the used module
%   
% \end{function}
%
%\stextest{
%   \begin{module}{UseTest1}
%      \symdecl{foo}
%   \end{module}
%   \begin{module}{UseTest2}
%     \usemodule{UseTest1}
%      \symdecl{bar}
%      Meaning:~\present\foo\\
%   \end{module}
%   \begin{module}{UseTest3}
%     \importmodule{UseTest2}
%     Meaning:~\present\foo\\
%     Meaning:~\present\bar\\
%
%     All modules: \ExplSyntaxOn
%       \seq_use:Nn \l_stex_all_modules_seq {,~} \\
%      All~symbols:~
%        \seq_use:Nn \l_stex_all_symbols_seq {,~}
%     \ExplSyntaxOff 
%   \end{module}
%}
%
%
%\stextest{
% Circular dependencies:
% \begin{module}{CircDep1}
%   \importmodule[Foo/Bar]{circular1?Circular1}
%   \importmodule[Bar/Foo]{circular2?Circular2}
%   \present\fooA\\
%   \present\fooB
% \end{module}
%}
%
% \begin{function}{\stex_import_module_uri:nn}
%   \begin{syntax} \cs{stex_import_module_uri:nn} \Arg{archive-ID} \Arg{module-path} \end{syntax}
%   Determines the URI of a module by splitting
%   \meta{module-path} into \meta{path}|?|\meta{name}. If \meta{module-path}
%   does \emph{not} contain a |?|-character, we consider it to be the \meta{name},
%   and \meta{path} to be empty.
%
%   If \meta{archive-ID} is empty, it is automatically set to the
%   ID of the current archive (if one exists).
%
%   \begin{enumerate}
%   \item If \meta{archive-ID} is empty:
%     \begin{enumerate}
%     \item If \meta{path} is empty, then
%         \meta{name} must have been declared earlier in the same file
%         and retrievable from \cs{g_stex_modules_in_file_seq}, or
%         a file with name \meta{name}|.|\meta{lang}|.tex| must exist
%         in the same folder, containing a module \meta{name}.
%
%         That module should have the same namespace as the current one.
%     \item If \meta{path} is not empty, it must point to the relative
%         path of the containing file as well as the namespace.
%     \end{enumerate}
%   \item Otherwise:
%      \begin{enumerate}
%         \item If \meta{path} is empty, then
%         \meta{name} must have been declared earlier in the same file
%         and retrievable from \cs{g_stex_modules_in_file_seq}, or
%         a file with name \meta{name}|.|\meta{lang}|.tex| must exist
%         in the top |source| folder of the archive, 
%         containing a module \meta{name}.
%
%         That module should lie directly in the namespace 
%         of the archive.
%     \item If \meta{path} is not empty, it must point to the
%         path of the containing file as well as the namespace,
%         relative to the namespace of the archive.
%
%         If a module by that namespace exists, it is returned.
%         Otherwise, we call \cs{stex_require_module:nn} 
%         on the |source| directory of the archive to find the
%         file.
%       \end{enumerate}
%   \end{enumerate}
% \end{function}
%
% \begin{function}{\stex_import_require_module:nnnn}
%    \begin{syntax} \Arg{ns} \Arg{archive-ID} \Arg{path} \Arg{name} \end{syntax}
%     Checks whether a module with URI \meta{ns}|?|\meta{name} already
%     exists. If not, it looks for a plausible file that declares
%     a module with that URI.
%
%     Finally, activates that module by executing its |content|-field.
% \end{function}
%
%
% \end{documentation}
%
% \begin{implementation}\label{pkg:inheritance:impl}
%
% \section{\sTeX-Module Inheritance Implementation}
%
%    \begin{macrocode}
%<*package>

%%%%%%%%%%%%%   inheritance.dtx   %%%%%%%%%%%%%

%    \end{macrocode}
%
% \subsection{SMS Mode}
%    \begin{macrocode}
%<@@=stex_smsmode>
%    \end{macrocode}
%
% \begin{variable}{
%   \g_stex_smsmode_allowedmacros_tl,
%   \g_stex_smsmode_allowedmacros_escape_tl,
%   \g_stex_smsmode_allowedenvs_seq
% }
%    \begin{macrocode}
\tl_new:N \g_stex_smsmode_allowedmacros_tl
\tl_new:N \g_stex_smsmode_allowedmacros_escape_tl
\seq_new:N \g_stex_smsmode_allowedenvs_seq

\tl_set:Nn \g_stex_smsmode_allowedmacros_tl {
  \makeatletter
  \makeatother
  \ExplSyntaxOn
  \ExplSyntaxOff
}

\tl_set:Nn \g_stex_smsmode_allowedmacros_escape_tl {
  \symdef
  \importmodule
  \notation
  \symdecl
  \STEXexport
}

\exp_args:NNx \seq_set_from_clist:Nn \g_stex_smsmode_allowedenvs_seq {
  \tl_to_str:n {
    module,
    @module
  }
}
%    \end{macrocode}
% \end{variable}
%
% \begin{macro}[pTF]{\stex_if_smsmode:}
%    \begin{macrocode}
\bool_new:N \g_@@_bool
\bool_set_false:N \g_@@_bool
\prg_new_conditional:Nnn \stex_if_smsmode: { p, T, F, TF } {
  \bool_if:NTF \g_@@_bool \prg_return_true: \prg_return_false:
}
%    \end{macrocode}
% \end{macro}
%
%
% \begin{macro}[pTF]{\_@@_if_catcodes:}
% Checks whether the SMS mode category code scheme is active.
%    \begin{macrocode}
\bool_new:N \g_@@_catcode_bool
\bool_set_false:N \g_@@_catcode_bool
\prg_new_conditional:Nnn \_@@_if_catcodes: { p, T, F, TF } {
  \bool_if:NTF \g_@@_catcode_bool 
    \prg_return_true: \prg_return_false:
}
%    \end{macrocode}
% \end{macro}
%
% \begin{macro}{\stex_smsmode_set_codes:}
%    \begin{macrocode}
\cs_new_protected:Nn \stex_smsmode_set_codes: {
  \stex_if_smsmode:T {
    \_@@_if_catcodes:F {
      \bool_gset_true:N \g_@@_catcode_bool
      \exp_after:wN \char_gset_active_eq:NN 
        \c_backslash_str \_@@_cs:
      \tex_global:D \char_set_catcode_active:N \\
      \tex_global:D \char_set_catcode_other:N $
      \tex_global:D \char_set_catcode_other:N ^
      \tex_global:D \char_set_catcode_other:N _
      \tex_global:D \char_set_catcode_other:N &
      \tex_global:D \char_set_catcode_other:N ##
    }
  }
} \iffalse $ \fi % to make syntax highlighting work again
%    \end{macrocode}
% \end{macro}
%
% \begin{macro}{\_@@_unset_codes:}
%   Sets category code scheme back from the one used in SMS mode.
%    \begin{macrocode}
\cs_new_protected:Nn \_@@_unset_codes: {
  \_@@_if_catcodes:T {
    \bool_gset_false:N \g_@@_catcode_bool
    \exp_after:wN \tex_global:D \exp_after:wN 
      \char_set_catcode_escape:N \c_backslash_str
    \tex_global:D \char_set_catcode_math_toggle:N $
    \tex_global:D \char_set_catcode_math_superscript:N ^
    \tex_global:D \char_set_catcode_math_subscript:N _
    \tex_global:D \char_set_catcode_alignment:N &
    \tex_global:D \char_set_catcode_parameter:N ##
  }
} \iffalse $ \fi % to make syntax highlighting work again
%    \end{macrocode}
% \end{macro}
%
% \begin{macro}{\stex_in_smsmode:nn}
%    \begin{macrocode}
\cs_new_protected:Nn \stex_in_smsmode:nn {
  \vbox_set:Nn \l_tmpa_box {
    \bool_set_eq:cN { l_@@_#1_bool } \g_@@_bool
    \bool_gset_true:N \g_@@_bool
    \stex_smsmode_set_codes:
    #2
    \bool_gset_eq:Nc \g_@@_bool { l_@@_#1_bool }
    \stex_if_smsmode:F {
      \_@@_unset_codes:
    }
  }
  \box_clear:N \l_tmpa_box
}
%    \end{macrocode}
% \end{macro}
%
% \begin{macro}{\_@@_cs:}
% is executed on encountering |\| in smsmode.
% It checks whether the corresponding command is allowed and executes
% or ignores it accordingly:
%    \begin{macrocode}
\cs_new_protected:Nn \_@@_cs: {
  \str_clear:N \l_tmpa_str
  \peek_analysis_map_inline:n {
    % #1: token (one expansion)
    % #2: charcode
    % #3 catcode
    \token_if_eq_charcode:NNTF ##3 B {
      % token is a letter
      \exp_args:NNo \str_put_right:Nn \l_tmpa_str { ##1 }
    } {
      \str_if_empty:NTF \l_tmpa_str {
        % we don't allow (or need) single non-letter CSs
        % for now
        \peek_analysis_map_break: 
      }{
        \str_if_eq:onTF \l_tmpa_str { begin } {
          \peek_analysis_map_break:n { 
            \exp_after:wN \_@@_checkbegin:n ##1
          }
        } {
          \str_if_eq:onTF \l_tmpa_str { end } {
            \peek_analysis_map_break:n { 
              \exp_after:wN \_@@_checkend:n ##1
            }
          } {
          \tl_set:Nn \l_tmpa_tl { \use:c{\l_tmpa_str} }
          \exp_args:NNNo \exp_args:NNo \tl_if_in:NnTF 
            \g_stex_smsmode_allowedmacros_tl 
              { \use:c{\l_tmpa_str} } {
              \stex_debug:nn{modules}{Executing~1:~\l_tmpa_str}
              \peek_analysis_map_break:n { 
                \exp_after:wN \l_tmpa_tl ##1
              }
            } {
              \exp_args:NNNo \exp_args:NNo \tl_if_in:NnTF 
              \g_stex_smsmode_allowedmacros_escape_tl 
                { \use:c{\l_tmpa_str} } {
                \_@@_unset_codes:
                \stex_debug:nn{modules}{Executing~2:~\l_tmpa_str}
                % TODO \_@@_rescan_cs:
%                \exp_after:wN \exp_after:wN \exp_after:wN
%                \token_if_eq_charcode:NNTF \exp_after:wN \c_backslash_str ##1 {
%                  \peek_analysis_map_break:n {
%                    \_@@_unset_codes:
%                    \_@@_rescan_cs:
%                  }
%                } {
                  \peek_analysis_map_break:n {
                    \exp_after:wN \l_tmpa_tl ##1
                  }
%                }
              } {
                \peek_analysis_map_break:n { ##1 }
              }
            }
          }
        }
      }
    }
  }
}
%    \end{macrocode}
% \end{macro}
%
%
% \begin{macro}{\_@@_rescan_cs:}
% If the last token gobbled by |\stex_smsmode_cs:| happened to be
% a |\|, we need to rescan the cs name and reinsert it into the input
% stream:
%    \begin{macrocode}
\cs_new_protected:Nn \_@@_rescan_cs: {
  \str_clear:N \l_tmpb_str
  \peek_analysis_map_inline:n {
    \token_if_eq_charcode:NNTF ##3 B {
      % token is a letter
      \exp_args:NNo \str_put_right:Nn \l_tmpb_str { ##1 }
    } {
      \peek_analysis_map_break:n {
        \exp_after:wN \use:c \exp_after:wN { 
          \exp_after:wN \l_tmpa_str\exp_after:wN 
        } \use:c { \l_tmpb_str \exp_after:wN } ##1
      }
    }
  }
}
%    \end{macrocode}
% \end{macro}
%
% \begin{macro}{\_@@_checkbegin:n}
% called on |\begin|; checks whether the environment being opened
% is allowed in SMS mode. 
%    \begin{macrocode}
\cs_new_protected:Nn \_@@_checkbegin:n {
  \str_set:Nn \l_tmpa_str { #1 }
  \seq_if_in:NoT \g_stex_smsmode_allowedenvs_seq \l_tmpa_str {
    \_@@_unset_codes:
    \begin{#1}
  }
}
%    \end{macrocode}
% \end{macro}
%
% \begin{macro}{\_@@_checkend:n}
% called on |\end|; checks whether the environment being opened
% is allowed in SMS mode. 
%    \begin{macrocode}
\cs_new_protected:Nn \_@@_checkend:n {
  \str_set:Nn \l_tmpa_str { #1 }
  \seq_if_in:NoT \g_stex_smsmode_allowedenvs_seq \l_tmpa_str {
    \end{#1}
  }
}
%    \end{macrocode}
% \end{macro}
%
% \subsection{Inheritance}
%    \begin{macrocode}
%<@@=stex_importmodule>
%    \end{macrocode}
%
% \begin{macro}{\stex_import_module_uri:nn}
%    \begin{macrocode}
\cs_new_protected:Nn \stex_import_module_uri:nn {
  \str_set:Nx \l_@@_archive_str { #1 }
  \str_set:Nn \l_@@_path_str { #2 }
  \str_if_empty:NT \l_@@_archive_str {
    \prop_if_empty:NF \l_stex_current_repository_prop {
      \prop_get:NnN \l_stex_current_repository_prop { id } \l_@@_archive_str
    }
  }

  \exp_args:NNNo \seq_set_split:Nnn \l_tmpb_seq ? { \l_@@_path_str }
  \seq_pop_right:NN \l_tmpb_seq \l_@@_name_str
  \str_set:Nx \l_@@_path_str { \seq_use:Nn \l_tmpb_seq ? }

  \str_if_empty:NTF \l_@@_archive_str {
    \stex_modules_current_namespace:
    \str_if_empty:NF \l_@@_path_str {
      \str_set:Nx \l_stex_module_ns_str {
        \l_stex_module_ns_str / \l_@@_path_str
      }
    }
  }{
    \stex_require_repository:n \l_@@_archive_str
    \prop_get:cnN { c_stex_mathhub_\l_@@_archive_str _manifest_prop } { ns }
      \l_stex_module_ns_str
    \str_if_empty:NF \l_@@_path_str {
      \str_set:Nx \l_stex_module_ns_str {
        \l_stex_module_ns_str / \l_@@_path_str
      }
    }
  }
}
%    \end{macrocode}
% \end{macro}
%
% \begin{variable}{
%   \l_@@_name_str,\l_@@_archive_str,\l_@@_path_str,\l_@@_file_str
% }
%   Store the return values of \cs{stex_import_module_uri:nn}.
%    \begin{macrocode}
\str_new:N \l_@@_name_str
\str_new:N \l_@@_archive_str
\str_new:N \l_@@_path_str
\str_new:N \g_@@_file_str
%    \end{macrocode}
% \end{variable}
%
% \begin{macro}{\stex_import_require_module:nnnn}
%    \begin{syntax} \Arg{ns} \Arg{archive-ID} \Arg{path} \Arg{name} \end{syntax}
%    \begin{macrocode}
\cs_new_protected:Nn \stex_import_require_module:nnnn {
  \exp_args:Nx \stex_if_module_exists:nF { #1 ? #4 } {

    % archive
    \str_set:Nx \l_tmpa_str { #2 }
    \str_if_empty:NTF \l_tmpa_str {
      \seq_set_eq:NN \l_tmpa_seq \g_stex_currentfile_seq
    } {
      \stex_path_from_string:Nn \l_tmpb_seq { \l_tmpa_str }
      \seq_concat:NNN \l_tmpa_seq \c_stex_mathhub_seq \l_tmpb_seq
      \seq_put_right:Nn \l_tmpa_seq { source }
    }

    % path
    \str_set:Nx \l_tmpb_str { #3 }
    \str_if_empty:NTF \l_tmpb_str {
      \str_set:Nx \l_tmpa_str { \stex_path_to_string:N \l_tmpa_seq / #4 }
      
      \ltx@ifpackageloaded{babel} { 
        \exp_args:NNx \prop_get:NnNF \c_stex_language_abbrevs_prop 
            { \languagename } \l_tmpb_str {
              \msg_error:nnn{stex}{error/unknownlanguage}{\languagename}
            }
      } {
        \str_clear:N \l_tmpb_str
      }

      \stex_debug:nn{modules}{Checking~\l_tmpa_str.\l_tmpb_str.tex}
      \IfFileExists{ \l_tmpa_str.\l_tmpb_str.tex }{
        \str_gset:Nx \g_@@_file_str { \l_tmpa_str.\l_tmpb_str.tex }
      }{
        \stex_debug:nn{modules}{Checking~\l_tmpa_str.tex}
        \IfFileExists{ \l_tmpa_str.tex }{
          \str_gset:Nx \g_@@_file_str { \l_tmpa_str.tex }
        }{
          % try english as default
          \stex_debug:nn{modules}{Checking~\l_tmpa_str.en.tex}
          \IfFileExists{ \l_tmpa_str.en.tex }{
            \str_gset:Nx \g_@@_file_str { \l_tmpa_str.en.tex }
          }{
            \msg_error:nnn{stex}{error/unknownmodule}{#1?#4}
          }
        }
      }

    } {
      \seq_set_split:NnV \l_tmpb_seq / \l_tmpb_str
      \seq_concat:NNN \l_tmpa_seq \l_tmpa_seq \l_tmpb_seq
      
      \ltx@ifpackageloaded{babel} { 
        \exp_args:NNx \prop_get:NnNF \c_stex_language_abbrevs_prop 
            { \languagename } \l_tmpb_str {
              \msg_error:nnn{stex}{error/unknownlanguage}{\languagename}
            }
      } {
        \str_clear:N \l_tmpb_str
      }

      \stex_path_to_string:NN \l_tmpa_seq \l_tmpa_str

      \stex_debug:nn{modules}{Checking~\l_tmpa_str/#4.\l_tmpb_str.tex}
      \IfFileExists{ \l_tmpa_str/#4.\l_tmpb_str.tex }{
        \str_gset:Nx \g_@@_file_str { \l_tmpa_str/#4.\l_tmpb_str.tex }
      }{
        \stex_debug:nn{modules}{Checking~\l_tmpa_str/#4.tex}
        \IfFileExists{ \l_tmpa_str/#4.tex }{
          \str_gset:Nx \g_@@_file_str { \l_tmpa_str/#4.tex }
        }{
          % try english as default
          \stex_debug:nn{modules}{Checking~\l_tmpa_str/#4.en.tex}
          \IfFileExists{ \l_tmpa_str/#4.en.tex }{
            \str_gset:Nx \g_@@_file_str { \l_tmpa_str/#4.en.tex }
          }{
            \stex_debug:nn{modules}{Checking~\l_tmpa_str.\l_tmpb_str.tex}
            \IfFileExists{ \l_tmpa_str.\l_tmpb_str.tex }{
              \str_gset:Nx \g_@@_file_str { \l_tmpa_str.\l_tmpb_str.tex }
            }{
              \stex_debug:nn{modules}{Checking~\l_tmpa_str.tex}
              \IfFileExists{ \l_tmpa_str.tex }{
                \str_gset:Nx \g_@@_file_str { \l_tmpa_str.tex }
              }{
                % try english as default
                \stex_debug:nn{modules}{Checking~\l_tmpa_str.en.tex}
                \IfFileExists{ \l_tmpa_str.en.tex }{
                  \str_gset:Nx \g_@@_file_str { \l_tmpa_str.en.tex }
                }{
                  \msg_error:nnn{stex}{error/unknownmodule}{#1?#4}
                }
              }
            }
          }
        }
      }
    }

    \seq_set_eq:NN \l_tmpa_seq \g_stex_modules_in_file_seq
    \seq_clear:N \g_stex_modules_in_file_seq
%    \exp_args:Nnx \use:nn {
      \exp_args:No \stex_in_smsmode:nn { \g_@@_file_str } {
        \seq_clear:N \l_stex_all_modules_seq
        \prop_clear:N \l_stex_current_module_prop
        \str_set:Nx \l_tmpb_str { #2 }
        \str_if_empty:NF \l_tmpb_str {
          \stex_set_current_repository:n { #2 }
        }
        \stex_debug:nn{modules}{Loading~\g_@@_file_str}
        \input { \g_@@_file_str }
      }
%    }{

%    }
    \prop_gput:Noo \g_stex_module_files_prop
    \g_@@_file_str \g_stex_modules_in_file_seq
    \seq_set_eq:NN \g_stex_modules_in_file_seq \l_tmpa_seq

    \stex_if_module_exists:nF { #1 ? #4 } {
      \msg_error:nnn{stex}{error/unknownmodule}{
        #1?#4~(in~file~\g_@@_file_str)
      }
    }
  }
  \stex_activate_module:n { #1 ? #4 }
}
%    \end{macrocode}
% \end{macro}
%
% \begin{macro}{\importmodule}
%    \begin{macrocode}
\NewDocumentCommand \importmodule { O{} m } {
  \stex_import_module_uri:nn { #1 } { #2 }
  \stex_debug:nn{modules}{Importing~module:~
    \l_stex_module_ns_str ? \l_@@_name_str
  }
  \stex_if_smsmode:F {
    \stex_import_require_module:nnnn 
    { \l_stex_module_ns_str } { \l_@@_archive_str } 
    { \l_@@_path_str } { \l_@@_name_str }
    \stex_annotate_invisible:nnn 
      {import} {\l_stex_module_ns_str ? \l_@@_name_str} {}
  }
  \exp_args:Nx \stex_add_to_current_module:n {
    \stex_import_require_module:nnnn 
    { \l_stex_module_ns_str } { \l_@@_archive_str } 
    { \l_@@_path_str } { \l_@@_name_str }
  }
  \exp_args:Nx \stex_add_import_to_current_module:n {
    \l_stex_module_ns_str ? \l_@@_name_str
  }
  \stex_smsmode_set_codes:
}
\stex_deactivate_macro:Nn \importmodule {module~environments}
%    \end{macrocode}
% \end{macro}
%
% \begin{macro}{\usemodule}
%    \begin{macrocode}
\NewDocumentCommand \usemodule { O{} m } {
  \stex_if_smsmode:F {
    \stex_import_module_uri:nn { #1 } { #2 }
    \stex_import_require_module:nnnn 
    { \l_stex_module_ns_str } { \l_@@_archive_str }
    { \l_@@_path_str } { \l_@@_name_str }
    \stex_annotate_invisible:nnn 
      {usemodule} {\l_stex_module_ns_str ? \l_@@_name_str} {}
  }
  \stex_smsmode_set_codes:
}
%    \end{macrocode}
% \end{macro}
%
%    \begin{macrocode}
%</package>
%    \end{macrocode}
%
% \end{implementation}
%
% \PrintIndex

  % \iffalse meta-comment
% An Infrastructure for Semantic Macros and Module Scoping
% Copyright (c) 2019 Michael Kohlhase, all rights reserved
%                this file is released under the
%                LaTeX Project Public License (LPPL)
% 
% The original of this file is in the public repository at 
% http://github.com/sLaTeX/sTeX/
%
% TODO update copyright  
%
%<*driver>
\providecommand\bibfolder{../../lib/bib}
\RequirePackage{paralist}
\documentclass[full,kernel]{l3doc}
\usepackage[dvipsnames]{xcolor}
\usepackage[utf8]{inputenc}
\usepackage[T1]{fontenc}
%\usepackage{document-structure}
\usepackage[showmods,debug=all,lang={de,en}, mathhub=./tests]{stex}
\usepackage{url,array,float,textcomp}
\usepackage[show]{ed}
\usepackage[hyperref=auto,style=alphabetic]{biblatex}
\addbibresource{\bibfolder/kwarcpubs.bib}
\addbibresource{\bibfolder/extpubs.bib}
\addbibresource{\bibfolder/kwarccrossrefs.bib}
\addbibresource{\bibfolder/extcrossrefs.bib}
\usepackage{amssymb}
\usepackage{amsfonts}
\usepackage{xspace}
\usepackage{hyperref}

\usepackage{morewrites}


\makeindex
\floatstyle{boxed}
\newfloat{exfig}{thp}{lop}
\floatname{exfig}{Example}

\usepackage{stex-tests}

\MakeShortVerb{\|}

\def\scsys#1{{{\sc #1}}\index{#1@{\sc #1}}\xspace}
\def\mmt{\textsc{Mmt}\xspace}
\def\xml{\scsys{Xml}}
\def\mathml{\scsys{MathML}}
\def\omdoc{\scsys{OMDoc}}
\def\openmath{\scsys{OpenMath}}
\def\latexml{\scsys{LaTeXML}}
\def\perl{\scsys{Perl}}
\def\cmathml{Content-{\sc MathML}\index{Content {\sc MathML}}\index{MathML@{\sc MathML}!content}}
\def\activemath{\scsys{ActiveMath}}
\def\twin#1#2{\index{#1!#2}\index{#2!#1}}
\def\twintoo#1#2{{#1 #2}\twin{#1}{#2}}
\def\atwin#1#2#3{\index{#1!#2!#3}\index{#3!#2 (#1)}}
\def\atwintoo#1#2#3{{#1 #2 #3}\atwin{#1}{#2}{#3}}
\def\cT{\mathcal{T}}\def\cD{\mathcal{D}}

\def\fileversion{3.0}
\def\filedate{\today}

\RequirePackage{pdfcomment}

\ExplSyntaxOn\makeatletter
\cs_set_protected:Npn \@comp #1 #2 {
  \pdftooltip {
    \textcolor{blue}{#1}
  } { #2 }
}

\cs_set_protected:Npn \@defemph #1 #2 {
  \pdftooltip { 
    \textbf{\textcolor{magenta}{#1}}
  } { #2 }
}
\makeatother\ExplSyntaxOff

\begin{document}
  \DocInput{\jobname.dtx}
\end{document}
%</driver>
% \fi
%
% \title{ \sTeX-Structural Features
% 	\thanks{Version {\fileversion} (last revised {\filedate})} 
% }
%
% \author{Michael Kohlhase, Dennis Müller\\
% 	FAU Erlangen-Nürnberg\\
% 	\url{http://kwarc.info/}
% }
%
% \maketitle
%
%\ifinfulldoc\else
% This is the documentation for the \pkg{stex-features} package.
% For a more high-level introduction, 
%  see \href{\basedocurl/manual.pdf}{the \sTeX Manual} or the
% \href{\basedocurl/stex.pdf}{full \sTeX documentation}.
%
% Given modules:

\stexexample{
    \begin{smodule}{magma}
        \symdef{universe}{\comp{\mathcal U}}
        \symdef{operation}[args=2,op=\circ]{#1 \comp\circ #2}
    \end{smodule}
    \begin{smodule}{monoid}
        \importmodule{magma}
        \symdef{unit}{\comp e}
    \end{smodule}
    \begin{smodule}{group}
        \importmodule{monoid}
        \symdef{inverse}[args=1]{{#1}^{\comp{-1}}}
    \end{smodule}
}

We can form a module for \emph{rings} by ``cloning''
an instance of |group| (for addition) and |monoid| (for multiplication),
respectively, and ``glueing them together'' to ensure they share the
same universe:

\stexexample{
    \begin{smodule}{ring}
        \begin{copymodule}{group}{addition}
            \renamedecl[name=universe]{universe}{runiverse}
            \renamedecl[name=plus]{operation}{rplus}
            \renamedecl[name=zero]{unit}{rzero}
            \renamedecl[name=uminus]{inverse}{ruminus}
        \end{copymodule}
        \notation*{rplus}[plus,op=+,prec=60]{#1 \comp+ #2}
        %\setnotation{rplus}{plus}
        \notation*{rzero}[zero]{\comp0}
        %\setnotation{rzero}{zero}
        \notation*{ruminus}[uminus,op=-]{\comp- #1}
        %\setnotation{ruminus}{uminus}
        \begin{copymodule}{monoid}{multiplication}
            \assign{universe}{\runiverse}
            \renamedecl[name=times]{operation}{rtimes}
            \renamedecl[name=one]{unit}{rone}
        \end{copymodule}
        \notation*{rtimes}[cdot,op=\cdot,prec=50]{#1 \comp\cdot #2}
        %\setnotation{rtimes}{cdot}
        \notation*{rone}[one]{\comp1}
        %\setnotation{rone}{one}
        Test: $\rtimes a{\rplus c{\rtimes de}}$
    \end{smodule}
}

\textcolor{red}{TODO: explain donotclone}


\stexexample{
    \begin{smodule}{int}
        \symdef{Integers}{\comp{\mathbb Z}}
        \symdef{plus}[args=2,op=+]{#1 \comp+ #2}
        \symdef{zero}{\comp0}
        \symdef{uminus}[args=1,op=-]{\comp-#1}

        \begin{interpretmodule}{group}{intisgroup}
            \assign{universe}{\Integers}
            \assign{operation}{\plus!}
            \assign{unit}{\zero}
            \assign{inverse}{\uminus!}
        \end{interpretmodule}
    \end{smodule}
}
% \fi
%
% \begin{documentation}\label{pkg:features:doc}
%
% Code related to structural features
%
% \section{Macros and Environments}\label{pkg:features:doc:macros}
%
% \subsection{Structures}
%
% \begin{environment}{mathstructure}
%   TODO
% \end{environment}
%
%^^A\stextest{
%^^A \begin{module}{StructureTest1}
%^^A   \begin{mathstructure}[name=Magma]{magma}
%^^A     \symdef{universe}{\comp M}
%^^A     \symdef[args=2]{op}{#1 \comp\circ #2}
%^^A       $\isa{\op ab}\universe$
%^^A   \end{mathstructure}
%^^A
%^^A     \ExplSyntaxOn
%^^A     \prop_get:NnN \g_stex_last_feature_prop {fields} \l_tmpa_seq
%^^A     \seq_use:Nn \l_tmpa_seq {,}
%^^A     \ExplSyntaxOff
%^^A
%^^A     \present\magma
%^^A
%^^A     \instantiate{magma}[
%^^A       universe ! {{\comp U}},
%^^A       op ! {{#1 \comp+ #2 }}
%^^A     ]{mM}
%^^A     \notation[op = U]{mM/universe}{\comp U}
%^^A     \notation[op = +]{mM/op}{#1 \comp+ #2}
%^^A
%^^A     Test: $\mM{op}ab$
%^^A
%^^A     Test2: $\mM{}$
%^^A \end{module}
%^^A}
%
%
% \end{documentation}
%
% \begin{implementation}\label{pkg:features:impl}
%
% \section{\sTeX-Structural Features Implementation}
%
%    \begin{macrocode}
%<*package>

%%%%%%%%%%%%%   features.dtx   %%%%%%%%%%%%%

%<@@=stex_features>
%    \end{macrocode}
%
% Warnings and error messages
%
%    \begin{macrocode}

%    \end{macrocode}
%
% \subsection{Imports with modification}
%
%    \begin{macrocode}
\cs_new_protected:Nn \stex_get_symbol_in_copymodule:n {
  \tl_if_head_eq_catcode:nNTF { #1 } \relax {
    \_@@_get_symbol_from_cs:n { #1 }
  }{
    % argument is a string
    % is it a command name?
    \cs_if_exist:cTF { #1 }{
      \cs_set_eq:Nc \l_tmpa_tl { #1 }
      \str_set:Nx \l_tmpa_str { \cs_argument_spec:N \l_tmpa_tl }
      \str_if_empty:NTF \l_tmpa_str {
        \exp_args:Nx \cs_if_eq:NNTF {
          \tl_head:N \l_tmpa_tl
        } \stex_invoke_symbol:n {
          \exp_args:No \_@@_get_symbol_from_cs:n { \use:c { #1 } }
        }{
          \_@@_get_symbol_from_string:n { #1 }
        }
      } {
        \_@@_get_symbol_from_string:n { #1 }
      }
    }{
      % argument is not a command name
      \_@@_get_symbol_from_string:n { #1 }
      % \l_stex_all_symbols_seq
    }
  }
}

\cs_new_protected:Nn \_@@_get_symbol_from_string:n {
  \str_set:Nn \l_tmpa_str { #1 }
  \bool_set_false:N \l_tmpa_bool
  \bool_if:NF \l_tmpa_bool {
    \tl_set:Nn \l_tmpa_tl {
      \msg_set:nnn{stex}{error/unknownsymbol}{
        No~symbol~#1~found!
      }
      \msg_error:nn{stex}{error/unknownsymbol}
    }
    \str_set:Nn \l_tmpa_str { #1 }
    \int_set:Nn \l_tmpa_int { \str_count:N \l_tmpa_str }
    \seq_map_inline:Nn \l_@@_copymodule_fields_seq {
      \str_set:Nn \l_tmpb_str { ##1 }
      \str_if_eq:eeT { \l_tmpa_str } {
        \str_range:Nnn \l_tmpb_str { -\l_tmpa_int } { -1 }
      } {
        \seq_map_break:n {
          \tl_set:Nn \l_tmpa_tl {
            \str_set:Nn \l_stex_get_symbol_uri_str {
              ##1
            }
            \_@@_get_symbol_check:
          }
        }
      }
    }
    \l_tmpa_tl
  }
}

\cs_new_protected:Nn \_@@_get_symbol_from_cs:n {
  \exp_args:NNx \tl_set:Nn \l_tmpa_tl 
    { \tl_tail:N \l_tmpa_tl }
  \tl_if_single:NTF \l_tmpa_tl {
    \exp_args:No \tl_if_head_is_group:nTF \l_tmpa_tl {
      \exp_after:wN \str_set:Nn \exp_after:wN
        \l_stex_get_symbol_uri_str \l_tmpa_tl
      \_@@_get_symbol_check:
    }{
      % TODO
      % tail is not a single group
    }
  }{
    % TODO
    % tail is not a single group
  }
}

\cs_new_protected:Nn \_@@_get_symbol_check: {
  \exp_args:NNno \seq_set_split:Nnn \l_tmpa_seq {?} \l_stex_get_symbol_uri_str
  \int_compare:nNnTF {\seq_count:N \l_tmpa_seq} = 3 {
    \seq_pop_right:NN \l_tmpa_seq \l_tmpb_str
    \str_set:Nx \l_tmpa_str {\seq_use:Nn \l_tmpa_seq ?}
    \seq_if_in:NoF \l_@@_copymodule_modules_seq {
      % TODO error
    }
  }{
    % TODO error
  }
}

\NewDocumentEnvironment {copymodule} { O{} m m}{
  \stex_import_module_uri:nn { #1 } { #2 }
  \stex_deactivate_macro:Nn \symdecl {module~environments}
  \stex_deactivate_macro:Nn \symdef {module~environments}
  \stex_deactivate_macro:Nn \notation {module~environments}
  \stex_reactivate_macro:N \assign
  \stex_reactivate_macro:N \renamedecl
  \stex_reactivate_macro:N \donotcopy
  \str_set:Nx \l_stex_current_copymodule_name_str {#3}
  %\let\notation\notation_in_copymodules:
  %\stex_module_setup:nn {}{ #3 }
  \stex_import_require_module:nnnn 
    { \l_stex_import_ns_str } { \l_stex_import_archive_str } 
    { \l_stex_import_path_str } { \l_stex_import_name_str }
  \stex_collect_imports:n {\l_stex_import_ns_str ?\l_stex_import_name_str }
  \seq_set_eq:NN \l_@@_copymodule_modules_seq \l_stex_collect_imports_seq
  \seq_clear:N \l_@@_copymodule_fields_seq
  \seq_map_inline:Nn \l_@@_copymodule_modules_seq {
    \seq_map_inline:cn {c_stex_module_##1_constants}{
      \exp_args:NNx \seq_put_right:Nn \l_@@_copymodule_fields_seq {
        ##1 ? ####1
      }
    }
  }
  \seq_clear:N \l_tmpa_seq
  \exp_args:NNx \prop_set_from_keyval:Nn \l_stex_current_copymodule_prop {
    name      = \l_stex_current_copymodule_name_str ,
    module    = \l_stex_current_module_str ,
    from      = \l_stex_import_ns_str ?\l_stex_import_name_str ,
    includes  = \l_tmpa_seq ,
    fields    = \l_tmpa_seq
  }
  \stex_debug:nn{copymodule}{cloning~module~{\l_stex_import_ns_str ?\l_stex_import_name_str}
    as~\l_stex_current_module_str?\l_stex_current_copymodule_name_str}
  \stex_debug:nn{copymodule}{fields:\seq_use:Nn \l_@@_copymodule_fields_seq {,~}}
  % todo

  \stex_if_smsmode:TF {
    \stex_smsmode_set_codes:
  } {
    \begin{stex_annotate_env} {structure} {
      \l_stex_current_module_str?\l_stex_current_copymodule_name_str
    }
    \stex_annotate_invisible:nnn{from}{\l_stex_import_ns_str ?\l_stex_import_name_str}{}
  }
  \bool_set_eq:NN \l_@@_oldhtml_bool \l_stex_html_do_output_bool
  \bool_set_false:N \l_stex_html_do_output_bool
}{
  \bool_set_eq:NN \l_stex_html_do_output_bool \l_@@_oldhtml_bool
  \tl_clear:N \l_tmpa_tl
  \prop_get:NnN \l_stex_current_copymodule_prop {fields} \l_tmpa_seq
  \seq_map_inline:Nn \l_@@_copymodule_modules_seq {
    \seq_map_inline:cn {c_stex_module_##1_constants}{\stex_annotate:nnn{assignment} {##1?####1} {
      \str_if_exist:cTF {l_@@_copymodule_##1?####1_name_str} {
        \tl_put_right:Nx \l_tmpa_tl {
          \prop_set_from_keyval:cn {
            l_stex_symdecl_\l_stex_current_module_str ? \use:c{l_@@_copymodule_##1?####1_name_str}_prop
          }{
            \exp_after:wN \prop_to_keyval:N \csname
              l_stex_symdecl_\l_stex_current_module_str ? \use:c{l_@@_copymodule_##1?####1_name_str}_prop
            \endcsname
          }
          \seq_clear:c {
            l_stex_symdecl_ 
            \l_stex_current_module_str ? \use:c{l_@@_copymodule_##1?####1_name_str}
            _notations
          }
        }
        \stex_annotate_invisible:nnn{alias}{\use:c{l_@@_copymodule_##1?####1_name_str}}{}
        \seq_put_right:Nx \l_tmpa_seq {\l_stex_current_module_str ? \use:c{l_@@_copymodule_##1?####1_name_str}}
        \str_if_exist:cT {l_@@_copymodule_##1?####1_macroname_str} {
          \stex_annotate_invisible:nnn{macroname}{\use:c{l_@@_copymodule_##1?####1_macroname_str}}{}
          \tl_put_right:Nx \l_tmpa_tl {
            \tl_set:cx {\use:c{l_@@_copymodule_##1?####1_macroname_str}}{
              \stex_invoke_symbol:n {
                \l_stex_current_module_str ? \use:c{l_@@_copymodule_##1?####1_name_str}
              }
            }
          }
        }
      }{
        \prop_set_eq:Nc \l_tmpa_prop {l_stex_symdecl_ ##1?####1 _prop}
        \prop_put:Nnx \l_tmpa_prop { name }{ \l_stex_current_copymodule_name_str / ####1 }
        \prop_put:Nnx \l_tmpa_prop { module }{ \l_stex_current_module_str }
        \tl_put_right:Nx \l_tmpa_tl {
          \prop_set_from_keyval:cn {
            l_stex_symdecl_\l_stex_current_module_str ? \l_stex_current_copymodule_name_str / ####1_prop
          }{
            \prop_to_keyval:N \l_tmpa_prop
          }
          \seq_clear:c {
            l_stex_symdecl_ 
            \l_stex_current_module_str ? \l_stex_current_copymodule_name_str / ####1
            _notations
          }
        }
        \seq_put_right:Nx \l_tmpa_seq {\l_stex_current_module_str ? \l_stex_current_copymodule_name_str / ####1 }
        \str_if_exist:cT {l_@@_copymodule_##1?####1_macroname_str} {
          \stex_annotate_invisible:nnn{macroname}{\use:c{l_@@_copymodule_##1?####1_macroname_str}}{}
          \tl_put_right:Nx \l_tmpa_tl {
            \tl_set:cx {\use:c{l_@@_copymodule_##1?####1_macroname_str}}{
              \stex_invoke_symbol:n {
                \l_stex_current_module_str ? \l_stex_current_copymodule_name_str / ####1
              }
            }
          }
        }
      }
      \tl_if_exist:cT {l_@@_copymodule_##1?####1_def_tl}{
        \stex_annotate_invisible:nnn{definiens}{}{$\use:c{l_@@_copymodule_##1?####1_def_tl}$}
      }
      % todo notations
    }}
  }
  \prop_put:Nno \l_stex_current_copymodule_prop {fields} \l_tmpa_seq
  \tl_put_left:Nx \l_tmpa_tl {
    \prop_set_from_keyval:cn {
      l_stex_copymodule_ \l_stex_current_module_str?\l_stex_current_copymodule_name_str _prop
    }{
      \prop_to_keyval:N \l_stex_current_copymodule_prop
    }
  }
  \exp_args:No \stex_add_to_current_module:n \l_tmpa_tl
  \stex_debug:nn{copymodule}{result:\meaning \l_tmpa_tl}
  \exp_args:Nx \stex_do_aftergroup:n { 
      \exp_args:No \exp_not:n \l_tmpa_tl
  }
  \stex_if_smsmode:F {
    \end{stex_annotate_env}
  }
}

\NewDocumentCommand \donotcopy { O{} m}{
  \stex_import_module_uri:nn { #1 } { #2 }
  \stex_collect_imports:n {\l_stex_import_ns_str ?\l_stex_import_name_str }
  \seq_map_inline:Nn \l_stex_collect_imports_seq {
    \seq_remove_all:Nn \l_@@_copymodule_modules_seq { ##1 }
    \seq_map_inline:cn {c_stex_module_##1_constants}{
      \seq_remove_all:Nn \l_@@_copymodule_fields_seq { ##1 ? ####1 }
      \bool_lazy_any_p:nT {
        { \cs_if_exist_p:c {l_@@_copymodule_##1?####1_name_str}}
        { \cs_if_exist_p:c {l_@@_copymodule_##1?####1_macroname_str}}
        { \cs_if_exist_p:c {l_@@_copymodule_##1?####1_def_tl}}
      }{
        % TODO throw error
      }
    }
  }

  \prop_get:NnN \l_stex_current_copymodule_prop { includes } \l_tmpa_seq
  \seq_put_right:Nx \l_tmpa_seq {\l_stex_import_ns_str ?\l_stex_import_name_str }
  \prop_put:Nnx \l_stex_current_copymodule_prop {includes} \l_tmpa_seq
}

\NewDocumentCommand \assign { m m }{
  \stex_get_symbol_in_copymodule:n {#1}
  \stex_debug:nn{assign}{defining~{\l_stex_get_symbol_uri_str}~as~\detokenize{#2}}
  \tl_set:cn {l_@@_copymodule_##1?####1_def_tl}{#2}
}

\keys_define:nn { stex / renamedecl } {
  name        .str_set_x:N  = \l_stex_renamedecl_name_str
}
\cs_new_protected:Nn \_@@_renamedecl_args:n {
  \str_clear:N \l_stex_renamedecl_name_str
  
  \keys_set:nn { stex / renamedecl } { #1 }
}

\NewDocumentCommand \renamedecl { O{} m m}{
  \_@@_renamedecl_args:n { #1 }
  \stex_get_symbol_in_copymodule:n {#2}
  \stex_debug:nn{renamedecl}{renaming~{\l_stex_get_symbol_uri_str}~to~#3}
  \str_set:cx {l_@@_copymodule_\l_stex_get_symbol_uri_str _macroname_str}{#3}
  \str_if_empty:NTF \l_stex_renamedecl_name_str {
    \tl_set:cx { #3 }{ \stex_invoke_symbol:n {
      \l_stex_get_symbol_uri_str
    } }
  } {
    \str_set:cx {l_@@_copymodule_\l_stex_get_symbol_uri_str _name_str}{\l_stex_renamedecl_name_str}
    \stex_debug:nn{renamedecl}{@~\l_stex_current_module_str ? \l_stex_renamedecl_name_str}
    \prop_set_eq:cc {l_stex_symdecl_ 
      \l_stex_current_module_str ? \l_stex_renamedecl_name_str
      _prop
    }{l_stex_symdecl_ \l_stex_get_symbol_uri_str _prop}
    \seq_set_eq:cc {l_stex_symdecl_ 
      \l_stex_current_module_str ? \l_stex_renamedecl_name_str
      _notations
    }{l_stex_symdecl_ \l_stex_get_symbol_uri_str _notations}
    \prop_put:cnx {l_stex_symdecl_ 
      \l_stex_current_module_str ? \l_stex_renamedecl_name_str
      _prop
    }{ name }{ \l_stex_renamedecl_name_str }
    \prop_put:cnx {l_stex_symdecl_ 
      \l_stex_current_module_str ? \l_stex_renamedecl_name_str
      _prop
    }{ module }{ \l_stex_current_module_str }
    \exp_args:NNx \seq_put_left:Nn \l_@@_copymodule_fields_seq {
      \l_stex_current_module_str ? \l_stex_renamedecl_name_str
    }
    \tl_set:cx { #3 }{ \stex_invoke_symbol:n {
      \l_stex_current_module_str ? \l_stex_renamedecl_name_str
    } }
  }
}
%\NewDocumentCommand \notation_in_copymodules: { O{} m } {
%  \_stex_notation_args:n { #1 }
%  \tl_clear:N \l_stex_symdecl_definiens_tl
%  \stex_get_symbol_in_copymodule:n { #2 }
%  \stex_notation_do:nn { \l_stex_get_symbol_uri_str }
%  % todo
%}
\stex_deactivate_macro:Nn \assign {copymodules}
\stex_deactivate_macro:Nn \renamedecl {copymodules}
\stex_deactivate_macro:Nn \donotcopy {copymodules}


%    \end{macrocode}
%
%
%
%    \begin{macrocode}
\seq_new:N \l_stex_implicit_morphisms_seq
\NewDocumentCommand \implicitmorphism { O{} m m}{
  \stex_import_module_uri:nn { #1 } { #2 }
  \stex_debug:nn{implicits}{
    Implicit~morphism:~
    \l_stex_module_ns_str ? \l_@@_name_str
  }
  \exp_args:NNx \seq_if_in:NnT \l_stex_all_modules_seq {
    \l_stex_module_ns_str ? \l_@@_name_str
  }{
    \msg_error:nnn{stex}{error/conflictingmodules}{
      \l_stex_module_ns_str ? \l_@@_name_str
    }
  }

  % TODO
  


  \seq_put_right:Nx \l_stex_implicit_morphisms_seq {
    \l_stex_module_ns_str ? \l_@@_name_str
  }
}

%    \end{macrocode}
%
%
% \subsection{The feature environment}
%
% \begin{environment}{structural@feature}
%    \begin{macrocode}

\NewDocumentEnvironment{structural@feature}{ m m m }{
  \stex_if_in_module:F {
    \msg_set:nnn{stex}{error/nomodule}{
      Structural~Feature~has~to~occur~in~a~module:\\
      Feature~#2~of~type~#1\\
      In~File:~\stex_path_to_string:N \g_stex_currentfile_seq
    }
    \msg_error:nn{stex}{error/nomodule}
  }

  \str_set:Nx \l_stex_module_name_str {
    \prop_item:Nn \l_stex_current_module_prop
      { name } / #2 - feature
  }
  
  \str_set:Nx \l_stex_module_ns_str {
    \prop_item:Nn \l_stex_current_module_prop
      { ns }
  }

  
  \str_clear:N \l_tmpa_str
  \seq_clear:N \l_tmpa_seq
  \tl_clear:N \l_tmpa_tl
  \exp_args:NNx \prop_set_from_keyval:Nn \l_stex_current_module_prop {
    origname  = #2,
    name      = \l_stex_module_name_str ,
    ns        = \l_stex_module_ns_str ,
    imports   = \exp_not:o { \l_tmpa_seq } ,
    constants = \exp_not:o { \l_tmpa_seq } ,
    content   = \exp_not:o { \l_tmpa_tl }  ,
    file      = \exp_not:o { \g_stex_currentfile_seq } ,
    lang      = \l_stex_module_lang_str ,
    sig       = \l_tmpa_str ,
    meta      = \l_tmpa_str ,
    feature   = #1 ,
  }

  \stex_if_smsmode:TF {
    \stex_smsmode_set_codes:
  } {
    \begin{stex_annotate_env}{ feature:#1 }{}
      \stex_annotate_invisible:nnn{header}{}{ #3 }
  }
}{  
  \str_set:Nx \l_tmpa_str {
    c_stex_feature_
    \prop_item:Nn \l_stex_current_module_prop { ns } ?
    \prop_item:Nn \l_stex_current_module_prop { name }
    _prop
  }
  \prop_gset_eq:cN { \l_tmpa_str } \l_stex_current_module_prop
  \prop_gset_eq:NN \g_stex_last_feature_prop \l_stex_current_module_prop
  \stex_if_smsmode:TF {
    \exp_args:Nx \stex_add_to_sms:n {
      \prop_gset_from_keyval:cn {
        c_stex_feature_
        \prop_item:Nn \l_stex_current_module_prop { ns } ?
        \prop_item:Nn \l_stex_current_module_prop { name }
        _prop
      } {
        origname  = #2,
        name      = \prop_item:cn { \l_tmpa_str } { name } ,
        ns        = \prop_item:cn { \l_tmpa_str } { ns } ,
        imports   = \prop_item:cn { \l_tmpa_str } { imports } ,
        constants = \prop_item:cn { \l_tmpa_str } { constants } ,
        content   = \prop_item:cn { \l_tmpa_str } { content } ,
        file      = \prop_item:cn { \l_tmpa_str } { file } ,
        lang      = \prop_item:cn { \l_tmpa_str } { lang } ,
        sig       = \prop_item:cn { \l_tmpa_str } { sig } ,
        meta      = \prop_item:cn { \l_tmpa_str } { meta } ,
        feature   = \prop_item:cn { \l_tmpa_str } { feature }
      }
    }
  } {
      \end{stex_annotate_env}
  }
}

%    \end{macrocode}
% \end{environment}
%
%
% \subsection{Features}
%
% \begin{environment}{structure}
%    \begin{macrocode}

\prop_new:N \l_stex_all_structures_prop

\keys_define:nn { stex / features / structure } {
  name         .str_set_x:N  = \l_@@_structure_name_str ,
}

\cs_new_protected:Nn \_@@_structure_args:n {
  \str_clear:N \l_@@_structure_name_str
  \keys_set:nn { stex / features / structure } { #1 }
}

%\stex_new_feature:nnnn { structure } { O{} m } {
%  \_@@_structure_args:n { ##1 }
%  \str_if_empty:NT \l_@@_structure_name_str {
%    \str_set:Nx \l_@@_structure_name_str { ##2 }
%  }
%} {
%
%}

\NewDocumentEnvironment{mathstructure}{ O{} m }{
  \_@@_structure_args:n { #1 }
  \str_if_empty:NT \l_@@_structure_name_str {
    \str_set:Nx \l_@@_structure_name_str { #2 }
  }
  \exp_args:Nnnx
  \begin{structural@feature}{ structure }
    { \l_@@_structure_name_str }{}
    \seq_clear:N \l_tmpa_seq
    \prop_put:Nno \l_stex_current_module_prop { fields } \l_tmpa_seq

}{
    \prop_get:NnN \l_stex_current_module_prop { constants } \l_tmpa_seq
    \prop_get:NnN \l_stex_current_module_prop { fields } \l_tmpb_seq
    \str_set:Nx \l_tmpa_str {
      \prop_item:Nn \l_stex_current_module_prop { ns } ?
      \prop_item:Nn \l_stex_current_module_prop { name }
    }
    \seq_map_inline:Nn \l_tmpa_seq {
      \exp_args:NNx \seq_put_right:Nn \l_tmpb_seq { \l_tmpa_str ? ##1 }
    }
    \prop_put:Nno \l_stex_current_module_prop { fields } { \l_tmpb_seq }
    \exp_args:Nnx
    \AddToHookNext { env / mathstructure / after }{
      \symdecl[type = \exp_not:N\collection,def={\STEXsymbol{module-type}{
        \_stex_term_math_oms:nnnn { \l_tmpa_str }{}{0}{}
      }}, name = \prop_item:Nn \l_stex_current_module_prop { origname }]{ #2 }
      \STEXexport {
        \prop_put:Nno \exp_not:N \l_stex_all_structures_prop 
          {\prop_item:Nn \l_stex_current_module_prop { origname }}
          {\l_tmpa_str}
          \prop_put:Nno \exp_not:N \l_stex_all_structures_prop 
            {#2}{\l_tmpa_str}
%        \seq_put_right:Nn \exp_not:N \l_stex_all_structures_seq {
%          \prop_item:Nn \l_stex_current_module_prop { origname },
%          \l_tmpa_str
%        }
%        \seq_put_right:Nn \exp_not:N \l_stex_all_structures_seq {
%          #2,\l_tmpa_str
%        }
%        \tl_set:cx { #2 } { 
%          \stex_invoke_structure:n { \l_tmpa_str }
      }
    }
    
  \end{structural@feature}
  % \g_stex_last_feature_prop
}
%    \end{macrocode}
% \end{environment}
%
%
% \begin{macro}{\instantiate}
%    \begin{macrocode}
\seq_new:N \l_@@_structure_field_seq
\str_new:N \l_@@_structure_field_str
\str_new:N \l_@@_structure_def_tl
\prop_new:N \l_@@_structure_prop
\NewDocumentCommand \instantiate { m O{} m }{
  \stex_smsmode_set_codes:
  \prop_get:NnN \l_stex_all_structures_prop {#1} \l_tmpa_str
  \prop_set_eq:Nc \l_@@_structure_prop {
    c_stex_feature_\l_tmpa_str _prop
  }
  \seq_set_from_clist:Nn \l_@@_structure_field_seq { #2 }
  \seq_map_inline:Nn \l_@@_structure_field_seq {
    \seq_set_split:Nnn \l_tmpa_seq{=}{ ##1 }
    \int_compare:nNnTF {\seq_count:N \l_tmpa_seq} > 1 {
      \seq_get_left:NN \l_tmpa_seq \l_tmpa_tl
      \exp_args:NNno \seq_set_split:Nnn \l_tmpb_seq
        {!} \l_tmpa_tl 
      \int_compare:nNnTF {\seq_count:N \l_tmpb_seq} > 1 {
        \str_set:Nx \l_@@_structure_field_str {\seq_item:Nn \l_tmpb_seq 1}
        \seq_get_right:NN \l_tmpb_seq \l_tmpb_tl
        \seq_get_right:NN \l_tmpa_seq \l_tmpa_tl
      }{
        \str_set:Nx \l_@@_structure_field_str \l_tmpa_tl
        \seq_get_right:NN \l_tmpa_seq \l_tmpa_tl
        \exp_args:NNno \seq_set_split:Nnn \l_tmpb_seq{!}
          \l_tmpa_tl 
        \int_compare:nNnTF {\seq_count:N \l_tmpb_seq} > 1 {
          \seq_get_left:NN \l_tmpb_seq \l_tmpa_tl
          \seq_get_right:NN \l_tmpb_seq \l_tmpb_tl
        }{
          \tl_clear:N \l_tmpb_tl
        }
      }
    }{
      \seq_set_split:Nnn \l_tmpa_seq{!}{ ##1 }
      \int_compare:nNnTF {\seq_count:N \l_tmpa_seq} > 1 {
        \str_set:Nx \l_@@_structure_field_str {\seq_item:Nn \l_tmpa_seq 1}
        \seq_get_right:NN \l_tmpa_seq \l_tmpb_tl
        \tl_clear:N \l_tmpa_tl
      }{
        % TODO throw error
      }
    }
    % \l_tmpa_str: name
    % \l_tmpa_tl: definiens
    % \l_tmpb_tl: notation
    \tl_if_empty:NT \l_@@_structure_field_str { 
      % TODO throw error
    }
    \str_clear:N \l_tmpb_str
    
    \prop_get:NnN \l_@@_structure_prop { fields } \l_tmpa_seq
    \seq_map_inline:Nn \l_tmpa_seq {
      \seq_set_split:Nnn \l_tmpb_seq ? { ####1 }
      \seq_get_right:NN \l_tmpb_seq \l_tmpb_str
      \str_if_eq:NNT \l_@@_structure_field_str \l_tmpb_str {
        \seq_map_break:n {
          \str_set:Nn \l_tmpb_str { ####1 }
        }
      }
    }
    \prop_get:cnN { l_stex_symdecl_ \l_tmpb_str _prop } {args}
      \l_tmpb_str

    \tl_if_empty:NTF \l_tmpb_tl {
      \tl_if_empty:NF \l_tmpa_tl {
        \exp_args:Nx \use:n {
          \symdecl[args=\l_tmpb_str,def={\exp_args:No\exp_not:n{\l_tmpa_tl}}]{#3/\l_@@_structure_field_str}
        }
      }
    }{
      \tl_if_empty:NTF \l_tmpa_tl {
        \exp_args:Nx \use:n {
          \symdef[args=\l_tmpb_str]{#3/\l_@@_structure_field_str}\exp_after:wN\exp_not:n\exp_after:wN{\l_tmpb_tl}
        }

      }{
        \exp_args:Nx \use:n {
          \symdef[args=\l_tmpb_str,def={\exp_args:No\exp_not:n{\l_tmpa_tl}}]{#3/\l_@@_structure_field_str}
          \exp_after:wN\exp_not:n\exp_after:wN{\l_tmpb_tl}
        }
      }
    }
%    \par \prop_item:Nn \l_stex_current_module_prop {ns} ?
%    \prop_item:Nn \l_stex_current_module_prop {name} ?
%    #3/\l_@@_structure_field_str
%    \par
%    \expandafter\present\csname
%      l_stex_symdecl_
%      \prop_item:Nn \l_stex_current_module_prop {ns} ?
%      \prop_item:Nn \l_stex_current_module_prop {name} ?
%      #3/\l_@@_structure_field_str
%      _prop
%    \endcsname
  }

  \tl_clear:N \l_@@_structure_def_tl

  \prop_get:NnN \l_@@_structure_prop { fields } \l_tmpa_seq
  \seq_map_inline:Nn \l_tmpa_seq {
    \seq_set_split:Nnn \l_tmpb_seq ? { ##1 }
    \seq_get_right:NN \l_tmpb_seq \l_tmpa_str
    \exp_args:Nx \use:n {
      \tl_put_right:Nn \exp_not:N \l_@@_structure_def_tl {

      }
    }

    \prop_if_exist:cF {
      l_stex_symdecl_
      \prop_item:Nn \l_stex_current_module_prop {ns} ?
      \prop_item:Nn \l_stex_current_module_prop {name} ?
      #3/\l_tmpa_str
      _prop
    }{
      \prop_get:cnN { l_stex_symdecl_ ##1 _prop } {args}
        \l_tmpb_str
      \exp_args:Nx \use:n {
        \symdecl[args=\l_tmpb_str]{#3/\l_tmpa_str}
      }
    }
  }

  \symdecl*[type={\STEXsymbol{module-type}{
    \_stex_term_math_oms:nnnn {
      \prop_item:Nn \l_@@_structure_prop {ns} ?
      \prop_item:Nn \l_@@_structure_prop {name}
      }{}{0}{}
  }}]{#3}
  
  % TODO: -> sms file

  \tl_set:cx{ #3 }{
    \stex_invoke_structure:nnn {
      \prop_item:Nn \l_stex_current_module_prop {ns} ?
      \prop_item:Nn \l_stex_current_module_prop {name} ? #3
    } {
      \prop_item:Nn \l_@@_structure_prop {ns} ?
      \prop_item:Nn \l_@@_structure_prop {name}
    }
  }

}
%    \end{macrocode}
% \end{macro}
%
%
%
% \begin{macro}{\stex_invoke_structure:nnn}
%    \begin{macrocode}
% #1: URI of the instance
% #2: URI of the instantiated module
\cs_new_protected:Nn \stex_invoke_structure:nnn {
  \tl_if_empty:nTF{ #3 }{  
    \prop_set_eq:Nc \l_@@_structure_prop {
      c_stex_feature_ #2 _prop
    }
    \tl_clear:N \l_tmpa_tl
    \prop_get:NnN \l_@@_structure_prop { fields } \l_tmpa_seq
    \seq_map_inline:Nn \l_tmpa_seq {
      \seq_set_split:Nnn \l_tmpb_seq ? { ##1 }
      \seq_get_right:NN \l_tmpb_seq \l_tmpa_str
      \cs_if_exist:cT {
        stex_notation_ #1/\l_tmpa_str \c_hash_str\c_hash_str _cs
      }{
        \tl_if_empty:NF \l_tmpa_tl {
          \tl_put_right:Nn \l_tmpa_tl {,}
        }
        \tl_put_right:Nx \l_tmpa_tl {
          \stex_invoke_symbol:n {#1/\l_tmpa_str}!
        }
      }
    }
    \exp_args:No \mathstruct \l_tmpa_tl
  }{
    \stex_invoke_symbol:n{#1/#3}
  }
}
%    \end{macrocode}
% \end{macro}
%
%    \begin{macrocode}
%</package>
%    \end{macrocode}
%
% \end{implementation}
%
% \PrintIndex

  \begin{omgroup}{Primitive Symbols (The \sTeX Metatheory)}
    % \iffalse meta-comment
% An Infrastructure for Semantic Macros and Module Scoping
% Copyright (c) 2019 Michael Kohlhase, all rights reserved
%                this file is released under the
%                LaTeX Project Public License (LPPL)
% 
% The original of this file is in the public repository at 
% http://github.com/sLaTeX/sTeX/
%
% TODO update copyright  
%
%<*driver>
\providecommand\bibfolder{../../lib/bib}
\RequirePackage{paralist}
\documentclass[full,kernel]{l3doc}
\usepackage[dvipsnames]{xcolor}
\usepackage[utf8]{inputenc}
\usepackage[T1]{fontenc}
%\usepackage{document-structure}
\usepackage[showmods,debug=all,lang={de,en}, mathhub=./tests]{stex}
\usepackage{url,array,float,textcomp}
\usepackage[show]{ed}
\usepackage[hyperref=auto,style=alphabetic]{biblatex}
\addbibresource{\bibfolder/kwarcpubs.bib}
\addbibresource{\bibfolder/extpubs.bib}
\addbibresource{\bibfolder/kwarccrossrefs.bib}
\addbibresource{\bibfolder/extcrossrefs.bib}
\usepackage{amssymb}
\usepackage{amsfonts}
\usepackage{xspace}
\usepackage{hyperref}

\usepackage{morewrites}


\makeindex
\floatstyle{boxed}
\newfloat{exfig}{thp}{lop}
\floatname{exfig}{Example}

\usepackage{stex-tests}

\MakeShortVerb{\|}

\def\scsys#1{{{\sc #1}}\index{#1@{\sc #1}}\xspace}
\def\mmt{\textsc{Mmt}\xspace}
\def\xml{\scsys{Xml}}
\def\mathml{\scsys{MathML}}
\def\omdoc{\scsys{OMDoc}}
\def\openmath{\scsys{OpenMath}}
\def\latexml{\scsys{LaTeXML}}
\def\perl{\scsys{Perl}}
\def\cmathml{Content-{\sc MathML}\index{Content {\sc MathML}}\index{MathML@{\sc MathML}!content}}
\def\activemath{\scsys{ActiveMath}}
\def\twin#1#2{\index{#1!#2}\index{#2!#1}}
\def\twintoo#1#2{{#1 #2}\twin{#1}{#2}}
\def\atwin#1#2#3{\index{#1!#2!#3}\index{#3!#2 (#1)}}
\def\atwintoo#1#2#3{{#1 #2 #3}\atwin{#1}{#2}{#3}}
\def\cT{\mathcal{T}}\def\cD{\mathcal{D}}

\def\fileversion{3.0}
\def\filedate{\today}

\RequirePackage{pdfcomment}

\ExplSyntaxOn\makeatletter
\cs_set_protected:Npn \@comp #1 #2 {
  \pdftooltip {
    \textcolor{blue}{#1}
  } { #2 }
}

\cs_set_protected:Npn \@defemph #1 #2 {
  \pdftooltip { 
    \textbf{\textcolor{magenta}{#1}}
  } { #2 }
}
\makeatother\ExplSyntaxOff

\begin{document}
  \DocInput{\jobname.dtx}
\end{document}
%</driver>
% \fi
%
% \title{ \sTeX-Metatheory
% 	\thanks{Version {\fileversion} (last revised {\filedate})} 
% }
%
% \author{Michael Kohlhase, Dennis Müller\\
% 	FAU Erlangen-Nürnberg\\
% 	\url{http://kwarc.info/}
% }
%
% \maketitle
%
%\ifinfulldoc\else
% This is the documentation for the \pkg{stex-metatheory} package.
% For a more high-level introduction, 
%  see \href{\basedocurl/manual.pdf}{the \sTeX Manual} or the
% \href{\basedocurl/stex.pdf}{full \sTeX documentation}.
%
% \textcolor{red}{TODO: metatheory documentation}
% \fi
%
% \begin{documentation}\label{pkg:metatheory:doc}
%
% The default meta theory for an \sTeX module. Contains
% symbols so ubiquitous, that it is virtually impossible
% to describe any flexiformal content without them, or
% that are required to annotate even the most primitive symbols
% with meaningful (foundation-independent) ``type''-annotations,
% or required for basic structuring principles (theorems, definitions).
%
% Foundations should ideally instantiate these symbols
% with their formal counterparts, e.g. |isa| corresponds
% to a typing operation in typed setting, or the $\in$-operator
% in set-theoretic contexts; |bind| corresponds to a universal
% quantifier in ($n$th-order) logic, or a $\Pi$ in dependent type
% theories.
%
% \section{Symbols}\label{pkg:metatheory:symbols}
%
% \end{documentation}
%
% \begin{implementation}\label{pkg:metatheory:impl}
%
% \section{\sTeX-Metatheory Implementation}
%
%    \begin{macrocode}
%<*package>
%<@@=stex_modules>

%%%%%%%%%%%%%   metatheory.dtx   %%%%%%%%%%%%%

\str_const:Nn \c_stex_metatheory_ns_str {http://mathhub.info/sTeX}
\begingroup
\stex_module_setup:nn{
  ns=\c_stex_metatheory_ns_str,
  meta=NONE
}{Metatheory}
\stex_reactivate_macro:N \symdecl
\stex_reactivate_macro:N \notation
\stex_reactivate_macro:N \symdef
\ExplSyntaxOff
\csname stex_suppress_html:n\endcsname{
  % is-a (a:A, a \in A, a is an A, etc.)
  \symdecl[args=ai]{isa}
  \notation[typed]{isa}{#1 \comp{:} #2}{##1 \comp, ##2}
  \notation[in]{isa}{#1 \comp\in #2}{##1 \comp, ##2}
  \notation[pred]{isa}{#2\comp(#1 \comp)}{##1 \comp, ##2}

  % bind (\forall, \Pi, \lambda etc.)
  \symdecl[args=Bi]{bind}
  \notation[forall]{bind}{\comp\forall #1.\;#2}{##1 \comp, ##2}
  \notation[Pi]{bind}{\comp\prod_{#1}#2}{##1 \comp, ##2}
  \notation[depfun]{bind}{\comp( #1 \comp{)\;\to\;} #2}{##1 \comp, ##2}

  % dummy variable
  \symdecl{dummyvar}
  \notation[underscore]{dummyvar}{\comp\_}
  \notation[dot]{dummyvar}{\comp\cdot}
  \notation[dash]{dummyvar}{\comp{{\rm --}}}

  %fromto (function space, Hom-set, implication etc.)
  \symdecl[args=ai]{fromto}
  \notation[xarrow]{fromto}{#1 \comp\to #2}{##1 \comp\times ##2}
  \notation[arrow]{fromto}{#1 \comp\to #2}{##1 \comp\to ##2}

  % mapto (lambda etc.)
  %\symdecl[args=Bi]{mapto}
  %\notation[mapsto]{mapto}{#1 \comp\mapsto #2}{#1 \comp, #2}
  %\notation[lambda]{mapto}{\comp\lambda #1 \comp.\; #2}{#1 \comp, #2}
  %\notation[lambdau]{mapto}{\comp\lambda_{#1} \comp.\; #2}{#1 \comp, #2}

  % function/operator application
  \symdecl[args=ia]{apply}
  \notation[prec=0;0x\infprec,parens]{apply}{#1 \comp( #2 \comp)}{##1 \comp, ##2}
  \notation[prec=0;0x\infprec,lambda]{apply}{#1 \; #2 }{##1 \; ##2}

  % ``type'' of all collections (sets,classes,types,kinds)
  \symdecl{collection}
  \notation[U]{collection}{\comp{\mathcal{U}}}
  \notation[set]{collection}{\comp{\textsf{Set}}}

  % sequences
  \symdecl[args=1]{seqtype}
  \notation[kleene]{seqtype}{#1^{\comp\ast}}

  \symdef[args=2,li,prec=nobrackets]{sequence-index}{{#1}_{#2}}
  \notation[ui,prec=nobrackets]{sequence-index}{{#1}^{#2}}

  %\symdef[args=3,li]{sequence-from-to}{#1_{#2}\comp{,\ellipses,}#1_{#3}}
  %\notation[ui]{sequence-from-to}{#1^{#2}\comp{,\ellipses,}#1^{#3}}
  % ^ superceded by \aseqfromto and \livar/\uivar

  \symdef[args=a,prec=nobrackets]{aseqdots}{#1\comp{,\ellipses}}{##1\comp,##2}
  \symdef[args=ai,prec=nobrackets]{aseqfromto}{#1\comp{,\ellipses,}#2}{##1\comp,##2}
  \symdef[args=aii,prec=nobrackets]{aseqfromtovia}{#1\comp{,\ellipses,}#2\comp{,\ellipses,}#3}{##1\comp,##2}

  % letin (``let'', local definitions, variable substitution)
  \symdecl[args=bii]{letin}
  \notation[let]{letin}{\comp{{\rm let}}\;#1\comp{=}#2\;\comp{{\rm in}}\;#3}
  \notation[subst]{letin}{#3 \comp[ #1 \comp/ #2 \comp]}
  \notation[frac]{letin}{#3 \comp[ \frac{#2}{#1} \comp]}

  % structures
  \symdecl*[args=1]{module-type}
  \notation{module-type}{\mathtt{MOD} #1}
  \symdecl[name=mathematical-structure,args=a]{mathstruct} % TODO
  \notation[angle,prec=nobrackets]{mathstruct}{\comp\langle #1 \comp\rangle}{##1 \comp, ##2}

}
  \ExplSyntaxOn
  \stex_add_to_current_module:n{
    \let\nappa\apply
    \def\nappli#1#2#3#4{\apply{#1}{\naseqli{#2}{#3}{#4}}}
    \def\nappui#1#2#3#4{\apply{#1}{\nasequi{#2}{#3}{#4}}}
    \def\livar{\csname sequence-index\endcsname[li]}
    \def\uivar{\csname sequence-index\endcsname[ui]}
    \def\naseqli#1#2#3{\aseqfromto{\livar{#1}{#2}}{\livar{#1}{#3}}}
    \def\nasequi#1#2#3{\aseqfromto{\uivar{#1}{#2}}{\uivar{#1}{#3}}}
    \def\nappe#1#2#3{\apply{#1}{\aseqfromto{#2}{#3}}}
  }
\_@@_end_module:
\endgroup
%    \end{macrocode}
%
%
%    \begin{macrocode}
%</package>
%    \end{macrocode}
%
% \end{implementation}
%
% \PrintIndex

% \endinput
% Local Variables:
% mode: doctex
% TeX-master: t
% End:

  \end{omgroup}
\end{omgroup}

\begin{omgroup}{\sTeX Statements (Definitions, Theorems, Examples, ...)}
  % \iffalse meta-comment
% An Infrastructure for Semantic Macros and Module Scoping
% Copyright (c) 2019 Michael Kohlhase, all rights reserved
%                this file is released under the
%                LaTeX Project Public License (LPPL)
% 
% The original of this file is in the public repository at 
% http://github.com/sLaTeX/sTeX/
%
% TODO update copyright  
%
%<*driver>
\providecommand\bibfolder{../../lib/bib}
\RequirePackage{paralist}
\documentclass[full,kernel]{l3doc}
\usepackage[dvipsnames]{xcolor}
\usepackage[utf8]{inputenc}
\usepackage[T1]{fontenc}
%\usepackage{document-structure}
\usepackage[showmods,debug=all,lang={de,en}, mathhub=./tests]{stex}
\usepackage{url,array,float,textcomp}
\usepackage[show]{ed}
\usepackage[hyperref=auto,style=alphabetic]{biblatex}
\addbibresource{\bibfolder/kwarcpubs.bib}
\addbibresource{\bibfolder/extpubs.bib}
\addbibresource{\bibfolder/kwarccrossrefs.bib}
\addbibresource{\bibfolder/extcrossrefs.bib}
\usepackage{amssymb}
\usepackage{amsfonts}
\usepackage{xspace}
\usepackage{hyperref}

\usepackage{morewrites}


\makeindex
\floatstyle{boxed}
\newfloat{exfig}{thp}{lop}
\floatname{exfig}{Example}

\usepackage{stex-tests}

\MakeShortVerb{\|}

\def\scsys#1{{{\sc #1}}\index{#1@{\sc #1}}\xspace}
\def\mmt{\textsc{Mmt}\xspace}
\def\xml{\scsys{Xml}}
\def\mathml{\scsys{MathML}}
\def\omdoc{\scsys{OMDoc}}
\def\openmath{\scsys{OpenMath}}
\def\latexml{\scsys{LaTeXML}}
\def\perl{\scsys{Perl}}
\def\cmathml{Content-{\sc MathML}\index{Content {\sc MathML}}\index{MathML@{\sc MathML}!content}}
\def\activemath{\scsys{ActiveMath}}
\def\twin#1#2{\index{#1!#2}\index{#2!#1}}
\def\twintoo#1#2{{#1 #2}\twin{#1}{#2}}
\def\atwin#1#2#3{\index{#1!#2!#3}\index{#3!#2 (#1)}}
\def\atwintoo#1#2#3{{#1 #2 #3}\atwin{#1}{#2}{#3}}
\def\cT{\mathcal{T}}\def\cD{\mathcal{D}}

\def\fileversion{3.0}
\def\filedate{\today}

\RequirePackage{pdfcomment}

\ExplSyntaxOn\makeatletter
\cs_set_protected:Npn \@comp #1 #2 {
  \pdftooltip {
    \textcolor{blue}{#1}
  } { #2 }
}

\cs_set_protected:Npn \@defemph #1 #2 {
  \pdftooltip { 
    \textbf{\textcolor{magenta}{#1}}
  } { #2 }
}
\makeatother\ExplSyntaxOff

\begin{document}
  \DocInput{\jobname.dtx}
\end{document}
%</driver>
% \fi
%
% \title{ \sTeX-Statements
% 	\thanks{Version {\fileversion} (last revised {\filedate})} 
% }
%
% \author{Michael Kohlhase, Dennis Müller\\
% 	FAU Erlangen-Nürnberg\\
% 	\url{http://kwarc.info/}
% }
%
% \maketitle
%
% \begin{documentation}\label{pkg:statements:doc}
%
% Code related to statements, e.g. definitions, theorems
%
% \section{Macros and Environments}\label{pkg:statements:doc:macros}
%
% \begin{environment}{symboldoc}
%    \begin{syntax} \cs{begin}\Arg{symboldoc}\Arg{symbols} \meta{text} \cs{end}\Arg{symboldoc} \end{syntax}
%  Declares \meta{text} to be a (natural language, encyclopaedic) description
% of \Arg{symbols} (a comma separated list of symbol identifiers).
% \end{environment}
%
% \end{documentation}
%
% \begin{implementation}\label{pkg:statements:impl}
%
% \section{\sTeX-Statements Implementation}
%
%    \begin{macrocode}
%<*package>

%%%%%%%%%%%%%   features.dtx   %%%%%%%%%%%%%

\protected\def\ignorespacesandpars{
  \begingroup\catcode13=10\relax
  \@ifnextchar\par{
    \endgroup\expandafter\ignorespacesandpars\@gobble
  }{
    \endgroup
  }
}

%<@@=stex_statements>
%    \end{macrocode}
%
% Warnings and error messages
%
%    \begin{macrocode}

%    \end{macrocode}
%
%    \begin{macrocode}
\def\titleemph#1{\textbf{#1}}
%    \end{macrocode}
%
% \begin{environment}{symboldoc}
%    \begin{macrocode}
\NewDocumentEnvironment{symboldoc}{ m }{
  \seq_set_split:Nnn \l_tmpa_seq , { #1 }
  \seq_clear:N \l_tmpb_seq
  \seq_map_inline:Nn \l_tmpa_seq {
    \str_if_eq:nnF{ ##1 }{}{
      \stex_get_symbol:n { ##1 }
      \exp_args:NNo \seq_put_right:Nn \l_tmpb_seq {
        \l_stex_get_symbol_uri_str
      }
    }
  }
  \par
  \exp_args:Nnnx
  \begin{stex_annotate_env}{symboldoc}{\seq_use:Nn \l_tmpb_seq {,}}
}{
  \end{stex_annotate_env}
}
%    \end{macrocode}
% \end{environment}
%
%    \begin{macrocode}
\seq_new:N \g_stex_statements_patched_seq

\cs_new_protected:Nn \stex_statements_set_patched:n {
  \seq_put_right:Nn \g_stex_statements_patched_seq {#1}
}

\cs_new_protected:Nn \stex_statements_patch:nn {
  \seq_if_in:NnF \g_stex_statements_patched_seq {#1} {
    \AddToHook{begindocument}{
      \cs_if_exist:cTF{end#1}{
        \AddToHook{env/#1/before}[stex]{\use:c{_@@_#2_begin:n}{}}
        \AddToHook{env/#1/after}[stex]{\use:c{_@@_#2_end:}}
      }{
        \NewDocumentEnvironment{#1}{O{}}{
          \use:c{_@@_#2_begin:n}{}
        }{
          \use:c{_@@_#2_end:}
        }
      }
    }
  }
}
%    \end{macrocode}
%
%
% \subsubsection{Definitions}
%
% \begin{environment}{definition}
%    \begin{macrocode}
\keys_define:nn {stex / definiendum }{
  post    .tl_set:N     = \l_@@_definiendum_post_tl,
  root    .str_set_x:N  = \l_@@_definiendum_root_str,
  gfa     .str_set_x:N  = \l_@@_definiendum_gfa_str
}
\cs_new_protected:Nn \_@@_definiendum_args:n {
  \str_clear:N \l_@@_definiendum_root_str
  \tl_clear:N \l_@@_definiendum_post_tl
  \str_clear:N \l_@@_definiendum_gfa_str
  \keys_set:nn { stex / definiendum }{ #1 }
}
\NewDocumentCommand \definiendum { O{} m m} {
  \_@@_definiendum_args:n { #1 }
  \stex_get_symbol:n { #2 }
  \stex_ref_new_sym_target:n \l_stex_get_symbol_uri_str
  \str_if_empty:NTF \l_@@_definiendum_root_str {
    \tl_if_empty:NTF \l_@@_definiendum_post_tl {
      \tl_set:Nn \l_tmpa_tl { #3 }
    } {
      \str_set:Nx \l_@@_definiendum_root_str { #3 }
      \tl_set:Nn \l_tmpa_tl {
        \l_@@_definiendum_root_str\l_@@_definiendum_post_tl
       }
    }
  } {
    \tl_set:Nn \l_tmpa_tl { #3 }
  }

  % TODO root
  \rustex_if:TF {
    \stex_annotate:nnn { definiendum } { \l_stex_get_symbol_uri_str } { \l_tmpa_tl }
  } {
    \exp_args:Nnx \defemph@uri { \l_tmpa_tl } { \l_stex_get_symbol_uri_str }
  }
}
\stex_deactivate_macro:Nn \definiendum {definition~environments}

\NewDocumentCommand \definame { O{} m } {
  \_@@_definiendum_args:n { #1 }
  % TODO: root
  \stex_get_symbol:n { #2 }
  \stex_ref_new_sym_target:n \l_stex_get_symbol_uri_str
  \str_set:Nx \l_tmpa_str {
    \prop_item:cn { g_stex_symdecl_ \l_stex_get_symbol_uri_str _prop } { name }
  }
  \exp_args:NNno \str_replace_all:Nnn \l_tmpa_str {-} {~}
  \rustex_if:TF {
    \stex_annotate:nnn { definiendum } { \l_stex_get_symbol_uri_str } { 
      \l_tmpa_str\l_@@_definiendum_post_tl
      }
  } {
    \defemph@uri {
      \l_tmpa_str\l_@@_definiendum_post_tl
    } { \l_stex_get_symbol_uri_str }
  }
}
\stex_deactivate_macro:Nn \definame {definition~environments}

\cs_new_protected:Nn \_@@_defi_begin:n {
  \stex_reactivate_macro:N \definiendum
  \stex_reactivate_macro:N \definame
  \seq_set_split:Nnn \l_tmpa_seq , { #1 }
  \seq_clear:N \l_tmpb_seq
  \seq_map_inline:Nn \l_tmpa_seq {
    \str_if_eq:nnF{ ##1 }{}{
      \stex_get_symbol:n { ##1 }
      \exp_args:NNo \seq_put_right:Nn \l_tmpb_seq {
        \l_stex_get_symbol_uri_str
      }
    }
  }
  \stex_smsmode_set_codes:
  \exp_args:Nnnx
  \begin{stex_annotate_env}{definition}{\seq_use:Nn \l_tmpb_seq {,}}
}

\cs_new_protected:Nn \_@@_defi_end: {
  \end{stex_annotate_env}
}
%    \end{macrocode}
% \end{environment}
%
%
% Hook:
%
%    \begin{macrocode}
\stex_statements_patch:nn{definition}{defi}
%    \end{macrocode}
%
% inline:
%    \begin{macrocode}
\NewDocumentCommand \inlinedef { m } {
  \begingroup
  \stex_reactivate_macro:N \definiendum
  \stex_reactivate_macro:N \definame
  \stex_ref_new_doc_target:n{}
  #1
  \endgroup
}
%    \end{macrocode}
%
% \subsubsection{Assertions}
%
% \begin{environment}{assertion}
%    \begin{macrocode}
\cs_new_protected:Nn \_@@_assertion_begin:n {
  \seq_set_split:Nnn \l_tmpa_seq , { #1 }
  \seq_clear:N \l_tmpb_seq
  \seq_map_inline:Nn \l_tmpa_seq {
    \str_if_eq:nnF{ ##1 }{}{
      \stex_get_symbol:n { ##1 }
      \exp_args:NNo \seq_put_right:Nn \l_tmpb_seq {
        \l_stex_get_symbol_uri_str
      }
    }
  }
  \titleemph{Assertion}~
  \stex_smsmode_set_codes:
  \exp_args:Nnnx
  \begin{stex_annotate_env}{assertion}{\seq_use:Nn \l_tmpb_seq {,}}
}

\cs_new_protected:Nn \_@@_assertion_end: {
  \end{stex_annotate_env}
}
%    \end{macrocode}
% \end{environment}
%
%
% Hook:
%
%    \begin{macrocode}
\stex_statements_patch:nn{assertion}{assertion}
%    \end{macrocode}
%
% inline:
%    \begin{macrocode}
\NewDocumentCommand \inlineass { m } {
  \begingroup
  \stex_ref_new_doc_target:n{}
  #1
  \endgroup
}
%    \end{macrocode}
%
% \begin{environment}{theorem}
%    \begin{macrocode}
\cs_new_protected:Nn \_@@_theorem_begin:n {
  \seq_set_split:Nnn \l_tmpa_seq , { #1 }
  \seq_clear:N \l_tmpb_seq
  \seq_map_inline:Nn \l_tmpa_seq {
    \str_if_eq:nnF{ ##1 }{}{
      \stex_get_symbol:n { ##1 }
      \exp_args:NNo \seq_put_right:Nn \l_tmpb_seq {
        \l_stex_get_symbol_uri_str
      }
    }
  }
  \titleemph{Theorem}~
  \stex_smsmode_set_codes:
  \exp_args:Nnnx
  \begin{stex_annotate_env}{assertion}{\seq_use:Nn \l_tmpb_seq {,}}
}

\cs_new_protected:Nn \_@@_theorem_end: {
  \end{stex_annotate_env}
}
%    \end{macrocode}
% \end{environment}
%
%
% Hook:
%
%    \begin{macrocode}
\stex_statements_patch:nn{theorem}{theorem}
%    \end{macrocode}
%
% \begin{environment}{lemma}
%    \begin{macrocode}
\cs_new_protected:Nn \_@@_lemma_begin:n {
  \seq_set_split:Nnn \l_tmpa_seq , { #1 }
  \seq_clear:N \l_tmpb_seq
  \seq_map_inline:Nn \l_tmpa_seq {
	\str_if_eq:nnF{ ##1 }{}{
      \stex_get_symbol:n { ##1 }
      \exp_args:NNo \seq_put_right:Nn \l_tmpb_seq {
        \l_stex_get_symbol_uri_str
      }
    }
  }
  \titleemph{Lemma}~
  \stex_smsmode_set_codes:
  \exp_args:Nnnx
  \begin{stex_annotate_env}{assertion}{\seq_use:Nn \l_tmpb_seq {,}}
}

\cs_new_protected:Nn \_@@_lemma_end: {
  \end{stex_annotate_env}
}
%    \end{macrocode}
% \end{environment}
%
%
% Hook:
%
%    \begin{macrocode}
\stex_statements_patch:nn{lemma}{lemma}
%    \end{macrocode}
%
% \begin{environment}{axiom}
%    \begin{macrocode}
\cs_new_protected:Nn \_@@_axiom_begin:n {
  \seq_set_split:Nnn \l_tmpa_seq , { #1 }
  \seq_clear:N \l_tmpb_seq
  \seq_map_inline:Nn \l_tmpa_seq {  	
    \str_if_eq:nnF{ ##1 }{}{
      \stex_get_symbol:n { ##1 }
      \exp_args:NNo \seq_put_right:Nn \l_tmpb_seq {
        \l_stex_get_symbol_uri_str
      }
    }
  }
  \titleemph{Axiom}~
  \stex_smsmode_set_codes:
  \exp_args:Nnnx
  \begin{stex_annotate_env}{assertion}{\seq_use:Nn \l_tmpb_seq {,}}
}

\cs_new_protected:Nn \_@@_axiom_end: {
  \end{stex_annotate_env}
}
%    \end{macrocode}
% \end{environment}
%
%
% Hook:
%
%    \begin{macrocode}
\stex_statements_patch:nn{axiom}{axiom}
%    \end{macrocode}
%
% \subsection{Examples}
%
% \begin{environment}{example}
%    \begin{macrocode}
\cs_new_protected:Nn \_@@_example_begin:n {
  \seq_set_split:Nnn \l_tmpa_seq , { #1 }
  \seq_clear:N \l_tmpb_seq
  \seq_map_inline:Nn \l_tmpa_seq {
  	\str_if_eq:nnF{ ##1 }{}{
      \stex_get_symbol:n { ##1 }
      \exp_args:NNo \seq_put_right:Nn \l_tmpb_seq {
        \l_stex_get_symbol_uri_str
      }
    }
  }
  \titleemph{Example}~
  \stex_smsmode_set_codes:
  \exp_args:Nnnx
  \begin{stex_annotate_env}{example}{\seq_use:Nn \l_tmpb_seq {,}}
}

\cs_new_protected:Nn \_@@_example_end: {
  \end{stex_annotate_env}
}
%    \end{macrocode}
% \end{environment}
%
%
% Hook:
%
%    \begin{macrocode}
\stex_statements_patch:nn{example}{example}
%    \end{macrocode}
%
% inline:
%    \begin{macrocode}
\NewDocumentCommand \inlineex { m } {
  \begingroup
  \stex_ref_new_doc_target:n{}
  #1
  \endgroup
}
%    \end{macrocode}
%
% \subsection{OMText}
%    \begin{macrocode}
\keys_define:nn { stex / omtext} {
  id      .str_set_x:N   = \l_stex_omtext_id_str , 
  title   .tl_set:N   = \l_stex_omtext_title_tl , 
  type    .tl_set_x:N   = \l_stex_omtext_type_tl ,
  for     .tl_set_x:N   = \l_stex_omtext_for_tl ,
  from    .tl_set_x:N   = \l_stex_omtext_from_tl ,
  start   .tl_set:N   = \l_stex_omtext_start_tl ,
}
\cs_new_protected:Nn \stex_omtext_args:n {
  \tl_clear:N \l_stex_omtext_title_tl
  \tl_clear:N \l_stex_omtext_start_tl
  \keys_set:nn { stex / omtext }{ #1 }
}
\newif\if@in@omtext\@in@omtextfalse
\NewDocumentEnvironment {omtext} { O{} } {
  \stex_omtext_args:n { #1 }
  \tl_if_empty:NTF \l_stex_omtext_start_tl {
    \tl_if_empty:NF \l_stex_omtext_title_tl {
      \titleemph{\l_stex_omtext_title_tl}:~
    }
  }{
    \titleemph{\l_stex_omtext_start_tl}~
  }
  \@in@omtexttrue

  \stex_ref_new_doc_target:n \l_stex_omtext_id_str
  \stex_smsmode_set_codes:
  \ignorespacesandpars
}{}
%    \end{macrocode}
%
%    \begin{macrocode}
%</package>
%    \end{macrocode}
%
% \end{implementation}
%
% \PrintIndex

% \endinput
% Local Variables:
% mode: doctex
% TeX-master: t
% End:

  \textcolor{red}{TODO: sproofs documentation}
\end{omgroup}

\begin{omgroup}{Additional Packages}
  %%
%% This is file `tikzinput.sty',
%% generated with the docstrip utility.
%%
%% The original source files were:
%%
%% stex.dtx  (with options: `tikzinput')
%% 
\ProvidesExplPackage{tikzinput}{2021/08/31}{1.9}{bla}
\RequirePackage{l3keys2e}

\keys_define:nn { tikzinput } {
  image   .bool_set:N   = \c_tikzinput_image_bool,
  image   .default:n    = false ,
}

\ProcessKeysOptions { tikzinput }

\bool_if:NTF \c_tikzinput_image_bool {
  \RequirePackage{graphicx}

  \providecommand\usetikzlibrary[]{}
  \newcommand\tikzinput[2][]{\includegraphics[#1]{#2}}
}{
  \RequirePackage{tikz}
  \RequirePackage{standalone}

  \newcommand \tikzinput [2] [] {
    \setkeys{Gin}{#1}
    \ifx \Gin@ewidth \Gin@exclamation
      \ifx \Gin@eheight \Gin@exclamation
        \input { #2 }
      \else
        \resizebox{!}{ \Gin@eheight }{
          \input { #2 }
        }
      \fi
    \else
      \ifx \Gin@eheight \Gin@exclamation
        \resizebox{ \Gin@ewidth }{!}{
          \input { #2 }
        }
      \else
        \resizebox{ \Gin@ewidth }{ \Gin@eheight }{
          \input { #2 }
        }
      \fi
    \fi
  }
}

\newcommand \ctikzinput [2] [] {
  \begin{center}
    \tikzinput [#1] {#2}
  \end{center}
}

\@ifpackageloaded{stex}{
  \RequirePackage{stex-tikzinput}
}{}


\endinput
%%
%% End of file `tikzinput.sty'.

  \begin{omgroup}{Modular Document Structuring}
    % \iffalse meta-comment
% A LaTeX Class and Package for OMDoc Document Structures
% Copyright (c) 2019 Michael Kohlhase, all rights reserved
%               this file is released under the
%               LaTeX Project Public License (LPPL)
%
% The original of this file is in the public repository at 
% http://github.com/sLaTeX/sTeX/
%
%
%<*driver>
\def\bibfolder{../../lib/bib}
\RequirePackage{paralist}
\ifcsname stexdocpath\endcsname\else\def\stexdocpath{.}\fi
\documentclass[full]{l3doc}
%\RequirePackage{document-structure}
\usepackage[hyperref=auto,style=alphabetic]{biblatex}
\usepackage[mathhub=\stexdocpath/mh,usesms]{stex}
\usepackage{stex-highlighting,stexthm}

\newif\ifhadtitle\hadtitlefalse

\def\fileversion{3.3.0}
\def\filedate{\today}
\def\stexdoctitle#1#2{\title{#1\thanks{Version {\fileversion} (last revised {\filedate})} }\def\thispkg{#2}}

\author{Michael Kohlhase, Dennis Müller\\
 	FAU Erlangen-Nürnberg\\
 	\url{http://kwarc.info/}
}

\def\stexmaketitle{\ifhadtitle\else\hadtitletrue\maketitle\fi}

\def\docmodule{
\begin{document}
  \EnableManual
  \DisableImplementation
  \DocInput{\jobname.dtx}
  \EnableImplementation
  \DisableDocumentation
  \DisableManual
  \DocInputAgain
  \clearpage
  \PrintIndex
\end{document}
}

\ExplSyntaxOn
  \bool_new:N \g_stexdoc_typeset_manual_bool
  \NewDocumentCommand \EnableManual {}{
    \bool_gset_true:N \g_stexdoc_typeset_manual_bool
  }
  \NewDocumentCommand \DisableManual {}{
    \bool_gset_false:N \g_stexdoc_typeset_manual_bool
  }
  \NewDocumentEnvironment {stexmanual} {} {
    \bool_if:NTF \g_stexdoc_typeset_manual_bool
      {\bool_set_false:N \l__codedoc_in_implementation_bool}
      {\comment}
  }{
    \bool_if:NF \g_stexdoc_typeset_manual_bool {\endcomment}
  }
\ExplSyntaxOff

%\usepackage{makeidx}
%\makeindex

%\usepackage{document-structure}
\ExplSyntaxOn
\int_new:N \l_stex_docheader_sect

\cs_new_protected:Nn \stexdoc_do_section:n {
  \int_case:nnF \l_stex_docheader_sect {
    {0}{\cs_if_exist:NTF \part {\part{#1}}{
      \int_incr:N \l_stex_docheader_sect
      \stexdoc_do_section:
    }}
    {1}{\cs_if_exist:NTF \chapter {\chapter{#1}}{
      \int_incr:N \l_stex_docheader_sect
      \stexdoc_do_section:
    }}
    {2}{\section{#1}}
    {3}{\subsection{#1}}
    {4}{\subsubsection{#1}}
  }{\paragraph{#1}}
  \int_incr:N \l_stex_docheader_sect
}
%\int_incr:N \l_stex_docheader_sect
\NewDocumentEnvironment{sfragment}{m}{
  \stexdoc_do_section:n{#1}
}{}

\cs_set_nopar:Nn \_stexdoc_do_cs:Nn {
  \stex_debug:nn{here}{\tl_to_str:n{#1},~#2}
  \cs_if_exist:cTF{s\tl_to_str:n{#2}}{
    \cs_if_exist:cTF{s\tl_to_str:n{#2}name}{
    \symref{#2-sym}{#1{#2}}
    }{#1{#2}}
  }{
    #1{#2}
  }
}
\let\my_old_cs\cs
\protected\def\cs#1{
  \_stexdoc_do_cs:Nn \my_old_cs{#1}
}
\let\my_old_cmd\cmd
\protected\def\cmd#1{
  \_stexdoc_do_cs:Nn \my_old_cmd{#1}
}

\ExplSyntaxOff

\mhinput[sTeX/Documentation]{../lib/examples.tex}

\begin{document}
  \DocInput{\jobname.dtx}
\end{document}
%</driver>
% \fi
% %^^A \changes{v0.1}{2006/1/17}{First Version}
%
% \title{document-structure: Semantic Markup for Open Mathematical Documents in {\LaTeX}
% 	\thanks{Version {\fileversion} (last revised {\filedate})} 
% }
%
% \author{Michael Kohlhase, Dennis Müller\\
% 	FAU Erlangen-Nürnberg\\
% 	\url{http://kwarc.info/}
% }
%
% \maketitle
%
%\ifinfulldoc\else
% This is the documentation for the \pkg{document-structure} package.
% For a more high-level introduction, 
%  see \href{\basedocurl/manual.pdf}{the \sTeX Manual} or the
% \href{\basedocurl/stex.pdf}{full \sTeX documentation}.
%
% \begin{sfragment}{Introduction}
The \pkg{document-structure} package supplies an infrastructure for writing {\omdoc} documents in {\LaTeX}.
This includes a simple structure sharing mechanism for \sTeX that allows to to move from
a copy-and-paste document development model to a copy-and-reference model, which
conserves space and simplifies document management. The augmented structure can be used
by MKM systems for added-value services, either directly from the \sTeX sources, or
after translation.


 The \pkg{document-structure} package supplies macros and environments that allow to label document
 fragments and to reference them later in the same document or in other documents. In
 essence, this enhances the document-as-trees model to
 documents-as-directed-acyclic-graphs (DAG) model. This structure can be used by MKM
 systems for added-value services, either directly from the \sTeX sources, or after
 translation. Currently, trans-document referencing provided by this package can only be
 used in the \sTeX collection.

 DAG models of documents allow to replace the ``Copy and Paste'' in the source document
 with a label-and-reference model where document are shared in the document source and the
 formatter does the copying during document formatting/presentation.
\end{sfragment}

\begin{sfragment}{Package Options}
The \pkg{document-structure} package accepts the following options:
\begin{center}
  \begin{tabular}{|l|p{10cm}|}\hline
    \texttt{class=\meta{name}} & load \meta{name}|.cls| instead of |article.cls|\\\hline 
    \texttt{topsect=\meta{sect}} & The top-level sectioning level; the default for
    \meta{sect} is \texttt{section}\\\hline 
  \end{tabular}
\end{center}
\end{sfragment}

\begin{sfragment}{Document Fragments}
\begin{environment}{sfragment}
  The structure of the document is given by nested |sfragment| environments. In the
  {\LaTeX} route, the |sfragment| environment is flexibly mapped to sectioning commands,
  inducing the proper sectioning level from the nesting of |sfragment|
  environments. Correspondingly, the |sfragment| environment takes an optional key/value
  argument for metadata followed by a regular argument for the (section) title of the
  sfragment. The optional metadata argument has the keys |id| for an identifier,
  |creators| and |contributors| for the Dublin Core metadata~\cite{DCMI:dmt03}. The option
  |short| allows to give a short title for the generated section. If the title contains
  semantic macros, they need to be protected by |\protect|\ednote{MK: still?}, and we need
  to give the |loadmodules| key it needs no value. For instance we would have
\begin{latexcode}
\begin{smodule}{foo}
  \symdef{bar}{B^a_r}
   ...
   \begin{sfragment}[id=sec.barderiv,loadmodules]
     {Introducing $\protect\bar$ Derivations}
\end{latexcode}

\sTeX automatically computes the sectioning level, from the nesting of |sfragment|
environments.
\end{environment}

But sometimes, we want to skip levels (e.g. to use a |\subsection*| as an introduction for
a chapter).

\begin{environment}{blindfragment}
  Therefore the \pkg{document-structure} package provides a variant |blindfragment| that
  does not produce markup, but increments the sectioning level and logically groups
  document parts that belong together, but where traditional document markup relies on
  convention rather than explicit markup. The |blindfragment| environment is useful
  e.g. for creating frontmatter at the correct level. The example below shows a typical
  setup for the outer document structure of a book with parts and chapters.
  
\begin{latexcode}
\begin{document}
\begin{blindfragment}
\begin{blindfragment}
\begin{frontmatter}
\maketitle\newpage
\begin{sfragment}{Preface}
... <<preface>> ...
\end{sfragment}
\clearpage\setcounter{tocdepth}{4}\tableofcontents\clearpage
\end{frontmatter}
\end{blindfragment}
... <<introductory remarks>> ...
\end{blindfragment}
\begin{sfragment}{Introduction}
... <<intro>> ...
\end{sfragment}
... <<more chapters>> ... 
\bibliographystyle{alpha}\bibliography{kwarc}
\end{document}
\end{latexcode}

Here we use two levels of |blindfragment|:
\begin{itemize}
\item The outer one groups the introductory parts of the book (which we assume to have a
  sectioning hierarchy topping at the part level). This |blindfragment| makes sure that
  the introductory remarks become a ``chapter'' instead of a ``part''.
\item The inner one groups the frontmatter\footnote{We shied away from redefining the
    |frontmatter| to induce a blindfragment, but this may be the ``right'' way to go in
    the future.} and makes the preface of the book a section-level construct.\ednote{MK:
    We need a substitute for the ``Note that here the |display=flow| on the |sfragment|
    environment prevents numbering as is traditional for prefaces.''}
\end{itemize}
\end{environment}

\begin{function}{\skipfragment}
  The |\skipfragment| ``skips an |sfragment|'', i.e. it just steps the respective sectioning
  counter. This macro is useful, when we want to keep two documents in sync structurally,
  so that section numbers match up: Any section that is left out in one becomes a
  |\skipfragment|.
\end{function}

\begin{function}{\currentsectionlevel,\CurrentSectionLevel}
  The |\currentsectionlevel| macro supplies the name of the current sectioning level,
  e.g. ``chapter'', or ``subsection''. |\CurrentSectionLevel| is the capitalized
  variant. They are useful to write something like ``In this |\currentsectionlevel|, we
  will\ldots'' in an |sfragment| environment, where we do not know which sectioning level we
  will end up.
\end{function}
\end{sfragment}  

\begin{sfragment}{Ending Documents Prematurely}
\begin{function}{\prematurestop,\afterprematurestop}
  For prematurely stopping the formatting of a document, \sTeX provides the
  |\prematurestop| macro. It can be used everywhere in a document and ignores all input
  after that -- backing out of the |sfragment| environments as needed. After that -- and
  before the implicit |\end{document}| it calls the internal |\afterprematurestop|, which
  can be customized to do additional cleanup or e.g. print the bibliography.

  |\prematurestop| is useful when one has a driver file, e.g. for a course taught multiple
  years and wants to generate course notes up to the current point in the lecture. Instead
  of commenting out the remaining parts, one can just move the |\prematurestop| macro.
  This is especially useful, if we need the rest of the file for processing, e.g. to
  generate a theory graph of the whole course with the already-covered parts marked up as
  an overview over the progress; see |import_graph.py| from the |lmhtools|
  utilities~\cite{lmhtools:github:on}.
\end{function}

Text fragments and modules can be made more re-usable by the use of global variables. For
instance, the admin section of a course can be made course-independent (and therefore
re-usable) by using variables (actually token registers) |courseAcronym| and |courseTitle|
instead of the text itself. The variables can then be set in the \sTeX preamble of the
course notes file.
\end{sfragment}

\begin{sfragment}{Global Document Variables}
  To make document fragments more reusable, we sometimes want to make the content depend
  on the context. We use \defemph{document variables} for that.

\begin{function}{\setSGvar,\useSGvar}
  |\setSGvar{|\meta{vname}|}{|\meta{text}|}| to set the global variable \meta{vname} to
  \meta{text} and |\useSGvar{|\meta{vname}|}| to reference it.
\end{function}
  
\begin{function}{\ifSGvar}
  With|\ifSGvar| we can test for the contents of a global variable: the macro call
  |\ifSGvar{|\meta{vname}|}{|\meta{val}|}{|\meta{ctext}|}| tests the content of the global
  variable \meta{vname}, only if (after expansion) it is equal to \meta{val}, the
  conditional text \meta{ctext} is formatted.
\end{function}
\end{sfragment}

%%% Local Variables:
%%% mode: latex
%%% TeX-master: "../stex-manual"
%%% End:

%  LocalWords:  article.cls topsect DCMI:dmt03 loadmodules lmhtools
%  LocalWords:  prematurestop afterprematurestop import_graph.py STRlabel STRcopy vname
%  LocalWords:  STRsemantics setSGvar ifSGvar ctext

% \fi
%
%
% \begin{documentation}\label{pkg:documentstructure:doc}
%
%   The |document-structure| package is part of the \sTeX collection, a version of {\TeX/\LaTeX} that
%   allows to markup {\TeX/\LaTeX} documents semantically without leaving the document
%   format, essentially turning {\TeX/\LaTeX} into a document format for mathematical
%   knowledge management (MKM).
%
%   This package supplies an infrastructure for writing {\omdoc} documents in {\LaTeX}.
%   This includes a simple structure sharing mechanism for \sTeX that allows to to move
%   from a copy-and-paste document development model to a copy-and-reference model, which
%   conserves space and simplifies document management. The augmented structure can be
%   used by MKM systems for added-value services, either directly from the \sTeX
%   sources, or after translation.
%   \begin{sfragment}[id=sec:STR]{Introduction}
% 
%  \sTeX is a version of {\TeX/\LaTeX} that allows to markup {\TeX/\LaTeX} documents
%  semantically without leaving the document format, essentially turning {\TeX/\LaTeX}
%  into a document format for mathematical knowledge management (MKM). The package
%  supports direct translation to the {\omdoc} format~\cite{Kohlhase:OMDoc1.2}
%
%  The |document-structure| package supplies macros and environments that allow to label document
%  fragments and to reference them later in the same document or in other documents. In
%  essence, this enhances the document-as-trees model to
%  documents-as-directed-acyclic-graphs (DAG) model. This structure can be used by MKM
%  systems for added-value services, either directly from the \sTeX sources, or after
%  translation. Currently, trans-document referencing provided by this package can only be
%  used in the \sTeX collection.
%
%  DAG models of documents allow to replace the ``Copy and Paste'' in the source document
%  with a label-and-reference model where document are shared in the document source and
%  the formatter does the copying during document
%  formatting/presentation.\ednote{integrate with latexml's XMRef in the Math mode.}
% \end{sfragment}
% 
% \begin{sfragment}[id=sec:user]{The User Interface}
% 
%   The |document-structure| package generates two files: |document-structure.cls|, and |document-structure.sty|. The {\omdoc}
%   class is a minimally changed variant of the standard |article| class that includes the
%   functionality provided by |document-structure.sty|. The rest of the documentation pertains to the
%   functionality introduced by |document-structure.sty|.
%
% \begin{sfragment}[id=sec:user:options]{Package and Class Options}
% 
%   The |document-strcture| class accept the following options: 
%   \begin{center}
%     \begin{tabular}{|l|p{10cm}|}\hline
%         \texttt{class=\meta{name}} & load \meta{name}|.cls| instead of |article.cls|\\\hline 
%         \texttt{topsect=\meta{sect}} & The top-level sectioning level; the default for
%         \meta{sect} is \texttt{section}\\\hline 
%         \texttt{showignores} & show the the contents of the |ignore| environment after all \\\hline
%         \texttt{showmeta} & show the metadata; see |metakeys.sty|\\\hline
%         \texttt{showmods} & show modules; see |modules.sty|\\\hline
%         \texttt{extrefs} & allow external references; see |sref.sty|\\\hline
%         \texttt{defindex} & index definienda; see |statements.sty|\\\hline
%         \texttt{minimal} & for testing; do not load any \sTeX packages\\\hline 
%     \end{tabular}
%   \end{center}
%   The |document-structure| package accepts the same except the first two.
% \end{sfragment}
% 
% \begin{sfragment}[id=sec:user:struct]{Document Structure}
% 
%   The top-level \DescribeEnv{document}|document| environment can be given key/value
%   information by the \DescribeMacro{\documentkeys}|\documentkeys| macro in the
%   preamble\footnote{We cannot patch the document environment to accept an optional
%   argument, since other packages we load already do; pity.}. This can be used to give
%   metadata about the document. For the moment only the \DescribeMacro{id}|id| key is
%   used to give an identifier to the \texttt{omdoc} element resulting from the {\latexml}
%   transformation.
% 
%   The structure of the document is given by the \DescribeEnv{sfragment}|omgroup|
%   environment just like in {\omdoc}. In the {\LaTeX} route, the |omgroup| environment is
%   flexibly mapped to sectioning commands, inducing the proper sectioning level from the
%   nesting of |omgroup| environments. Correspondingly, the |omgroup| environment takes an
%   optional key/value argument for metadata followed by a regular argument for the
%   (section) title of the omgroup. The optional metadata argument has the keys
%   \DescribeMacro{id}|id| for an identifier, \DescribeMacro{creators}|creators| and
%   \DescribeMacro{contributors}|contributors| for the Dublin Core
%   metadata~\cite{DCMI:dmt03}; see~\cite{Kohlhase:dcm:git} for details of the format. The
%   \DescribeMacro{short}|short| allows to give a short title for the generated
%   section. If the title contains semantic macros, they need to be protected by
%   |\protect|, and we need to give the \DescribeMacro{loadmodules}|loadmodules| key it
%   needs no value. For instance we would have
%   \begin{verbatim}
%   \begin{smodule}{foo}
%   \symdef{bar}{B^a_r}
%    ...
%   \begin{sfragment}[id=sec.barderiv,loadmodules]{Introducing $\protect\bar$ Derivations}
%   \end{verbatim}
% 
%   \sTeX automatically computes the sectioning level, from the nesting of |omgroup|
%   environments. But sometimes, we want to skip levels (e.g. to use a subsection* as an
%   introduction for a chapter). Therefore the |document-structure| package provides a variant
%   \DescribeEnv{blindfragment}|blindomgroup| that does not produce markup, but increments
%   the sectioning level and logically groups document parts that belong together, but
%   where traditional document markup relies on convention rather than explicit
%   markup. The |blindomgroup| environment is useful e.g. for creating frontmatter at the
%   correct level. Example~\ref{fig:docstruct} shows a typical setup for the outer
%   document structure of a book with parts and chapters. We use two levels of
%   |blindomgroup|:
%   \begin{itemize}
%   \item The outer one groups the introductory parts of the book (which we assume to have
%     a sectioning hierarchy topping at the part level). This |blindomgroup| makes sure
%     that the introductory remarks become a ``chapter'' instead of a ``part''.
%   \item Th inner one groups the frontmatter\footnote{We shied away from redefining the
%     |frontmatter| to induce a blindomgroup, but this may be the ``right'' way to go in
%     the future.} and makes the preface of the book a section-level construct. Note that
%     here the |display=flow| on the |omgroup| environment prevents numbering as is
%     traditional for prefaces. 
%   \end{itemize}
%   \begin{exfig}
% \begin{verbatim}
% \begin{document}
% \begin{blindfragment}
% \begin{blindfragment}
% \begin{frontmatter}
% \maketitle\newpage
% \begin{sfragment}[display=flow]{Preface}
% ... <<preface>> ...
% \end{sfragment}
% \clearpage\setcounter{tocdepth}{4}\tableofcontents\clearpage
% \end{frontmatter}
% \end{blindfragment}
% ... <<introductory remarks>> ...
% \end{blindfragment}
% \begin{sfragment}{Introduction}
% ... <<intro>> ...
% \end{sfragment}
% ... <<more chapters>> ... 
% \bibliographystyle{alpha}\bibliography{kwarc}
% \end{document}
% \end{verbatim}
% \vspace*{-2em}
%   \caption{A typical Document Structure of a Book}\label{fig:docstruct}
% \end{exfig}
% 
% The \DescribeMacro{\skipomgroup}|\skipomgroup| ``skips an |omgroup|'', i.e. it just
% steps the respective sectioning counter. This macro is useful, when we want to keep two
% documents in sync structurally, so that section numbers match up: Any section that is
% left out in one becomes a |\skipomgroup|.
%
%   The \DescribeMacro{\currentsectionlevel}|\currentsectionlevel| macro supplies the name
%   of the current sectioning level, e.g. ``chapter'', or
%   ``subsection''. \DescribeMacro{\CurrentSectionLevel}|\CurrentSectionLevel| is the
%   capitalized variant. They are useful to write something like ``In this
%   |\currentsectionlevel|, we will\ldots'' in an |omgroup| environment, where we do not
%   know which sectioning level we will end up.
% \end{sfragment}
% 
% \begin{sfragment}[id=sec:user:ignore]{Ignoring Inputs}
% 
% The \DescribeEnv{ignore}|ignore| environment can be used for hiding text parts from the
% document structure. The body of the environment is not PDF or DVI output unless the
% \DescribeMacro{showignores}|showignores| option is given to the |document-structure| class or
% |package|. But in the generated {\omdoc} result, the body is marked up with a |ignore|
% element. This is useful in two situations. For
% \begin{description}
% \item[editing] One may want to hide unfinished or obsolete parts of a document
% \item[narrative/content markup] In {\stex} we mark up narrative-structured documents. In
%   the generated {\omdoc} documents we want to be able to cache content objects that are
%   not directly visible. For instance in the |statements|
%   package~\cite{Kohlhase:smms:git} we use the |\inlinedef| macro to mark up phrase-level
%   definitions, which verbalize more formal definitions. The latter can be hidden by an
%   ignore and referenced by the |verbalizes| key in |\inlinedef|.
% \end{description}
%
% For prematurely stopping the formatting of a document, \sTeX provides the
% \DescribeMacro{\prematurestop}|\prematurestop| macro. It can be used everywhere in a
% document and ignores all input after that -- backing out of the omgroup environment as
% needed. After that -- and before the implicit |\end{document}| it calls the internal
% \DescribeMacro{\afterprematurestop}|\afterprematurestop|, which can be customized to do
% additional cleanup or e.g. print the bibliography.
%
% |\prematurestop| is useful when one has a driver file, e.g. for a course taught multiple
% years and wants to generate course notes up to the current point in the lecture. Instead
% of commenting out the remaining parts, one can just move the |\prematurestop| macro.
% This is especially useful, if we need the rest of the file for processing, e.g. to
% generate a theory graph of the whole course with the already-covered parts marked up as
% an overview over the progress; see |import_graph.py| from the |lmhtools|
% utilities~\cite{lmhtools:github:on}.
% \end{sfragment}
%
% \begin{sfragment}[id=sec:user:sharing]{Structure Sharing}
%
%   The \DescribeMacro{\STRlabel}|\STRlabel| macro takes two arguments: a label and the
%   content and stores the the content for later use by
%   \DescribeMacro{\STRcopy}|\STRcopy[|\meta{URL}|]{|\meta{label}|}|, which expands to the
%   previously stored content. If the |\STRlabel| macro was in a different file, then we
%   can give a URL \meta{URL} that lets {\latexml} generate the correct reference.
%
%   \DescribeMacro{\STRsemantics} The |\STRlabel| macro has a variant |\STRsemantics|,
%   where the label argument is optional, and which takes a third argument, which is
%   ignored in {\LaTeX}. This allows to specify the meaning of the content (whatever that
%   may mean) in cases, where the source document is not formatted for presentation, but
%   is transformed into some content markup format.\ednote{document LMID und LMXREf here
%   if we decide to keep them.}
% \end{sfragment}
% 
% \begin{sfragment}[id=sec:user:gvars]{Global Variables}
% 
%   Text fragments and modules can be made more re-usable by the use of global
%   variables. For instance, the admin section of a course can be made course-independent
%   (and therefore re-usable) by using variables (actually token registers)
%   |courseAcronym| and |courseTitle| instead of the text itself. The variables can then
%   be set in the \sTeX preamble of the course notes file.
%   \DescribeMacro{\setSGvar}|\setSGvar{|\meta{vname}|}{|\meta{text}|}| to set the global
%   variable \meta{vname} to \meta{text} and
%   \DescribeMacro{\useSGvar}|\useSGvar{|\meta{vname}|}| to reference it.
%
%   With \DescribeMacro{\ifSGvar}|\ifSGvar| we can test for the contents of a global
%   variable: the macro call |\ifSGvar{|\meta{vname}|}{|\meta{val}|}{|\meta{ctext}|}|
%   tests the content of the global variable \meta{vname}, only if (after expansion) it is
%   equal to \meta{val}, the conditional text \meta{ctext} is formatted.
% \end{sfragment}
%
% \begin{sfragment}[id=sec:user:colors]{Colors}
% 
%   For convenience, the |document-structure| package defines a couple of color macros for the |color|
%   package: For instance \DescribeMacro{\blue}|\blue| abbreviates |\textcolor{blue}|, so
%   that |\blue{|\meta{something}|}| writes \meta{something} in blue. The macros
%   \DescribeMacro{\red}|\red| \DescribeMacro{...}|\green|, |\cyan|, |\magenta|, |\brown|,
%   |\yellow|, |\orange|, |\gray|, and finally \DescribeMacro{\black}|\black| are
%   analogous.
% \end{sfragment}
% \end{sfragment}
%
% \begin{sfragment}[id=sec:limitations]{Limitations}
% 
% In this section we document known limitations. If you want to help alleviate them,
% please feel free to contact the package author. Some of them are currently discussed in
% the \sTeX GitHub repository~\cite{sTeX:github:on}. 
% \begin{enumerate}
% \item when option |book| which uses |\pagestyle{headings}| is given and semantic macros
%   are given in the |omgroup| titles, then they sometimes are not defined by the time the
%   heading is formatted. Need to look into how the headings are made. 
% \end{enumerate}
% \end{sfragment}
% 
% \end{documentation}
%
% \begin{implementation}\label{pkg:documentstructure:impl}
%
% \begin{sfragment}{document-structure.sty Implementation}
%
% \begin{sfragment}[id=sec:impl:cls]{The document-structure Class}
%
%   The functionality is spread over the |document-structure| class and package. The class provides the
%   |document| environment and the |document-structure| element corresponds to it, whereas the
%   package provides the concrete functionality.
%
%    \begin{macrocode}
%<*cls>
%<@@=document_structure>
\ProvidesExplClass{document-structure}{2022/02/24}{3.0.0}{Modular Document Structure Class}
\RequirePackage{l3keys2e}
%    \end{macrocode}
%
% \begin{sfragment}[id=sec:impl:cls:options]{Class Options}
%   To initialize the |document-structure| class, we declare and process the necessary options using
%   the |kvoptions| package for  key/value options handling. For
%   |omdoc.cls| this is quite simple. We have options |report| and |book|, which set the
%   \DescribeMacro{\omdoc@cls@class}|\omdoc@cls@class| macro and pass on the macro to |omdoc.sty|
%   for further processing. 
%
%    \begin{macrocode}
\keys_define:nn{ document-structure / pkg }{
  class       .str_set_x:N  = \c_document_structure_class_str,
  minimal     .bool_set:N   = \c_document_structure_minimal_bool,
  report      .code:n       = {
    \ClassWarning{document-structure}{the option 'report' is deprecated, use 'class=report', instead}
    \str_set:Nn \c_document_structure_class_str {report}
  },
  book        .code:n       = {
    \ClassWarning{document-structure}{the option 'book' is deprecated, use 'class=book', instead}
    \str_set:Nn \c_document_structure_class_str {book}
  },
  bookpart    .code:n       = {
    \ClassWarning{document-structure}{the option 'bookpart' is deprecated, use 'class=book,topsect=chapter', instead}
    \str_set:Nn \c_document_structure_class_str {book}
    \str_set:Nn \c_document_structure_topsect_str {chapter}
  },
  docopt      .str_set_x:N  = \c_document_structure_docopt_str,
  unknown     .code:n       = {
    \PassOptionsToPackage{ \CurrentOption }{ document-structure }
  }
}
\ProcessKeysOptions{ document-structure / pkg }
\str_if_empty:NT \c_document_structure_class_str {
  \str_set:Nn \c_document_structure_class_str {article}
}
\exp_after:wN\LoadClass\exp_after:wN[\c_document_structure_docopt_str]
  {\c_document_structure_class_str}

%    \end{macrocode}
% \end{sfragment}
%
% \begin{sfragment}[id=sec:impl:cls:document]{Beefing up the \texttt{document} environment}
%
%   Now, -- unless the option |minimal| is defined -- we include the |stex| package 
%
%    \begin{macrocode}
\RequirePackage{document-structure}
\bool_if:NF \c_document_structure_minimal_bool {
%    \end{macrocode}
%
% And define the environments we need.  The top-level one is the |document| environment,
% which we redefined so that we can provide keyval arguments.
%
% \begin{environment}{document}
%   For the moment we do not use them on the {\LaTeX} level, but the document identifier
%   is picked up by {\latexml}.\ednote{faking documentkeys for now. @HANG, please implement}
%    \begin{macrocode}
\keys_define:nn { document-structure / document }{
  id .str_set_x:N = \c_document_structure_document_id_str
}
\let\_@@_orig_document=\document
\renewcommand{\document}[1][]{
  \keys_set:nn{ document-structure / document }{ #1 }
  \stex_ref_new_doc_target:n { \c_document_structure_document_id_str }
  \_@@_orig_document
}
%    \end{macrocode}
% \end{environment}
%
% Finally, we end the test for the |minimal| option. 
% 
%    \begin{macrocode}
}
%</cls>
%    \end{macrocode}
% \end{sfragment}
% \end{sfragment}
% 
% \begin{sfragment}[id=sec:impl:sty]{Implementation: document-structure Package}
%
%    \begin{macrocode}
%<*package>
\ProvidesExplPackage{document-structure}{2022/02/24}{3.0.0}{Modular Document Structure}
\RequirePackage{l3keys2e}
%    \end{macrocode}
%
% \begin{sfragment}[id=sec:impl:options]{Package Options}
%
%   We declare some switches which will modify the behavior according to the package
%   options. Generally, an option |xxx| will just set the appropriate switches to true
%   (otherwise they stay false).
%
%
%    \begin{macrocode}

\keys_define:nn{ document-structure / pkg }{
  class       .str_set_x:N  = \c_document_structure_class_str,
  topsect     .str_set_x:N  = \c_document_structure_topsect_str,
%  showignores .bool_set:N   = \c_document_structure_showignores_bool,
}
\ProcessKeysOptions{ document-structure / pkg }
\str_if_empty:NT \c_document_structure_class_str {
  \str_set:Nn \c_document_structure_class_str {article}
}
\str_if_empty:NT \c_document_structure_topsect_str {
  \str_set:Nn \c_document_structure_topsect_str {section}
}
%    \end{macrocode}
%
% Then we need to set up the packages by requiring the |sref| package to be loaded,
% and set up triggers for other languages
%
%    \begin{macrocode}
\RequirePackage{xspace}
\RequirePackage{comment}
\AddToHook{begindocument}{
	\ltx@ifpackageloaded{babel}{
    \clist_set:Nx \l_tmpa_clist {\bbl@loaded}
    \clist_if_in:NnT \l_tmpa_clist {ngerman}{
      \makeatletter\input{document-structure-ngerman.ldf}\makeatother
    }
  }{}
}
%    \end{macrocode}
%
% Finally, we set the \DescribeMacro{\section@level}|\section@level| macro that governs
% sectioning. The default is two (corresponding to the |article| class), then we set the
% defaults for the standard classes |book| and |report| and then we take care of the
% levels passed in via the |topsect| option.
%
%    \begin{macrocode}
\int_new:N \l_document_structure_section_level_int
\str_case:VnF \c_document_structure_topsect_str {
  {part}{
    \int_set:Nn \l_document_structure_section_level_int {0}
  }
  {chapter}{
    \int_set:Nn \l_document_structure_section_level_int {1}
  }
}{
  \str_case:VnF \c_document_structure_class_str {
    {book}{
      \int_set:Nn \l_document_structure_section_level_int {0}
    }
    {report}{
      \int_set:Nn \l_document_structure_section_level_int {0}
    }
  }{
    \int_set:Nn \l_document_structure_section_level_int {2}
  }
}
%    \end{macrocode}
%
% \end{sfragment}
% 
% \begin{sfragment}[id=sec:impl:struct]{Document Structure}
% 
%   The structure of the document is given by the |omgroup| environment just like in
%   OMDoc. The hierarchy is adjusted automatically according to the {\LaTeX} class in
%   effect. 
% \begin{macro}{\currentsectionlevel}
%   For the |\currentsectionlevel| and |\Currentsectionlevel| macros we use an internal
%   macro |\current@section@level| that only contains the keyword (no markup). We
%   initialize it with ``document'' as a default. In the generated OMDoc, we only generate
%   a text element of class |omdoc_currentsectionlevel|, wich will be instantiated by CSS
%   later.\ednote{MK: we may have to experiment with the more powerful uppercasing macro
%   from \texttt{mfirstuc.sty} once we internationalize.}
%    \begin{macrocode}
\def\current@section@level{document}%
\newcommand\currentsectionlevel{\lowercase\expandafter{\current@section@level}\xspace}%
\newcommand\Currentsectionlevel{\expandafter\MakeUppercase\current@section@level\xspace}%
%    \end{macrocode}
% \end{macro}
%
% \begin{macro}{\skipomgroup}
%    \begin{macrocode}
\cs_new_protected:Npn \skipomgroup {
  \ifcase\l_document_structure_section_level_int
  \or\stepcounter{part}
  \or\stepcounter{chapter}
  \or\stepcounter{section}
  \or\stepcounter{subsection}
  \or\stepcounter{subsubsection}
  \or\stepcounter{paragraph}
  \or\stepcounter{subparagraph}
  \fi
}
%    \end{macrocode}
% \end{macro}
%
% \begin{environment}{blindfragment}
%    \begin{macrocode}
\newcommand\at@begin@blindomgroup[1]{}
\newenvironment{blindfragment}
{
  \int_incr:N\l_document_structure_section_level_int
  \at@begin@blindomgroup\l_document_structure_section_level_int
}{}
%    \end{macrocode}
% \end{environment}
%
% \begin{macro}{\omgroup@nonum}
%   convenience macro: |\omgroup@nonum{|\meta{level}|}{|\meta{title}|}| makes an unnumbered
%   sectioning with title \meta{title} at level \meta{level}.
%    \begin{macrocode}
\newcommand\omgroup@nonum[2]{
  \ifx\hyper@anchor\@undefined\else\phantomsection\fi
  \addcontentsline{toc}{#1}{#2}\@nameuse{#1}*{#2}
}
%    \end{macrocode}
% \end{macro}
%
% \begin{macro}{\omgroup@num}
%   convenience macro: |\omgroup@nonum{|\meta{level}|}{|\meta{title}|}| makes numbered
%   sectioning with title \meta{title} at level \meta{level}. We have to check the |short|
%   key was given in the |omgroup| environment and -- if it is use it. But how to do that
%   depends on whether the |rdfmeta| package has been loaded. In the end we call
%   |\sref@label@id| to enable crossreferencing.
%    \begin{macrocode}
\newcommand\omgroup@num[2]{
  \tl_if_empty:NTF \l_@@_omgroup_short_tl {
    \@nameuse{#1}{#2}
  }{
    \cs_if_exist:NTF\rdfmeta@sectioning{
      \@nameuse{rdfmeta@#1@old}[\l_@@_omgroup_short_tl]{#2}
    }{
      \@nameuse{#1}[\l_@@_omgroup_short_tl]{#2}
    }
  }
%\sref@label@id@arg{\omdoc@sect@name~\@nameuse{the#1}}\omgroup@id
}
%    \end{macrocode}
% \end{macro}
%
% \begin{environment}{sfragment}
%    \begin{macrocode}
\keys_define:nn { document-structure / omgroup }{
  id            .str_set_x:N = \l_@@_omgroup_id_str,
  date          .str_set_x:N = \l_@@_omgroup_date_str,
  creators      .clist_set:N = \l_@@_omgroup_creators_clist,
  contributors  .clist_set:N = \l_@@_omgroup_contributors_clist,
  srccite       .tl_set:N    = \l_@@_omgroup_srccite_tl,
  type          .tl_set:N    = \l_@@_omgroup_type_tl,
  short         .tl_set:N    = \l_@@_omgroup_short_tl,
  display       .tl_set:N    = \l_@@_omgroup_display_tl,
  intro         .tl_set:N    = \l_@@_omgroup_intro_tl,
  loadmodules   .bool_set:N  = \l_@@_omgroup_loadmodules_bool
}
\cs_new_protected:Nn \_@@_omgroup_args:n {
  \str_clear:N \l_@@_omgroup_id_str
  \str_clear:N \l_@@_omgroup_date_str
  \clist_clear:N \l_@@_omgroup_creators_clist
  \clist_clear:N \l_@@_omgroup_contributors_clist
  \tl_clear:N \l_@@_omgroup_srccite_tl
  \tl_clear:N \l_@@_omgroup_type_tl
  \tl_clear:N \l_@@_omgroup_short_tl
  \tl_clear:N \l_@@_omgroup_display_tl
  \tl_clear:N \l_@@_omgroup_intro_tl
  \bool_set_false:N \l_@@_omgroup_loadmodules_bool
  \keys_set:nn { document-structure / omgroup } { #1 }
}
%    \end{macrocode}
% we define a switch for numbering lines and a hook for the beginning of groups: The
% \DescribeMacro{\at@begin@omgroup}|\at@begin@omgroup| macro allows customization. It is
% run at the beginning of the |omgroup|, i.e. after the section heading.
%    \begin{macrocode}
\newif\if@mainmatter\@mainmattertrue
\newcommand\at@begin@omgroup[3][]{}
%    \end{macrocode}
%
% Then we define a helper macro that takes care of the sectioning magic. It comes with its
% own key/value interface for customization.
%
%    \begin{macrocode}
\keys_define:nn { document-structure / sectioning }{
  name    .str_set_x:N  = \l_@@_sect_name_str   ,
  ref     .str_set_x:N  = \l_@@_sect_ref_str    ,
  clear   .bool_set:N   = \l_@@_sect_clear_bool ,
  clear   .default:n    = {true}                ,
  num     .bool_set:N   = \l_@@_sect_num_bool   ,
  num     .default:n    = {true}
}
\cs_new_protected:Nn \_@@_sect_args:n {
  \str_clear:N \l_@@_sect_name_str
  \str_clear:N \l_@@_sect_ref_str
  \bool_set_false:N \l_@@_sect_clear_bool
  \bool_set_false:N \l_@@_sect_num_bool
  \keys_set:nn { document-structure / sectioning } { #1 }
}
\newcommand\omdoc@sectioning[3][]{
  \_@@_sect_args:n {#1 }
  \let\omdoc@sect@name\l_@@_sect_name_str
  \bool_if:NT \l_@@_sect_clear_bool { \cleardoublepage }
  \if@mainmatter% numbering not overridden by frontmatter, etc.
    \bool_if:NTF \l_@@_sect_num_bool {
      \omgroup@num{#2}{#3}
    }{
      \omgroup@nonum{#2}{#3}
    }
    \def\current@section@level{\omdoc@sect@name}
  \else
    \omgroup@nonum{#2}{#3}
  \fi
}% if@mainmatter
%    \end{macrocode}
% and another one, if redefines the |\addtocontentsline| macro of {\LaTeX} to import the
% respective macros. It takes as an argument a list of module names.
%    \begin{macrocode}
\newcommand\omgroup@redefine@addtocontents[1]{%
%\edef\@@import{#1}%
%\@for\@I:=\@@import\do{%
%\edef\@path{\csname module@\@I  @path\endcsname}%
%\@ifundefined{tf@toc}\relax%
%     {\protected@write\tf@toc{}{\string\@requiremodules{\@path}}}}
%\ifx\hyper@anchor\@undefined% hyperref.sty loaded?
%\def\addcontentsline##1##2##3{%
%\addtocontents{##1}{\protect\contentsline{##2}{\string\withusedmodules{#1}{##3}}{\thepage}}}
%\else% hyperref.sty not loaded
%\def\addcontentsline##1##2##3{%
%\addtocontents{##1}{\protect\contentsline{##2}{\string\withusedmodules{#1}{##3}}{\thepage}{\@currentHref}}}%
%\fi
}% hypreref.sty loaded?
%    \end{macrocode}
% now the |omgroup| environment itself. This takes care of the table of contents via the
% helper macro above and then selects the appropriate sectioning command from
% |article.cls|. It also registeres the current level of omgroups in the |\omgroup@level|
% counter. 
%    \begin{macrocode}
\newenvironment{sfragment}[2][]% keys, title
{
  \_@@_omgroup_args:n { #1 }%\sref@target%
%    \end{macrocode}
% If the |loadmodules| key is set on |\begin{sfragment}|, we redefine the |\addcontetsline|
%   macro that determines how the sectioning commands below construct the entries for the
%   table of contents.
%    \begin{macrocode}
  \bool_if:NT \l_@@_omgroup_loadmodules_bool {
    \omgroup@redefine@addtocontents{
      %\@ifundefined{module@id}\used@modules%
      %{\@ifundefined{module@\module@id @path}{\used@modules}\module@id}
    }
  }
%    \end{macrocode}
% now we only need to construct the right sectioning depending on the value of
% |\section@level|.
%    \begin{macrocode}
  \int_incr:N\l_document_structure_section_level_int
  \ifcase\l_document_structure_section_level_int
    \or\omdoc@sectioning[name=\omdoc@part@kw,clear,num]{part}{#2}
    \or\omdoc@sectioning[name=\omdoc@chapter@kw,clear,num]{chapter}{#2}
    \or\omdoc@sectioning[name=\omdoc@section@kw,num]{section}{#2}
    \or\omdoc@sectioning[name=\omdoc@subsection@kw,num]{subsection}{#2}
    \or\omdoc@sectioning[name=\omdoc@subsubsection@kw,num]{subsubsection}{#2}
    \or\omdoc@sectioning[name=\omdoc@paragraph@kw,ref=this \omdoc@paragraph@kw]{paragraph}{#2}
    \or\omdoc@sectioning[name=\omdoc@subparagraph@kw,ref=this \omdoc@subparagraph@kw]{paragraph}{#2}
  \fi
  \at@begin@omgroup[#1]\l_document_structure_section_level_int{#2}
  \str_if_empty:NF \l_@@_omgroup_id_str {
    \stex_ref_new_doc_target:n\l_@@_omgroup_id_str
  }
}% for customization
{}
%    \end{macrocode}
% \end{environment}
%
% and finally, we localize the sections
%    \begin{macrocode}
\newcommand\omdoc@part@kw{Part}
\newcommand\omdoc@chapter@kw{Chapter}
\newcommand\omdoc@section@kw{Section}
\newcommand\omdoc@subsection@kw{Subsection}
\newcommand\omdoc@subsubsection@kw{Subsubsection}
\newcommand\omdoc@paragraph@kw{paragraph}
\newcommand\omdoc@subparagraph@kw{subparagraph}
%    \end{macrocode}
%
% \end{sfragment}
%
% \begin{sfragment}[id=sec:user:docmatter]{Front and Backmatter}
% 
%   Index markup is provided by the |omtext| package~\cite{Kohlhase:smmtf:git}, so in the
%   |document-structure| package we only need to supply the corresponding |\printindex| command, if it
%   is not already defined
% \begin{macro}{\printindex}
%    \begin{macrocode}
\providecommand\printindex{\IfFileExists{\jobname.ind}{\input{\jobname.ind}}{}}
%    \end{macrocode}
% \end{macro}
% 
% some classes (e.g. |book.cls|) already have |\frontmatter|, |\mainmatter|, and
% |\backmatter| macros. As we want to define |frontmatter| and |backmatter| environments,
% we save their behavior (possibly defining it) in |orig@*matter| macros and make them
% undefined (so that we can define the environments).
%    \begin{macrocode}
\cs_if_exist:NTF\frontmatter{
  \let\_@@_orig_frontmatter\frontmatter
  \let\frontmatter\relax
}{
  \tl_set:Nn\_@@_orig_frontmatter{
    \clearpage
    \@mainmatterfalse
    \pagenumbering{roman}
  }
}
\cs_if_exist:NTF\backmatter{
  \let\_@@_orig_backmatter\backmatter
  \let\backmatter\relax
}{
  \tl_set:Nn\_@@_orig_backmatter{
    \clearpage
    \@mainmatterfalse
    \pagenumbering{roman}
  }
}
%    \end{macrocode}
%
% Using these, we can now define the |frontmatter| and |backmatter| environments
% 
% \begin{environment}{frontmatter}
%  we use the |\orig@frontmatter| macro defined above and |\mainmatter| if it exists,
%  otherwise we define it.  
%    \begin{macrocode}
\newenvironment{frontmatter}{
  \_@@_orig_frontmatter
}{
  \cs_if_exist:NTF\mainmatter{
    \mainmatter
  }{
    \clearpage
    \@mainmattertrue
    \pagenumbering{arabic}
  }
}
%    \end{macrocode}
% \end{environment}
%
% \begin{environment}{backmatter}
%   As backmatter is at the end of the document, we do nothing for |\endbackmatter|. 
%    \begin{macrocode}
\newenvironment{backmatter}{
  \_@@_orig_backmatter
}{
  \cs_if_exist:NTF\mainmatter{
    \mainmatter
  }{
    \clearpage
    \@mainmattertrue
    \pagenumbering{arabic}
  }
}
%    \end{macrocode}
%
% finally, we make sure that page numbering is arabic and we have main matter as the default
%
%    \begin{macrocode}
\@mainmattertrue\pagenumbering{arabic}
%    \end{macrocode}
% \end{environment}
% \end{sfragment}
%
% \begin{macro}{\prematurestop}
%   We initialize |\afterprematurestop|, and provide 
%   |\prematurestop@endomgroup| which looks up |\omgroup@level| and recursively ends
%   enough |{sfragment}|s. 
%    \begin{macrocode}
\def \c_@@_document_str{document} 
\newcommand\afterprematurestop{}
\def\prematurestop@endomgroup{
  \unless\ifx\@currenvir\c_@@_document_str
    \expandafter\expandafter\expandafter\end\expandafter\expandafter\expandafter{\expandafter\@currenvir\expandafter}
    \expandafter\prematurestop@endomgroup
  \fi
}
\providecommand\prematurestop{
  \message{Stopping~sTeX~processing~prematurely}
  \prematurestop@endomgroup
  \afterprematurestop
  \end{document}
}
%    \end{macrocode}
% \end{macro}
%
% \iffalse
% \begin{sfragment}[id=sec:impl:share]{Structure Sharing}
%   \ednote{The following is simply copied over from the |latexml| package, which we
%   eliminated, we should integrate better.}
%    \begin{macrocode}
\iffalse
\providecommand{\lxDocumentID}[1]{}%
\def\LXMID#1#2{\expandafter\gdef\csname xmarg#1\endcsname{#2}\csname xmarg#1\endcsname}
\def\LXMRef#1{\csname xmarg#1\endcsname}
%    \end{macrocode}
%
% \begin{macro}{\STRlabel}
%    The main macro, it it used to attach a label to some text expansion. Later on, using the
%    |\STRcopy| macro, the author can use this label to get the expansion originally assigned.
%    \begin{macrocode}
\long\def\STRlabel#1#2{\STRlabeldef{#1}{#2}{#2}}
%    \end{macrocode}
% \end{macro}
% 
% \begin{macro}{\STRcopy}
%   The |\STRcopy| macro is used to call the expansion of a given label. In case the label
%   is not defined it will issue a warning.\ednote{MK: we need to do something about the
%   ref!}
%    \begin{macrocode}
\newcommand\STRcopy[2][]{\expandafter\ifx\csname STR@#2\endcsname\relax
\message{STR warning: reference #2 undefined!}
\else\csname STR@#2\endcsname\fi}
%    \end{macrocode}
% \end{macro}
%
% \begin{macro}{\STRsemantics}
%    if we have a presentation form and a semantic form, then we can use
%    \begin{macrocode}
\newcommand\STRsemantics[3][]{#2\def\@test{#1}\ifx\@test\@empty\STRlabeldef{#1}{#2}\fi}
%    \end{macrocode}
% \end{macro}
%
% \begin{macro}{\STRlabeldef}
%    This is the macro that does the actual labeling. Is it called inside |\STRlabel|
%    \begin{macrocode}
\def\STRlabeldef#1{\expandafter\gdef\csname STR@#1\endcsname}
\fi
%    \end{macrocode}
% \end{macro}
% \end{sfragment}
%\fi
% 
% \begin{sfragment}[id=sec:impl:gvars]{Global Variables}
%
% \begin{macro}{\setSGvar}
%   set a global variable
%    \begin{macrocode}
\RequirePackage{etoolbox}
\newcommand\setSGvar[1]{\@namedef{sTeX@Gvar@#1}}
%    \end{macrocode}
% \end{macro}
%
% \begin{macro}{\useSGvar}
%   use a global variable
%    \begin{macrocode}
\newrobustcmd\useSGvar[1]{%
  \@ifundefined{sTeX@Gvar@#1}
  {\PackageError{document-structure}
    {The sTeX Global variable #1 is undefined}
    {set it with \protect\setSGvar}}
\@nameuse{sTeX@Gvar@#1}}
%    \end{macrocode}
% \end{macro}
%
% \begin{macro}{\ifSGvar}
%   execute something conditionally based on the state of the global variable. 
%    \begin{macrocode}
\newrobustcmd\ifSGvar[3]{\def\@test{#2}%
  \@ifundefined{sTeX@Gvar@#1}
  {\PackageError{document-structure}
    {The sTeX Global variable #1 is undefined}
    {set it with \protect\setSGvar}}
  {\expandafter\ifx\csname sTeX@Gvar@#1\endcsname\@test #3\fi}}
%    \end{macrocode}
% \end{macro}
%
% \end{sfragment}
% \end{sfragment}
% \end{sfragment}
%
% \end{implementation}
\endinput
% \iffalse
%%% Local Variables: 
%%% mode: doctex
%%% TeX-master: t
%%% End: 
% \fi
% LocalWords:  GPL structuresharing STR omdoc dtx stex CPERL LoadClass url dc filedate om 
% LocalWords:  RequirePackage RegisterNamespace namespace xsl DocType ltxml dtd DAG hline
% LocalWords:  ltx DefEnvironment beforeDigest AssignValue inPreamble getGullet rangle
% LocalWords:  afterDigest keyval omgroup DefKeyVal Semiverbatim KeyVal srcf frontmatter
% LocalWords:  OptionalKeyVals DefParameterType IfBeginFollows skipSpaces CMP rangle xdef
% LocalWords:  ifNext DefMacro needwrapper unlist DefConstructor omtext bgroup showmods
% LocalWords:  useCMPItemizations RefStepItemCounter egroup beginItemize li di pathsuris
% LocalWords:  beforeDigestEnd dt autoclose ul ol dl env showignores srcref Cwd rdfmeta
% LocalWords:  afterOpen LastSeenCMP autoClose DefCMPEnvironment proto ToString st@flow
% LocalWords:  addAttribute nlex nlcex omdocColorMacro args tok MergeFont qw setion@level
% LocalWords:  TokenizeInternal toString isMath foreach maybeCloseElement id'd Backmatter
% LocalWords:  autoOpen minipage footnotesize scriptsize numberIt whatsit href endinput
% LocalWords:  getAttribute setAttribute OMDoc RelaxNGSchema noindex xml lec KeyVals
% LocalWords:  Subsubsection useDefaultItemizations refundefinedtrue sblockquote defindex
% LocalWords:  DefCMPConstructor sinlinequote idx idt ide idp emph  extrefs sref Tokenize
% LocalWords:  flushleft flushright DeclareOption PassOptions undef cls iffalse noauxreq
% LocalWords:  ProcessOptions subparagraph ignoresfalse ignorestrue texttt ttin behavior
% LocalWords:  texttt latexml fileversion maketitle newpage tableofcontents cwd srccite
% LocalWords:  newpage ednote ctancite dmt03 smms inlinedef STRlabel STRcopy loadmodules
% LocalWords:  STRlabel STRsemantics STRsemantics textcolor printbibliography loadmodules
% LocalWords:  textsf langle textsf langle respetively orig renewcommand cdir capitalized
% LocalWords:  baseuri baseuri baselocal baselocal SOURCEFILE cooluri newif ifx tf@toc
% LocalWords:  SOURCEBASE chapterfalse partfalse newcount ifshow chaptertrue initialize
% LocalWords:  parttrue srefaddidkey newenvironment textbf compactenum showmeta tf@toc
% LocalWords:  noindent noindent ignorespaces ifnum thepart thechapter regexp color.sty
% LocalWords:  thesection thesubsection thesubsubsection needswrapper itshape xmarg xmarg
% LocalWords:  textgreater renewenvironment excludecomment STRlabeldef csname Section,num
% LocalWords:  expandafter endcsname xref newcommand gdef doctex metakeys Hacky arabic
% LocalWords:  metasetkeys addmetakey printindex providecommand jobname.ind importmodules
% LocalWords:  jobname.ind tocdepth hateq ensuremath xspace hatequiv equiv NeedsTeXFormat
% LocalWords:  textleadsto leadsto etoolbox blindomgroup blindomgroup docstruct setSGvar
% LocalWords:  compactitem exfig vspace currentsectionlevel currentsectionlevel setSGvar
% LocalWords:  ldots URLBASE ifclass bookfalse booktrue currentsetionlevel thedocument@ID
% LocalWords:  nonum phantomsection nameuse numtrue numfalse contentsline unnum vname
% LocalWords:  thepage hypreref.sty ifcase cleardoublepage frontmatterfalse customization
% LocalWords:  frontmattertrue pagenumbering setcounter hyperref.sty addcontetsline ctext
%  LocalWords:  mfirstuc.sty internationalize documentkeys withusedmodules Part,clear,num
% \endinput
% Local Variables:
% mode: doctex
% TeX-master: t
% End:
%  LocalWords:  crossreferencing Chapter,clear,num Subsection,num Subsubsection,num cslet
%  LocalWords:  Paragraph,ref Subparagraph,ref useSGvar useSGvar ifSGvar ifSGvar topsect
%  LocalWords:  sTeX@Gvar kvoptions omdoc@cls,prefix book,topsect xappto omdoc@sty,prefix
%  LocalWords:  ifdefstring ifcsdef cslet localization ngerman omdoc-ngerman.ldf omgroups
%  LocalWords:  Kohlhase:smmtf endbackmatter prematurestop prematurestop clear,num
%  LocalWords:  prematurestop@endomgroup textbackslash import_graph.py lmhtools bibfolder
%  LocalWords:  stepcounter jobname.dtx clist_set:Nx l_tmpa_clist clist_if_in:NnT
%  LocalWords:  ExplSyntaxOff stex_ref_new_doc_target:n

  \end{omgroup}
  \begin{omgroup}{Slides and Course Notes}
    \textcolor{red}{TODO: notesslides documentation}
  \end{omgroup}
  \begin{omgroup}{Homework, Problems and Exams}
    %%
%% This is file `problem.sty',
%% generated with the docstrip utility.
%%
%% The original source files were:
%%
%% problem.dtx  (with options: `package')
%% 
\ProvidesExplPackage{problem}{2022/02/24}{3.0.0}{Semantic Markup for Problems}
\RequirePackage{l3keys2e,stex}

\keys_define:nn { problem / pkg }{
  notes     .default:n    = { true },
  notes     .bool_set:N   = \c__problems_notes_bool,
  gnotes    .default:n    = { true },
  gnotes    .bool_set:N   = \c__problems_gnotes_bool,
  hints     .default:n    = { true },
  hints     .bool_set:N   = \c__problems_hints_bool,
  solutions .default:n    = { true },
  solutions .bool_set:N   = \c__problems_solutions_bool,
  pts       .default:n    = { true },
  pts       .bool_set:N   = \c__problems_pts_bool,
  min       .default:n    = { true },
  min       .bool_set:N   = \c__problems_min_bool,
  boxed     .default:n    = { true },
  boxed     .bool_set:N   = \c__problems_boxed_bool,
  unknown   .code:n       = {}
}
\newif\ifsolutions

\ProcessKeysOptions{ problem / pkg }
\bool_if:NTF \c__problems_solutions_bool {
  \solutionstrue
}{
  \solutionsfalse
}
\RequirePackage{comment}
\bool_if:NT \c__problems_boxed_bool { \RequirePackage{mdframed} }
\def\prob@problem@kw{Problem}
\def\prob@solution@kw{Solution}
\def\prob@hint@kw{Hint}
\def\prob@note@kw{Note}
\def\prob@gnote@kw{Grading}
\def\prob@pt@kw{pt}
\def\prob@min@kw{min}
\AddToHook{begindocument}{
  \ltx@ifpackageloaded{babel}{
      \makeatletter
      \clist_set:Nx \l_tmpa_clist {\bbl@loaded}
      \clist_if_in:NnT \l_tmpa_clist {ngerman}{
        \input{problem-ngerman.ldf}
      }
      \clist_if_in:NnT \l_tmpa_clist {finnish}{
        \input{problem-finnish.ldf}
      }
      \clist_if_in:NnT \l_tmpa_clist {french}{
        \input{problem-french.ldf}
      }
      \clist_if_in:NnT \l_tmpa_clist {russian}{
        \input{problem-russian.ldf}
      }
      \makeatother
  }{}
}
\keys_define:nn{ problem / problem }{
  id      .str_set_x:N  = \l__problems_prob_id_str,
  pts     .tl_set:N     = \l__problems_prob_pts_tl,
  min     .tl_set:N     = \l__problems_prob_min_tl,
  title   .tl_set:N     = \l__problems_prob_title_tl,
  type    .tl_set:N     = \l__problems_prob_type_tl,
  refnum  .int_set:N    = \l__problems_prob_refnum_int
}
\cs_new_protected:Nn \__problems_prob_args:n {
  \str_clear:N \l__problems_prob_id_str
  \tl_clear:N \l__problems_prob_pts_tl
  \tl_clear:N \l__problems_prob_min_tl
  \tl_clear:N \l__problems_prob_title_tl
  \tl_clear:N \l__problems_prob_type_tl
  \int_zero_new:N \l__problems_prob_refnum_int
  \keys_set:nn { problem / problem }{ #1 }
  \int_compare:nNnT \l__problems_prob_refnum_int = 0 {
    \let\l__problems_prob_refnum_int\undefined
  }
}
\newcounter{problem}
\newcommand\numberproblemsin[1]{\@addtoreset{problem}{#1}}
\newcommand\prob@label[1]{#1}
\newcommand\prob@number{
  \int_if_exist:NTF \l__problems_inclprob_refnum_int {
    \prob@label{\int_use:N \l__problems_inclprob_refnum_int }
  }{
    \int_if_exist:NTF \l__problems_prob_refnum_int {
      \prob@label{\int_use:N \l__problems_prob_refnum_int }
    }{
        \prob@label\theproblem
    }
  }
}
\newcommand\prob@title[3]{%
  \tl_if_exist:NTF \l__problems_inclprob_title_tl {
    #2 \l__problems_inclprob_title_tl #3
  }{
    \tl_if_exist:NTF \l__problems_prob_title_tl {
      #2 \l__problems_prob_title_tl #3
    }{
      #1
    }
  }
}
\def\prob@heading{
  {\prob@problem@kw}\ \prob@number\prob@title{~}{~(}{)\strut}
  %\sref@label@id{\prob@problem@kw~\prob@number}{}
}
\newenvironment{sproblem}[1][]{
  \__problems_prob_args:n{#1}%\sref@target%
  \@in@omtexttrue% we are in a statement (for inline definitions)
  \stepcounter{problem}\record@problem
  \def\current@section@level{\prob@problem@kw}
  \tl_if_exist:NTF \l__problems_inclprob_type_tl {
    \tl_set_eq:NN \sproblemtype \l__problems_inclprob_type_tl
  }{
    \tl_set_eq:NN \sproblemtype \l__problems_prob_type_tl
  }
  \str_if_exist:NTF \l__problems_inclprob_id_str {
    \str_set_eq:NN \sproblemid \l__problems_inclprob_id_str
  }{
    \str_set_eq:NN \sproblemid \l__problems_prob_id_str
  }

  \clist_set:No \l_tmpa_clist \sproblemtype
  \tl_clear:N \l_tmpa_tl
  \clist_map_inline:Nn \l_tmpa_clist {
    \tl_if_exist:cT {__problems_sproblem_##1_start:}{
      \tl_set:Nn \l_tmpa_tl {\use:c{__problems_sproblem_##1_start:}}
    }
  }
  \tl_if_empty:NTF \l_tmpa_tl {
    \__problems_sproblem_start:
  }{
    \l_tmpa_tl
  }
  \stex_ref_new_doc_target:n \sproblemid
}{
  \clist_set:No \l_tmpa_clist \sproblemtype
  \tl_clear:N \l_tmpa_tl
  \clist_map_inline:Nn \l_tmpa_clist {
    \tl_if_exist:cT {__problems_sproblem_##1_end:}{
      \tl_set:Nn \l_tmpa_tl {\use:c{__problems_sproblem_##1_end:}}
    }
  }
  \tl_if_empty:NTF \l_tmpa_tl {
    \__problems_sproblem_end:
  }{
    \l_tmpa_tl
  }

  \smallskip
}

\cs_new_protected:Nn \__problems_sproblem_start: {
  \par\noindent\textbf\prob@heading\show@pts\show@min\\\ignorespacesandpars
}
\cs_new_protected:Nn \__problems_sproblem_end: {\par\smallskip}

\newcommand\stexpatchproblem[3][] {
    \str_set:Nx \l_tmpa_str{ #1 }
    \str_if_empty:NTF \l_tmpa_str {
      \tl_set:Nn \__problems_sproblem_start: { #2 }
      \tl_set:Nn \__problems_sproblem_end: { #3 }
    }{
      \exp_after:wN \tl_set:Nn \csname __problems_sproblem_#1_start:\endcsname{ #2 }
      \exp_after:wN \tl_set:Nn \csname __problems_sproblem_#1_end:\endcsname{ #3 }
    }
}

\bool_if:NT \c__problems_boxed_bool {
  \surroundwithmdframed{problem}
}
\def\record@problem{
  \protected@write\@auxout{}
  {
    \string\@problem{\prob@number}
    {
      \tl_if_exist:NTF \l__problems_inclprob_pts_tl {
        \l__problems_inclprob_pts_tl
      }{
        \l__problems_prob_pts_tl
      }
    }%
    {
      \tl_if_exist:NTF \l__problems_inclprob_min_tl {
        \l__problems_inclprob_min_tl
      }{
        \l__problems_prob_min_tl
      }
    }
  }
}
\def\@problem#1#2#3{}
\keys_define:nn { problem / solution }{
  id            .str_set_x:N  = \l__problems_solution_id_str ,
  for           .tl_set:N     = \l__problems_solution_for_tl ,
  height        .dim_set:N    = \l__problems_solution_height_dim ,
  creators      .clist_set:N  = \l__problems_solution_creators_clist ,
  contributors  .clist_set:N  = \l__problems_solution_contributors_clist ,
  srccite       .tl_set:N     = \l__problems_solution_srccite_tl
}
\cs_new_protected:Nn \__problems_solution_args:n {
  \str_clear:N \l__problems_solution_id_str
  \tl_clear:N \l__problems_solution_for_tl
  \tl_clear:N \l__problems_solution_srccite_tl
  \clist_clear:N \l__problems_solution_creators_clist
  \clist_clear:N \l__problems_solution_contributors_clist
  \dim_zero:N \l__problems_solution_height_dim
  \keys_set:nn { problem / solution }{ #1 }
}
\newcommand\@startsolution[1][]{
  \__problems_solution_args:n { #1 }
  \@in@omtexttrue% we are in a statement.
  \bool_if:NF \c__problems_boxed_bool { \hrule }
  \smallskip\noindent
  {\textbf\prob@solution@kw :\enspace}
  \begin{small}
  \def\current@section@level{\prob@solution@kw}
  \ignorespacesandpars
}
\newcommand\startsolutions{
  \specialcomment{solution}{\@startsolution}{
    \bool_if:NF \c__problems_boxed_bool {
      \hrule\medskip
    }
    \end{small}%
  }
  \bool_if:NT \c__problems_boxed_bool {
    \surroundwithmdframed{solution}
  }
}
\newcommand\stopsolutions{\excludecomment{solution}}
\ifsolutions
  \startsolutions
\else
  \stopsolutions
\fi
\bool_if:NTF \c__problems_notes_bool {
  \newenvironment{exnote}[1][]{
    \par\smallskip\hrule\smallskip
    \noindent\textbf{\prob@note@kw : }\small
  }{
    \smallskip\hrule
  }
}{
  \excludecomment{exnote}
}
\bool_if:NTF \c__problems_notes_bool {
  \newenvironment{hint}[1][]{
    \par\smallskip\hrule\smallskip
    \noindent\textbf{\prob@hint@kw :~ }\small
  }{
    \smallskip\hrule
  }
  \newenvironment{exhint}[1][]{
    \par\smallskip\hrule\smallskip
    \noindent\textbf{\prob@hint@kw :~ }\small
  }{
    \smallskip\hrule
  }
}{
  \excludecomment{hint}
  \excludecomment{exhint}
}
\bool_if:NTF \c__problems_notes_bool {
  \newenvironment{gnote}[1][]{
    \par\smallskip\hrule\smallskip
    \noindent\textbf{\prob@gnote@kw : }\small
  }{
    \smallskip\hrule
  }
}{
  \excludecomment{gnote}
}
\newenvironment{mcb}{
  \begin{enumerate}
}{
  \end{enumerate}
}
\cs_new_protected:Nn \__problems_do_yes_param:Nn {
  \exp_args:Nx \str_if_eq:nnTF { \str_lowercase:n{ #2 } }{ yes }{
    \bool_set_true:N #1
  }{
    \bool_set_false:N #1
  }
}
\keys_define:nn { problem / mcc }{
  id        .str_set_x:N  = \l__problems_mcc_id_str ,
  feedback  .tl_set:N     = \l__problems_mcc_feedback_tl ,
  T         .default:n    = { true } ,
  T         .bool_set:N   = \l__problems_mcc_t_bool ,
  F         .default:n    = { true } ,
  F         .bool_set:N   = \l__problems_mcc_f_bool ,
  Ttext     .code:n       = {
    \__problems_do_yes_param:Nn \l__problems_mcc_Ttext_bool { #1 }
  } ,
  Ftext     .code:n       = {
    \__problems_do_yes_param:Nn \l__problems_mcc_Ftext_bool { #1 }
  }
}
\cs_new_protected:Nn \l__problems_mcc_args:n {
  \str_clear:N \l__problems_mcc_id_str
  \tl_clear:N \l__problems_mcc_feedback_tl
  \bool_set_true:N \l__problems_mcc_t_bool
  \bool_set_true:N \l__problems_mcc_f_bool
  \bool_set_true:N \l__problems_mcc_Ttext_bool
  \bool_set_false:N \l__problems_mcc_Ftext_bool
  \keys_set:nn { problem / mcc }{ #1 }
}
\newcommand\mcc[2][]{
  \l__problems_mcc_args:n{ #1 }
  \item #2
  \ifsolutions
    \\
    \bool_if:NT \l__problems_mcc_t_bool {
      % TODO!
      % \ifcsstring{mcc@T}{T}{}{\mcc@Ttext}%
    }
    \bool_if:NT \l__problems_mcc_f_bool {
      % TODO!
      % \ifcsstring{mcc@F}{F}{}{\mcc@Ftext}%
    }
    \tl_if_empty:NTF \l__problems_mcc_feedback_tl {
      !
    }{
      \l__problems_mcc_feedback_tl
    }
  \fi
} %solutions

\keys_define:nn{ problem / inclproblem }{
  id      .str_set_x:N  = \l__problems_inclprob_id_str,
  pts     .tl_set:N     = \l__problems_inclprob_pts_tl,
  min     .tl_set:N     = \l__problems_inclprob_min_tl,
  title   .tl_set:N     = \l__problems_inclprob_title_tl,
  refnum  .int_set:N    = \l__problems_inclprob_refnum_int,
  type    .tl_set:N     = \l__problems_inclprob_type_tl,
  mhrepos .str_set_x:N  = \l__problems_inclprob_mhrepos_str
}
\cs_new_protected:Nn \__problems_inclprob_args:n {
  \str_clear:N \l__problems_prob_id_str
  \tl_clear:N \l__problems_inclprob_pts_tl
  \tl_clear:N \l__problems_inclprob_min_tl
  \tl_clear:N \l__problems_inclprob_title_tl
  \tl_clear:N \l__problems_inclprob_type_tl
  \int_zero_new:N \l__problems_inclprob_refnum_int
  \str_clear:N \l__problems_inclprob_mhrepos_str
  \keys_set:nn { problem / inclproblem }{ #1 }
  \tl_if_empty:NT \l__problems_inclprob_pts_tl {
    \let\l__problems_inclprob_pts_tl\undefined
  }
  \tl_if_empty:NT \l__problems_inclprob_min_tl {
    \let\l__problems_inclprob_min_tl\undefined
  }
  \tl_if_empty:NT \l__problems_inclprob_title_tl {
    \let\l__problems_inclprob_title_tl\undefined
  }
  \tl_if_empty:NT \l__problems_inclprob_type_tl {
    \let\l__problems_inclprob_type_tl\undefined
  }
  \int_compare:nNnT \l__problems_inclprob_refnum_int = 0 {
    \let\l__problems_inclprob_refnum_int\undefined
  }
}

\cs_new_protected:Nn \__problems_inclprob_clear: {
  \let\l__problems_inclprob_id_str\undefined
  \let\l__problems_inclprob_pts_tl\undefined
  \let\l__problems_inclprob_min_tl\undefined
  \let\l__problems_inclprob_title_tl\undefined
  \let\l__problems_inclprob_type_tl\undefined
  \let\l__problems_inclprob_refnum_int\undefined
  \let\l__problems_inclprob_mhrepos_str\undefined
}
\__problems_inclprob_clear:

\newcommand\includeproblem[2][]{
  \__problems_inclprob_args:n{ #1 }
  \str_if_empty:NTF \l__problems_inclprob_mhrepos_str {
    \input{#2}
  }{
    \stex_in_repository:nn{\l__problems_inclprob_mhrepos_str}{
      \input{\mhpath{\l__problems_inclprob_mhrepos_str}{#2}}
    }
  }
  \__problems_inclprob_clear:
}
\AddToHook{enddocument}{
  \bool_if:NT \c__problems_pts_bool {
    \message{Total:~\arabic{pts}~points}
  }
  \bool_if:NT \c__problems_min_bool {
    \message{Total:~\arabic{min}~minutes}
  }
}
\def\pts#1{
  \bool_if:NT \c__problems_pts_bool {
    \marginpar{#1~\prob@pt@kw}
  }
}
\def\min#1{
  \bool_if:NT \c__problems_min_bool {
    \marginpar{#1~\prob@min@kw}
  }
}
\newcounter{pts}
\def\show@pts{
  \tl_if_exist:NTF \l__problems_inclprob_pts_tl {
    \bool_if:NT \c__problems_pts_bool {
      \marginpar{\l__problems_inclprob_pts_tl\ \prob@pt@kw\smallskip}
      \addtocounter{pts}{\l__problems_inclprob_pts_tl}
    }
  }{
    \tl_if_exist:NT \l__problems_prob_pts_tl {
      \bool_if:NT \c__problems_pts_bool {
        \marginpar{\l__problems_prob_pts_tl\ \prob@pt@kw\smallskip}
        \addtocounter{pts}{\l__problems_prob_pts_tl}
      }
    }
  }
}
\newcounter{min}
\def\show@min{
  \tl_if_exist:NTF \l__problems_inclprob_min_tl {
    \bool_if:NT \c__problems_min_bool {
      \marginpar{\l__problems_inclprob_pts_tl\ min}
      \addtocounter{min}{\l__problems_inclprob_min_tl}
    }
  }{
    \tl_if_exist:NT \l__problems_prob_min_tl {
      \bool_if:NT \c__problems_min_bool {
        \marginpar{\l__problems_prob_min_tl\ min}
        \addtocounter{min}{\l__problems_prob_min_tl}
      }
    }
  }
}
\endinput
%%
%% End of file `problem.sty'.

    %%
%% This is file `hwexam.sty',
%% generated with the docstrip utility.
%%
%% The original source files were:
%%
%% hwexam.dtx  (with options: `package')
%% 
\ProvidesExplPackage{hwexam}{2022/05/24}{3.1.0}{homework assignments and exams}
\RequirePackage{l3keys2e}

\newif\iftest\testfalse
\DeclareOption{test}{\testtrue}
\newif\ifmultiple\multiplefalse
\DeclareOption{multiple}{\multipletrue}
\DeclareOption{lang}{\PassOptionsToPackage{\CurrentOption}{stex}}
\DeclareOption*{\PassOptionsToPackage{\CurrentOption}{problem}}
\ProcessOptions
\RequirePackage{keyval}[1997/11/10]
\RequirePackage{problem}
\newcommand\hwexam@assignment@kw{Assignment}
\newcommand\hwexam@given@kw{Given}
\newcommand\hwexam@due@kw{Due}
\newcommand\hwexam@testemptypage@kw{This~page~was~intentionally~left~blank~for~extra~space}
\newcommand\hwexam@minutes@kw{minutes}
\newcommand\correction@probs@kw{prob.}
\newcommand\correction@pts@kw{total}
\newcommand\correction@reached@kw{reached}
\newcommand\correction@sum@kw{Sum}
\newcommand\correction@grade@kw{grade}
\newcommand\correction@forgrading@kw{To~be~used~for~grading,~do~not~write~here}
\AddToHook{begindocument}{
\ltx@ifpackageloaded{babel}{
\makeatletter
\clist_set:Nx \l_tmpa_clist {\bbl@loaded}
\clist_if_in:NnT \l_tmpa_clist {ngerman}{
  \input{hwexam-ngerman.ldf}
}
\clist_if_in:NnT \l_tmpa_clist {finnish}{
  \input{hwexam-finnish.ldf}
}
\clist_if_in:NnT \l_tmpa_clist {french}{
  \input{hwexam-french.ldf}
}
\clist_if_in:NnT \l_tmpa_clist {russian}{
  \input{hwexam-russian.ldf}
}
\makeatother
}{}
}

\newcounter{assignment}
\keys_define:nn { hwexam / assignment } {
id  .str_set_x:N = \l__problems_assign_id_str,
number  .int_set:N  = \l__problems_assign_number_int,
title  .tl_set:N  = \l__problems_assign_title_tl,
type  .tl_set:N  = \l__problems_assign_type_tl,
given .tl_set:N  = \l__problems_assign_given_tl,
due .tl_set:N  = \l__problems_assign_due_tl,
loadmodules .code:n  = {
\bool_set_true:N \l__problems_assign_loadmodules_bool
}
}
\cs_new_protected:Nn \__problems_assignment_args:n {
\str_clear:N \l__problems_assign_id_str
\int_set:Nn \l__problems_assign_number_int {-1}
\tl_clear:N \l__problems_assign_title_tl
\tl_clear:N \l__problems_assign_type_tl
\tl_clear:N \l__problems_assign_given_tl
\tl_clear:N \l__problems_assign_due_tl
\bool_set_false:N \l__problems_assign_loadmodules_bool
\keys_set:nn { hwexam / assignment }{ #1 }
}
\newcommand\given@due[2]{
\bool_lazy_all:nF {
{\tl_if_empty_p:V \l__problems_inclassign_given_tl}
{\tl_if_empty_p:V \l__problems_assign_given_tl}
{\tl_if_empty_p:V \l__problems_inclassign_due_tl}
{\tl_if_empty_p:V \l__problems_assign_due_tl}
}{ #1 }

\tl_if_empty:NTF \l__problems_inclassign_given_tl {
\tl_if_empty:NF \l__problems_assign_given_tl {
\hwexam@given@kw\xspace\l__problems_assign_given_tl
}
}{
\hwexam@given@kw\xspace\l__problems_inclassign_given_tl
}

\bool_lazy_or:nnF {
\bool_lazy_and_p:nn {
\tl_if_empty_p:V \l__problems_inclassign_due_tl
}{
\tl_if_empty_p:V \l__problems_assign_due_tl
}
}{
\bool_lazy_and_p:nn {
\tl_if_empty_p:V \l__problems_inclassign_due_tl
}{
\tl_if_empty_p:V \l__problems_assign_due_tl
}
}{ ,~ }

\tl_if_empty:NTF \l__problems_inclassign_due_tl {
\tl_if_empty:NF \l__problems_assign_due_tl {
\hwexam@due@kw\xspace \l__problems_assign_due_tl
}
}{
\hwexam@due@kw\xspace \l__problems_inclassign_due_tl
}

\bool_lazy_all:nF {
{ \tl_if_empty_p:V \l__problems_inclassign_given_tl }
{ \tl_if_empty_p:V \l__problems_assign_given_tl }
{ \tl_if_empty_p:V \l__problems_inclassign_due_tl }
{ \tl_if_empty_p:V \l__problems_assign_due_tl }
}{ #2 }
}
\newcommand\assignment@title[3]{
\tl_if_empty:NTF \l__problems_inclassign_title_tl {
\tl_if_empty:NTF \l__problems_assign_title_tl {
#1
}{
#2\l__problems_assign_title_tl#3
}
}{
#2\l__problems_inclassign_title_tl#3
}
}
\newcommand\assignment@number{
\int_compare:nNnTF \l__problems_inclassign_number_int = {-1} {
\int_compare:nNnTF \l__problems_assign_number_int = {-1} {
\arabic{assignment}
} {
\int_use:N \l__problems_assign_number_int
}
}{
\int_use:N \l__problems_inclassign_number_int
}
}
\newenvironment{assignment}[1][]{
\__problems_assignment_args:n { #1 } 
\int_compare:nNnTF \l__problems_assign_number_int = {-1} {
\global\stepcounter{assignment}
}{
\global\setcounter{assignment}{\int_use:N\l__problems_assign_number_int}
}
\setcounter{sproblem}{0}
\renewcommand\prob@label[1]{\assignment@number.##1}
\def\current@section@level{\document@hwexamtype}
\begin{@assignment}
}{
\end{@assignment}
}
\def\ass@title{
{\protect\document@hwexamtype}~\arabic{assignment}
\assignment@title{}{\;(}{)\;} -- \given@due{}{}
}
\ifmultiple
\newenvironment{@assignment}{
\bool_if:NTF \l__problems_assign_loadmodules_bool {
\begin{sfragment}[loadmodules]{\ass@title}
}{
\begin{sfragment}{\ass@title}
}
}{
\end{sfragment}
}
\else
\newenvironment{@assignment}{
\begin{center}\bf
\Large\@title\strut\\
\document@hwexamtype~\arabic{assignment}\assignment@title{\;}{:\;}{\\}
\large\given@due{--\;}{\;--}
\end{center}
}{}
\fi% multiple
\keys_define:nn { hwexam / inclassignment } {
number  .int_set:N  = \l__problems_inclassign_number_int,
title  .tl_set:N  = \l__problems_inclassign_title_tl,
type  .tl_set:N  = \l__problems_inclassign_type_tl,
given .tl_set:N  = \l__problems_inclassign_given_tl,
due .tl_set:N  = \l__problems_inclassign_due_tl,
mhrepos  .str_set_x:N = \l__problems_inclassign_mhrepos_str
}
\cs_new_protected:Nn \__problems_inclassignment_args:n {
\int_set:Nn \l__problems_inclassign_number_int {-1}
\tl_clear:N \l__problems_inclassign_title_tl
\tl_clear:N \l__problems_inclassign_type_tl
\tl_clear:N \l__problems_inclassign_given_tl
\tl_clear:N \l__problems_inclassign_due_tl
\str_clear:N \l__problems_inclassign_mhrepos_str
\keys_set:nn { hwexam / inclassignment }{ #1 }
}
\__problems_inclassignment_args:n {}

\newcommand\inputassignment[2][]{
\__problems_inclassignment_args:n { #1 }
\str_if_empty:NTF \l__problems_inclassign_mhrepos_str {
\input{#2}
}{
\stex_in_repository:nn{\l__problems_inclassign_mhrepos_str}{
\input{\mhpath{\l__problems_inclassign_mhrepos_str}{#2}}
}
}
\__problems_inclassignment_args:n {}
}
\newcommand\includeassignment[2][]{
\newpage
\inputassignment[#1]{#2}
}
\ExplSyntaxOff
\newcommand\quizheading[1]{%
\def\@tas{#1}%
\large\noindent NAME: \hspace{8cm}  MAILBOX:\\[2ex]%
\ifx\@tas\@empty\else%
\noindent TA:~\@for\@I:=\@tas\do{{\Large$\Box$}\@I\hspace*{1em}}\\[2ex]%
\fi%
}
\ExplSyntaxOn

\def\hwexamheader{\input{hwexam-default.header}}

\def\hwexamminutes{
\tl_if_empty:NTF \testheading@duration {
{\testheading@min}~\hwexam@minutes@kw
}{
\testheading@duration
}
}

\keys_define:nn { hwexam / testheading } {
min  .tl_set:N  = \testheading@min,
duration .tl_set:N  = \testheading@duration,
reqpts .tl_set:N  = \testheading@reqpts,
tools .tl_set:N  = \testheading@tools
}
\cs_new_protected:Nn \__problems_testheading_args:n {
\tl_clear:N \testheading@min
\tl_clear:N \testheading@duration
\tl_clear:N \testheading@reqpts
\tl_clear:N \testheading@tools
\keys_set:nn { hwexam / testheading }{ #1 }
}
\newenvironment{testheading}[1][]{
\__problems_testheading_args:n{ #1 }
\newcount\check@time\check@time=\testheading@min
\advance\check@time by -\theassignment@totalmin
\newif\if@bonuspoints
\tl_if_empty:NTF \testheading@reqpts {
\@bonuspointsfalse
}{
\newcount\bonus@pts
\bonus@pts=\theassignment@totalpts
\advance\bonus@pts by -\testheading@reqpts
\edef\bonus@pts{\the\bonus@pts}
\@bonuspointstrue
}
\edef\check@time{\the\check@time}

\makeatletter\hwexamheader\makeatother
}{
\newpage
}
\newcommand\testspace[1]{\iftest\vspace*{#1}\fi}
\newcommand\testnewpage{\iftest\newpage\fi}
\newcommand\testemptypage[1][]{\iftest\begin{center}\hwexam@testemptypage@kw\end{center}\vfill\eject\else\fi}
\renewcommand\@problem[3]{
\stepcounter{assignment@probs}
\def\__problemspts{#2}
\ifx\__problemspts\@empty\else
\addtocounter{assignment@totalpts}{#2}
\fi
\def\__problemsmin{#3}\ifx\__problemsmin\@empty\else\addtocounter{assignment@totalmin}{#3}\fi
\xdef\correction@probs{\correction@probs & #1}%
\xdef\correction@pts{\correction@pts & #2}
\xdef\correction@reached{\correction@reached &}
}
\newcounter{assignment@probs}
\newcounter{assignment@totalpts}
\newcounter{assignment@totalmin}
\def\correction@probs{\correction@probs@kw}
\def\correction@pts{\correction@pts@kw}
\def\correction@reached{\correction@reached@kw}
\stepcounter{assignment@probs}
\newcommand\correction@table{
\resizebox{\textwidth}{!}{%
\begin{tabular}{|l|*{\theassignment@probs}{c|}|l|}\hline%
&\multicolumn{\theassignment@probs}{c||}%|
{\footnotesize\correction@forgrading@kw} &\\\hline
\correction@probs & \correction@sum@kw & \correction@grade@kw\\\hline
\correction@pts &\theassignment@totalpts & \\\hline
\correction@reached & & \\[.7cm]\hline
\end{tabular}}}
\endinput
%%
%% End of file `hwexam.sty'.

  \end{omgroup}

\end{omgroup}

% \iffalse meta-comment
% An Infrastructure for Semantic Macros and Module Scoping
% Copyright (c) 2019 Michael Kohlhase, all rights reserved
%                this file is released under the
%                LaTeX Project Public License (LPPL)
% 
% The original of this file is in the public repository at 
% http://github.com/sLaTeX/sTeX/
%
% TODO update copyright  
%
%<*driver>
\providecommand\bibfolder{../../lib/bib}
\RequirePackage{paralist}
\documentclass[full,kernel]{l3doc}
\usepackage[dvipsnames]{xcolor}
\usepackage[utf8]{inputenc}
\usepackage[T1]{fontenc}
%\usepackage{document-structure}
\usepackage[showmods,debug=all,lang={de,en}, mathhub=./tests]{stex}
\usepackage{url,array,float,textcomp}
\usepackage[show]{ed}
\usepackage[hyperref=auto,style=alphabetic]{biblatex}
\addbibresource{\bibfolder/kwarcpubs.bib}
\addbibresource{\bibfolder/extpubs.bib}
\addbibresource{\bibfolder/kwarccrossrefs.bib}
\addbibresource{\bibfolder/extcrossrefs.bib}
\usepackage{amssymb}
\usepackage{amsfonts}
\usepackage{xspace}
\usepackage{hyperref}

\usepackage{morewrites}


\makeindex
\floatstyle{boxed}
\newfloat{exfig}{thp}{lop}
\floatname{exfig}{Example}

\usepackage{stex-tests}

\MakeShortVerb{\|}

\def\scsys#1{{{\sc #1}}\index{#1@{\sc #1}}\xspace}
\def\mmt{\textsc{Mmt}\xspace}
\def\xml{\scsys{Xml}}
\def\mathml{\scsys{MathML}}
\def\omdoc{\scsys{OMDoc}}
\def\openmath{\scsys{OpenMath}}
\def\latexml{\scsys{LaTeXML}}
\def\perl{\scsys{Perl}}
\def\cmathml{Content-{\sc MathML}\index{Content {\sc MathML}}\index{MathML@{\sc MathML}!content}}
\def\activemath{\scsys{ActiveMath}}
\def\twin#1#2{\index{#1!#2}\index{#2!#1}}
\def\twintoo#1#2{{#1 #2}\twin{#1}{#2}}
\def\atwin#1#2#3{\index{#1!#2!#3}\index{#3!#2 (#1)}}
\def\atwintoo#1#2#3{{#1 #2 #3}\atwin{#1}{#2}{#3}}
\def\cT{\mathcal{T}}\def\cD{\mathcal{D}}

\def\fileversion{3.0}
\def\filedate{\today}

\RequirePackage{pdfcomment}

\ExplSyntaxOn\makeatletter
\cs_set_protected:Npn \@comp #1 #2 {
  \pdftooltip {
    \textcolor{blue}{#1}
  } { #2 }
}

\cs_set_protected:Npn \@defemph #1 #2 {
  \pdftooltip { 
    \textbf{\textcolor{magenta}{#1}}
  } { #2 }
}
\makeatother\ExplSyntaxOff

\begin{document}
  \DocInput{\jobname.dtx}
\end{document}
%</driver>
% \fi
%
% \title{ \sTeX-Basics
% 	\thanks{Version {\fileversion} (last revised {\filedate})} 
% }
%
% \author{Michael Kohlhase, Dennis Müller\\
% 	FAU Erlangen-Nürnberg\\
% 	\url{http://kwarc.info/}
% }
%
% \maketitle
%
%\ifinfulldoc\else
% This is the documentation for the \pkg{stex-basics} package.
% For a more high-level introduction, 
%  see \href{\basedocurl/manual.pdf}{the \sTeX Manual} or the
% \href{\basedocurl/stex.pdf}{full \sTeX documentation}.
%
% Both the \pkg{stex} package and document class offer the following
options:

\begin{description}
   \item[\texttt{lang}] (\meta{language}$\ast$) Languages
     to load with the \pkg{babel} package.
   \item[\texttt{mathhub}] (\meta{directory}) MathHub folder
     to search for repositories.
   \item[\texttt{sms}] (\meta{boolean}) use \emph{persisted}
     mode (not yet implemented).
   \item[\texttt{image}] (\meta{boolean}) passed on to
     \pkg{tikzinput}.
   \item[\texttt{debug}] (\meta{log-prefix}$\ast$) Logs debugging
     information with the given prefixes to the terminal,
     or all if |all| is given.
\end{description}
% \fi
%
% \begin{documentation}\label{pkg:basics:doc}
%
% Both the \sTeX package and class offer the following
% package options:
%
% \begin{description}
%   \item[\texttt{debug}] (\meta{log-prefix}$\ast$) Logs debugging
%     information with the given prefixes to the terminal,
%     or all if |all| is given.
%   \item[\texttt{showmods}] (\meta{boolean}) Shows explicit
%     module information at the document margins.
%   \item[\texttt{lang}] (\meta{language}$\ast$) Languages
%     to load with the \pkg{babel} package.
%   \item[\texttt{mathhub}] (\meta{directory}) MathHub folder
%     to search for repositories.
%   \item[\texttt{sms}] (\meta{boolean}) use \emph{persisted}
%     mode (see ???).
%   \item[\texttt{image}] (\meta{boolean}) passed on to
%     \pkg{tikzinput}.
% \end{description}
%
% \section{Macros and Environments}\label{pkg:basics:doc:macros}
%
% \begin{function}{\sTeX , \stex}
%   Both print this \stex logo.
% \end{function}
%
% \begin{function}{\stex_debug:nn}
%   \begin{syntax}
%     \cs{stex_debug:nn} \Arg{log-prefix} \Arg{message} ^^A \meta{comma list}
%   \end{syntax}
% Logs \meta{message}, if the package option |debug| contains \meta{log-prefix}.
% \end{function}
%
% \begin{function}{\stex_add_to_sms:n}
% Adds the provided code to the |.sms|-file of the document.
% \end{function}
%
% \begin{function}{\if@latexml,\latexml_if_p:,\latexml_if:T,\latexml_if:F,\latexml_if:TF}
%   \LaTeX2e and \LaTeX3 conditionals for \latexml.
% \end{function}
%
% We have four macros for annotating generated HTML (via \latexml
% or \rustex) with attributes:
%
% \begin{function}{\stex_annotate:nnn, \stex_annotate_invisible:nnn,
%   \stex_annotate_invisible:n}
%   \begin{syntax} \cs{stex_annotate:nnn} \Arg{property} \Arg{resource} \Arg{content} \end{syntax}
% Annotates the HTML generated by \meta{content} with\\
% \begin{center}
%  |property="stex:|\meta{property}|", resource="|\meta{resource}|"|.
% \end{center}
%
% \cs{stex_annotate_invisible:n} adds the attributes\\
% \begin{center}
% |stex:visible="false", style="display:none"|.
% \end{center}
%
% \cs{stex_annotate_invisible:nnn} combines the functionality of both.
% \end{function}
%
% \begin{environment}{stex_annotate_env}
%   \begin{syntax} \cs{begin}|{stex_annotate_env}|\Arg{property}\Arg{resource}
%       \meta{content}
%     \cs{end}|{stex_annotate_env}|
%\end{syntax}
% behaves like \cs{stex_annotate:nnn} \Arg{property} \Arg{resource}
%     \Arg{content}.
% \end{environment}
%
% \begin{variable}{\c_stex_languages_prop,\c_stex_language_abbrevs_prop}
%   Map language abbreviations to their full babel names and vice versa.
%   e.g. \cs{c_stex_languages_prop}|{en}| yields |english|, and
%   \cs{c_stex_language_abbrevs_prop}|{english}| yields |en|.
% \end{variable}
%
% \begin{function}{\stex_deactivate_macro:Nn , \stex_reactivate_macro:N}
%   \begin{syntax}\cs{stex_deactivate_macro:Nn}\meta{cs}\Arg{environments}\end{syntax}
%   Makes the macro \meta{cs} throw an error, indicating that it
%   is only allowed in the context of \meta{environments}.
%
%   \cs{stex_reactivate_macro:N}\meta{cs} reactivates it again, i.e.
%   this happens ideally in the \meta{begin}-code of the associated
%   environments.
% \end{function}
%
% \begin{function}{\MSC}
%   \begin{syntax}\cs{MSC}\Arg{msc} \end{syntax}
%   Designates the \emph{math subject classifier} of the current module / file.
% \end{function}
%
% \end{documentation}
%
% \begin{implementation}
%
% \section{\sTeX-Basics Implementation}\label{pkg:basics:impl}
%
%   \subsection{The \sTeX Document Class}
%
% The \cls{stex} document class is pretty straight-forward: It largely extends the \cls{standalone} package
% and loads the \pkg{stex} package, passing all provided options on to the package.
%
%    \begin{macrocode}
%<*cls>

%%%%%%%%%%%%%   basics.dtx   %%%%%%%%%%%%%

\RequirePackage{expl3,l3keys2e}
\ProvidesExplClass{stex}{2021/08/01}{1.9}{bla}
\LoadClass[border=1px,varwidth]{standalone}
\setlength\textwidth{15cm}

\DeclareOption*{\PassOptionsToPackage{\CurrentOption}{stex}}
\ProcessOptions

\RequirePackage{stex}
%</cls>
%    \end{macrocode}
%
% \subsection{Preliminaries}
%
%    \begin{macrocode}
%<*package>

%%%%%%%%%%%%%   basics.dtx   %%%%%%%%%%%%%

\RequirePackage{expl3,l3keys2e,ltxcmds}
\ProvidesExplPackage{stex}{2021/08/01}{1.9}{bla}
\RequirePackage{expl-keystr-compat}

%\RequirePackage{morewrites}
%\RequirePackage{amsmath}

%    \end{macrocode}
%
% Package options:
%
%    \begin{macrocode}
\keys_define:nn { stex } {
  debug     .clist_set:N  = \c_stex_debug_clist ,
  showmods  .bool_set:N   = \c_stex_showmods_bool ,
  lang      .clist_set:N  = \c_stex_languages_clist ,
  mathhub   .tl_set_x:N   = \mathhub ,
  sms       .bool_set:N   = \c_stex_persist_mode_bool ,
  image     .bool_set:N   = \c_tikzinput_image_bool,
  unknown   .code:n       = {}
}
\ProcessKeysOptions { stex }
%    \end{macrocode}
%
% \begin{macro}{\stex,\sTeX}
%   The \sTeX logo:
%
%    \begin{macrocode}
\protected\def\stex{%
  \@ifundefined{texorpdfstring}%
  {\let\texorpdfstring\@firstoftwo}%
  {}%
  \texorpdfstring{\raisebox{-.5ex}S\kern-.5ex\TeX}{sTeX}\xspace%
}
\def\sTeX{\stex}
%    \end{macrocode}
% \end{macro}
%
%
% \subsection{Messages and logging}
%
%    \begin{macrocode}
%<@@=stex_log>
%    \end{macrocode}
%
% Warnings and error messages
%
%    \begin{macrocode}
\msg_new:nnn{stex}{error/unknownlanguage}{
  Unknown~language:~#1
}
\msg_new:nnn{stex}{warning/nomathhub}{
  MATHHUB~system~variable~not~found~and~no~
  \detokenize{\mathhub}-value~set!
}
\msg_new:nnn{stex}{error/deactivated-macro}{
  The~\detokenize{#1}~command~is~only~allowed~in~#2!
}
%    \end{macrocode}
% 
% \begin{macro}{\stex_debug:nn}
%
%  A simple macro issuing package messages with subpath.
%
%    \begin{macrocode}
\cs_new_protected:Nn \stex_debug:nn {
  \clist_if_in:NnTF \c_stex_debug_clist { all } {
    \exp_args:Nnnx\msg_set:nnn{stex}{debug / #1}{
      \\Debug~#1:~#2\\
    }
    \msg_none:nn{stex}{debug / #1}
  }{
    \clist_if_in:NnT \c_stex_debug_clist { #1 } {
      \exp_args:Nnnx\msg_set:nnn{stex}{debug / #1}{
        \\Debug~#1:~#2\\
      }
      \msg_none:nn{stex}{debug / #1}
    }  
  }
}
%    \end{macrocode}
% \end{macro}
%
% Redirecting messages:
%
%    \begin{macrocode}
\clist_if_in:NnTF \c_stex_debug_clist {all} {
    \msg_redirect_module:nnn{ stex }{ none }{ term }
}{
  \clist_map_inline:Nn \c_stex_debug_clist {
    \msg_redirect_name:nnn{ stex }{ debug / ##1 }{ term }
  }
}

\stex_debug:nn{log}{debug~mode~on}
%    \end{macrocode}
%
% \subsection{Persistence}
%
%    \begin{macrocode}
%<@@=stex_persist>
%    \end{macrocode}
%
% \begin{variable}{\c_@@_sms_iow}
%
%   File variable used for the sms-File
%
%    \begin{macrocode}
\iow_new:N \c_@@_sms_iow
\AddToHook{begindocument}{
  \bool_if:NTF \c_stex_persist_mode_bool {
    \ExplSyntaxOn \input{\jobname.sms} \ExplSyntaxOff
  } {
%    \iow_open:Nn \c_@@_sms_iow {\jobname.sms}
  }
}
\AddToHook{enddocument}{
  \bool_if:NF \c_stex_persist_mode_bool {
%    \iow_close:N \c_@@_sms_iow
  }
}
%    \end{macrocode}
% \end{variable}
%
% \begin{macro}{\stex_add_to_sms:n}
%
% Adds the provided code to the |.sms|-file of the document.
%
%    \begin{macrocode}
\cs_new_protected:Nn \stex_add_to_sms:n {
  \bool_if:NF \c_stex_persist_mode_bool {
%    \iow_now:Nn \c_@@_sms_iow { #1 }
  }
}
%    \end{macrocode}
% \end{macro}
%
% \subsection{HTML Annotations}
%    \begin{macrocode}
%<@@=stex_annotate>
\RequirePackage{rustex}
%    \end{macrocode}
%
% We add the namespace abbreviation |ns:stex="http://kwarc.info/ns/sTeX"| to \rustex:
%
%    \begin{macrocode}
\rustex_add_Namespace:nn{stex}{http://kwarc.info/ns/sTeX}
%    \end{macrocode}
%
% \begin{macro}{\if@latexml}
% \begin{macro}[pTF]{\latexml_if:}
%
% Conditionals for \latexml:
%
%    \begin{macrocode}
\ifcsname if@latexml\endcsname\else
    \expandafter\newif\csname if@latexml\endcsname\@latexmlfalse
\fi

\prg_new_conditional:Nnn \latexml_if: {p, T, F, TF} {
  \if@latexml
    \prg_return_true:
  \else:
    \prg_return_false:
  \fi:
}
%    \end{macrocode}
% \end{macro}
% \end{macro}
%
% \begin{variable}{\l_@@_arg_tl, \c_@@_emptyarg_tl}
%
% Used by annotation macros to ensure that the HTML output to annotate
% is not empty.
%
%    \begin{macrocode}
\tl_new:N \l_@@_arg_tl
\tl_const:Nx \c_@@_emptyarg_tl {
  \rustex_if:TF {
    \rustex_direct_HTML:n { \c_ampersand_str lrm; }
  }{~}
}
%    \end{macrocode}
% \end{variable}
%
% \begin{macro}{\_@@_checkempty:n}
%    \begin{macrocode}
\cs_new_protected:Nn \_@@_checkempty:n {
  \tl_set:Nn \l_@@_arg_tl { #1 }
  \tl_if_empty:NT \l_@@_arg_tl {
    \tl_set_eq:NN \l_@@_arg_tl \c_@@_emptyarg_tl
  }
}
%    \end{macrocode}
% \end{macro}
%
% \begin{macro}{\l_stex_html_do_output_bool,\stex_if_do_html:}
%  Whether to (locally) produce HTML output
%    \begin{macrocode}
\bool_new:N \l_stex_html_do_output_bool
\bool_set_true:N \l_stex_html_do_output_bool
\prg_new_conditional:Nnn \stex_if_do_html: {p,T,F,TF} {
  \bool_if:nTF \l_stex_html_do_output_bool
    \prg_return_true: \prg_return_false:
}
%    \end{macrocode}
% \end{macro}
%
% \begin{macro}{\stex_suppress_html:n}
%  Whether to (locally) produce HTML output
%    \begin{macrocode}
\cs_new_protected:Nn \stex_suppress_html:n {
  \exp_args:Nne \use:nn {
    \bool_set_false:N \l_stex_html_do_output_bool
    #1
  }{
    \stex_if_do_html:T {
      \bool_set_true:N \l_stex_html_do_output_bool
    }
  }
}
%    \end{macrocode}
% \end{macro}
%
%
% \begin{environment}{stex_annotate_env}
% \begin{macro}{\stex_annotate:nnn, \stex_annotate_invisible:n,
%    \stex_annotate_invisible:nnn}
%
% We define four macros for introducing attributes in the HTML
% output. The definitions depend on the ``backend'' used
% (\latexml, \rustex, \texttt{pdflatex}). 
%
% The \texttt{pdflatex}-macros largely do nothing; the
% \rustex-implementations are pretty clear in what they do,
%  the \latexml-implementations resort to perl bindings.
%
%    \begin{macrocode}
\rustex_if:TF{
  \cs_new_protected:Nn \stex_annotate:nnn {
    \_@@_checkempty:n { #3 }
    \rustex_annotate_HTML:nn {
      property="stex:#1" ~
      resource="#2"
    } {
      \mode_if_vertical:TF{
        \tl_use:N \l_@@_arg_tl\par
      }{
        \tl_use:N \l_@@_arg_tl
      }
    }
  }
  \cs_new_protected:Nn \stex_annotate_invisible:n {
    \_@@_checkempty:n { #1 }
    \rustex_annotate_HTML:nn {
      stex:visible="false" ~
      style:display="none"
    } {
      \mode_if_vertical:TF{
        \tl_use:N \l_@@_arg_tl\par
      }{
        \tl_use:N \l_@@_arg_tl
      }
    }
  }
  \cs_new_protected:Nn \stex_annotate_invisible:nnn {
    \_@@_checkempty:n { #3 }
    \rustex_annotate_HTML:nn {
      property="stex:#1" ~
      resource="#2" ~
      stex:visible="false" ~
      style:display="none"
    } {
      \mode_if_vertical:TF{
        \tl_use:N \l_@@_arg_tl\par
      }{
        \tl_use:N \l_@@_arg_tl
      }
    }
  }
  \NewDocumentEnvironment{stex_annotate_env} { m m } {
    \par
    \rustex_annotate_HTML_begin:n {
      property="stex:#1" ~
      resource="#2"
    }
  }{
    \par\rustex_annotate_HTML_end:
  }
}{
  \latexml_if:TF {
    \cs_new_protected:Nn \stex_annotate:nnn {
      \_@@_checkempty:n { #3 }
      \mode_if_math:TF {
        \cs:w latexml@annotate@math\cs_end:{#1}{#2}{
          \tl_use:N \l_@@_arg_tl
        }
      }{
        \cs:w latexml@annotate@text\cs_end:{#1}{#2}{
          \tl_use:N \l_@@_arg_tl
        }
      }
    }
    \cs_new_protected:Nn \stex_annotate_invisible:n {
      \_@@_checkempty:n { #1 }
      \mode_if_math:TF {
        \cs:w latexml@invisible@math\cs_end:{
          \tl_use:N \l_@@_arg_tl
        }
      } {
        \cs:w latexml@invisible@text\cs_end:{
          \tl_use:N \l_@@_arg_tl
        }
      }
    }
    \cs_new_protected:Nn \stex_annotate_invisible:nnn {
      \_@@_checkempty:n { #3 }
      \cs:w latexml@annotate@invisible\cs_end:{#1}{#2}{
        \tl_use:N \l_@@_arg_tl
      }
    }
    \NewDocumentEnvironment{stex_annotate_env} { m m } {
      \par\begin{latexml@annotateenv}{#1}{#2}
    }{
      \par\end{latexml@annotateenv}
    }
  }{
    \cs_new_protected:Nn \stex_annotate:nnn {#3}
    \cs_new_protected:Nn \stex_annotate_invisible:n {}
    \cs_new_protected:Nn \stex_annotate_invisible:nnn {}
    \NewDocumentEnvironment{stex_annotate_env} { m m } {}{}
  }
}
%    \end{macrocode}
% \end{macro}
% \end{environment}
%
% \subsection{Languages}
%    \begin{macrocode}
%<@@=stex_language>
%    \end{macrocode}
%
% \begin{variable}{\c_stex_languages_prop,\c_stex_language_abbrevs_prop}
%
% We store language abbreviations in two (mutually inverse) 
% property lists:
%    \begin{macrocode}
\prop_const_from_keyval:Nn \c_stex_languages_prop {
  en = english ,
  de = ngerman ,
  ar = arabic ,
  bg = bulgarian ,
  ru = russian ,
  fi = finnish ,
  ro = romanian ,
  tr = turkish ,
  fr = french
}

\prop_const_from_keyval:Nn \c_stex_language_abbrevs_prop {
  english   = en ,
  ngerman   = de ,
  arabic    = ar ,
  bulgarian = bg ,
  russian   = ru ,
  finnish   = fi ,
  romanian  = ro ,
  turkish   = tr ,
  french    = fr
}
% todo: chinese simplified (zhs)
%       chinese traditional (zht)
%    \end{macrocode}
% \end{variable}
%
% we use the |lang|-package option to load the corresponding
% babel languages:
%
%    \begin{macrocode}
\clist_if_empty:NF \c_stex_languages_clist {
  \clist_clear:N \l_tmpa_clist
  \clist_map_inline:Nn \c_stex_languages_clist {
    \prop_get:NnNTF \c_stex_languages_prop { #1 } \l_tmpa_str {
      \clist_put_right:No \l_tmpa_clist \l_tmpa_str
    } {
      \msg_error:nnx{stex}{error/unknownlanguage}{\l_tmpa_str}
    }
  }
  \stex_debug:nn{lang} {Languages:~\clist_use:Nn \l_tmpa_clist {,~} }
  \RequirePackage[\clist_use:Nn \l_tmpa_clist,]{babel}
}
%    \end{macrocode}
%
% \subsection{Activating/Deactivating Macros}
%
% \begin{macro}{\stex_deactivate_macro:Nn}
%    \begin{macrocode}
\cs_new_protected:Nn \stex_deactivate_macro:Nn {
  \exp_after:wN\let\csname \detokenize{#1} - orig\endcsname#1
  \def#1{
    \msg_error:nnxx{stex}{error/deactivated-macro}{#1}{#2}
  }
}
%    \end{macrocode}
% \end{macro}
%
% \begin{macro}{\stex_reactivate_macro:N}
%    \begin{macrocode}
\cs_new_protected:Nn \stex_reactivate_macro:N {
  \exp_after:wN\let\exp_after:wN#1\csname \detokenize{#1} - orig\endcsname
}
%    \end{macrocode}
% \end{macro}
%
%
%    \begin{macrocode}
%</package>
%    \end{macrocode}
%
% \end{implementation}
%
% \PrintIndex


\chapter{Stuff}

\section{Modules}


\begin{function}{\sTeX , \stex}
  Both print this \stex logo.
\end{function}

 \subsection{Semantic Macros and Notations}

 Semantic macros invoke a formally declared symbol.

 To declare a symbol (in a module), we use \cs{symdecl},
 which takes as argument the name of the corresponding
 semantic macro, e.g. |\symdecl{foo}| introduces the macro
 \cs{foo}. Additionally, \cs{symdecl} takes several options,
 the most important one being its arity. |foo| as declared above
 yields a \emph{constant} symbol. To introduce an \emph{operator}
 which takes arguments, we have to specify which arguments it takes.

 \begin{@module}{SemanticMacrosExample}
   For example, to introduce binary multiplication,
   we can do |\symdecl[args=2]{mult}|. We can then supply
   the semantic macro with arbitrarily many notations, such as
   |\notation{mult}{#1 #2}|.
   
   \stexexample{
 \symdecl[args=2]{mult}
 \notation{mult}{#1 #2}
 $\mult{a}{b}$
}

 Since usually, a freshly introduced symbol also comes with a
 notation from the start, the \cs{symdef} command combines
 \cs{symdecl} and \cs{notation}. So instead of the above,
 we could have also written
 \begin{center} |\symdef[args=2]{mult}{#1 #2}| \end{center}

 \symdecl[args=2]{mult}
 \notation{mult}{#1 #2}

   \notation[cdot]{mult}{#1 \comp{\cdot} #2}
   \notation[times]{mult}{#1 \comp{\times} #2}
   Adding more notations like
   |\notation[cdot]{mult}{#1 \comp{\cdot} #2}| or 
   |\notation[times]{mult}{#1 \comp{\times} #2}|
   allows us to write |$\mult[cdot]{a}{b}$| and
   |$\mult[times]{a}{b}$|:
   \stexexample{
   \notation[cdot]{mult}{#1 \comp{\cdot} #2}
   \notation[times]{mult}{#1 \comp{\times} #2}
 $\mult[cdot]{a}{b}$ and $\mult[times]{a}{b}$
}
   \notation[cdot]{mult}{#1 \comp{\cdot} #2}
   \notation[times]{mult}{#1 \comp{\times} #2}

   Not using an explicit option with a semantic macro yields
   the first declared notation, unless changed\ednote{TODO}.

   Outside of math mode, or by using the starred variant
   |\foo*|, allows to provide a custom notation, where
   notational (or textual) components can be given
   explicitly in square brackets.
   \stexexample{
 $\mult*{a}[\comp{\ast}]{b}$ is the 
 \mult[\comp{product of} ]{$a$}[ \comp{and} ]{$b$}
}

   In custom mode, prefixing an argument with a star will not
   print that argument, but still export it to \omdoc:
   \stexexample{
 \mult[\comp{Multiplying}]*{$\mult{a}{b}$}[ again by ]{$b$} yields...
}
   The syntax |*[|\meta{int}|]| allows switching
   the order of arguments. For example, given a 2-ary semantic
   macro |\forevery| with exemplary notation
   |\forall #1. #2|, we can write
   \stexexample{
     \symdecl[args=2]{forevery}
     \forevery*[2]{The proposition $P$}[ \comp{holds for every} ]*[1]{$x\in A$}
}

 When using |*[|$n$|]|, after reading the provided ($n$th) argument,
  the ``argument counter'' automatically 
 continues where we left off, so the |*[1]| in the above example
 can be omitted.

   For a macro with arity $>0$, we can refer to the operator
   \emph{itself} semantically by suffixing the semantic macro
   with an exclamation point |!| in either text or math mode.
   For that reason \cs{notation} (and thus \cs{symdef}) take an
   additional optional argument |op=|, which allows to assign
   a notation for the operator itself. e.g.
   \stexexample{
     \symdef[args=2,op={+}]{add}{#1 \comp+ #2}
     The operator $\add!$ adds two elements, as in $\add ab$.
   }

  |*| is composable with |!| for custom notations, as in:

   \stexexample{
 \mult![\comp{Multiplication}] (denoted by $\mult*![\comp\cdot]$) is defined by...
}

 The macro \cs{comp} as used everywhere above is responsible
 for highlighting, linking, and tooltips, and should be wrapped
 around the notation (or text) components that should be treated
 accordingly. While it is attractive to just wrap a whole notation,
 this would also wrap around e.g. the arguments themselves, so
 instead, the user is tasked with marking the notation components
 themself.

 The precise behaviour of \cs{comp} is governed by
 the macro \cs{@comp}, which takes two arguments: The tex code
 of the text
 (unexpanded) to highlight, and the URI of the current symbol.
 \cs{@comp} can be safely redefined to customize the behaviour.


 The starred variant |\symdecl*{foo}| does not introduce a semantic
 macro, but still declares a corresponding symbol. |foo| (like
 any other symbol, for that matter) can
 then be accessed via \cs{STEXsymbol}|{foo}| or (if |foo| was declared
 in a module |Foo|) via \cs{STEXModule}|{Foo}?{foo}|.

 both \cs{STEXsymbol} and \cs{STEXModule} take any
 arbitrary ending segment of a full URI to determine
 which symbol or module is meant. e.g.
 \cs{STEXsymbol}|{Foo?foo}| is also valid, as are e.g.
 \cs{STEXModule}|{path?Foo}?{foo}| or
 \cs{STEXsymbol}|{path?Foo?foo}|

 There's also a convient shortcut \cs{symref}|{?foo}{some text}| for
 \cs{STEXsymbol}|{?foo}![some text]|.

 \end{@module}

 \subsubsection{Other Argument Types}

 So far, we have stated the arity of a semantic macro directly.
 This works if we only have ``normal'' (or more precisely: |i|-type) arguments.
  To make use of other argument types, instead of providing the arity
 numerically, we can provide it as a sequence of characters representing
 the argument types -- e.g. instead of writing |args=2|, we
 can equivalently write |args=ii|, indicating that the macro
 takes two |i|-type arguments.

 Besides |i|-type arguments, \sTeX has two other types, which we will
 discuss now.

 The first are \emph{binding} (|b|-type) arguments, representing
 variables that are \emph{bound} by the operator. This is the
 case for example in the above \cs{forevery}-macro:
 The first argument is not actually an argument that the
 |forevery| ``function'' is ``applied'' to; rather, the first argument
 is a new variable (e.g. $x$) that is \emph{bound} in the subsequent
 argument. More accurately, the macro should therefore have been
 implemented thusly:
   \begin{center}|\symdef[args=bi]{forevery}{\forall #1.\; #2}|\end{center}

 \begin{@module}{OtherArgs}
 |b|-type arguments are indistinguishable from |i|-type arguments
 within \sTeX, but are treated very differently in \omdoc and by \mmt.
 More interesting \emph{within} \sTeX are |a|-type arguments,
 which represent (associative) arguments of flexible arity, which are
 provided as comma-separated lists.
 This allows e.g. better representing the \cs{mult}-macro above:
 
   \stexexample{
 \symdef[args=a]{mult}{#1}{#1 \comp\cdot #2}
 $\mult{a,b,c,{d^e},f}$
}
 As the example above shows, notations get a little more complicated
 for associative arguments. For every |a|-type argument, the
 \cs{notation}-macro takes an additional argument that declares
 how individual entries in an |a|-type argument list are aggregated.
 The first notation argument then describes how the aggregated
 expression is combined into the full representation.

 For a more interesting example, consider a flexary operator
 for ordered sequences in ordered set, that taking 
 arguments |{a,b,c}| and |\mathbb{R}| prints
 $a \leq b \leq c\in \mathbb R$. This operator takes
 two arguments (an |a|-type argument and an |i|-type argument),
 aggregates the individuals of the associative argument using |\leq|,
 and combines the result with |\in| and the second argument thusly:

   \stexexample{
 \symdef[args=ai]{numseq}{#1 \comp\in #2}{#1 \comp\leq #2}
 $\numseq{a,b,c}{\mathbb R}$
}

 Finally, |B|-type arguments combine the functionalities of |a|
 and |b|, i.e. they represent flexary binding operator arguments.

\ednote{what about e.g. \detokenize{\int_x\int_y\int_z f dx dy dz}?}
\ednote{``decompose'' a-type arguments into fixed-arity operators?}

 \end{@module}

 \subsubsection{Precedences}

 Every notation has an (upwards) \emph{operator precedence} and
 for each argument a (downwards) \emph{argument precedence}
 used for automated bracketing. For example, a notation
 for a binary operator \cs{foo} could be declared like this:
 \begin{center} |\notation[prec=200;500x600]{foo}{#1 \comp{+} #2}| \end{center}
 assigning an operator precedence of 200, an argument precedence
 of 500 for the first argument, and an argument precedence of 600
 for the second argument.

 \sTeX insert brackets thusly: Upon encountering a semantic
 macro (such as \cs{foo}), its operator precedence (e.g. 200)
 is compared to the current downwards precedence (initially 
 \cs{neginfprec}). If the operator precedence is \emph{larger}
 than the current downwards precedence, parentheses are inserted
 around the semantic macro.

 Notations for symbols of arity 0 have a default precedence of \cs{infprec},
 i.e. by default, parentheses are never inserted around constants.
 Notations for symbols with arity $>0$ have a default operator
 precedence of $0$.
 If no argument precedences are explicitly provided, then by
 default they are equal to the operator precedence.

 Consequently, if some operator $A$ should bind stronger than
 some operator $B$, then $A$s operator precedence should be
 smaller than $B$s argument precedences.

 For example:
 \begin{@module}{NotationsEx}
 \symdecl[args=2]{plus}
 \symdecl[args=2]{times}
 \stexexample{
\notation[prec=100]{plus}{#1 \comp{+} #2}
\notation[prec=50]{times}{#1 \comp{\cdot} #2}
$\plus{a}{\times{b}{c}}$ and $\times{a}{\plus{b}{c}}$
}


 \end{@module}

 \subsection{Archives and Imports}

 \subsubsection{Namespaces}
   Ideally, \sTeX would use arbitrary URIs for modules, with no
   forced relationships between the \emph{logical} namespace
   of a module and the \emph{physical} location of the file
   declaring the module -- like \mmt does things.

   Unfortunately, \TeX\ only provides very restricted access to
   the file system, so we are forced to generate namespaces
   systematically in such a way that they reflect the physical
   location of the associated files, so that \sTeX can resolve
   them accordingly. Largely, users need not concern themselves
   with namespaces at all, but for completenesses sake, we describe
   how they are constructed:

   \begin{itemize}
     \item If \cs{begin}|{module}{Foo}| occurs in a file
       |/path/to/file/Foo[.|\meta{lang}|].tex| which does not belong
       to an archive, the namespace is |file://path/to/file|.
     \item If the same statement occurs in a file
       |/path/to/file/bar[.|\meta{lang}|].tex|, the namespace is 
       |file://path/to/file/bar|.
   \end{itemize}

   In other words: outside of archives, the namespace corresponds to
   the file URI with the filename dropped iff it is equal to the
   module name, and ignoring the (optional) language suffix^^A
   \footnote{which is internally attached to the module name instead,
   but a user need not worry about that.}.

   If the current file is in an archive, the procedure is the same
   except that the initial segment of the file path up to the archive's
   |source|-folder is replaced by the archive's namespace URI.

 \subsubsection{Paths in Import-Statements}

 Conversely, here is how namespaces/URIs and file paths are computed
 in import statements, examplary \cs{importmodule}:

 \begin{itemize}
   \item \cs{importmodule}|{Foo}| outside of an archive refers 
     to module |Foo| in the current namespace. Consequently, |Foo|
     must have been declared earlier in the same document or, if not,
     in a file |Foo[.|\meta{lang}|].tex| in the same directory.
   \item The same statement \emph{within} an archive refers to either
     the module |Foo| declared earlier in the same document, or
     otherwise to the module |Foo| in the archive's top-level namespace.
     In the latter case, is has to be declared in a file |Foo[.|\meta{lang}|].tex|
     directly in the archive's |source|-folder.
   \item Similarly, in \cs{importmodule}|{some/path?Foo}| the path
     |some/path| refers to either the sub-directory and relative 
     namespace path of the current directory and namespace outside of an archive,
     or relative to the current archive's top-level namespace and |source|-folder,
     respectively.

     The module |Foo| must either be declared in the file
     \meta{top-directory}|/some/path/Foo[.|\meta{lang}|].tex|, or in
     \meta{top-directory}|/some/path[.|\meta{lang}|].tex| (which are
     checked in that order).
   \item Similarly, \cs{importmodule}|[Some/Archive]{some/path?Foo}|
     is resolved like the previous cases, but relative to the archive
     |Some/Archive| in the mathhub-directory.
   \item Finally, \cs{importmodule}|{full://uri?Foo}| naturally refers to the
     module |Foo| in the namespace |full://uri|. Since the file this module
     is declared in can not be determined directly from the URI, the module
     must be in memory already, e.g. by being referenced earlier in the
     same document.

     Since this is less compatible with a modular development, using full
     URIs directly is discouraged.

 \end{itemize} 


	
	
\csname if@infulldoc\endcsname\else\end{document}\fi
