\begin{sfragment}{Setting up the \sTeX Package}

    \begin{sfragment}[id=sec.minimal-setup]{Minimal Setup for the \sTeX Package}
        In the best of all worlds, there is no setup, as you already have a new version of
        {\TeX}Live on your system as a {\LaTeX} enthusiast. If not now is the time to
        install it; see \cite{TeXLive:on}. You can usually update {\TeX}Live via a package
        manager or the {\TeX}Live manager \textbf{tlmgr}.
        \sTeX requires a \TeX{} kernel newer than February 2022. 

        Alternatively, you can install \sTeX from CTAN, the Comprehensive {\TeX} Archive
        Network; see \cite{stexCTAN:on} for details. We
        assume you have the \sTeX package in at least version 3.2 (September 2022).
    \end{sfragment}

    \begin{sfragment}[id=sec.git-setup]{GIT-based Setup for the \sTeX Development Version}
        If you want use the latest and greatest \sTeX packages
        that have not even been released to CTAN, 
        then you can directly clone them from the \sTeX development
        repository \cite{sTeX:github:on} by the following command-line instructions: 
        \begin{lstlisting}[language=bash]
        cd <stexdir>
        git clone https://github.com/slatex/sTeX.git
        \end{lstlisting}
        and keep it updated by pulling updates via \lstinline|git pull| in the cloned \sTeX
        directory.
        Make sure to either clone the \sTeX repository into a local texmf-tree or to update your \lstinline|TEXINPUTS| environment variable, e.g. by placing the following line in your \lstinline|.bashrc|:
        \begin{lstlisting}[language=bash]
        export TEXINPUTS="$(TEXINPUTS):<sTeXDIR>//:"
        \end{lstlisting}       
    \end{sfragment}

      \begin{sfragment}{Setting your MathHub Directory}
    One of \sTeX's features is a proper \emph{module system}
    of interconnected document snippets for mathematical
    content. Analogously to \emph{object-oriented programming},
    it allows for ``object-oriented mathematics'' via individual
    combinable and, importantly, \emph{reusable} modules, developed
    collaboratively.

    To make use of such modules, the \sTeX system needs to be told
    where to find them. There are several ways to do so (see
    \sref[file=stex-mathhub]{sec:localmh}[in=../stex-manual,
        title={\href{\basedocurl/stex-manual.pdf}{the \sTeX{}3 Manual}}]),
    but the most convenient way to do so is via a system variable.

    To do so, create a directory \texttt{MathHub} somewhere on
    your local file system and set the environment
    variable \texttt{MATHHUB} to the file path to that directory.

    In linux, you can do so by writing 
    \begin{lstlisting}[language=bash]
        export MATHHUB="/path/to/your/MathHub"
    \end{lstlisting}
    in your \verb|~/.profile| (for all shells) or \verb|~/.bashrc|
    (for the bash terminal only) file.
  \end{sfragment}
    
\end{sfragment}