Given modules:

\stexexample{
    \begin{module}{magma}
        \symdef{universe}{\comp{\mathcal U}}
        \symdef[args=2,op=\circ]{operation}{#1 \comp\circ #2}
    \end{module}
    \begin{module}{monoid}
        \importmodule{magma}
        \symdef{unit}{\comp e}
    \end{module}
    \begin{module}{group}
        \importmodule{monoid}
        \symdef[args=1]{inverse}{{#1}^{\comp{-1}}}
    \end{module}
}

We can form a module for \emph{rings} by ``cloning''
an instance of |group| (for addition) and |monoid| (for multiplication),
respectively, and ``glueing them together'' to ensure they share the
same universe:

\stexexample{
    \begin{module}{ring}
        \begin{copymodule}{group}{addition}
            \renamedecl[name=universe]{universe}{runiverse}
            \renamedecl[name=plus]{operation}{rplus}
            \renamedecl[name=zero]{unit}{rzero}
            \renamedecl[name=uminus]{inverse}{ruminus}
        \end{copymodule}
        \notation[plus,op=+,prec=60]{rplus}{#1 \comp+ #2}
        \notation[zero]{rzero}{\comp0}
        \notation[uminus,op=-]{ruminus}{\comp- #1}
        \begin{copymodule}{monoid}{multiplication}
            \assign{universe}{\runiverse}
            \renamedecl[name=times]{operation}{rtimes}
            \renamedecl[name=one]{unit}{rone}
        \end{copymodule}
        \notation[cdot,op=\cdot,prec=50]{rtimes}{#1 \comp\cdot #2}
        \notation[one]{rone}{\comp1}

        Test: $\rtimes a{\rplus c{\rtimes de}}$
    \end{module}
}

\textcolor{red}{TODO: explain donotclone}


\stexexample{
    \begin{module}{int}
        \symdef{Integers}{\comp{\mathbb Z}}
        \symdef[args=2,op=+]{plus}{#1 \comp+ #2}
        \symdef{zero}{\comp0}
        \symdef[args=1,op=-]{uminus}{\comp-#1}

        \begin{interpretmodule}{group}{intisgroup}
            \assign{universe}{\Integers}
            \assign{operation}{\plus!}
            \assign{unit}{\zero}
            \assign{inverse}{\uminus!}
        \end{interpretmodule}
    \end{module}
}