\begin{sfragment}{The \texttt{mathstructure} Environment}
\begin{smodule}[ns=https://github.com/slatex/sTeX/doc]{MathStructures}
    A common occurence in mathematics is bundling several
    interrelated ``declarations'' together into \emph{structures}.
    For example:
    \begin{itemize}
        \item A \emph{monoid} is a structure $\mathstruct{M,\circ,e}$
            with $\circ:M\times M\to M$ and $e\in M$ such that...
        \item A \emph{topological space} is a structure
            $\mathstruct{X,\mathcal T}$ where $X$ is a set and
            $\mathcal T$ is a topology on $X$
        \item A \emph{partial order} is a structure $\mathstruct{S,\leq}$
            where $\leq$ is a binary relation on $S$ such that...
    \end{itemize}

    This phenomenon is important and common enough to warrant special
    support, in particular because it requires being able
    to \emph{instantiate} such structures (or, ratherer,
    structure \emph{signatures}) in order to talk about (concrete
    or variable) \emph{particular} monoids, topological spaces,
    partial orders etc.

    \begin{environment}{mathstructure}
        The \stexcode"mathstructure" environment allows us to do
        exactly that. It behaves exactly like the
        \stexcode"smodule" environment, but is itself only allowed
        inside an \stexcode"smodule" environment, and allows
        for instantiation later on.
    \end{environment}

    How this works is again best demonstrated by example:
        \symdef{funtype}[args=ai]{#1 \comp\to #2}{##1 \comp\times ##2}
        \symdef{fun}[args=bi]{#1 \comp\mapsto #2}
        \symdecl{set}

        \stexexample{
\begin{mathstructure}{monoid}
    \symdef{universe}[type=\set]{\comp{U}}
    \symdef{op}[
        args=2,
        type=\funtype{\universe,\universe}{\universe},
        op=\circ
    ]{#1 \comp{\circ} #2}
    \symdef{unit}[type=\universe]{\comp{e}}
\end{mathstructure}

A \symname{monoid} is...
        }
        Note that the \stexcode"\symname{monoid}" is appropriately
        highlighted and (depending on your pdf viewer)
        shows a URI on hovering -- implying that the \stexcode"mathstructure"
        environment has generated a \emph{symbol} |monoid| for us.
        It has not generated a semantic macro though, since
        we can not use the |monoid|-symbol \emph{directly}. Instead,
        we can instantiate it, for example for integers:

        \stexexample{
\symdef{Int}[type=\set]{\comp{\mathbb Z}}
\symdef{addition}[
    type=\funtype{\Int,\Int}{\Int},
    args=2,
    op=+
]{##1 \comp{+} ##2}
\symdef{zero}[type=\Int]{\comp{0}}

$\mathstruct{\Int,\addition!,\zero}$ is a \symname{monoid}.
        }

        So far, we have not actually instantiated |monoid|, but now
        that we have all the symbols to do so, we can:

        \stexexample{
\instantiate{intmonoid}{
    universe = Int ,
    op = addition ,
    unit = zero
}{monoid}{\mathbb{Z}_{+,0}}

    $\intmonoid{universe}$, $\intmonoid{unit}$ and $\intmonoid{op}{a}{b}$.

    Also: $\intmonoid!$
        }
        \begin{function}{\instantiate}
            So summarizing:
            \stexcode"\instantiate" takes four arguments: The 
            (macro-)name of the instance, a key-value pair assigning
            declarations in the corresponding \stexcode"mathstructure"
            to symbols currently in scope, the name of the \stexcode"mathstructure"
            to instantiate, and lastly a notation for the instance itself.

            It then generates a semantic macro that takes as argument
            the name of a declaration in the instantiated \stexcode"mathstructure"
            and resolves it to the corresponding instance of that particular declaration.
        \end{function}

        \begin{mmtbox}
            \stexcode"\instantiate" and \stexcode"mathstructure"
            make use of the \emph{Theories-as-Types} paradigm:

            \stexcode"mathstructure{<name>}" does in fact simply create
            a nested theory with name |<name>-structure|. The \emph{constant}
            |<name>| is defined as |Mod(<name>-structure)| -- 
            a \emph{dependent record type with manifest fields}, the fields
            of which are generated from (and correspond to) the constants
            in |<name>-structure|.

            \stexcode"\instantiate" appropriately generates a constant
            whose definiens is a record term of type |Mod(<name>-structure)|,
            with the fields assigned appropriately based on the key-value-list.
        \end{mmtbox}

        Notably, \stexcode"\instantiate" throws an error if not \emph{every}
        declaration in the instantiated \stexcode"mathstructure" is being assigned.
        
        You might consequently ask what the usefulness of \stexcode"mathstructure"
        even is.

        \begin{function}{\varinstantiate}
            The answer is that we can also instantiate a 
            \stexcode"mathstructure" with a \emph{variable}.
            The syntax of \stexcode"\varianstantiate" is equivalent
            to that of \stexcode"\instantiate", but all of the key-value-pairs
            are optional, and if not explicitly assigned (to a symbol \emph{or}
            a variable declared with \stexcode"\vardef") inherit their notation
            from the one in the \stexcode"mathstructure" environment.
        \end{function}

        This allows us to do things like:

        \stexexample{
\varinstantiate{varM}{}{monoid}{M}

A \symname{monoid} is a structure 
$\varM!:=\mathstruct{\varM{universe},\varM{op}!,\varM{unit}}$
such that 
$\varM{op}!:\funtype{\varM{universe},\varM{universe}}{\varM{universe}}$
 and...
        }

        We will return to this example later, when we also know
        how to handle the \emph{axioms} of a monoid.
\end{smodule}
\end{sfragment}

\begin{sfragment}{The \texttt{copymodule} Environment}

    \textcolor{red}{TODO: explain}

Given modules:

\stexexample{
\begin{smodule}{magma}
    \symdef{universe}{\comp{\mathcal U}}
    \symdef{operation}[args=2,op=\circ]{#1 \comp\circ #2}
\end{smodule}
\begin{smodule}{monoid}
    \importmodule{magma}
    \symdef{unit}{\comp e}
\end{smodule}
\begin{smodule}{group}
    \importmodule{monoid}
    \symdef{inverse}[args=1]{{#1}^{\comp{-1}}}
\end{smodule}
}

We can form a module for \emph{rings} by ``cloning''
an instance of |group| (for addition) and |monoid| (for multiplication),
respectively, and ``glueing them together'' to ensure they share the
same universe:

\stexexample{
\begin{smodule}{ring}
    \begin{copymodule}{group}{addition}
        \renamedecl[name=universe]{universe}{runiverse}
        \renamedecl[name=plus]{operation}{rplus}
        \renamedecl[name=zero]{unit}{rzero}
        \renamedecl[name=uminus]{inverse}{ruminus}
    \end{copymodule}
    \notation*{rplus}[plus,op=+,prec=60]{#1 \comp+ #2}
    %\setnotation{rplus}{plus}
    \notation*{rzero}[zero]{\comp0}
    %\setnotation{rzero}{zero}
    \notation*{ruminus}[uminus,op=-]{\comp- #1}
    %\setnotation{ruminus}{uminus}
    \begin{copymodule}{monoid}{multiplication}
        \assign{universe}{\runiverse}
        \renamedecl[name=times]{operation}{rtimes}
        \renamedecl[name=one]{unit}{rone}
    \end{copymodule}
    \notation*{rtimes}[cdot,op=\cdot,prec=50]{#1 \comp\cdot #2}
    %\setnotation{rtimes}{cdot}
    \notation*{rone}[one]{\comp1}
    %\setnotation{rone}{one}
    Test: $\rtimes a{\rplus c{\rtimes de}}$
\end{smodule}
}

\textcolor{red}{TODO: explain donotclone}
    
\end{sfragment}

\begin{sfragment}{The \texttt{interpretmodule} Environment}

    \textcolor{red}{TODO: explain}

\stexexample{
\begin{smodule}{int}
    \symdef{Integers}{\comp{\mathbb Z}}
    \symdef{plus}[args=2,op=+]{#1 \comp+ #2}
    \symdef{zero}{\comp0}
    \symdef{uminus}[args=1,op=-]{\comp-#1}

    \begin{interpretmodule}{group}{intisgroup}
        \assign{universe}{\Integers}
        \assign{operation}{\plus!}
        \assign{unit}{\zero}
        \assign{inverse}{\uminus!}
    \end{interpretmodule}
\end{smodule}
}
    
\end{sfragment}