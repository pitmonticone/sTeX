The \pkg{stex-proof} package supplies macros and environment that allow to annotate the
structure of mathematical proofs in \sTeX document. This structure can be used by MKM
systems for added-value services, either directly from the \sTeX sources, or after
translation.

We will go over the general intuition by way of a running example: 

\begin{latexcode}
\begin{sproof}[id=simple-proof]
   {We prove that $\sum_{i=1}^n{2i-1}=n^{2}$ by induction over $n$}
  \begin{spfcases}{For the induction we have to consider three cases:}
   \begin{spfcase}{$n=1$}
    \begin{spfstep}[type=inline] then we compute $1=1^2$\end{spfstep}
   \end{spfcase}
   \begin{spfcase}{$n=2$}
      \begin{spfcomment}[type=inline]
        This case is not really necessary, but we do it for the
        fun of it (and to get more intuition).
      \end{spfcomment}
      \begin{spfstep}[type=inline] We compute $1+3=2^{2}=4$.\end{spfstep}
   \end{spfcase}
   \begin{spfcase}{$n>1$}
      \begin{spfstep}[type=assumption,id=ind-hyp]
        Now, we assume that the assertion is true for a certain $k\geq 1$,
        i.e. $\sum_{i=1}^k{(2i-1)}=k^{2}$.
      \end{spfstep}
      \begin{spfcomment}
        We have to show that we can derive the assertion for $n=k+1$ from
        this assumption, i.e. $\sum_{i=1}^{k+1}{(2i-1)}=(k+1)^{2}$.
      \end{spfcomment}
      \begin{spfstep}
        We obtain $\sum_{i=1}^{k+1}{2i-1}=\sum_{i=1}^k{2i-1}+2(k+1)-1$
        \spfjust[method=arith:split-sum]{by splitting the sum}.
      \end{spfstep}
      \begin{spfstep}
        Thus we have $\sum_{i=1}^{k+1}{(2i-1)}=k^2+2k+1$
        \spfjust[method=fertilize]{by inductive hypothesis}.
      \end{spfstep}
      \begin{spfstep}[type=conclusion]
        We can \spfjust[method=simplify]{simplify} the right-hand side to
        ${k+1}^2$, which proves the assertion.
      \end{spfstep}
   \end{spfcase}
    \begin{spfstep}[type=conclusion]
      We have considered all the cases, so we have proven the assertion.
    \end{spfstep}
  \end{spfcases}
\end{sproof}
\end{latexcode}

This yields the following result: 

\begin{mdframed}
  \begin{sproof}[id=simple-proof]
  {We prove that $\sum_{i=1}^n{2i-1}=n^{2}$ by induction over $n$}
  \begin{spfcases}{For the induction we have to consider the following cases:}
    \begin{spfcase}{$n=1$}
      \begin{spfstep}[type=inline] then we compute $1=1^2$\end{spfstep}
    \end{spfcase}
    \begin{spfcase}{$n=2$}
      \begin{spfcomment}[type=inline]
         This case is not really necessary, but we do it for the fun
         of it (and to get more intuition).
      \end{spfcomment}
      \begin{spfstep}[type=inline]
         We compute $1+3=2^{2}=4$
      \end{spfstep}
    \end{spfcase}
    \begin{spfcase}{$n>1$}
      \begin{spfstep}[type=hypothesis,id=ind-hyp]
        Now, we assume that the assertion is true for a certain $k\geq 1$, i.e.
        $\sum_{i=1}^k{(2i-1)}=k^{2}$.
      \end{spfstep}
      \begin{spfcomment}
        We have to show that we can derive the assertion for $n=k+1$ from this
        assumption, i.e.  $\sum_{i=1}^{k+1}{(2i-1)}=(k+1)^{2}$.
      \end{spfcomment}
      \begin{spfstep}[id=splitit]
        We obtain $\sum_{i=1}^{k+1}{(2i-1)}=\sum_{i=1}^k{(2i-1)}+2(k+1)-1$
       \spfjust[method=arith:split-sum]{by splitting the sum}.
     \end{spfstep}
     \begin{spfstep}[id=byindhyp]
       Thus we have $\sum_{i=1}^{k+1}{(2i-1)}=k^2+2k+1$
       \spfjust[method=fertilize]{by \premise[ind-hyp]{inductive hypothesis}}.
     \end{spfstep}
     \begin{spfstep}[type=conclusion]
       We can \spfjust[method=simplify-eq]{simplify the \justarg[rhs]{right-hand side}} to
       $(k+1)^2$, which proves the assertion.
     \end{spfstep}
   \end{spfcase}
   \begin{spfstep}[type=conclusion]
     We have considered all the cases, so we have proven the assertion.
   \end{spfstep}
  \end{spfcases}
\end{sproof}
\end{mdframed}

\begin{environment}{sproof}
  The |sproof| environment is the main container for proofs. It takes an optional |KeyVal|
  argument that allows to specify the |id| (identifier) and |for| (for which assertion is
  this a proof) keys. The regular argument of the |proof| environment contains an
  introductory comment, that may be used to announce the proof style. The |proof|
  environment contains a sequence of |spfstep|, |spfcomment|, and |spfcases| environments
  that are used to markup the proof steps.
\end{environment}
  
\begin{function}{\spfidea}
  The |\spfidea| macro allows to give a one-paragraph description of the proof idea.
\end{function}

\begin{function}{\spfsketch}
  For one-line proof sketches, we use the |\spfsketch| macro, which takes the same
  optional argument as |sproof| and another one: a natural language text that sketches
  the proof.
\end{function}

\begin{environment}{spfstep}
  Regular proof steps are marked up with the |step| environment, which takes an optional
  |KeyVal| argument for annotations. A proof step usually contains a local assertion
  (the text of the step) together with some kind of evidence that this can be derived
  from already established assertions.
\end{environment}

\begin{function}{\spfjust}
  This evidence is marked up with the |\spfjust| macro in the \pkg{stex-proofs}
  package. This environment totally invisible to the formatted result; it wraps the text
  in the proof step that corresponds to the evidence. The environment takes an optional
  |KeyVal| argument, which can have the |method| key, whose value is the name of a proof
  method (this will only need to mean something to the application that consumes the
  semantic annotations). Furthermore, the justification can contain ``premises''
  (specifications to assertions that were used justify the step) and ``arguments''
  (other information taken into account by the proof method).
\end{function}

\begin{function}{\premise}
  The |\premise| macro allows to mark up part of the text as reference to an assertion
  that is used in the argumentation. In the running example we have used the |\premise|
  macro to identify the inductive hypothesis.
\end{function}

\begin{function}{\justarg}
  The |\justarg| macro is very similar to |\premise| with the difference that it is used
  to mark up arguments to the proof method. Therefore the content of the first argument
  is interpreted as a mathematical object rather than as an identifier as in the case of
  |\premise|. In our example, we specified that the simplification should take place on
  the right hand side of the equation. Other examples include proof methods that
  instantiate. Here we would indicate the substituted object in a |\justarg| macro.
\end{function}

Note that both |\premise| and |\justarg| can be used with an empty second argument to
mark up premises and arguments that are not explicitly mentioned in the text.

\begin{environment}{subproof}
  The |spfcases| environment is used to mark up a subproof. This environment takes an
  optional |KeyVal| argument for semantic annotations and a second argument that allows
  to specify an introductory comment (just like in the |proof| environment). The
  |method| key can be used to give the name of the proof method
  executed to make this subproof.
\end{environment}

\begin{environment}{spfcases}
  The |spfcases| environment is used to mark up a proof by cases. Technically it is a
  variant of the |subproof| where the |method| is |by-cases|. Its contents are |spfcase|
  environments that mark up the cases one by one.
\end{environment}

\begin{environment}{spfcase}
  The content of a |spfcases| environment are a sequence of case proofs marked up in the
  |spfcase| environment, which takes an optional |KeyVal| argument for semantic
  annotations. The second argument is used to specify the the description of the case
  under consideration. The content of a |spfcase| environment is the same as that of a
  |sproof|, i.e. |spfstep|s, |spfcomment|s, and |spfcases| environments.
\end{environment}

\begin{function}{\spfcasesketch}
  |\spfcasesketch| is a variant of the |spfcase| environment that takes the same
  arguments, but instead of the |spfstep|s in the body uses a third argument for a proof
  sketch.
\end{function}

\begin{environment}{spfcomment}
  The |spfcomment| environment is much like a |step|, only that it does not have an
  object-level assertion of its own. Rather than asserting some fact that is relevant
  for the proof, it is used to explain where the proof is going, what we are attempting
  to to, or what we have achieved so far. As such, it cannot be the target of a
  |\premise|.
\end{environment}

\begin{function}{\sproofend}
  Traditionally, the end of a mathematical proof is marked with a little box at the end of
  the last line of the proof (if there is space and on the end of the next line if there
  isn't), like so:\sproofend

  The \pkg{stex-proofs} package provides the |\sproofend| macro for this.
\end{function}
  
\begin{variable}{\sProofEndSymbol}
  If a different symbol for the proof end is to be used (e.g. {\sl{q.e.d}}), then this can
  be obtained by specifying it using the |\sProofEndSymbol| configuration macro (e.g. by
  specifying |\sProofEndSymbol{q.e.d}|).
\end{variable}
  
Some of the proof structuring macros above will insert proof end symbols for sub-proofs,
in most cases, this is desirable to make the proof structure explicit, but sometimes this
wastes space (especially, if a proof ends in a case analysis which will supply its own
proof end marker). To suppress it locally, just set |proofend={}| in them or use use
|\sProofEndSymbol{}|.

%%% Local Variables:
%%% mode: latex
%%% TeX-master: "../stex-manual"
%%% End:

%  LocalWords:  hypothesis,id geq splitit arith:split-sum byindhyp rhs proofend
