We can use \sTeX by simply including the package with |\usepackage{stex}|,
or -- primarily for individual fragments to be included in other
documents -- by using the \sTeX document class with |\documentclass{stex}|
which combines the \pkg{standalone} document class with the \pkg{stex}
package.

Both the \pkg{stex} package and document class offer the following
options:

\begin{description}
   \item[\texttt{lang}] (\meta{language}$\ast$) Languages
     to load with the \pkg{babel} package.
   \item[\texttt{mathhub}] (\meta{directory}) MathHub folder
     to search for repositories -- this is not necessary if the
     |MATHHUB| system variable is set.
   \item[\texttt{sms}] (\meta{boolean}) use \emph{persisted}
     mode (not yet implemented).
   \item[\texttt{image}] (\meta{boolean}) passed on to
     \pkg{tikzinput}.
   \item[\texttt{debug}] (\meta{log-prefix}$\ast$) Logs debugging
     information with the given prefixes to the terminal,
     or all if |all| is given. Largely irrelevant for the
     majority of users.
\end{description}

%%% Local Variables:
%%% mode: latex
%%% TeX-master: "../stex-manual"
%%% End:
