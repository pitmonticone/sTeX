\begin{sfragment}{Definitions, Theorems, Examples, Paragraphs}
\begin{smodule}{Statements}
    As mentioned earlier, we can semantically mark-up
    \emph{statements} such as definitions, theorems, lemmata, examples, etc.

    The corresponding environments for that are:
    \begin{itemize}
        \item \stexcode"sdefinition" for definitions,
        \item \stexcode"sassertion" for assertions, i.e.
            propositions that are declared to be \emph{true},
            such as theorems, lemmata, axioms,
        \item \stexcode"sexample" for examples, and
        \item \stexcode"sparagraph" for other semantic paragraphs,
            such as comments, remarks, conjectures, etc.
    \end{itemize}

    The \emph{presentation} of these environments can be customized
    to use e.g. predefined |theorem|-environments, see \sref{sec.customhighlight}
    for details.

    All of these environments take optional arguments
    in the form of |key=value|-pairs. Common to all of them are
    the keys |id=| (for cross-referencing, see \sref{sec.references}),
    |type=| for customization (see \sref{sec.customhighlight})
    and additional information (e.g. definition principles,
    ``difficulty'' etc), |title=|, and |for=|.

    The |for=| key expects a comma-separated list of existing
    symbols, allowing for e.g. things like
    \symdef{addition}[args=a,prec=100]{#1}{##1 \comp+ ##2}
    \symdef{multiplication}[args=a,prec=50]{#1}{##1 \comp\cdot ##2}
    \stexexample{
\begin{sexample}[
    id=additionandmultiplication.ex,
    for={addition,multiplication},
    type={trivial,boring},
    title={An Example}
]
    $\addition{2,3}$ is $5$, $\multiplication{2,3}$ is $6$.
\end{sexample}
    }

    \begin{function}{\definiendum,\definame,\definiens}
        \stexcode"sdefinition" (and \stexcode"sparagraph" with 
        |type=symdoc|) introduce three new macros:
        \stexcode"definiendum" behaves like \stexcode"symref"
        (and \stexcode"definame" like \stexcode"symname"),
        but highlights the references symbol as \emph{being defined}
        in the current definition.

        \stexcode"\definiens[<optional symbolname>]{<code>}"
        marks up |<code>| as being the explicit \emph{definiens}
        of |<optional symbolname>| (in case |for=| has multiple
        symbols).
    \end{function}

    \begin{mmtbox}
        The special |type=symdoc| for \stexcode"sparagraph"
    is intended to be used for ``informal definitions'', or
    encyclopedia-style descriptions for symbols. 
    
    The \mmt-system can use those
    (in lieu of an actual \stexcode"sdefinition" in scope)
    to present to users, e.g. when hovering over symbols.
    \end{mmtbox}

    All four environments also take an optional parameter
    |name=| -- if this one is given a value, the environment
    will generate a \emph{symbol} by that name (but with no semantic
    macro). Not only does this allow for \stexcode"\symref" et al,
    it allows us to resume our earlier example for monoids 
    much more nicely:

    \symdef{set}{\comp{\texttt{Set}}}
    \symdef{equal}[args=2]{#1 \comp= #2}
    \symdef{inset}[args=2]{#1 \comp\in #2}
    \symdef{funtype}[args=ai]{#1 \comp\to #2}{##1 \comp\times ##2}

    \stexexample{
\begin{mathstructure}{monoid}
    \symdef{universe}[type=\set]{\comp{U}}
    \symdef{op}[
        args=2,
        type=\funtype{\universe,\universe}{\universe},
        op=\circ
    ]{#1 \comp{\circ} #2}
    \symdef{unit}[type=\universe]{\comp{e}}

    \begin{sparagraph}[type=symdoc,for=monoid]
        A \definame{monoid} is a structure
        $\mathstruct{\universe,\op!,\unit}$
        where $\op!:\funtype{\universe}{\universe}$ and
        $\inset{\unit}{\universe}$ such that

        \begin{sassertion}[name=associative,
            type=axiom,
            title=Associativity]
            $\op!$ is associative
        \end{sassertion}
        \begin{sassertion}[name=isunit,
            type=axiom,
            title=Unit]
            $\equal{\op{\svar{x}}{\unit}}{\svar{x}}$
            for all $\inset{\svar{x}}{\universe}$
        \end{sassertion}
    \end{sparagraph}
\end{mathstructure}

An example for a \symname{monoid} is...
    }

    Now the \stexcode"mathstructure" \symname{monoid} contains
    two additional symbols, namely the axioms for associativity
    and that $e$ is a unit. Note that both symbols do not
    represent the mere \emph{propositions} that e.g. 
    $\circ$ is associative, but \emph{the assertion that it is
    actually true} that $\circ$ is associative.
    
    If we now want to instantiate |monoid| (unless with a variable,
    of course), we also need to assign |associative| and |neutral|
    to analogous assertions. So the earlier example
    \begin{latexcode}[gobble=8]
        \instantiate{intmonoid}{
            universe = Int ,
            op = addition ,
            unit = zero
        }{monoid}{\mathbb{Z}_{+,0}}
    \end{latexcode}
    ...will not work anymore. We now need to give assertions that
    |addition| is associative and that |zero| is a unit with respect
    to addition.\footnote{Of course, \sTeX can not check that
    the assertions are the ``correct'' ones -- but if
    the assertions (both in |monoid| as well as those for addition and
    zero) are properly marked up, \mmt can. \textcolor{red}{TODO: should}}

\end{smodule}
\end{sfragment}