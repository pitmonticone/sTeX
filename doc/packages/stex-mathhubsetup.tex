  \begin{sfragment}{Setting your MathHub Directory}
    One of \sTeX's features is a proper \emph{module system}
    of interconnected document snippets for mathematical
    content. Analogously to \emph{object-oriented programming},
    it allows for ``object-oriented mathematics'' via individual
    combinable and, importantly, \emph{reusable} modules, developed
    collaboratively.

    To make use of such modules, the \sTeX system needs to be told
    where to find them. There are several ways to do so (see
    \sref[file=stex-mathhub]{sec:localmh}[in=../stex-manual,
        title={\href{\basedocurl/stex-manual.pdf}{the \sTeX{}3 Manual}}]),
    but the most convenient way to do so is via a system variable.

    To do so, create a directory \texttt{MathHub} somewhere on
    your local file system and set the environment
    variable \texttt{MATHHUB} to the file path to that directory.

    In linux, you can do so by writing 
    \begin{lstlisting}[language=bash]
        export MATHHUB="/path/to/your/MathHub"
    \end{lstlisting}
    in your \verb|~/.profile| (for all shells) or \verb|~/.bashrc|
    (for the bash terminal only) file.
  \end{sfragment}